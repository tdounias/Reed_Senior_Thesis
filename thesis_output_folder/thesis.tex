% This is the Reed College LaTeX thesis template. Most of the work
% for the document class was done by Sam Noble (SN), as well as this
% template. Later comments etc. by Ben Salzberg (BTS). Additional
% restructuring and APA support by Jess Youngberg (JY).
% Your comments and suggestions are more than welcome; please email
% them to cus@reed.edu
%
% See http://web.reed.edu/cis/help/latex.html for help. There are a
% great bunch of help pages there, with notes on
% getting started, bibtex, etc. Go there and read it if you're not
% already familiar with LaTeX.
%
% Any line that starts with a percent symbol is a comment.
% They won't show up in the document, and are useful for notes
% to yourself and explaining commands.
% Commenting also removes a line from the document;
% very handy for troubleshooting problems. -BTS

% As far as I know, this follows the requirements laid out in
% the 2002-2003 Senior Handbook. Ask a librarian to check the
% document before binding. -SN

%%
%% Preamble
%%
% \documentclass{<something>} must begin each LaTeX document
\documentclass[12pt,twoside]{reedthesis}
% Packages are extensions to the basic LaTeX functions. Whatever you
% want to typeset, there is probably a package out there for it.
% Chemistry (chemtex), screenplays, you name it.
% Check out CTAN to see: http://www.ctan.org/
%%
\usepackage{graphicx,latexsym}
\usepackage{amsmath}
\usepackage{amssymb,amsthm}
\usepackage{longtable,booktabs,setspace}
\usepackage{chemarr} %% Useful for one reaction arrow, useless if you're not a chem major
\usepackage[hyphens]{url}
% Added by CII
\usepackage{hyperref}
\usepackage{lmodern}
% End of CII addition
\usepackage{rotating}

% Next line commented out by CII
%%% \usepackage{natbib}
% Comment out the natbib line above and uncomment the following two lines to use the new 
% biblatex-chicago style, for Chicago A. Also make some changes at the end where the 
% bibliography is included. 
%\usepackage{biblatex-chicago}
%\bibliography{thesis}


% Added by CII (Thanks, Hadley!)
% Use ref for internal links
\renewcommand{\hyperref}[2][???]{\autoref{#1}}
\def\chapterautorefname{Chapter}
\def\sectionautorefname{Section}
\def\subsectionautorefname{Subsection}
% End of CII addition

% Added by CII 
\usepackage{caption}
\captionsetup{width=5in}
% End of CII addition

% \usepackage{times} % other fonts are available like times, bookman, charter, palatino


% To pass between YAML and LaTeX the dollar signs are added by CII
\title{Turnout and Mail Voting in Colorado; or How I Learned to Stop Worrying
and Love Voter Registration Files}
\author{Theodore Dounias}
% The month and year that you submit your FINAL draft TO THE LIBRARY (May or December)
\date{December 2018}
\division{Mathematics and Natural Sciences and History and Social Sciences}
\advisor{Paul Gronke}
%If you have two advisors for some reason, you can use the following
% Uncommented out by CII
\altadvisor{Andrew Bray} 
% End of CII addition

%%% Remember to use the correct department!
\department{Mathematics and Political Science}
% if you're writing a thesis in an interdisciplinary major,
% uncomment the line below and change the text as appropriate.
% check the Senior Handbook if unsure.
%\thedivisionof{The Established Interdisciplinary Committee for}
% if you want the approval page to say "Approved for the Committee",
% uncomment the next line
%\approvedforthe{Committee}

% Added by CII
%%% Copied from knitr
%% maxwidth is the original width if it's less than linewidth
%% otherwise use linewidth (to make sure the graphics do not exceed the margin)
\makeatletter
\def\maxwidth{ %
  \ifdim\Gin@nat@width>\linewidth
    \linewidth
  \else
    \Gin@nat@width
  \fi
}
\makeatother

\renewcommand{\contentsname}{Table of Contents}
% End of CII addition

\setlength{\parskip}{0pt}

% Added by CII

\providecommand{\tightlist}{%
  \setlength{\itemsep}{0pt}\setlength{\parskip}{0pt}}

\Acknowledgements{

}

\Dedication{

}

\Preface{
I started my second attempt at this preface by reading through my
maternal grandfather's eulogy. In my bloodline I have researchers,
engineers, teachers, social scientists, and literary scholars, all with
astoundingly different approaches to stemming their curiosity about the
world around them. All of them, though, have in common that they have
fought long and hard for me to have what I do now, in my privilege to
study at Reed College and live a financially secure life, but also in my
ability to determine my future in a democratic state. They all fought
for Greece to be as free as it is today, even though our elections may
often deliver painful results. I find it very fitting that the way I
honor their legacy is by studying the system that they helped
create\footnote{And by also veering wildly away from any of their own
  previous research interests, much to the occasional chagrin of my
  close relatives, and much in the same fashion that they seem to have
  steered clear of their own parents.}. \par Of course, my object of
study is not Greek, but American elections. I made this choice first by
necessity, since the data access and previous literature is
exponentially greater. I soon realized, however, that the puzzles that
US elections present are both generalizable and deeply intriguing. I was
taken in by the complexity of translating the will of the people for the
world's greatest economic and military power into political action and
by how the most initially mundane policy choices could have massive
impacts on representation. As I mention again in my introduction,
democracies are based on procedures as much as principles, and I am now
very happy to have completed my first pass at adding to the scientific
knowledge available for one such procedure. \par Here I must mention my
first thesis adviser, and in fact my academic adviser and guide
throughout my years at Reed, Professor Paul Gronke. Paul as a person
transmits emphatic dedication and love to the subject of his research,
to the extent that it is hard not to be taken along for the ride. He
helped me get back my sense of direction that I had lost before
transferring to Reed, he was the first to suggest I do an
interdisciplinary thesis, and he stuck with me and worked incredibly
hard to guide me through the process. While at times not giving me the
answer I wanted to hear, he also helped ground me and calm me down when
my own sense of anxiety was getting the better of me, and I can't thank
him enough for putting up with that as well. \par I am, of course, an
interdisciplinary major. This is partly by virtue of the academic
credits I gained from accidentally trying to be an engineer for two
years after graduating high school, but mostly because of the people who
have taught me mathematics. My mathematical education was always deeply
grounded in applicability and the elegance of statistically
approximating real-world phenomena. At Reed College I met my second
adviser, Andrew Bray, whose calm and deeply sincere admiration for
statistical applications is infectious. Andrew often pushed me to
accomplish tasks that I felt were beyond my capacity, but despite my
protests I always managed to get done with his help. Because of him I
feel that I not only know how to apply the methods I use, but that I
really \emph{understand} them (or at least understand how much I
\emph{do not} understand). He has worked incredibly hard, sacrificing
time from his sabbatical to oversee my senior thesis, and I cannot thank
him enough. \par I would like to thank Andrew Menger, Judd Choate, and
Robert Stein. Andrew Menger, Postdoctoral Fellow at the Weidenbaum
Center on the Economy, Government, and Public Policy at Washington
University, was gracious enough to share the data that he and his team
collected and are currently working on; he saved my thesis by giving me
access to voter files from 2012-2016, while I only had access to the
2017 file. Without his help, I would not have been able to complete my
analysis. Judd Choate, Director of Elections for Colorado was
instrumental in helping me procure data for 2017, from which I started
my thesis. He also helped me troubleshoot data issues by connecting me
with his team. Robert Stein, Lena Goldman Fox Professor of Political
Science at Rice University, also gave me guidance when I ran into data
problems common in his field. \par I would like to also show my
appreciation to the Reed College Computer User Services that loaned me
the computer I am using to run my models, and saving me what I can
safely assume to be several months worth of listening to my own laptop's
fans screaming for mercy while I age, considerably. Paul Manson and my
fellow Reedie Jay Lee were also helpful in solving issues with
presentation, mapping, tables, and the R Markdown thesis template. The
template itself was coded by Chester Ismay. I thank all of them.
\par There are so many people at Reed and back home that have been
important to me that I will almost certainly fail at mentioning them
all. Consider the following a woefully incomplete list. To GameDEV, my
dorm, you eternally empathetic, thoughtful, and warm friends, with whom
I lived out my Reed years in full force. There is not much to say other
than that I will forever miss being part of your community, and I am
proud of you all. To the Reed Mock Trial Team, who helped me
procrastinate and escape campus with good company when I needed it most.
You are all awesome and I am sure you will get to ORCS this year!
\par To Kathy Saitas, who gave me a home away from home from my first
week here. To all my Professors that took interest in me and helped me
along the way. To Joanne Buzaglo and her family, that taught me to be a
Philly sports fan and gave me a home for fall breaks. I thank all of you
for your kindness. \par To my innumerable friends back home, I apologize
for not having the space for all of you here. Suffice to say, if you
ever played DotA with me while I was here, if we played soccer back
home, or shared a beer, or sang, or discussed this year's Fantasy
Premier League, then consider this as an expression of gratitude for
being there for me. I miss you all. \par Last, but most importantly, my
family: Theodoros, Eftychia, Julius, Poly, my uncle and aunt Christos
and Christina, my cousin (now Dr.) Philip and my parents George and
Marina. You have given me a life worth living, a home worth missing, and
a world worth fighting for.
\(\Theta o \delta \omega \varrho\acute\eta\varsigma~N\tau o\upsilon \nu \iota \acute{\alpha}\varsigma\)
\par
}

\Abstract{
Mail voting in the United States was conceived and first implemented to
serve absentee voters during the Civil War (Fortier, 2006) and has
persisted until the present day, becoming one of the key reforms
associated with ``convenience voting'' and the expansion of the
democratic franchise (Gronke, Galanes-Rosenbaum, Miller, \& Toffey,
2008). In 2013, Colorado implemented the latest in a series of in-state
election reforms and became the third state in the nation with universal
mail voting for all elections, after Oregon and Washington. Despite
claims by policymakers that mail voting should have a strong, positive
effect on voter turnout, a recent series of studies on Oregon,
Washington, parts of California, and Colorado have failed to show
consistent results, disagreeing both on the scale and the direction
(positive or negative) that this effect has. This thesis aims at
following this series of studies by examining Colorado voter
registration files for recent elections (2010-2016). These files consist
of a registration file with voter information and a history file with
voter participation data in Colorado elections, and provide all
information necessary for a comprehensive study of turnout. By
describing, fitting, and interpreting multilevel general additive
regression models of voter turnout based on these data, I show that
there is a small positive effect of mail voting on turnout in national
elections at the county level. This thesis also contributes to the
literature by presenting a description of modeling and data wrangling
difficulties associated with voter registration files, and giving a
series of potential solutions, as well as an extensive coding library to
aid future research on the subject.
}

% End of CII addition
%%
%% End Preamble
%%
%

\begin{document}

% Everything below added by CII
      \maketitle
  
  \frontmatter % this stuff will be roman-numbered
  \pagestyle{empty} % this removes page numbers from the frontmatter

  
      \begin{preface}
      I started my second attempt at this preface by reading through my
      maternal grandfather's eulogy. In my bloodline I have researchers,
      engineers, teachers, social scientists, and literary scholars, all with
      astoundingly different approaches to stemming their curiosity about the
      world around them. All of them, though, have in common that they have
      fought long and hard for me to have what I do now, in my privilege to
      study at Reed College and live a financially secure life, but also in my
      ability to determine my future in a democratic state. They all fought
      for Greece to be as free as it is today, even though our elections may
      often deliver painful results. I find it very fitting that the way I
      honor their legacy is by studying the system that they helped
      create\footnote{And by also veering wildly away from any of their own
        previous research interests, much to the occasional chagrin of my
        close relatives, and much in the same fashion that they seem to have
        steered clear of their own parents.}. \par Of course, my object of
      study is not Greek, but American elections. I made this choice first by
      necessity, since the data access and previous literature is
      exponentially greater. I soon realized, however, that the puzzles that
      US elections present are both generalizable and deeply intriguing. I was
      taken in by the complexity of translating the will of the people for the
      world's greatest economic and military power into political action and
      by how the most initially mundane policy choices could have massive
      impacts on representation. As I mention again in my introduction,
      democracies are based on procedures as much as principles, and I am now
      very happy to have completed my first pass at adding to the scientific
      knowledge available for one such procedure. \par Here I must mention my
      first thesis adviser, and in fact my academic adviser and guide
      throughout my years at Reed, Professor Paul Gronke. Paul as a person
      transmits emphatic dedication and love to the subject of his research,
      to the extent that it is hard not to be taken along for the ride. He
      helped me get back my sense of direction that I had lost before
      transferring to Reed, he was the first to suggest I do an
      interdisciplinary thesis, and he stuck with me and worked incredibly
      hard to guide me through the process. While at times not giving me the
      answer I wanted to hear, he also helped ground me and calm me down when
      my own sense of anxiety was getting the better of me, and I can't thank
      him enough for putting up with that as well. \par I am, of course, an
      interdisciplinary major. This is partly by virtue of the academic
      credits I gained from accidentally trying to be an engineer for two
      years after graduating high school, but mostly because of the people who
      have taught me mathematics. My mathematical education was always deeply
      grounded in applicability and the elegance of statistically
      approximating real-world phenomena. At Reed College I met my second
      adviser, Andrew Bray, whose calm and deeply sincere admiration for
      statistical applications is infectious. Andrew often pushed me to
      accomplish tasks that I felt were beyond my capacity, but despite my
      protests I always managed to get done with his help. Because of him I
      feel that I not only know how to apply the methods I use, but that I
      really \emph{understand} them (or at least understand how much I
      \emph{do not} understand). He has worked incredibly hard, sacrificing
      time from his sabbatical to oversee my senior thesis, and I cannot thank
      him enough. \par I would like to thank Andrew Menger, Judd Choate, and
      Robert Stein. Andrew Menger, Postdoctoral Fellow at the Weidenbaum
      Center on the Economy, Government, and Public Policy at Washington
      University, was gracious enough to share the data that he and his team
      collected and are currently working on; he saved my thesis by giving me
      access to voter files from 2012-2016, while I only had access to the
      2017 file. Without his help, I would not have been able to complete my
      analysis. Judd Choate, Director of Elections for Colorado was
      instrumental in helping me procure data for 2017, from which I started
      my thesis. He also helped me troubleshoot data issues by connecting me
      with his team. Robert Stein, Lena Goldman Fox Professor of Political
      Science at Rice University, also gave me guidance when I ran into data
      problems common in his field. \par I would like to also show my
      appreciation to the Reed College Computer User Services that loaned me
      the computer I am using to run my models, and saving me what I can
      safely assume to be several months worth of listening to my own laptop's
      fans screaming for mercy while I age, considerably. Paul Manson and my
      fellow Reedie Jay Lee were also helpful in solving issues with
      presentation, mapping, tables, and the R Markdown thesis template. The
      template itself was coded by Chester Ismay. I thank all of them.
      \par There are so many people at Reed and back home that have been
      important to me that I will almost certainly fail at mentioning them
      all. Consider the following a woefully incomplete list. To GameDEV, my
      dorm, you eternally empathetic, thoughtful, and warm friends, with whom
      I lived out my Reed years in full force. There is not much to say other
      than that I will forever miss being part of your community, and I am
      proud of you all. To the Reed Mock Trial Team, who helped me
      procrastinate and escape campus with good company when I needed it most.
      You are all awesome and I am sure you will get to ORCS this year!
      \par To Kathy Saitas, who gave me a home away from home from my first
      week here. To all my Professors that took interest in me and helped me
      along the way. To Joanne Buzaglo and her family, that taught me to be a
      Philly sports fan and gave me a home for fall breaks. I thank all of you
      for your kindness. \par To my innumerable friends back home, I apologize
      for not having the space for all of you here. Suffice to say, if you
      ever played DotA with me while I was here, if we played soccer back
      home, or shared a beer, or sang, or discussed this year's Fantasy
      Premier League, then consider this as an expression of gratitude for
      being there for me. I miss you all. \par Last, but most importantly, my
      family: Theodoros, Eftychia, Julius, Poly, my uncle and aunt Christos
      and Christina, my cousin (now Dr.) Philip and my parents George and
      Marina. You have given me a life worth living, a home worth missing, and
      a world worth fighting for.
      \(\Theta o \delta \omega \varrho\acute\eta\varsigma~N\tau o\upsilon \nu \iota \acute{\alpha}\varsigma\)
      \par
    \end{preface}
  
      \hypersetup{linkcolor=black}
    \setcounter{tocdepth}{2}
    \tableofcontents
  
      \listoftables
  
      \listoffigures
  
      \begin{abstract}
      Mail voting in the United States was conceived and first implemented to
      serve absentee voters during the Civil War (Fortier, 2006) and has
      persisted until the present day, becoming one of the key reforms
      associated with ``convenience voting'' and the expansion of the
      democratic franchise (Gronke, Galanes-Rosenbaum, Miller, \& Toffey,
      2008). In 2013, Colorado implemented the latest in a series of in-state
      election reforms and became the third state in the nation with universal
      mail voting for all elections, after Oregon and Washington. Despite
      claims by policymakers that mail voting should have a strong, positive
      effect on voter turnout, a recent series of studies on Oregon,
      Washington, parts of California, and Colorado have failed to show
      consistent results, disagreeing both on the scale and the direction
      (positive or negative) that this effect has. This thesis aims at
      following this series of studies by examining Colorado voter
      registration files for recent elections (2010-2016). These files consist
      of a registration file with voter information and a history file with
      voter participation data in Colorado elections, and provide all
      information necessary for a comprehensive study of turnout. By
      describing, fitting, and interpreting multilevel general additive
      regression models of voter turnout based on these data, I show that
      there is a small positive effect of mail voting on turnout in national
      elections at the county level. This thesis also contributes to the
      literature by presenting a description of modeling and data wrangling
      difficulties associated with voter registration files, and giving a
      series of potential solutions, as well as an extensive coding library to
      aid future research on the subject.
    \end{abstract}
  
  
  \mainmatter % here the regular arabic numbering starts
  \pagestyle{fancyplain} % turns page numbering back on

  \chapter*{Introduction}\label{introduction}
  \addcontentsline{toc}{chapter}{Introduction}
  
  Elections shift the course of history in ways that are sometimes small,
  but often enormous. Particularly in the United States, but really in any
  democracy across the world, elections have the capacity to shape the
  existence of millions of people for decades to come. Elections are
  written in the history books, regardless of the final outcome. It is
  truly wondrous that this power is in \emph{our} hands, that every
  seismic shift in the course of our states comes from the tremors each of
  us cause with our ballots. This is a power that people have suffered and
  continue to suffer for, a power my own parents and grandparents risked
  their lives for. The way we give people access to this power, the way
  nations like the United States or my birthplace of Greece choose to
  conduct their elections and the enfranchisement and disenfranchisement
  such choices may cause has never ceased to be contentious.
  
  What quickly becomes apparent is that beyond the soaring rhetoric of
  democratic power, the setup of our elections systems is not
  straightforward in the slightest. Starting from the beginning, there is
  still no consensus on \emph{who} gets to vote. Do felons get the vote?
  Florida (as of 2018) says yes, as does much of Europe, but other states
  in the US say no. How about permanent residents? Should people have to
  pass a basic political knowledge test to vote? Is one informed vote more
  important than a vote from someone who lacks knowledge of current
  affairs and government? Should people be forced to show ID when voting,
  or is this restrictive and undemocratic? \emph{When} people get to vote
  is similarly contentious. Is early voting something we should support,
  or does it dilute the effect of election day turnout? Should voting day
  be a national holiday? How about refferenda? Should the people be
  directly consulted by their governments on seminal issues? Finally,
  there is the question of \emph{how} or \emph{where} people vote. How
  many polling places are enough? Should we use paper ballots or voting
  machines? Should people be automatically registered to vote? Should they
  be allowed to register on election day? Do we freeze registrations
  before election day, and if so how long before? Should we allow voting
  by mail? Permanent absentee status? All-mail elections? This is all even
  before making choices on subjects like the allocation of representatives
  by state or prefecture, or the total number of parliamentarians, or how
  many chambers congress should have.
  
  The 2000 Presidential election between Al Gore and George W. Bush was
  decided in Florida by a razor thin margin of ballots. Following the
  final result, which occurred in the halls of the United States Supreme
  Court whose decision in \emph{Bush v. Gore} halted all Florida recounts
  that were in progress, researchers claimed that the voting machines and
  ballot cards used were confusing and misleading. Many voters, for
  example, mistakenly voted for a different candidate because of the
  confusing layout of outdated punch-card voting machines(Wand et al.,
  2001). 22,000 voters in Duval County had their ballots rejected due to
  ``over-voting'', because they were given ballots that implied the
  necessity to vote on multiple pages for the same candidate: an action
  that spoiled their ballot(Saltman, 2009).
  
  A democratic system is based on procedures as much as on principles.
  Elections are about translating the vote of the people into political
  power, government action, and fair representation by chosen
  representatives. The way we vote can often be critical to the outcome.
  Thus the design and implementation of voting systems is far from being
  neutral; the decisions made on who votes, and how, when, and where they
  do so often serves to alter the course of history. Underlying those
  decisions is a nebulous, inconclusively answered question: are elections
  fair, and how can we make them more so?
  
  The United States has long grappled with translating this goal into
  policy. During the Civil War, several states like Virginia pioneered the
  use of absentee mail ballots to serve military personnel and displaced
  residents; this same policy was expanded nationwide in 1942, to
  accommodate for soldiers and factory workers serving US efforts in World
  War II. In 1871 Congress passed the Enforcement Act, which clearly
  defined acts like voter impersonation or intimidation as crimes, and set
  up a clear, federal structure of supervisors and marshals tasked with
  overseeing all elections. This was the first in what was to become over
  a century of sweeping federal election reforms, in an attempt to at
  least partially centralize an election process whose control had been
  left to localities. The Civil Rights Acts of '57, '60, and '64 along
  with the Voting Rights Act of '65 collectively aimed to prevent
  discriminatory policies at the local level. They abolished literacy
  tests, made tampering with voter registration rolls a federal crime, and
  mandated federal oversight of local authorities with a history of
  discriminatory voter registration practices.
  
  In 1993 Congress passed the National Voter Registration Act (or NVRA),
  which increased oversight of local registration policies by mandating
  by-mail registration and registration at the DMV (otherwise known as the
  ``motor voter'' program), and by strictly regulating local registration
  forms. The NVRA also imposed strict regulations against the practice of
  ``purging'' registered voters from the rolls, and mandated that states
  routinely check their registration data to ensure integrity. After the
  events of 2000 in Florida, the NVRA was followed by the Help America
  Vote Act (HAVA) of 2002. HAVA included sweeping changes to how American
  election administration was conducted. For the first time in history,
  the federal government started spending money on local election
  administration. The act also banned certain types of voting machines,
  mandated that local officials accept provisional ballots, and created
  the Elections Administration Commission (EAC) to gather information and
  assist state and national elections. Quite importantly for researchers
  and voters alike (and for this thesis), HAVA mandated that states
  compile and maintain voter registration and history data \emph{at the
  state level}, providing some centralized oversight for elections(Ewald,
  2009).
  
  Apart from serving local election administration, voter files
  consolidated by HAVA have allowed researchers to make concrete
  inferences of individual characteristics (E. D. Hersh, 2015). Voting
  related theories derived from political science are now commonly tested
  using advanced statistical methods and huge amounts of data; both
  disciplines tackle these data to face joint problems such as quantifying
  the quality of voter registration files (Ansolabehere \& Hersh, 2010),
  or linking disparate voter records (Ansolabehere \& Hersh, 2017). In my
  thesis I take advantage of such centralized files from the state of
  Colorado\footnote{Voter files commonly include two distinct parts: Voter
    Registration Files (VRF) and Voter History Files. The former contain
    an entry for each registered individual with all necessary demographic
    and personal data needed to correctly identify which ballots they
    should complete, like where they live, their party registration, their
    local legislative district, precinct, school board etc. The later
    contain information on each \emph{ballot} cast, including who cast it,
    what election it was cast in, how it was cast (mail vote, absentee,
    etc.), and what county it was cast in.}.
  
  The purpose of research into policies like the ones described above is
  to both to model and understand the behavior of the voters, and to
  formulate an idea on what the optimal policy is. Research conclusions on
  their own obviously cannot sway policy; such decisions are made,
  fittingly, with the input of the people or their representatives through
  processes of policy-making, whose study and function is beyond the scope
  of this thesis. In my thesis, I wish to focus on just one of the
  multitude of elections policies that are either discussed or enacted in
  the US at present: Vote By Mail. My purpose is to add to the existing
  literature of quantitative studies on how Vote By Mail affects voter
  turnout, and through this process draw conclusions on what behavioral
  model of voter choice best fits the reality of mail voting.
  
  Apart from the data, such research is also dependent on the statistical
  methods used to draw inferences. In my thesis, I construct both county-
  and individual-level models of turnout by using methods such as logistic
  regression, hierarchical modelling, and natural splines. The use of
  hierarchical modelling here is particularly significant, as the vast
  majority of previous studies on mail voting have employed fixed effects
  models. Hierarchical modelling is particularly salient when data exhibit
  different ``levels'' of grouping, here represented by the counties in
  which individual voters are registered. While fixed-effects regression
  would assume independence of effects between counties, hierarchical
  modeling will allow me to do away with that assumption, thus also vastly
  increasing the inferential potential of my models(Gelman \& Hill, 2006).
  
  This thesis should be viewed as a combination of three factors:
  questions, data, and methods. My fundamental question is one of the most
  common for elections science\footnote{Sometimes referred to as
    \emph{psephology}, the study of the \emph{psephos}
    (\(\psi \acute \eta \phi o \varsigma\)) or ``vote'' in Greek.}: how do
  we increase electoral participation in the form of turnout, and
  consequently how do voters choose when to vote? My data comes from
  Colorado Voter Registration Files: a complete, all-inclusive collection
  of all current registrants in the State along with their voter
  histories. To these data, with the purpose of answering the fundamental
  question I set, I apply hierarchical models; a practice commonly applied
  in statistical studies but relatively new and exciting as a development
  in mail voting research.
  
  \chapter{The State of the Literature}\label{rmd-basics}
  
  I start my thesis with an examination of the literature on turnout,
  deciding to vote, and mail voting. I begin with an examination of
  turnout: what it is, its use as a metric of participation, and how it is
  estimated. I continue by presenting a comprehensive list of current
  theories on voter decision and participation; these offer conflicting
  descriptions as to what variables are important when trying to predict
  the turnout effects of elections policy. I then provide a run-down of
  current voting methods in the US, with a particular focus on mail
  voting. I then make a series of predictions on what the expected effect
  of mail voting on turnout is, based on each of the aforementioned
  theories. I end on a presentation of past studies that have tried to
  statistically estimate this effect.
  
  The most basic form of regression is \(y\sim x\), where \(y\) is the
  response and \(x\) is the predictor. By the end of this chapter, I aim
  to clarify the theoretical framework behind turnout (response), mail
  voting (main predictor), and the predicted correlation between the two
  (value of regression coefficients).
  
  \section{Turnout and Political
  Participation}\label{turnout-and-political-participation}
  
  In their seminal work \emph{Participation in America}, Verba and Nie
  divide the modes of political participation into electoral activity
  (voting or campaigning), and non-electoral activity (cooperative
  activity or citizen-initiated contact of political operatives). As Verba
  and Nie point out, of these forms of participation voting is the most
  widespread and regularized. All voting-eligible citizens are given
  specific instructions on how to vote, where to do so, and at what time.
  Verba and Nie refer to voting as a ``high pressure'' concern for elected
  officials, since their continued service in their positions is directly
  dependent on this form of participation. They do, however, point out
  that voting does not convey as much information to government actors as
  other forms of participation; interest groups and direct contact are
  significantly more targeted. The conclusion they draw is that while
  voting may not be the most specific of ways to participate, but it is
  the most measurable, uniform, and general indicator of political
  engagement(Verba \& Nie, 1972).
  
  In subsequent portions of their book they use two measures of voting:
  turnout and frequency(Verba \& Nie, 1972). Frequency refers to the
  proportion of elections that a particular voter participated in over the
  total amount of elections they were eligible to vote in. Turnout is the
  ratio of ballots cast over a measure of the voting population, as in the
  following equation:\\
  \[ \% ~Turnout = \frac{Total~Ballots~Cast}{Measure~of~Voting~Population}\times100\%\]\\
  In this thesis, I will use turnout as a measure of electoral political
  participation participation. I make this choice for three reasons:
  first, there is substantial literature on the relationship between mail
  voting (the subject of my study) and turnout; second, there is also
  substantial literature behind the choice that individuals make between
  turning out to vote and not participating, meaning that I have a strong
  theoretical background from which to build hypotheses; third, the data I
  have available includes complete individual electoral history, making
  calculating turnout possible.
  
  \subsection{Calculating Turnout}\label{calculating-turnout}
  
  The choice of numerator seems fairly straightforward at first glance,
  but can become complex. There are two issues to consider. In any given
  election, there can be several candidates and issues on the ballot,
  ranging from US Senate races to local school-board contests. Therefore a
  first issue is ``undervoting'', or only completing some parts of a
  ballot and not voting in every contest(Saltman, 2009). This means that
  there can be different turnout counts produced by electoral contest for
  any given election date: one for US Senate, one for state legislative
  contests, etc. Alternatively, turnout can be calculated for that
  election day, and not for any particular race, where the numerator is
  every ballot cast. I use this second conception of turnout, calculating
  at the election level. A second issue is that a fair amount of voters
  may turn out (by physically going to a polling place or mailing their
  ballot), but have their ballot rejected for registration issues, concern
  for authenticity, or a range of other issues. Since these voters did
  take action to participate, should a metric of turnout as political
  participation include them? I would tend to answer yes, but the data I
  have available do not include information on voters having their ballots
  rejected; this means that I am only able to include such participants
  whose ballots were officially counted in my final calculations.
  
  The three main statistics used for the denominator are the total voting
  age population (VAP), voting eligible population (VEP), and the number
  of registered voters in a certain geographical location. The total
  voting age population (all individuals over 18 years of age) can be
  measured using data from the US Census. However, such an interpretation
  of VAP counts individuals of age that are not allowed to vote, like
  people with severe mental illnesses or felons, and does not count
  oversees voters or military personnel. Michael McDonald offers an
  alternative to VAP he calls ``voting eligible population'', which
  corrects for such individuals (McDonald, 2007).
  
  Counts of registered voters are also a useful tool for calculation of
  turnout, as they usually require no estimation. These counts can simply
  be extracted from voter registration files. Such a metric, however, is
  only representative of the electoral political participation level of a
  group that has already taken a step towards participating: registering
  to vote. Researchers should be careful to also include registration
  level by income, race, gender, or other metrics if using this form of
  turnout calculation to infer levels of participation by social or
  demographic characteristics.
  
  The punch line here is that calculation of turnout is not an obvious
  choice, and will have an impact on what conclusions are drawn. To give
  one example, consider Oregon's Motor Voter program, that automatically
  registers voters when they interact with the DMV. It is conceivable that
  this reform will \emph{decrease} turnout when measured as a percentage
  of the total registered voter count, but \emph{increase} turnout when
  measured against total population. This happens if more people register
  to vote, but do not actually do so--in other words, both number of
  registrants and number of ballots cast are increasing, but the former
  increases at a larger rate than the latter. In my thesis I will use
  registered voter counts as the turnout denominator.
  
  \subsection{Turnout and Voting Probability
  Models}\label{turnout-and-voting-probability-models}
  
  Statistical models of turnout can be constructed at either the
  individual or group level. At the individual level, a model is built to
  predict the probability of voting for every member of a group, which
  then can sum over the members to create an estimate for turnout. This is
  a classification problem that can be solved with models such as Probit
  or Logit generalized linear regressions. Turnout can be calculated for
  several groups: counties, precincts, racial characteristics, etc. The
  unit of observation is a single group, and a regression model is fit on
  turnout as a continuous variable. In this thesis, I estimate models at
  the county level, as well as the individual level.
  
  Both these models include a standard set of societal variables at the
  individual and/or group level, administrative variables (whether the
  district uses Postal Voting, whether Voter ID requirements are
  particularly strict), election-specific variables (closeness of election
  or campaign expenditure) and sometimes time-series data, like previous
  levels of turnout. This type of analysis is used either to predict
  turnout or draw inferences on the effects that certain explanatory
  variables have on electoral participation.
  
  Meta-analyses conducted by Geys (2006), Geys and Cancela (2016), and
  Smets (2013) provide a clearer picture on which predictors are relevant
  to turnout. Geys includes 83 studies of national US elections in his
  initial meta-analysis (Geys, 2006), later increasing that number to 185
  (Geys and Cancela, 2016) and adding local elections. On aggregate-level
  models for national elections they conclude that competitiveness,
  campaign financing, and registration policy have the most pronounced
  effects, while on the sub-national level there are more pronounced
  effects for societal variables and characteristics of election
  administration (spending, voting policy, etc.). Smets and Van Ham (2013)
  examine individual-level predictors for turnout in a similar
  meta-analysis, and conclude that ``age and age squared, education,
  residential mobility, region, media exposure, mobilization (partisan and
  nonpartisan), vote in previous election, party identification, political
  interest, and political knowledge'' (Smets \& Ham, 2013) are the most
  significant explanatory variables for turnout, along with income and
  race. I will specify the model I will use for turnout in the second
  chapter.
  
  \subsection{Deciding to Vote}\label{deciding-to-vote}
  
  Here I take one step back from turnout, and examine the theories
  surrounding individual choices to vote or abstain. There are three main
  theories outlined in the literature on why individuals chose to vote:
  
  \begin{itemize}
  \item
    \emph{Decision ``at the margins''}: In his 1993 study, Aldrich posits
    that voting is a low cost-low benefit behavior. Voting is a decision
    that individuals make ``at the margins''; in most people, the urge to
    vote is not overwhelmingly strong, and therefore individuals will vote
    when it is convenient to them, when they are motivated by a
    competitive race, when policies are put in place to help them, and
    when they are subjected to GOTV (Get-Out-the-Vote) efforts. For
    Aldrich, this is corroborated by the fact that most turnout models
    present consistent, yet weak, relational variables; if decisions are
    made ``at the margins'', then no single predictor would have an
    overwhelming result. This is also supported by Matsusaka (1997), and
    Burden \& Neiheisel (2012). Matsusaka expresses support for a more
    ``random'' process of voting, where turnout models are ambiguous
    because of the difficulty that predicting ``at the margins'' entails
    (Matsusaka \& Palda, 1999). Burden \& Neiheisel (2012) use data from
    Wisconsin to calculate a net \emph{negative} effect of 2\% on turnout
    following the expansion of early voting access in the state. While
    this can be read as an Aldrich effect on turnout, the authors claim
    that their findings should be attributed to a lack of ``election-day
    effects''(Aldrich, 1993; Neiheisel \& Burden, 2012).
  \item
    \emph{Habitual Voting}: While Aldrich supports that there is no single
    overwhelming predictor of turnout, Fowler (2006) posits that future
    voting behavior can be strongly predicted using individual voting
    history. This leads to the conclusion that individuals are set to
    either be habitual voters, or habitual non-voters (Plutzer, 2002) by
    their upbringing and social circumstances, locking them into distinct
    groups. (Fowler, 2006)
  \item
    \emph{Social/Structural Voting}: Close to habitual voting are those
    that support a model of social and structural voting; these
    researchers claim that the decision to vote or not is deeply rooted in
    socioeconomic factors, which means that the divide between
    traditionally voting and non-voting groups can only be bridged by
    directly dealing with the socioeconomic divide between them (Berinsky,
    2005; Edlin, Gelman, \& Kaplan, 2007 ). They address habitual voting
    claims by arguing that they are too short-sighted; individuals
    themselves might be habitually voting, but their decision to do so is
    rooted in strong societal and policy factors.
  \item
    \emph{Resources and Organization}: To some extent growing from
    structural theories of voting, resources and organizations theory
    emphasizes the interaction of personal political and societal
    characteristics of voters, and actions taken by politicians to
    mobilize participation. This theory is very broad in the inputs it
    assesses for voter participation, ranging from practical issues of
    access and resources (how easy it is for someone to vote if they so
    choose), to public policy feedback effects and signaling (how the
    government's policies affect the people and how they react), to how
    political parties and groups choose to mobilize and approach voters
    (Rosenstone, 2003). Apart from Rosenstone and Hansen's work (2013),
    there have been several studies examining voter participation based on
    resources and organizations theory, a lot of which come from the
    public policy side of political science. Some examples are Chen's
    study of how distributive benefits like federal emergency aid affect
    participation among recipients, after controlling for partisan
    characteristics (2012), or Mettler and Stonecash's examination of
    correlation between welfare program participation and political
    mobilization (2008), or Campbell's analysis of social security
    recipients and their voting patterns (2002). The punchline in all
    these studies is that public policy is correlated with trends in
    participation, either because recipients of benefits wish to protect
    such programs, or because of the interaction between partisanship and
    government support, or because of access related to resources and
    voting laws (Campbell, 2002; Chen, 2013; Mettler \& Stonecash, 2008).
  \end{itemize}
  
  \section{From Theory to Policy}\label{from-theory-to-policy}
  
  \subsection{Voting Methods}\label{voting-methods}
  
  I have already flagged in my introduction the reason why theories behind
  voting choice matter: each construct an image of the electorate that
  reacts differently to policy change around voting. They are all an
  answer to the fact that elections policy, and how we conduct elections,
  is not value neutral but has implications for turnout, which in turn has
  implications on the franchise of democracy.
  
  In trying to respond to the issues set up by theoretical paradigms,
  different states--both in the global and US contexts--have adapted to
  different ways of conducting elections. In the US, voting styles can be
  simplified into three categories:
  
  \begin{itemize}
  \item
    \emph{In-Person Election Day}, for which all individuals are required
    to vote at a polling place on election day. Standard accommodations
    for overseas or excused absentee voters apply, but the vast majority
    of people will have to be present to vote in a particular time frame.
  \item
    \emph{In-Person Early Voting}, for which all individuals must vote in
    person at a polling place, but the timeframe for voting extends for
    around two weeks and not a single day.
  \item
    \emph{Mail Voting}, for which individuals have a
    no-excuse-necessary\footnote{Individuals do not need to present proof
      that they will be out of state or unable to physically be present at
      a polling place, but can just ask to be mailed their ballot.} option
    for not being present when they vote, or for filling in a mailed
    ballot and dropping it off at designated locations.
  \end{itemize}
  
  Note that some form of voting via mail ballot exists in all three cases;
  the distinction between Mail Voting and the other two is that there is
  no excuse necessary to cast a mail ballot. In such systems mail voting
  is not an exception, but exists somewhere between a common practice and
  a universal means of casting a ballot\footnote{It is also commonly
    argued that full postal voting is a 4th category in and of itself. I
    don't make this distinction here, but in the following section I break
    down Mail Voting into different categories, one of which is full
    postal voting.}.
  
  In-person election day voting is, historically speaking, the way the
  vast majority of democracies have conducted their elections. Therefore
  it has been of interest for researchers to examine if other systems can
  outperform that baseline for some metric of participation. Specifically,
  it is most interesting to examine voting styles that are heralded for
  their expansion of turnout, to see whether popular beliefs on their
  benefits and drawbacks hold; if they are different from the base model
  of conducting American elections, or if they present new challenges and
  unique selling points. Mail voting is particularly interesting because
  it is quickly becoming a trend in state-level elections administration.
  
  \subsection{What is VBM?}\label{what-is-vbm}
  
  Vote-By-Mail (or as it is commonly referred to by most in the field,
  VBM) is a process by which voters receive a ballot delivered by mail to
  their homes. Voters then have a variety of options on how to return
  these ballots, ranging from dropping them off at pre-designated
  locations, to mailing them in, to bringing them to a polling place. The
  two first options are most commonly implemented, with a very small
  number of states still operating polling places for mail ballots. This
  varies across states that have implemented VBM. Some common forms of the
  VBM policy are:
  
  \begin{itemize}
  \item
    \emph{Postal Voting}: All voters receive a ballot by mail, which can
    then be returned to a pre-designated location or mailed in to be
    counted. All-mail elections currently occur in Oregon, Washington, and
    Colorado.
  \item
    \emph{No-Excuse Absentee}: Voters can choose to register as absentee
    voters without giving any reason related to disability, health,
    distance to polling place etc. This is the case in 27 states and the
    District of Columbia.
  \item
    \emph{Permanent No-Excuse Absentee}: This is similar to the previous
    system, but allows voters to register as absentees indefinitely,
    without having to renew their registration each year; they become de
    facto all-mail voters. This is in place in a very large number of the
    no-excuse absentee voting states like Utah, California, Montana,
    Arizona, New Jersey, and others.
  \item
    \emph{Hybrid Systems}: In hybrid systems, voters receive a mail ballot
    but can choose to disregard it and vote conventionally. This is the
    case in Colorado. In other states postal voting is not mandatory for
    all counties, as some are still allowed to conduct their elections
    conventionally; these states keep an in-person option open as a part
    of their elections administration system. This is the case in
    California, Utah, and Montana.
  \end{itemize}
  
  Vote-By-Mail is also commonly considered a type of early voting, since
  voters receive their ballots around two weeks in advance of election
  day; they are also able to return that ballot whenever they wish within
  that time-frame. This means that Vote-By-Mail can be counted as a
  ``convenience voting'' reform (Gronke et al., 2008). These are usually
  implemented by state and local governments with the argument that they
  either expand the democratic franchise by bringing in new voters, or by
  making it more likely that current registered voters participate in the
  electoral process (``Absentee and Early Voting,'' 2018).
  
  \subsection{How Theories Apply to VBM}\label{how-theories-apply-to-vbm}
  
  Under Aldrich's paradigm, vote by mail would not effect significant
  change in voting behavior. The whole concept of a decision ``at the
  margins'' is that the forces at play when an individual decides to vote
  are overwhelmingly strong both ways, so any effect that policy can have
  will minimaly shift these margins. If, for example, we take a
  presidential election the forces at play include the media, national
  committees, social effects etc. In this environment, some added
  convenience does not significantly add to an individual's decision to
  turn out. However, this would indicate that at a local level, where
  national and media effects are less strong, the effect of VBM on turnout
  might be more significant. The effect would be present for all groups,
  not only those currently registered, since voting would be easier
  uniformly.
  
  If we asume habitual voting, the conclusion on VBM would differ
  significantly. In this case, the effect to be considered is how VBM
  impacts already formed habits around voting. It could be argued that VBM
  has no effect, which follows if we assume that voting habits formed do
  not shift if the mode of voting changes. It could also be argued that
  VBM might have a negative effect on turnout in the short term, because
  it disrupts the habit of election day for a readjustment period, before
  people settle into new groups of habitual voters and non-voters, adapted
  to the new policy context.
  
  Under social and structural voting contexts, VBM retains rather than
  stimulates new voters (Berinsky, 2005). This means that already
  registered and semi-active voters are more likely to participate, but
  there is no significant change in the amount of new voters entering the
  franchise. This would mean that traditional forms of voting policy that
  emphasize access to the polls will do nothing to bring in
  disenfranchised people, and potentially hide the problem under an
  inflated turnout statistic calculated on registered voters. Berinsky in
  particular emphasizes the need for a shift towards voter education,
  rather than early voting or VBM policies (Berinsky, 2016).
  
  Vote-by-Mail is obviously not a welfare or spending program, but it does
  increase individual resources in terms of voting capacity. Filling out a
  ballot at your kitchen table does not include spending time to go to a
  polling center, or standing in line to vote, or factoring in significant
  changes to your daily schedule for election day. A mail ballot can
  usually be dropped off at several locations, or can be mailed in along
  with any other mail a voter may have to send. Less time and effort is
  spent on casting your vote. This in turn has both a practical
  effect--building capacity--and a more behavioral effect--a feeling of
  inclusion, an interaction with the process of voting that comes through
  a ballot at your doorstep that would not exist if you had chosen not to
  a polling place (Schneider \& Ingram, 1990). Under a resources and
  organizations framework, both these effects are most likely to be net
  positive to political participation, and as such would predict a strong,
  positive effect of VBM on turnout.
  
  \subsection{General Results on VBM}\label{general-results-on-vbm}
  
  I will start with studies that show a negative effect on turnout.
  Bergman (2011) uses a series of logit models of individual voting
  probability in California, during a period where part of the state
  conducted VBM elections, while others maintained traditional voting.
  This is called a ``quasi-experiment'', and it is common throughout the
  literature. Bergman's results show a statistically significant drop in
  voting probability in VBM counties (Bergman \& Yates, 2011). Using a
  similar method, Keele (2018) takes a single city in Colorado, Basalt
  City, which is divided into two different voting districts using
  different voting systems. The conclusion is, again, a 2-4\% drop in
  turnout along the VBM part of the city (Keele \& Titiunik, 2017). Burden
  et al. (2014) take a different approach, using country-wide election
  data from 2004 and 2008 presidential elections, and compares districts
  based on early voting practices. Their results show a significant drop
  in turnout, which can be associated to VBM as well due to its closeness
  to EV (Burden, Canon, Mayer, \& Moynihan, 2014).
  
  In contrast, Gerber et al. (2013), applying both individual and
  county-level models for the state of Washington, reach the conclusion
  that VBM increases turnout by around 2-4\%; they use the same
  quasi-experimental model that offers itself to researchers in states
  that are under transitional systems (Gerber, Huber, \& Hill, 2013). R.M.
  Stein also reaches a similar conclusion when examining Colorado's
  practice of ``vote centers'', which are non-precinct attached polling
  places, which can service multiple counties (Stein \& Vonnahme, 2008). I
  include this paper here due to the link that voting centers have with
  VBM, as they serve as drop-off points for mail-in ballots. Richey (2008)
  examines the effects that Oregon's VBM program has on turnout by using
  past elections data, concluding a 10\% positive trend associated with
  the policy (Richey Sean, 2008). This effect is studied again by Gronke
  et al.(2012) who find a similar positive effect with much lower
  magnitude, which might point to a novelty effect: the existence of
  diminishing returns in turnout after the implementation of this policy
  (Gronke \& Miller, 2012). Gronke et al. (2017), again studying Oregon
  but focusing on Oregon's Motor Voter program, find evidence of positive
  association to turnout. I include these effects due to Oregon being an
  exclusively VBM state, and because this paper uses a ``synthetic control
  group'' model, a particularly interesting statistical technique. Lastly
  I include a study conducted by Pantheon Analytics on Colorado, which
  compares actual turnout to predicted levels for VBM counties in
  Colorado. The results show a positive effect of approximately 3.3\% due
  to VBM ({\textbf{???}}).
  
  The conclusion to be drawn is that results on VBM vary significantly.
  There are multiple studies, using multiple methods, on multiple states,
  with multiple results. This only adds to the importance of being careful
  when constructing models and hypotheses to test VBM's effects on
  turnout, as assumptions made in the process can critically impact the
  results.
  
  \chapter{Hypotheses and Methods}\label{hypotheses-and-methods}
  
  \section{Hypotheses}\label{hypotheses}
  
  \subsection{Questions}\label{questions}
  
  Before moving in to outlining hypotheses, the first step necessary is to
  frame a series of questions, which the hypotheses will flow from. Based
  on relevant research, the most obvious first question to ask would be:
  
  \begin{quotation}
  Q1: \textit{What is the effect of mail voting on turnout?}
  \end{quotation}
  
  I went through this question substantially in the previous chapter; it
  should be clear that depending on which paradigm of participation choice
  is present, the answer here can be radically different. In order to best
  answer the previous question, it is necessary to establish some
  conditions on importance of effect. Therefore it is also necessary to
  ask the following question:
  
  \begin{quotation}   
  Q2: \textit{Does this effect persist when accounting for other relevant predictors of turnout?}
  \end{quotation}
  
  The last question asked in this thesis is more specific to a particular
  formulation of Aldrich's hypothesis on voting ``at the margins''. I
  mentioned in the previous section that VBM could be theorized to have a
  more significant effect when discussing elections at the local level, or
  the regional level, rather than national general elections. Therefore a
  third question is:
  
  \begin{quotation}   
  Q3: \textit{Is  the  effect  of  VBM on turnout more  pronounced as significant, national determinants become less strong?}
  \end{quotation}
  
  \subsection{Hypotheses}\label{hypotheses-1}
  
  Using the above questions I can now move on to formulate more clear
  hypotheses. The hypotheses in this section are meant to test theories of
  voter choice from the perspective of the theory of voting ``at the
  margins'' as introduced by Aldrich.
  
  In response to Q1, Q2, a first hypothesis is:
  
  \begin{quotation}  
  H1: \textit{Mail voting is another incremental effect on voting decisions, and therefore
  does not significantly affect turnout}
  \end{quotation}
  
  The alternative hypothesis would be:
  
  \begin{quotation}  
  H1$'$: \textit{Mail  voting  significantly  affects  turnout,  even  compared  to  other  metrics}
  \end{quotation}
  
  Similarly, for the third question, a corresponding hypothesis derived
  from Aldrich's paradigm is:
  
  \begin{quotation}  
  H2: \textit{The  effect  of  VBM  on  turnout  is  larger  as  national  effects  become less pronounced}
  \end{quotation}
  
  The alternative hypothesis is:
  
  \begin{quotation}  
  H2$'$: \textit{The  effect  of  VBM  on  turnout  is either independent of or proportional to the presence of national effects}
  \end{quotation}
  
  \subsection{Criteria}\label{criteria}
  
  A first, glaring issue that needs to be clarified is the apparent
  contradictions between my two hypothesized results. This becomes clear,
  however, if I define ``significant effect'' in the context of my first
  hypothesis. Aldrich's paradigm does state that ``conveniences'' like
  mail voting should not have significant effects, but those effects are
  defined in the context of huge, clashing forces that vastly outweigh
  them. This does not necessarily mean that they are literally
  non-existent, but that they are poor indicators of consistently
  increased turnout. Therefore, I will confirm my first hypothesis not
  only if the effect of mail voting on turnout is vanishingly small, but
  also if it is relatively small in comparison to the effects of other
  variables I include. I will confirm the alternative hypothesis if,
  across multiple of the models I will parametrize and fit, VBM retains a
  consistent, significant effect on turnout. If the effect is negative,
  this may point to a habitual or structural voting paradigm being
  present. If the effect is positive, this may be a signifier that issues
  of convenience in voting--having a mail delivered ballot, voting from
  your kitchen table etc.--have a particularly strong effect in the
  examined elections.
  
  Moving on to the second hypothesis. It is extremely hard to correctly
  operationalize and account for all variables going into turnout.
  Therefore, instead of trying to include all possibly relevant effects
  into a model and try to see how they interact with VBM, I will test my
  hypothesis on different levels of elections: local, midterm,
  presidential, and primary. National effects on turnout should be
  especially present in presidential races, since a specific set of
  candidates is running across the whole nation. These effects should also
  be present in midterm and primary elections to some extent, as the
  results of local races are aggregated in control of congress or
  high-profile governorship. Apart from a ballot measure that garners
  national interest, or a singularly high-profile race, local off-year
  election turnout should have a negligible relation to national effects.
  Therefore I will use election level as a stand-in for the prominence of
  national turnout effects. The following is an alternative formulation of
  the second hypothesis, made more specific to the criteria I have set:
  
  \begin{center}    
  H3: \textit{The  effect  of  VBM  on  turnout  is  more  pronounced  in  local  or  off year elections}
  \end{center}
  
  I will confirm this hypothesis if mail voting has substantially larger
  positive effects on turnout in smaller, local elections.
  
  \subsection{Importance of Hypotheses}\label{importance-of-hypotheses}
  
  The importance of these hypotheses is intrinsically tied to the
  importance of different theories of electoral participation. Confirming
  or rejecting each hypothesis--even when only applied to a single
  state--serves as an argument for or against one of the aforementioned
  theories. The theories in and of themselves are important, since they
  form a part of a broader literature on elections, democracy, and
  electoral processes, that can be said to be foundational to political
  science as a whole. As mentioned in my introduction, elections
  themselves are significant, since they are the process by which
  political power and representation is allocated, given the will of the
  people.
  
  Additionally, from a public policy perspective, these hypotheses are
  significant since they are connected to the effectiveness of mail voting
  as an electoral reform. Whether, in general, mail voting increases
  turnout is directly connected to whether it is successful in expanding
  the democratic franchise. If it is not, questions can be raised as to
  the effectiveness of expanding voter access through elections
  administration, rather than education, or even measures like
  voting-day-holidays or local transportation to polling places. In local
  elections in particular, significant effects of mail voting could be
  precursors to more general involvement of individuals in their local
  politics. This may open the way to numerous comparative studies on local
  politics between states that apply VBM and states that do not.
  
  Lastly, from a narrower perspective specific to the study of early and
  mail voting, my first hypothesis can still be said to be quite
  important, yet mundane. It does its job according to the particular
  state I chose to look at--in this case Colorado--to add to existing
  literature on mail voting effects in different parts of the country.
  However, my second and third hypotheses are much more unique in their
  scope. There have not been many studies that look at VBM at a local
  level, and any addition to the literature on this front could be
  significant.
  
  \section{Methodology}\label{methodology}
  
  Before directly defining the models I will later use in writing this
  thesis, I will go through each type of method to provide some background
  on the statistics behind the models. In the next chapter, I will
  introduce the data and fully outline my models. This section should
  serve as a general introduction to the methods. I will not extensively
  go through the statistics behind linear or multiple regression, but will
  assume that it is common knowledge. For an extensive introduction to
  such methods, James et al.(2017) or Chihara and Hesterberg (2011) are
  particularly useful.
  
  \subsection{Logistic Regression}\label{logistic-regression}
  
  Let function \(f : [0, 1] \to \mathbb{R}\) be defined as:
  \[f(p) = \text{logit}(p) = \text{log}\left( \frac{p}{1-p} \right)\] This
  is called the logit function or, when \(p\) refers to a probability, the
  log-odds function. When modelling a binary response Y, which follows a
  Bernoulli distribution: \[Y \sim \text{Bernoulli}(p),\] the logit
  function can be used as a link function to model Y in a generalized
  linear model. The generic form of a generalized linear model looks
  like:\\
  \[f(Y) = X\beta ,\] where Y is a vector of response variable values, X
  is a matrix of predictors, and B is a vector of coefficients to be
  estimated. The function \(f\) is called a link function, because it
  ``links'' the response variable with the set of predictors included in
  the model. This is typically done to ensure that the range of values
  outputted by the model are consistent with the range of the response
  variable.\footnote{Or, in this case, the range of the parameter defining
    the distribution of the response, which is p for the Bernoulli
    distribution} When wanting to compute a model on a binary response
  through its corresponding Bernoulli distribution probability parameter,
  the inverse logit function should be a perfect fit for a link function,
  since it maps values from all real numbers to a range between 0 and 1.
  Using the inverse logit function, we arrive at the final form of
  logistic regression, which is:\\
  \[\mathbb{P} (Y = 1) = \text{logit}^{-1} (XB)\]
  
  Conveniently, despite the use of a link function, there is an easy way
  to interpret the coefficients of such a regression. While obviously
  individual values from the \(B\) vector will not be particularly
  helpful, \(e^B\) can be used as a vector of multiplicative, one-unit
  shifts in the value of the probability that \(Y = 1\). This means that a
  one unit increase in any predictor will cause an effect equal to
  multiplying p by the exponent of the corresponding coefficient\footnote{This
    can be simplified even more, if one approximates the process of
    exponentiation with just dividing the coefficient by 4. Crude, yet
    effective for a quick scan of the results}. (James, Witten, Hastie, \&
  Tibshirani, 2017)
  
  \subsection{Generalized Additive
  Models}\label{generalized-additive-models}
  
  In simple logistic or linear regression, there is an assumption made on
  the functional form of the relationship between predictors and response
  variable. These are called parametric models, where the data is
  exclusively used to estimate values for coefficients. Non-parametric
  models, on the other hand, use the data to estimate both coefficients
  and the function that serves to connect response to predictors. While on
  the surface this seems like a great idea (more reliance on your data and
  fewer assumptions!), such an exclusively non-parametric model would
  suffer greatly from the curse of dimensionality--where the addition of
  multiple predictors or over-reliance on data leads to substantial
  over-fitting.
  
  One solution is the Generalized Additive Model, or GAM. This model lets
  us fit a different functional form to each predictor, allowing for
  assumptions to be made on the data where it is safe to do so, and for
  non-parametric fitting when it is necessary. This model looks like:
  
  \[y_i = \alpha + \sum_{j = 1}^p \beta_j f_j(x_{ij}), ~~j \in \{1,2,...p\}, i \in \{1,2,...n\} \]
  
  where \(y_i\) the i-th response variable, \(\alpha\) is the intercept
  term, \(f_j, \beta_j\) a series of \(p\) functions and coefficients, and
  \(x_{ij}\) the i-th observation for the j-th predictor. Note that for
  \(f(x_{ij}) = x_{ij}\), this is a multilinear regression! (James et al.,
  2017)
  
  A type of most commonly fit functions--and the type I will make use
  of--are smoothing splines. These are functions connected at specific
  points called ``knots'', with the limitation that the full function must
  be continuous and smooth, and have a continuous first and second
  derivative. Between knots, different functional forms are fit to the
  data, within some constraints; they may, for example, all have to be
  cubic polynomials. These are particularly useful when modeling time
  variables, as they can be fitted to variables like years or months in
  order to distinguish a secular trend from a general trend over time
  (Barr, Diez, Wang, Dominici, \& Samet, 2012). In terms of this thesis,
  this will help when responding to Q2 as it was framed earlier in this
  chapter.
  
  \subsection{Multilevel Models}\label{multilevel-models}
  
  Multilevel models (otherwise known as hierarchical or ``mixed effects''
  models) can be intuitively pictured in two ways: either as a set of
  models working on different ``levels'', where one is calculated first,
  with its effects having implications for the second, or as a model where
  some of the parameters are estimated under a particular series of
  constraints. Multilevel models are, in essence, a compromise between
  levels of ``pooling'' data. If the dataset on which parameters are being
  estimate operates in different units of observation--say on the
  individual and county level--you could run a model that treats all
  individuals as coming from the same larger group; this would be a
  complete pooling model. You could also add indicator variables for each
  and every group, de facto estimating \(n\) different models for \(n\)
  groups; this would be a no pooling model. Multilevel modelling offers
  partial pooling (Gelman \& Hill, 2006).
  
  To consider what this model looks like, let's assume a dataset
  comprising of a vector of values for the response variable \(Y\), a
  matrix of \(i\) individual level predictors \(X\), a matrix of \(j\)
  group level predictors \(U\), intercept terms \(\alpha\), individual
  level coefficients \(B\), and group level coefficients \(\Gamma\). Based
  on this, a multilevel model with intercept terms varying by group looks
  like:
  
  \[Y_i = \alpha_{[i], j} + X_iB~,~~~~\alpha_{[i], j} \sim N(U_{j[i]}\Gamma, \sigma_{\alpha}^2)\]
  
  Multilevel models can be fit using the \texttt{lme4} \textit{R} package
  that uses restricted maximum likelihood calculations for estimating
  coefficients (Bates, Mächler, Bolker, \& Walker, 2015). Multilevel
  modelling also works perfectly well with general additive models! In
  \textit{R} this can be accomplished with the \texttt{gamm4} package (S.
  Wood \& Scheipl, 2017).
  
  \subsection{Model Accurracy and Quality of
  Fit}\label{model-accurracy-and-quality-of-fit}
  
  \subsubsection{Mean Squared Error (MSE)}\label{mean-squared-error-mse}
  
  For all generalized linear regression models (including GAMs, mixed and
  fixed effects models) I use Mean Squared Error to asses the accuracy of
  the fit. Assuming a dataset
  \(\{(y_0, x_0^1, x_0^2, ..., x_0^m),...,(y_n, x_n^1, x_n^2, ..., x_n^m)\}\)
  of n observations and m predictors, with \(X_i\) a vector of the
  predictors for the i-th observation, and \(f:R^m \to R\) the true
  multivariate function connecting the predictors and response, mean
  squared error is calculated as follows:
  \[\text{MSE} = \frac{1}{n}\sum_{i=1}^{n}(y_i - \hat{f}(X_i))^2\]
  
  MSE can be calculated either using the same dataset used in estimating
  the model coefficients, or on a new dataset. In the later case it is
  called predictive or test MSE. Despite prediction not being the purpose
  of the models presented in this thesis, I use test MSE because of the
  independence such a calculation method brings from the data used for the
  fit, compensating in a way for over-fitting(James et al., 2017). To
  calculate test MSE I use five-fold cross-validation, which will be
  analyzed shortly.
  
  \subsubsection{Area Under the Curve
  (AUC)}\label{area-under-the-curve-auc}
  
  Logistic regression models estimate the probability of a binary variable
  heing equal to 1, or alternatively an indicator variable taking a
  ``TRUE'' value. The predictive output of such a model will be a series
  of probabilities. These probabilities can then be used to approximate a
  dataset of positive and negative values for the response variable (in my
  case, voting). Based on the true values of the response, one can
  calculate the counts of true positive, true negative, false positive,
  and false negative predictions. To make this calculation, a probability
  threshold is set over which the prediction for the response is positive.
  Positive predictive values of the response are assigned based on the
  following statement:
  
  \[\text{P}(y_i = 1|X_i) > p\]
  
  where \(y_i, X_i\) can be assumed to be the same as in the previous
  section, and \(p\) is the threshold. A common and intuitive threshold is
  \(0.5\), but any number in \((0,1)\) can be used. After getting counts
  for true/false negative/positive values, one can then calculate
  \emph{specificity} and \emph{sensitivity} for the model. These are:
  
  \[\text{specificity} = \frac{\text{True Positive}}{\text{False Negative + True Positive}}\]
  
  \[\text{sensitivity} = \frac{\text{True Negative}}{\text{False Positive + True Negative}}\]
  
  Using sensitivity, specificity, and probability threshold it's possible
  to create an ROC curve, which is one of the most widely used diagnostic
  plots for classification models\footnote{The ROC curve takes its name
    from a term in communications science, the \emph{receiver operating
    characteristics curve}. The name is historic, and not relevant to its
    statistical application.}. The ROC curve has \(1-\text{specificity}\)
  on the x-axis, \(\text{sensitivity}\) on the y-axis, and each point
  describes a pair of x-y values for each value of the probability
  threshold. Using this plot, it's possible to measure the \emph{area
  under the (ROC) curve}, or AUC, which serves as a goodness-of-fit
  measure for classification models. The AUC is a number in the \([0,1]\)
  range and should be maximized; a \(.5\) AUC is representative of an ROC
  curve on the \(y = x\) line, which is a coin-toss no-information
  classifier(James et al., 2017). Plot 2.1 is an example of an ROC curve.
  
  \begin{figure}
  
  {\centering \includegraphics[width=0.5\linewidth]{/Users/tdounias/Desktop/Reed_Senior_Thesis/plots/roc_example} 
  
  }
  
  \caption[Example of an ROC curve]{Example of an ROC curve}\label{fig:roc example}
  \end{figure}
  
  Similarly to MSE, there is value in calculating AUC from a test dataset,
  rather than the dataset used to train the model. Therefore I also use
  5-fold cross-validation for AUC as well\footnote{This also compensates
    for models not converging, as some of mine do.}.
  
  \subsubsection{k-Folds Cross Validation}\label{k-folds-cross-validation}
  
  The goal of statistical modeling is to approximate the true function
  that links predictors to response. While the final model's coefficients
  should be estimated using as much data as possible, when assessing how
  good a fit that model is there can be better uses of the power that
  large amounts of data give us. k-Folds cross validation allows for
  better approximations of goodness-of-of-fit metrics, by partitioning the
  data into training datasets and test datasets. The fundamental idea is
  that the data is split into k different subsets, which are then
  sequentially used to fit the model and calculate the value of some
  metric(James et al., 2017). The algorithm goes as follows:
  
  \begin{enumerate}
  \item Partition data into k folds
  \item Fit model on all but the i-th fold
  \item Calculate goodness-of-fit metric on the i-th fold
  \item Repeat 2 and 3 for i$\in [0,k]$
  \item Calculate the average of all obtained goodness-of-fit measurements
  \end{enumerate}
  
  I perform 5-fold cross validation to calculate MSE and AUC for all
  models which I estimate in R.
  
  \chapter{Case Selection, Data, Model
  Parametrization}\label{case-selection-data-model-parametrization}
  
  \section{The Centennial State and Its
  Voters}\label{the-centennial-state-and-its-voters}
  
  \subsection{Demographics}\label{demographics}
  
  Colorado--named the Centennial State due to assuming statehood on the
  centennial of the Union--lies in the Southwestern United States, with
  its Western half squarely atop the Rocky Mountains. Based on its
  estimated population of just over 5.5 million, Colorado is the 21st most
  populous state, and ranks 37th in population density. The vast majority
  of that population is gathered in a series of urban areas that comprise
  a North-to-South strip in the middle of the state, containing the
  Denver-Aurora-Lakewood Metro Area, Colorado Springs, Pueblo, and Fort
  Collins. Apart from the Western town of Grand Junction, the rest of the
  population resides in vast rural areas.
  
  \begin{figure}
  
  {\centering \includegraphics[width=0.8\linewidth]{/Users/tdounias/Desktop/Reed_Senior_Thesis/maps/pct_white_county_map} 
  
  }
  
  \caption[White voters per Colorado county]{White voters per Colorado county}\label{fig:white pct map}
  \end{figure}
  
  Colorado is landlocked, and far from any coastal town; in place of
  seaside resorts, Colorado attracts a substantial amount of tourists to
  its mountains every year. Therefore the more mountainous regions have
  developed skiing and mountaineering resorts. They also heavily depend on
  federal money and protection for national parks. These importance of
  these characteristics will become apparent in the following sections.
  
  Colorado has a median age of 34.3 and median household income of
  \$65,685. Colorado's population is mostly white, with a higher minority
  group population density in its Southern regions, as shown in figure
  3.1. (``U.S. Census Bureau QuickFacts,'' 2010) The conclusion here is
  that Colorado is a relatively young, mostly white, and fairly well-off
  state that is increasingly getting more diverse, particularly in the
  South. These factors are important as they serve to associate Colorado
  with other states; such associations are useful for the replication of
  this study or the generalization of my results.
  
  The State Capital is Denver; Colorado is split into 64 Counties, of
  which the most populous are, in no particular order, the following: El
  Paso, Denver, Arapahoe, Jefferson, Adams, Larimer, Boulder, and Douglas.
  These counties comprise 73\% of the total population of Colorado.
  
  \begin{longtable}[]{@{}lccl@{}}
  \caption{Colorado population for largest counties
  \label{tab:pop_table}}\tabularnewline
  \toprule
  \begin{minipage}[b]{0.13\columnwidth}\raggedright\strut
  County\strut
  \end{minipage} & \begin{minipage}[b]{0.21\columnwidth}\centering\strut
  Total Population\strut
  \end{minipage} & \begin{minipage}[b]{0.20\columnwidth}\centering\strut
  CO Population \%\strut
  \end{minipage} & \begin{minipage}[b]{0.34\columnwidth}\raggedright\strut
  Largest Metro Area\strut
  \end{minipage}\tabularnewline
  \midrule
  \endfirsthead
  \toprule
  \begin{minipage}[b]{0.13\columnwidth}\raggedright\strut
  County\strut
  \end{minipage} & \begin{minipage}[b]{0.21\columnwidth}\centering\strut
  Total Population\strut
  \end{minipage} & \begin{minipage}[b]{0.20\columnwidth}\centering\strut
  CO Population \%\strut
  \end{minipage} & \begin{minipage}[b]{0.34\columnwidth}\raggedright\strut
  Largest Metro Area\strut
  \end{minipage}\tabularnewline
  \midrule
  \endhead
  \begin{minipage}[t]{0.13\columnwidth}\raggedright\strut
  Adams\strut
  \end{minipage} & \begin{minipage}[t]{0.21\columnwidth}\centering\strut
  441603\strut
  \end{minipage} & \begin{minipage}[t]{0.20\columnwidth}\centering\strut
  0.08781\strut
  \end{minipage} & \begin{minipage}[t]{0.34\columnwidth}\raggedright\strut
  Denver-Aurora-Lakewood\strut
  \end{minipage}\tabularnewline
  \begin{minipage}[t]{0.13\columnwidth}\raggedright\strut
  Arapahoe\strut
  \end{minipage} & \begin{minipage}[t]{0.21\columnwidth}\centering\strut
  572003\strut
  \end{minipage} & \begin{minipage}[t]{0.20\columnwidth}\centering\strut
  0.1137\strut
  \end{minipage} & \begin{minipage}[t]{0.34\columnwidth}\raggedright\strut
  Denver-Aurora-Lakewood\strut
  \end{minipage}\tabularnewline
  \begin{minipage}[t]{0.13\columnwidth}\raggedright\strut
  Boulder\strut
  \end{minipage} & \begin{minipage}[t]{0.21\columnwidth}\centering\strut
  294567\strut
  \end{minipage} & \begin{minipage}[t]{0.20\columnwidth}\centering\strut
  0.05857\strut
  \end{minipage} & \begin{minipage}[t]{0.34\columnwidth}\raggedright\strut
  Boulder\strut
  \end{minipage}\tabularnewline
  \begin{minipage}[t]{0.13\columnwidth}\raggedright\strut
  Denver\strut
  \end{minipage} & \begin{minipage}[t]{0.21\columnwidth}\centering\strut
  600158\strut
  \end{minipage} & \begin{minipage}[t]{0.20\columnwidth}\centering\strut
  0.1193\strut
  \end{minipage} & \begin{minipage}[t]{0.34\columnwidth}\raggedright\strut
  Denver\strut
  \end{minipage}\tabularnewline
  \begin{minipage}[t]{0.13\columnwidth}\raggedright\strut
  Douglas\strut
  \end{minipage} & \begin{minipage}[t]{0.21\columnwidth}\centering\strut
  285465\strut
  \end{minipage} & \begin{minipage}[t]{0.20\columnwidth}\centering\strut
  0.05676\strut
  \end{minipage} & \begin{minipage}[t]{0.34\columnwidth}\raggedright\strut
  Denver-Aurora-Lakewood\strut
  \end{minipage}\tabularnewline
  \begin{minipage}[t]{0.13\columnwidth}\raggedright\strut
  El Paso\strut
  \end{minipage} & \begin{minipage}[t]{0.21\columnwidth}\centering\strut
  622263\strut
  \end{minipage} & \begin{minipage}[t]{0.20\columnwidth}\centering\strut
  0.1237\strut
  \end{minipage} & \begin{minipage}[t]{0.34\columnwidth}\raggedright\strut
  Colorado Springs\strut
  \end{minipage}\tabularnewline
  \begin{minipage}[t]{0.13\columnwidth}\raggedright\strut
  Jefferson\strut
  \end{minipage} & \begin{minipage}[t]{0.21\columnwidth}\centering\strut
  534543\strut
  \end{minipage} & \begin{minipage}[t]{0.20\columnwidth}\centering\strut
  0.1063\strut
  \end{minipage} & \begin{minipage}[t]{0.34\columnwidth}\raggedright\strut
  Denver-Aurora-Lakewood\strut
  \end{minipage}\tabularnewline
  \begin{minipage}[t]{0.13\columnwidth}\raggedright\strut
  Larimer\strut
  \end{minipage} & \begin{minipage}[t]{0.21\columnwidth}\centering\strut
  299630\strut
  \end{minipage} & \begin{minipage}[t]{0.20\columnwidth}\centering\strut
  0.05958\strut
  \end{minipage} & \begin{minipage}[t]{0.34\columnwidth}\raggedright\strut
  Fort Collins\strut
  \end{minipage}\tabularnewline
  \begin{minipage}[t]{0.13\columnwidth}\raggedright\strut
  Other\strut
  \end{minipage} & \begin{minipage}[t]{0.21\columnwidth}\centering\strut
  1378964\strut
  \end{minipage} & \begin{minipage}[t]{0.20\columnwidth}\centering\strut
  0.2742\strut
  \end{minipage} & \begin{minipage}[t]{0.34\columnwidth}\raggedright\strut
  \strut
  \end{minipage}\tabularnewline
  \begin{minipage}[t]{0.13\columnwidth}\raggedright\strut
  Colorado\strut
  \end{minipage} & \begin{minipage}[t]{0.21\columnwidth}\centering\strut
  5029196\strut
  \end{minipage} & \begin{minipage}[t]{0.20\columnwidth}\centering\strut
  100\strut
  \end{minipage} & \begin{minipage}[t]{0.34\columnwidth}\raggedright\strut
  \strut
  \end{minipage}\tabularnewline
  \bottomrule
  \end{longtable}
  
  \clearpage
  
  \subsection{The Politics of Colorado}\label{the-politics-of-colorado}
  
  Curtis Martin (1962) notes that Colorado, due to its status as a
  frontier state, has always been fiercely democratic and independent. He
  connects this fact with Colorado's past, by pointing out that its
  political institutions were deeply rooted in mining culture, ordinary
  citizens' participation,a strong feeling of being ``far away'' from
  sources of centralized power in the coasts, and a wish for the
  protection and preservation of Colorado's natural environment. As such,
  Colorado can be described as a populist state with a strong libertarian
  streak, that highly values democratic processes when they serve the
  people or protect and fund national parks, but staunchly opposes state
  intervention when it is unwarranted. (Martin, 1962)
  
  This 1964 study of Colorado politics rings true to this day. One needs
  not search for long to see instances when Colorado honored this
  description. One example is TABOR, or the Taxpayer's Bill of Rights; a
  strongly libertarian, small-government, populist series of regulations
  that mandated a referendum for any measure that would increase state
  taxes, and caped government spending. TABOR was passed by referendum in
  1992, and later amended in 2005 after the dot com economic crisis
  exposed the fact that inability to spend is very bad for a state
  government trying to jump start its economy. (Staff, 2009)
  
  Similarly, Amendment 64 passed in 2012 made Colorado one of the first
  states to legalize the selling, possession, and consumption of
  recreational marijuana--a policy advocated by progressives and
  libertarians alike. Colorado was also the staging ground for what has
  been coined the ``Sagebrush Rebellion'': a movement primarily consisting
  on ranchers in dispute with the federal government over land use laws
  and wildlife protection. While this ``rebellion'' primarily consisted of
  battles in local legislatures or elections in the 1970s, its echoes can
  be heard till today in events like the Bundy Standoff, with ranchers
  taking up arms against federal employees and occupying federal land
  (Thompson, 2016).
  
  Setting policy aside, this description of Colorado is also confirmed by
  polling data and election results. While being traditionally more
  conservative, inflows of immigration from the South coupled with
  increasing urban liberalization and tourism has led the state from
  leaning republican to being aggressively purple: the quintessential
  swing state. Colorado voted both for and later against Bill Clinton,
  voted for G.W. Bush twice, and has supported democratic presidential
  candidates since (Hamm, 2017). The trend is also, maybe surprisingly,
  consistent when considering both rural and urban voters; the divide that
  is said to plague other states seems to have passed by Colorado.
  Additionally, when polled on trust of federal or local governments,
  Colorado residents are systematically skeptical; in a random sample poll
  conducted by Cronin and Loevy (2012) in 2010, 56\% stated that their
  state officials were lazy, wasteful, and inefficient. However--again
  indicating a libertarian, independent streak--most Coloradoans from 1988
  to today consistently believe that their state is ``on the right
  track.''\footnote{Colorado College Citizens Polls, taken from Cronin et
    al. (Cronin \& Loevy, 2012)}
  
  \subsection{Voting in Colorado}\label{voting-in-colorado}
  
  Each County individually administers local, coordinated, primary, and
  general elections, under the supervision of the Colorado Secretary of
  State. This means that each county individually handles the voters
  registered in that county. Unsurprisingly, the same eight most populous
  counties are also the counties with the majority of registered voters,
  as their registrants comprise 73\% of total Colorado registered voters
  (as of November 2017). As table 3.2 shows, these eight counties have a
  registration rate between 60-80\%, compared to a Colorado-wide rate of
  about 67\%. Registration rates for all counties are also graphically
  depicted in figure 3.2. In terms of Party registration, Colorado as a
  whole leans democratic by a very narrow margin (figure 3.3).
  
  \begin{longtable}[]{@{}lccc@{}}
  \caption{Colorado voter registration for largest counties
  \label{tab:voter_reg}}\tabularnewline
  \toprule
  \begin{minipage}[b]{0.10\columnwidth}\raggedright\strut
  County\strut
  \end{minipage} & \begin{minipage}[b]{0.24\columnwidth}\centering\strut
  Total Registered Voters\strut
  \end{minipage} & \begin{minipage}[b]{0.29\columnwidth}\centering\strut
  County Registration Rate\strut
  \end{minipage} & \begin{minipage}[b]{0.25\columnwidth}\centering\strut
  \% of Statewide Registrants\strut
  \end{minipage}\tabularnewline
  \midrule
  \endfirsthead
  \toprule
  \begin{minipage}[b]{0.10\columnwidth}\raggedright\strut
  County\strut
  \end{minipage} & \begin{minipage}[b]{0.24\columnwidth}\centering\strut
  Total Registered Voters\strut
  \end{minipage} & \begin{minipage}[b]{0.29\columnwidth}\centering\strut
  County Registration Rate\strut
  \end{minipage} & \begin{minipage}[b]{0.25\columnwidth}\centering\strut
  \% of Statewide Registrants\strut
  \end{minipage}\tabularnewline
  \midrule
  \endhead
  \begin{minipage}[t]{0.10\columnwidth}\raggedright\strut
  Adams\strut
  \end{minipage} & \begin{minipage}[t]{0.24\columnwidth}\centering\strut
  270,303\strut
  \end{minipage} & \begin{minipage}[t]{0.29\columnwidth}\centering\strut
  0.61\strut
  \end{minipage} & \begin{minipage}[t]{0.25\columnwidth}\centering\strut
  0.07\strut
  \end{minipage}\tabularnewline
  \begin{minipage}[t]{0.10\columnwidth}\raggedright\strut
  Arapahoe\strut
  \end{minipage} & \begin{minipage}[t]{0.24\columnwidth}\centering\strut
  410,546\strut
  \end{minipage} & \begin{minipage}[t]{0.29\columnwidth}\centering\strut
  0.72\strut
  \end{minipage} & \begin{minipage}[t]{0.25\columnwidth}\centering\strut
  0.11\strut
  \end{minipage}\tabularnewline
  \begin{minipage}[t]{0.10\columnwidth}\raggedright\strut
  Boulder\strut
  \end{minipage} & \begin{minipage}[t]{0.24\columnwidth}\centering\strut
  237,091\strut
  \end{minipage} & \begin{minipage}[t]{0.29\columnwidth}\centering\strut
  0.80\strut
  \end{minipage} & \begin{minipage}[t]{0.25\columnwidth}\centering\strut
  0.06\strut
  \end{minipage}\tabularnewline
  \begin{minipage}[t]{0.10\columnwidth}\raggedright\strut
  Denver\strut
  \end{minipage} & \begin{minipage}[t]{0.24\columnwidth}\centering\strut
  450,616\strut
  \end{minipage} & \begin{minipage}[t]{0.29\columnwidth}\centering\strut
  0.75\strut
  \end{minipage} & \begin{minipage}[t]{0.25\columnwidth}\centering\strut
  0.12\strut
  \end{minipage}\tabularnewline
  \begin{minipage}[t]{0.10\columnwidth}\raggedright\strut
  Douglas\strut
  \end{minipage} & \begin{minipage}[t]{0.24\columnwidth}\centering\strut
  237,659\strut
  \end{minipage} & \begin{minipage}[t]{0.29\columnwidth}\centering\strut
  0.83\strut
  \end{minipage} & \begin{minipage}[t]{0.25\columnwidth}\centering\strut
  0.06\strut
  \end{minipage}\tabularnewline
  \begin{minipage}[t]{0.10\columnwidth}\raggedright\strut
  El Paso\strut
  \end{minipage} & \begin{minipage}[t]{0.24\columnwidth}\centering\strut
  445,708\strut
  \end{minipage} & \begin{minipage}[t]{0.29\columnwidth}\centering\strut
  0.71\strut
  \end{minipage} & \begin{minipage}[t]{0.25\columnwidth}\centering\strut
  0.12\strut
  \end{minipage}\tabularnewline
  \begin{minipage}[t]{0.10\columnwidth}\raggedright\strut
  Jefferson\strut
  \end{minipage} & \begin{minipage}[t]{0.24\columnwidth}\centering\strut
  422,362\strut
  \end{minipage} & \begin{minipage}[t]{0.29\columnwidth}\centering\strut
  0.79\strut
  \end{minipage} & \begin{minipage}[t]{0.25\columnwidth}\centering\strut
  0.11\strut
  \end{minipage}\tabularnewline
  \begin{minipage}[t]{0.10\columnwidth}\raggedright\strut
  Larimer\strut
  \end{minipage} & \begin{minipage}[t]{0.24\columnwidth}\centering\strut
  250,626\strut
  \end{minipage} & \begin{minipage}[t]{0.29\columnwidth}\centering\strut
  0.84\strut
  \end{minipage} & \begin{minipage}[t]{0.25\columnwidth}\centering\strut
  0.06\strut
  \end{minipage}\tabularnewline
  \begin{minipage}[t]{0.10\columnwidth}\raggedright\strut
  Other\strut
  \end{minipage} & \begin{minipage}[t]{0.24\columnwidth}\centering\strut
  1,009,392\strut
  \end{minipage} & \begin{minipage}[t]{0.29\columnwidth}\centering\strut
  ---\strut
  \end{minipage} & \begin{minipage}[t]{0.25\columnwidth}\centering\strut
  0.27\strut
  \end{minipage}\tabularnewline
  \begin{minipage}[t]{0.10\columnwidth}\raggedright\strut
  Colorado\strut
  \end{minipage} & \begin{minipage}[t]{0.24\columnwidth}\centering\strut
  3,734,303\strut
  \end{minipage} & \begin{minipage}[t]{0.29\columnwidth}\centering\strut
  0.67\strut
  \end{minipage} & \begin{minipage}[t]{0.25\columnwidth}\centering\strut
  1.00\strut
  \end{minipage}\tabularnewline
  \bottomrule
  \end{longtable}
  
  \begin{figure}
  
  {\centering \includegraphics[width=0.8\linewidth]{/Users/tdounias/Desktop/Reed_Senior_Thesis/maps/pct_registered_county_map} 
  
  }
  
  \caption[Registration rates per Colorado county]{Registration rates per Colorado county}\label{fig:reg per county map}
  \end{figure}
  
  \begin{figure}
  
  {\centering \includegraphics[width=0.8\linewidth]{/Users/tdounias/Desktop/Reed_Senior_Thesis/maps/party_affiliation_county_map} 
  
  }
  
  \caption[Democratic/Republican party lean per Colorado county]{Democratic/Republican party lean per Colorado county}\label{fig:party reg per county map}
  \end{figure}
  
  In the past 25 years, there have been a series of key changes in the way
  Colorado administers elections, in relation to Vote By Mail and other
  reforms targeted and expanding the democratic franchise. In 1992,
  Colorado introduced no-excuse absentee voting, allowing voters to either
  physically pick up a mail ballot at a Vote Center or County Office, or
  have a ballot mailed to them prior to election day. In 2008, this reform
  was expanded to a permanent Vote-By-Mail system, which gave counties the
  option to allow voters to be permanently put on a list of addresses that
  received mail ballots prior to the election. The State also entered a
  transitional status to full mail elections, giving counties the option
  to make all coordinated local elections, general elections, and primary
  elections exclusively VBM. In 2013, the Colorado State Legislature
  passed HB13-1303: The Voter Access and Modernized Elections Act, which
  mandated that every voter currently registered receive a mail ballot for
  all future elections. The Act also expanded the use of Vote Centers
  instead of traditional polling places, instituted same-day voter
  registration, and revamped the way active and inactive voter status was
  designated on voter rolls--more on this in future sections (Hullinghorst
  \& Pabon, 2013). These changes are summarized in Table 3.3.
  
  \begin{longtable}[]{@{}cl@{}}
  \caption{Key changes to Colorado elections policy
  \label{tab:elect_policy}}\tabularnewline
  \toprule
  \begin{minipage}[b]{0.07\columnwidth}\centering\strut
  Year\strut
  \end{minipage} & \begin{minipage}[b]{0.87\columnwidth}\raggedright\strut
  Key Changes\strut
  \end{minipage}\tabularnewline
  \midrule
  \endfirsthead
  \toprule
  \begin{minipage}[b]{0.07\columnwidth}\centering\strut
  Year\strut
  \end{minipage} & \begin{minipage}[b]{0.87\columnwidth}\raggedright\strut
  Key Changes\strut
  \end{minipage}\tabularnewline
  \midrule
  \endhead
  \begin{minipage}[t]{0.07\columnwidth}\centering\strut
  1992\strut
  \end{minipage} & \begin{minipage}[t]{0.87\columnwidth}\raggedright\strut
  No Excuse Absentee Statewide Implementation\strut
  \end{minipage}\tabularnewline
  \begin{minipage}[t]{0.07\columnwidth}\centering\strut
  2008\strut
  \end{minipage} & \begin{minipage}[t]{0.87\columnwidth}\raggedright\strut
  Permanent No-Excuse VBM Lists, Option of Full-VBM Elections\strut
  \end{minipage}\tabularnewline
  \begin{minipage}[t]{0.07\columnwidth}\centering\strut
  2013\strut
  \end{minipage} & \begin{minipage}[t]{0.87\columnwidth}\raggedright\strut
  Automatic Mail Ballot System Implemented Statewide, Established Vote
  Centers\strut
  \end{minipage}\tabularnewline
  \bottomrule
  \end{longtable}
  
  \clearpage
  
  \subsection{Colorado as a Case for this
  Thesis}\label{colorado-as-a-case-for-this-thesis}
  
  Colorado presents such an interesting case for research on Vote By Mail
  exactly because it has gone through such a long transitional process to
  reach its current elections system. It has steadily developed voting
  policy through a mixture of state mandates, county action, and outside
  policy motivations. Colorado's streak of independence and direct
  democracy is also very apparent in this shift in electoral practices,
  since they have been passing policies trying to expand participation for
  a very long time. It gives researchers access to approximately 22 years
  during which at least part of the state conducted elections partially by
  mail, making comparative, county- or individual- level case studies
  particularly alluring. Colorado's streak of independence and direct
  democracy is also very apparent in this shift in electoral practices,
  since they have been passing policies trying to expand participation for
  a very long time.
  
  On a more general level, Colorado is interesting exactly because it is
  ``typical'' but with a wild streak. It is typical rocky mountain
  country, great planes country, and liberal urban city but all \emph{in
  one state}. In is libertarian yet increasingly Democratic. It heavily
  relies on state funding for national parks, yet rebels against federal
  land use laws. It is a frontier state with traditional values, that
  overwhelmingly supports marijuana legalization. It is also a consistent
  purple state, with a Democratic Governor and House, but Republican
  Attorney General, Secretary of State and senate. This means that
  Colorado is a combination of distinct national effects, but also local
  effects that make it significantly different from national trends as a
  whole. In this environment, predicting results of policy can be
  difficult, but extremely salient as multiple effects can be tested
  against each other.
  
  \section{The Data}\label{the-data}
  
  This thesis relies on county and individual level models to draw
  conclusions on voting behaviors, and how they are affected by voting
  method. As such, the data I need will optimally contain the following:
  
  \begin{itemize}
  \item
    \textbf{County and individual level demographic characteristics}:
    race, gender, urban population
  \item
    \textbf{County and individual level voting data}: turnout, party
    registration, total registrants
  \item
    \textbf{Information on individual elections}: date, ballots cast,
    voting methods, county, election descriptions
  \end{itemize}
  
  In the process of my research, I have acquired sufficient data to cover
  the second and third of these areas. I was unable to procure individual
  level data on demographic characteristics apart from gender, age, and
  party registration. However, reasonable conclusions can still be drawn
  from county or precinct aggregates.
  
  \subsection{Sources and first glance}\label{sources-and-first-glance}
  
  I used two sources of data: Colorado voter records gathered by the
  Colorado Secretary of State's office, and demographic data from the 2010
  US Census. In the process of procuring these data I was aided by a
  series of other researchers and professionals with experience in the
  field of elections administration. Andrew Menger, Postdoctoral Fellow at
  the Weidenbaum Center on the Economy, Government, and Public Policy at
  Washington University, was kind enough to give me access to data files
  for Colorado for the years of 2012-2016 that he had already collected
  for his research\footnote{Doctor Menger's website with links to his
    research can be found at www.andrewmenger.com}. I directly obtained
  the 2017 from the Colorado Secretary of State's office, with the help of
  Mr.~Judd Choate, Director of Elections.
  
  \subsubsection{2010 US Census}\label{us-census}
  
  The US Census is conducted country-wide every ten years, with the goal
  of procuring accurate data on the demographic characteristics of the
  population. The Census uses a combination of federal field workers
  conducting door-to-door canvassing and statistical methods for data
  aggregation. From the 2010 Census--which is publicly available online--I
  get total population counts, characteristics on race, and rural/urban
  population counts for Colorado.
  
  I use two datasets from the Census. For both, the unit of observation is
  one of the 64 counties of Colorado, and both include the same total
  population counts. One contains racial demographic characteristics and
  the other contain percentages of rural and urban populations in each
  county. The racial demographic dataset needed some wrangling work to
  extract a percentage of white residents in each county. Individuals were
  coded as ``white'' when they identified as exclusively white--this
  doesn't include mixed-race individuals reporting white ancestry.
  
  \subsubsection{Colorado Voter Files}\label{colorado-voter-files}
  
  As mandated by HAVA, Colorado maintains a statewide registry of all
  currently registered voters. This registry is typically under the
  purview of the Secretary of State. Voter Registration Files are
  constantly updated with new information on existing voters, new voters,
  or with the removal of inactive or otherwise ineligible voters.
  Therefore, this file will be different every time it is accessed or
  shared. Based on when this file is accessed, only a ``snapshot'' of the
  file can be obtained. Similarly with VRFs, a Voter History File is
  maintained and constantly updated by the state. This file is uniquely
  connected to its VRF: only voters showing up as registrants will have
  their histories included. I have both Voter Registration and History
  files for the years between 2012-2017, obtained with the help of Judd
  Choate and Andrew Menger.
  
  In the Voter Registration files, the unit of observation is the
  individual voter, and all variables are initially coded as character
  strings. Each voter is assigned a unique voter ID, which serves as a
  point of reference between the two files. Broadly speaking, data in this
  file can be divided between three categories: first, personal
  identification information like address, ZIP code, or phone number;
  second, demographic information like age and gender; third, information
  pertinent to elections administration like congressional district, local
  elections for which the individual should receive a ballot, voter ID,
  and party registration. I will further elaborate on relevant variables
  in the wrangling section.
  
  In the Voter History files, the unit of observation here is a single
  ballot cast, and all variables are initially coded as character strings.
  This means that for each voter registered--and so included in the
  VRF--the history file should contain an observation for each time they
  voted. This file includes two types of data: first, identifiers for the
  election like county, date, description, and type; second, identifiers
  for the individual vote including voter ID and voting method.
  
  \subsection{Why Voter Registration
  Files?}\label{why-voter-registration-files}
  
  Voter Registration and History files are suitable sources for my
  analysis because they contain all the data that is necessary for a first
  pass at testing my hypotheses: voting method, county, election level,
  active registration, registration dates, and a series of individual
  characteristics like party registration, age, or gender. These data, and
  the demographic data in particular, are also in the most basic unit of
  observation: the ballot level\footnote{A good heuristic for what
    ``ballot'' level means is a specific individual at the time when they
    cast a specific ballot. Between ballots all characteristics may
    change: age, gender, party registration etc., which is why the
    ``ballot'' level is distinct from the individual level.}. This means
  that I do not need to establish any process to infer individual
  characteristics from population-wide statistics.
  
  In statistical science, sampling is the process by which individual
  units are selected from a population. The sample selected should be
  \emph{representative} of the whole so that it can be used to infer
  characteristics of the general population. Despite a vast array of
  techniques to ensure that sampling is representative, there is always
  room for error. Voter Registration Files are an excellent source of data
  because they do not involve any sampling whatsoever; they include
  \emph{all} currently registered voters and voter histories. This
  characteristic helps cut down on data-related errors and on the
  complexity of data extraction.
  
  An additional characteristic of these data is that they are concentrated
  and relatively uniform. Data transfer errors still exist, and the
  wrangling process is never entirely straightforward. However, getting
  the accuracy and uniformity exhibited in these files in the way they
  include dates, election types, parties, and other data is staggering.
  Over thirty five million observations are included in my final,
  cumulative voter history dataset, and all of them have, for example, the
  same types of entry for party registration (REP, DEM, UAF etc.).
  Admittedly, this may just be a characteristic of the Colorado files,
  since they are the only ones I used for my research.
  
  A last benefit of using Voter Registration Files comes from the
  replicability they allow for. These files are generally public, with
  access to them including only data transfer and administrative costs.
  This makes peer-review and replication less complicated than if, say, I
  was using private survey data. It also allows for expansion that goes
  beyond the State of Colorado; my code can be adapted to fit different
  data, making future comparative studies more likelly and less
  time-consuming than starting from scratch.
  
  \section{Wrangling the Data}\label{wrangling-the-data}
  
  The process of ``wrangling'' refers to manipulating the data into a form
  that can then be used for graphing, exploratory data analysis,
  modelling, or presentation. In this case, wrangling also included
  aggregating data across multiple sources and datasets. For this purpose,
  I made heavy use of the tidyverse R package, and in particular the dplyr
  package. In this section I will go through some of the key problems
  encountered during the wrangling of these data, and then discuss the
  final form each variable takes.
  
  \subsection{Initial Problems with the 2017 Voter File and
  Solution}\label{initial-problems-with-the-2017-voter-file-and-solution}
  
  In my initial research, I intended to only use the 2017 snapshots of the
  Colorado Registration and History file. The major issue I
  encountered--which merits discussion in its own section, and
  necessitated that I search for more data--comes from the aforementioned
  fact that the records I had access to are ``snapshots''. What this
  means, is that for each person in each year of voter registration files,
  I have their corresponding history files for all ballots they have cast
  in Colorado, but not their own history of registration and migration.
  If, say, a voter moved from Boulder County to Summit County, I would
  have their votes in Boulder County show up in the voter history file,
  but them being registered in Summit. If you recall the turnout
  calculations specified earlier on, this implies an overestimation when
  looking back at elections that happened some time before the date of the
  ``snapshot''. Additionally, ``snapshots'' of current voter files do not
  reflect voters dropping off the rolls for whatever reason--death, moving
  out of the state, long term inactivity, non-confirmable personal data
  etc. Since for these voters the history files would also not be
  included, the issue created is less one of overestimation of turnout
  like before, but just the inclusion of additional room for error that is
  created when subtracting one from the denominator and enumerator of
  turnout.
  
  After going through turnout calculations with the 2017 files, a
  significant majority of counties appeared to have turnout exceeding
  80\%, particularly for years between 2000 and 2012. This was, to put it
  mildly, concerning. With the aforementioned help, I was given access to
  similar ``snapshots'' for each year between 2012-2016. After similar
  calculations, I returned figure 3.4 for the eight most populous counties
  as described above, including different shapes for election type, colors
  for county, and a vertical line at 2013 to signify the latest major
  change in how Colorado administers elections.
  
  \begin{figure}
  
  {\centering \includegraphics[width=0.8\linewidth]{/Users/tdounias/Desktop/Reed_Senior_Thesis/plots/colorado_bigeight_turnout_graph} 
  
  }
  
  \caption[Turnout plot for eight largest Colorado counties, 2012-2016]{Turnout plot for eight largest Colorado counties, 2012-2016}\label{fig:big eight turnout plot}
  \end{figure}
  
  To also further illustrate the in-county migration and dropped voter
  problem, I created a graph that includes logged total counts of
  registered voters calculated using the 2017 and the 2012-2016 files. The
  plot also includes a line at y=x. If in-Colorado migration and dropped
  voters are not an issue, most points on this graph should be at this
  line.
  
  \begin{figure}
  
  {\centering \includegraphics[width=0.8\linewidth]{/Users/tdounias/Desktop/Reed_Senior_Thesis/plots/county_migration_A} 
  
  }
  
  \caption[Comparison of registration count methods]{Comparison of registration count methods}\label{fig:county migration A}
  \end{figure}
  
  Two things should be clear from figure 3.5. First, there is significant
  deviation between the counts using just the 2017 file and all files
  across years. Specifically, the 2017 count consistently underestimates
  the total amount of registered voters--this is shown by the red linear
  model smoothing line. This consistent difference means that it is close
  to impossible to generate safe conclusions on my hypotheses using only
  the 2017 files and the methods I have outlined in Chapter 2. Second,
  counts get more accurate the closer to 2017 we get. This should be even
  more apparent in figure 3.6, which limits the scale to only some high
  registration counties, and adds a shape indicator for county.
  
  \begin{figure}
  
  {\centering \includegraphics[width=0.8\linewidth]{/Users/tdounias/Desktop/Reed_Senior_Thesis/plots/county_migration_B} 
  
  }
  
  \caption[Comparison of registration count methods only for a few counties, 2012-2016]{Comparison of registration count methods only for a few counties, 2012-2016}\label{fig:county migration B}
  \end{figure}
  
  Here the structure of the data becomes clear: for each county, there are
  a series of almost vertically distributed points, which get closer to
  the y = x line the closer the counts get to 2017. Through this series of
  tests, it became clear that using multiple years of data was necessary
  in order to conduct an accurate test of my hypotheses. My selection was
  later vindicated, when looking at comparisons between reported rates of
  turnout\footnote{Turnout is calculated over all registered voters} and
  turnout calculated through my dataset for the 2014 midterm election (see
  fig. 3.7).
  
  \begin{figure}
  
  {\centering \includegraphics[width=0.8\linewidth]{/Users/tdounias/Desktop/Reed_Senior_Thesis/plots/Calc_vs_rep_turnout} 
  
  }
  
  \caption[Comparison of reported and calculated turnout for 2014 midterms across county]{Comparison of reported and calculated turnout for 2014 midterms across county}\label{fig:comp turnout 2014}
  \end{figure}
  
  The differences are insignificant. They exist because of ``noise'' added
  on because of errors in the data, misreporting, registration records
  redacted due to privacy concerns, voters dropped before the ``snapshot''
  occurred, and other similar factors.
  
  \subsection{Other Wrangling Issues}\label{other-wrangling-issues}
  
  Wrangling the data was the majority of the work that went into this
  thesis. As will become clear in this section, apart from accurately
  processing, diagnosing, and merging the data, the process of wrangling
  includes several non-trivial decisions about how to treat missing values
  and variable specification. Doing a full account would probably read
  like the world's most cliche crime novel: a series of elusive final
  datasets, a plucky yet occasionally naive young detective, two wisened
  mentors, clues, dead ends, frustration, compromise,
  and\ldots{}spreadsheets. I will spare the reader the whole story, but I
  will include a non-comprehensive list of some of the difficulties
  associated with wrangling voter files, as it was a crucial part of the
  learning process I underwent while doing my research.
  
  \textbf{Missing Values}: The decision on how to deal with missing
  values--or NAs--in a dataset is a lot more important than it may
  initially seem. A first, intuitive reaction might be to just disregard
  them; however this works under the assumption that there is no structure
  inherent to why these data are missing! To give just two examples, in
  the data I have collected, the PARTY value for the 2015 voter
  registration file is missing. If I excluded all observations with
  missing PARTY values, I would be excluding a fifth of my data. Missing
  values were also present in the VOTING\_METHOD variable of the voter
  history files. While this may have seemed troubling, after closer
  examination it was revealed that the vast majority of such missing
  values was concentrated in Jefferson County, and in elections prior to
  2002. Therefore, these observations could be ignored, since they played
  no role in my final dataset. The conclusion should be that choices made
  on exclusion, inclusion, or estimation of missing data are very
  important, and should be taken with much care and consideration for the
  underlying structure of the data.
  
  \textbf{Data Input Errors}: Is ``QATAR'' a political party in Colorado?
  State records say not. However, ``QATAR'' did show up as a value in the
  Party variable for my 2016 voter registration file snapshot. This may
  occur for a number of reasons, the most likely of which is the
  introduction of errors when transferring these data. The data I have has
  been read and written by multiple operating systems (iOS and Windows)
  and programming platforms (STATA, R); they have also been uploaded,
  downloaded, and written unto CDs, as well as transferred between County
  and Colorado Secretary of State's Office when they were created.
  Characters that would be normally read into one platform as line or
  value delimiters may have been misinterpreted by another platform, with
  no operator error involved. In my analysis I treated all values that
  seemed more likely than not to be errors as NAs. There were not many of
  these--less than .001\% of my data--but they were a hassle to find,
  analyze, and then recode into some useful value.
  
  \textbf{Data Size}: Nothing to write home about here, just an
  observation that multiple voter registration files can be \emph{huge},
  which puts considerable strain on a computer's processing power. This
  means that wrangling has to comprise of a series of careful, deliberate
  moves. Brute force should be discouraged, as a dead end means several
  hours of melodic computer fan panic.
  
  \textbf{Joining, Merging, Spreading, and the Multiplicity of Levels}:
  For the data to end up in any functional shape, it eventually becomes
  necessary to start joining datasets. Thankfully, a clear division of
  modelling tasks between county and individual level models means that
  joining on COUNTY or VOTER\_ID is ideal, and fairly straightforward. As
  will become clear in later sections, I also had to consider the variety
  of different units of observation, specifically: county, individual,
  ballot, election, county-by-election.
  
  \subsection{Final Variable
  Specifications}\label{final-variable-specifications}
  
  After the conclusion of the wrangling process, the resulting dataset
  included a series of discrete and continuous variables. I will briefly
  outline them here, along with their range and values.
  
  \begin{itemize}
  \tightlist
  \item
    VOTER\_ID: Discrete variable, unique value given to each individual
    voter. Useful for merging.
  \item
    COUNTY: Discrete variable, the 64 counties of Colorado.
  \item
    REGISTRATION\_DATE: Discrete variable, date of registration for each
    registrant. Useful to get total registrants on election day.
  \item
    TURNOUT: Continuous variable, in the range {[}0,1{]}. The response
    variable for my county-level models.
  \item
    ELECTION\_TYPE: Discrete variable, the four types of elections:
    Primary, Coordinated, Midterm, Presidential.
  \item
    ELECTION\_DATE: Discrete variable, self-explanatory.
  \item
    VBM\_PCT: Continuous variable, in the range {[}0,1{]}. This is the
    focus of my analysis, as it counts the percentage of total ballots
    that were mail ballots.
  \item
    PCT\_WHITE: Continuous variable, in the range {[}0,1{]}. Percentage of
    white residents per county.
  \item
    PCT\_URBAN: Continuous variable, in the range {[}0,1{]}. Percentage of
    urban residents per county.
  \item
    PARTY: Discrete variable. For each voter, the party they are
    registered with. Can be: Republican, Democrat, Other, or Unaffiliated.
  \item
    GENDER: Discrete binary variable, Male or Female.
  \item
    AGE: The age of the individual registrant.
  \item
    VOTING\_METHOD: The method used by an individual voter to cast their
    ballot. Is coded as either VBM or In Person, according to Table 3.4:
  \end{itemize}
  
  \begin{longtable}[]{@{}lll@{}}
  \caption{Voting methods coding table
  \label{tab:voting_methods_table}}\tabularnewline
  \toprule
  \begin{minipage}[b]{0.22\columnwidth}\raggedright\strut
  Voting Method\strut
  \end{minipage} & \begin{minipage}[b]{0.42\columnwidth}\raggedright\strut
  Description of Method\strut
  \end{minipage} & \begin{minipage}[b]{0.18\columnwidth}\raggedright\strut
  Designation\strut
  \end{minipage}\tabularnewline
  \midrule
  \endfirsthead
  \toprule
  \begin{minipage}[b]{0.22\columnwidth}\raggedright\strut
  Voting Method\strut
  \end{minipage} & \begin{minipage}[b]{0.42\columnwidth}\raggedright\strut
  Description of Method\strut
  \end{minipage} & \begin{minipage}[b]{0.18\columnwidth}\raggedright\strut
  Designation\strut
  \end{minipage}\tabularnewline
  \midrule
  \endhead
  \begin{minipage}[t]{0.22\columnwidth}\raggedright\strut
  Absentee Carry\strut
  \end{minipage} & \begin{minipage}[t]{0.42\columnwidth}\raggedright\strut
  Voters who carried an absentee ballot with them from an early voting
  location or county office\strut
  \end{minipage} & \begin{minipage}[t]{0.18\columnwidth}\raggedright\strut
  VBM\strut
  \end{minipage}\tabularnewline
  \begin{minipage}[t]{0.22\columnwidth}\raggedright\strut
  Absentee Mail\strut
  \end{minipage} & \begin{minipage}[t]{0.42\columnwidth}\raggedright\strut
  Voters who were sent an absentee ballot, and mailed it in\strut
  \end{minipage} & \begin{minipage}[t]{0.18\columnwidth}\raggedright\strut
  VBM\strut
  \end{minipage}\tabularnewline
  \begin{minipage}[t]{0.22\columnwidth}\raggedright\strut
  Early Voting\strut
  \end{minipage} & \begin{minipage}[t]{0.42\columnwidth}\raggedright\strut
  Voters who physically went to an Early Voting location and voted\strut
  \end{minipage} & \begin{minipage}[t]{0.18\columnwidth}\raggedright\strut
  In Person\strut
  \end{minipage}\tabularnewline
  \begin{minipage}[t]{0.22\columnwidth}\raggedright\strut
  In Person\strut
  \end{minipage} & \begin{minipage}[t]{0.42\columnwidth}\raggedright\strut
  Voters who physically went to a polling place and voted on paper\strut
  \end{minipage} & \begin{minipage}[t]{0.18\columnwidth}\raggedright\strut
  In Person\strut
  \end{minipage}\tabularnewline
  \begin{minipage}[t]{0.22\columnwidth}\raggedright\strut
  Mail Ballot\strut
  \end{minipage} & \begin{minipage}[t]{0.42\columnwidth}\raggedright\strut
  Vote By Mail\strut
  \end{minipage} & \begin{minipage}[t]{0.18\columnwidth}\raggedright\strut
  VBM\strut
  \end{minipage}\tabularnewline
  \begin{minipage}[t]{0.22\columnwidth}\raggedright\strut
  Polling Place\strut
  \end{minipage} & \begin{minipage}[t]{0.42\columnwidth}\raggedright\strut
  Traditional polling place voting, discontinued in 2013\strut
  \end{minipage} & \begin{minipage}[t]{0.18\columnwidth}\raggedright\strut
  In Person\strut
  \end{minipage}\tabularnewline
  \begin{minipage}[t]{0.22\columnwidth}\raggedright\strut
  Vote Center\strut
  \end{minipage} & \begin{minipage}[t]{0.42\columnwidth}\raggedright\strut
  Voters who cast their ballots at Vote Centers\strut
  \end{minipage} & \begin{minipage}[t]{0.18\columnwidth}\raggedright\strut
  In Person\strut
  \end{minipage}\tabularnewline
  \bottomrule
  \end{longtable}
  
  \chapter{Model Specification and
  Results}\label{model-specification-and-results}
  
  The goal of this chapter is apply inferential statistical modeling to
  the data. This task can be divided into three steps: specifying the
  models mathematically, fitting the models, and interpreting the
  results\footnote{In theory there is also the step of translating the
    models from their mathematical specification into some sort of
    algorithmic process that produces estimates of coefficients and error
    terms. This process is arduous and long, so it is not included in this
    chapter. Appendix A deals with some of the techniques involved with
    model estimation}. However, before jumping into this process, it is
  worth going into some key problems with the models produced. As a
  consequence of these issues, some of the models are not estimated to the
  standards of convergence that are commonly set.
  
  \section{Modelling Issues}\label{modelling-issues}
  
  \subsection{Lack of variability}\label{lack-of-variability}
  
  To put it very simply, it's not enough to have hundreds of thousands of
  observations if they are all almost identical to each other. If, for
  example, my data included a thousand people in Jefferson county, and 63
  in all other counties of Colorado combined (one in each remaining
  county), then I would not be able to leverage my data to draw
  conclusions on county-level effects.
  
  As previously stated, the data available includes registration files
  going back to 2012. From these files, I have extracted data for
  elections going back to 2010.\footnote{See section 3.3.1; the extracted
    data is limited to this time period to avoid accuracy issues with
    migration and removal of inactive/unavailable voters.} In order to
  make inferences on VBM and turnout effects it is necessary to have
  extensive and varied data. Specifically, it is necessary to have data
  that include a large enough sample of the voters in Colorado, with a
  substantial portion of them using different voting methods, from
  different counties, or in different election years etc.
  
  The data are extensive (over 35 million observations at the individual
  level) but substantially lack variance in voting method. Put simply, the
  vast majority of registrants in Colorado from 2010 onward either did not
  vote at all, or voted by mail. If you recall the changes in Colorado
  election law, in 2008 counties were allowed to conduct all mail
  elections, and no-excuse permanent absentee voting was implemented
  state-wide; then in 2013 Colorado transitioned to full VBM for all
  elections. This means that few people were still using traditional
  polling places or vote centers to cast their ballots. Figure 4.1 shows
  how, after 2013, and even before that in 2011--the coordinated, local
  election for which mail ballots were more convenient for counties--over
  95\% of ballots cast were mail ballots. Only in the general elections of
  2010 and 2012 is there some variance, but mail ballots account for well
  over two thirds of total votes.
  
  \begin{figure}
  
  {\centering \includegraphics[width=0.8\linewidth]{/Users/tdounias/Desktop/Reed_Senior_Thesis/plots/vbm_county_graph} 
  
  }
  
  \caption[Percentage of mail ballots over total ballots by year]{Percentage of mail ballots over total ballots by year}\label{fig:vbm png}
  \end{figure}
  
  This issue is not completely fatal for county level models. There is
  still variance between counties that have 100\% mail ballots and those
  that are around the 75-85\% margin. For individual level models, where I
  am estimating voting probability, VBM will be an almost perfect
  predictor for voting, and therefore will not present me with any
  substantial analytical result on how it affects voting probability.
  There are some ways to compensate for this issue, which I outline; due
  to time or data constraints, not all of these will be implemented in
  this thesis:
  
  \begin{itemize}
  \item
    \emph{More (Diverse) Data}: It would be very useful to get snapshot
    data of Colorado voter files from, say, 2004 to today, because it
    would allow for an extensive study on how the 2008 and 2013 election
    laws re-shaped voting decisions in the state. It would be useful, but
    also expensive and very time consuming, involving several purchases of
    data from the Secretary of State of Colorado. Voter registration files
    also tend to get messier the further back one goes, which means that
    the process of cleaning up the data would get substantially harder. It
    would also require more processing power to handle more observations.
    My research here does not do this, as the scope of a senior thesis is
    a lot more limited than such an overarching study that would probably
    be conducted by multiple researchers with several assistants. I do
    however present several replicable materials for such a study, through
    the creation of an \textit{R} package I include on my GitHub page
    along with the final results of this thesis. This thesis does not go
    that far, but it may help similar studies in the future.
  \item
    \emph{Localized, Natural Experiment Studies}: A natural experiment is
    when, due to policy changes and circumstances, a ``control'' and
    ``treatment'' group of such a policy are created in the same
    approximate geographical area. This happens when, for example, only
    some of the counties in a state enact a specific change. Several such
    studies exist already, with some even tackling VBM in Colorado (Keele
    \& Titiunik, 2017), or how turnout rates are affected by new,
    restrictive registration laws (Burden \& Neiheisel, 2013). This method
    would allow for more accuracy in both the individual and county level
    models, and through the existence of a treatment and control group
    would guarantee the variability that I am currently lacking.
  \item
    \emph{Synthetic Control Group}: The synthetic control group method is
    a way of creating a control group when no such group seems to exist.
    It involves gathering a set of characteristics from the treatment
    group members and then using statistical methods to combine them into
    making the appropriate control (McClelland \& Gault, 2017). I will not
    go into the particulars of this method (the sources cited should
    provide a decent introduction), but this method has been successful in
    assessing policy effects such as anti-smoking laws (Barr et al.,
    2012), or even motor voter laws in Oregon(Gronke, McGhee, Romero, \&
    Griffin, 2017).
  \end{itemize}
  
  \subsection{Computational
  Considerations}\label{computational-considerations}
  
  The process of computing estimates for model coefficients can often be
  very computationally intensive. This issue was particularly present
  during estimation of individual level models, which alongside complex
  hierarchical structure also draw on a huge dataset of 35 million
  observations. This computationally intensive procedure requires more
  processing power than I currently have available. For now, I compensated
  for this problem by using stratified sampling to sample a subset of my
  observations\footnote{A long term solution to this issue could be the
    use of a more powerful local RStudio server, or Amazon Web Services
    (AWS).}.
  
  The form of stratified sampling I am using is very simple; based on
  county, mail vote, and electoral participation, I use \texttt{dplyr} in
  \textit{R} to draw a sample that contains equal proportions of every
  combination of values of these variables to those in the original
  dataset. If, for example, the original dataset had 2\% of entries being
  voters from Jefferson county that participated using a mail ballot, the
  sampled dataset would have a proportion that is approximately equal to
  2\% (Chihara \& Hesterberg, 2011). In this way I draw a sample of around
  400,000 observations from my initial ballot dataset, on which I run all
  my individual models. After checking the variable ratios in sampled and
  population datasets, I found that the differences between ratios had a
  mean and standard deviation of less than a hundredth of a percentile.
  Therefore this sampled dataset could serve as a decent approximation of
  my population.
  
  \section{Variable Specification}\label{variable-specification}
  
  I will not go through each individual variable in this section, but will
  briefly describe my notation for the following models. I will include
  more comments whenever they seem necessary under each model. In this
  thesis I include predictors on a series of variables that can be divided
  into five categories based on unit of observation: county, election,
  individual, local result, and ballot. The last two are functions of
  other units: local result units are equal to the product of elections
  and counties, while ballot units are equal to the number of unique
  individuals multiplied by the number of elections each of them was
  registered in. For notation, I follow this set of rules:
  
  \begin{enumerate}
  \def\labelenumi{\arabic{enumi}.}
  \tightlist
  \item
    If the variable is a response, it is coded \(y\).
  \item
    If the variable is a predictor, it is coded \(x\)
  \item
    The variable's superscript will provide information on what it
    represents, else it will be explained.
  \item
    All variables represent a single value (scalar) of that variable
    unless stated otherwise.
  \item
    Unit of observation will also be specified in subscript, according to
    the indices described in Table 4.1. These indices are also used in sum
    notation.
  \item
    All Greek characters represent coefficients to be calculated.
  \item
    By \(k[j]\) I represent the k-value corresponding to the
    j-observation. In this case, this would be the county that an
    individual is registered in.
  \item
    Note that for Local Result level variables, I use \(k,l\) as an index.
    This is because there are very few variables at this level, it is a
    direct Cartesian product of two other units, and this notation avoids
    confusion with even more index types.
  \end{enumerate}
  
  \begin{longtable}[]{@{}lc@{}}
  \caption{Variable indices per unit of observation
  \label{tab:units_vars}}\tabularnewline
  \toprule
  \begin{minipage}[b]{0.27\columnwidth}\raggedright\strut
  Units\strut
  \end{minipage} & \begin{minipage}[b]{0.16\columnwidth}\centering\strut
  Index\strut
  \end{minipage}\tabularnewline
  \midrule
  \endfirsthead
  \toprule
  \begin{minipage}[b]{0.27\columnwidth}\raggedright\strut
  Units\strut
  \end{minipage} & \begin{minipage}[b]{0.16\columnwidth}\centering\strut
  Index\strut
  \end{minipage}\tabularnewline
  \midrule
  \endhead
  \begin{minipage}[t]{0.27\columnwidth}\raggedright\strut
  Ballot\strut
  \end{minipage} & \begin{minipage}[t]{0.16\columnwidth}\centering\strut
  i\strut
  \end{minipage}\tabularnewline
  \begin{minipage}[t]{0.27\columnwidth}\raggedright\strut
  Individual\strut
  \end{minipage} & \begin{minipage}[t]{0.16\columnwidth}\centering\strut
  j\strut
  \end{minipage}\tabularnewline
  \begin{minipage}[t]{0.27\columnwidth}\raggedright\strut
  County\strut
  \end{minipage} & \begin{minipage}[t]{0.16\columnwidth}\centering\strut
  k\strut
  \end{minipage}\tabularnewline
  \begin{minipage}[t]{0.27\columnwidth}\raggedright\strut
  Election\strut
  \end{minipage} & \begin{minipage}[t]{0.16\columnwidth}\centering\strut
  l\strut
  \end{minipage}\tabularnewline
  \begin{minipage}[t]{0.27\columnwidth}\raggedright\strut
  General Index\strut
  \end{minipage} & \begin{minipage}[t]{0.16\columnwidth}\centering\strut
  v\strut
  \end{minipage}\tabularnewline
  \bottomrule
  \end{longtable}
  
  \section{County Level Models}\label{county-level-models}
  
  \subsection{Specifications}\label{specifications}
  
  In this section I will go through a step-by step creation of models at
  the county level. County level models use a series of variables at the
  election, county, and local result levels. The response variable is
  always turnout in one county after a particular election. With no other
  information, this model could be thought of as an assignment of voting
  tendencies across counties; each county independent of election has a
  unique range of turnout results. In this way it is possible to build a
  naive, baseline model of turnout as follows:
  
  \begin{equation} \tag{Model 1}
  Y^{turnout}_{k,l} = \beta_0 + (\sum_{k=1}^{64}\beta_kx_k^{county}) + \epsilon,\ \epsilon \sim N(0,\sigma^2)
  \end{equation}
  
  where \(x_k^{county}\) is a series of 64 dummy variables for each county
  of Colorado. Here differences between elections come from normally
  distributed error terms, rather than predictors. I name this
  \textbf{Model 1}, and it does not fit the data particularly well. First
  off, this model includes the assumption that counties are independent of
  one another, which is probably false; just consider that these counties
  are areas of the same state, in the same country, with populations
  moving between them at regular intervals, and many of them covering the
  same metropolitan area or congressional district. Additionally, the
  model matrix here is rank deficient; there are two county coefficients
  that are perfect linear combinations of other coefficients. This means
  they will be dropped by \textit{R} when the model is called in the
  \texttt{lm()} function.
  
  A way to fix both these issues is to use a multilevel model with mixed
  effects for county. By constraining coefficients at the county level to
  a set distribution, this model does away with the assumption of
  independence. The other county level predictors help to explain some of
  the unexplained group level variation, which reduces the standard
  deviation of county coefficients and helps provide more exact estimates
  (Gelman \& Hill, 2006). I call this \textbf{Model 2}, which can be
  written as:
  
  \begin{equation} \tag{Model 2}
  Y^{turnout}_{k,l} = a_{k} + \beta_{1}x_k^{\%white} + \beta_{2}x_k^{\%urban} + \epsilon,
  \end{equation}
  
  \[a_{k} \sim N (\gamma_0, \sigma_{\alpha}^2)\]
  \[\epsilon \sim N(0, \sigma^2)\] This model provides a more reasonable
  set of estimates for each county, but still fails to provide any
  information as to secular trends, time-specific effects, election type
  effects, or mail voting--the variable of interest. I will amend this by
  adding a set of variables at the election and local result levels:
  election type and an interaction term between election type and mail
  voting. This variable should reflect whether turnout effects of mail
  voting are more pronounced in a specific type of election. I call this
  \textbf{Model 3} and it can be specified as follows:
  
  \begin{multline} \tag{Model 3}
  Y^{turnout}_{k,l} = a_{k} + \beta_{1}x_k^{\% white} + \beta_{2}x_k^{\% urban} + \overbrace{(\sum_{v=3}^{6}\beta_{v}x_{v}^{election type} x_{k,l}^{\% mail~vote})}^\text{Interaction Effect with Type} + \\ \overbrace{(\sum_{v=7}^{10}\beta_{v}x_{v}^{election type})}^\text{Main Effect of Election Type} + \epsilon,
  \end{multline}
  
  \[a_{k} \sim N(\gamma_0, \sigma_{\alpha}^2)\]
  \[\epsilon \sim N(0, \sigma^2)\]
  
  where \(x_{v}^{election type}\) is a series of four dummy variables for
  each type of election (General, Primary, Coordinated, Midterm). This
  model reflects nearly all the information I have available, apart from
  election date. For the incorporation of election dates there are two
  possible alternatives. First, I can simply add a dummy variable for each
  year. This would assume independence between each year, as it would
  specify different, independent ``slopes'' for the seven years I have
  data for--this is like calculating seven different models, one for each
  year. This is not particularly elegant as a solution nor does it reflect
  the fact that years actually are interconnected; of course there can be
  massive shifts in national or regional political climates, but those
  shifts happened \emph{from some baseline}, which is reflected in
  previous years.
  
  These elections can be thought of as systems for which prior condition
  affects future outcomes, and therefore time cannot be modeled as a
  series of independent effects. The solution here is adding a spline
  function for time, using a general additive multilevel model. The most
  commonly used spline function, and the default in the \texttt{gamm4}
  \textit{R} package is a thin plate regression spline, which I also use
  here (S. N. Wood, 2006). More on the subject of splines can be found in
  the Wood (2006) textbook. The model, which I call \textbf{Model 4} can
  be written as follows:
  
  \begin{multline}\tag{Model 4}
  Y^{turnout}_{k,l} = a_{k} + \beta_{1}x_k^{\% white} + \beta_{2}x_k^{\% urban} + \overbrace{(\sum_{v=3}^{6}\beta_{v}x_{v}^{election type} x_{k,l}^{\% mail~vote})}^\text{Interaction Effect with Type} + \\ \overbrace{(\sum_{v=7}^{10}\beta_{v}x_{v}^{election type})}^\text{Main Effect of Election Type} + s(x^{year}_{l}) + \epsilon,
  \end{multline}
  
  \[a_{k} \sim N(\gamma_0, \sigma_{\alpha}^2)\]
  \[\epsilon \sim N(0, \sigma^2)\]
  
  where \(s()\) is a natural cubic regression spline function with seven
  knots--equal to the number of years.\footnote{I used the
    \texttt{gam.check()} function that is present in the \texttt{mgcv}
    \textit{R} package, whose call determined that the number of knots
    here may be too low. However, given the data available to me, I was
    limited to the inclusion of seven years and as such cannot increase
    the number of knots any further. Setting the number of knots to seven
    also gave the lowest CV MSE.} A summary of these four models is
  provided in the following table:
  
  \begin{longtable}[]{@{}ll@{}}
  \caption{County level model descriptions
  \label{tab:model_desc_county}}\tabularnewline
  \toprule
  \begin{minipage}[b]{0.15\columnwidth}\raggedright\strut
  Model No\strut
  \end{minipage} & \begin{minipage}[b]{0.80\columnwidth}\raggedright\strut
  Model Description\strut
  \end{minipage}\tabularnewline
  \midrule
  \endfirsthead
  \toprule
  \begin{minipage}[b]{0.15\columnwidth}\raggedright\strut
  Model No\strut
  \end{minipage} & \begin{minipage}[b]{0.80\columnwidth}\raggedright\strut
  Model Description\strut
  \end{minipage}\tabularnewline
  \midrule
  \endhead
  \begin{minipage}[t]{0.15\columnwidth}\raggedright\strut
  Model 1\strut
  \end{minipage} & \begin{minipage}[t]{0.80\columnwidth}\raggedright\strut
  Baseline model with only county specific effects\strut
  \end{minipage}\tabularnewline
  \begin{minipage}[t]{0.15\columnwidth}\raggedright\strut
  Model 2\strut
  \end{minipage} & \begin{minipage}[t]{0.80\columnwidth}\raggedright\strut
  Multilevel model; added county level predictors\strut
  \end{minipage}\tabularnewline
  \begin{minipage}[t]{0.15\columnwidth}\raggedright\strut
  Model 3\strut
  \end{minipage} & \begin{minipage}[t]{0.80\columnwidth}\raggedright\strut
  Multilevel model; added VBM, interaction terms, and election fixed
  effects\strut
  \end{minipage}\tabularnewline
  \begin{minipage}[t]{0.15\columnwidth}\raggedright\strut
  Model 4\strut
  \end{minipage} & \begin{minipage}[t]{0.80\columnwidth}\raggedright\strut
  Multilevel General Additive model; added spline function for election
  year\strut
  \end{minipage}\tabularnewline
  \bottomrule
  \end{longtable}
  
  \subsection{Results}\label{results}
  
  The table in this section presents coefficients and standard errors for
  all four county level models. This table does not include any metrics
  for county--either mixed or fixed effects. I have chosen to omit these
  because they firstly are not very relevant to my hypotheses, and
  secondly because they are very extensive--64 coefficients for each of
  the four models. I have also not included any metric for time--here
  measured in years and used only in the fourth model. Both the mixed
  effects for county and the measure for time should be considered as
  controls: the first controls for county-specific trends while still
  restricting these to allow for non-independence, and the second makes
  sure that my results are indicative of a secular trend, accounting for
  any shifts along time.
  
  In terms of goodness-of-fit, I use 5-fold cross-validated Mean Squared
  Error (MSE) for all of the models. There is a significant drop-off in
  MSE between Models 1, 2 and Models 3, 4 of around \(.35\), which shows
  that the variables introduced in the later models substantially increase
  how well the models explain variability in the data. There is also a
  small increase of CV MSE between models 3 and 4, but the numbers are
  very comparable\footnote{If I was trying to make a model for predictive
    purposes I would probably choose Model 3; however, there is still
    value in comparing Models 3 and 4, even if the later doesn't fit the
    data better than 3. The difference in coefficient values, after
    controlling for time, is a particularly interesting result.}.
  
  \begin{longtable}[]{@{}lcccc@{}}
  \caption{Estimated county level coefficients (**Significant at the .01
  level *at the .05 level) \label{tab:county_coef}}\tabularnewline
  \toprule
  \begin{minipage}[b]{0.26\columnwidth}\raggedright\strut
  Variables\strut
  \end{minipage} & \begin{minipage}[b]{0.12\columnwidth}\centering\strut
  Model 1\strut
  \end{minipage} & \begin{minipage}[b]{0.13\columnwidth}\centering\strut
  Model 2\strut
  \end{minipage} & \begin{minipage}[b]{0.14\columnwidth}\centering\strut
  Model 3\strut
  \end{minipage} & \begin{minipage}[b]{0.14\columnwidth}\centering\strut
  Model 4\strut
  \end{minipage}\tabularnewline
  \midrule
  \endfirsthead
  \toprule
  \begin{minipage}[b]{0.26\columnwidth}\raggedright\strut
  Variables\strut
  \end{minipage} & \begin{minipage}[b]{0.12\columnwidth}\centering\strut
  Model 1\strut
  \end{minipage} & \begin{minipage}[b]{0.13\columnwidth}\centering\strut
  Model 2\strut
  \end{minipage} & \begin{minipage}[b]{0.14\columnwidth}\centering\strut
  Model 3\strut
  \end{minipage} & \begin{minipage}[b]{0.14\columnwidth}\centering\strut
  Model 4\strut
  \end{minipage}\tabularnewline
  \midrule
  \endhead
  \begin{minipage}[t]{0.26\columnwidth}\raggedright\strut
  (Intercept)\strut
  \end{minipage} & \begin{minipage}[t]{0.12\columnwidth}\centering\strut
  0.369\strut
  \end{minipage} & \begin{minipage}[t]{0.13\columnwidth}\centering\strut
  0.492**\strut
  \end{minipage} & \begin{minipage}[t]{0.14\columnwidth}\centering\strut
  0.455\strut
  \end{minipage} & \begin{minipage}[t]{0.14\columnwidth}\centering\strut
  0.470**\strut
  \end{minipage}\tabularnewline
  \begin{minipage}[t]{0.26\columnwidth}\raggedright\strut
  \strut
  \end{minipage} & \begin{minipage}[t]{0.12\columnwidth}\centering\strut
  (0.60)\strut
  \end{minipage} & \begin{minipage}[t]{0.13\columnwidth}\centering\strut
  (0.045)\strut
  \end{minipage} & \begin{minipage}[t]{0.14\columnwidth}\centering\strut
  (0.078)**\strut
  \end{minipage} & \begin{minipage}[t]{0.14\columnwidth}\centering\strut
  (0.072)\strut
  \end{minipage}\tabularnewline
  \begin{minipage}[t]{0.26\columnwidth}\raggedright\strut
  Pct\_white\strut
  \end{minipage} & \begin{minipage}[t]{0.12\columnwidth}\centering\strut
  \strut
  \end{minipage} & \begin{minipage}[t]{0.13\columnwidth}\centering\strut
  0.034\strut
  \end{minipage} & \begin{minipage}[t]{0.14\columnwidth}\centering\strut
  0.033\strut
  \end{minipage} & \begin{minipage}[t]{0.14\columnwidth}\centering\strut
  0.031\strut
  \end{minipage}\tabularnewline
  \begin{minipage}[t]{0.26\columnwidth}\raggedright\strut
  \strut
  \end{minipage} & \begin{minipage}[t]{0.12\columnwidth}\centering\strut
  \strut
  \end{minipage} & \begin{minipage}[t]{0.13\columnwidth}\centering\strut
  (0.053)\strut
  \end{minipage} & \begin{minipage}[t]{0.14\columnwidth}\centering\strut
  (0.050)\strut
  \end{minipage} & \begin{minipage}[t]{0.14\columnwidth}\centering\strut
  (0.050)\strut
  \end{minipage}\tabularnewline
  \begin{minipage}[t]{0.26\columnwidth}\raggedright\strut
  Pct\_urban\strut
  \end{minipage} & \begin{minipage}[t]{0.12\columnwidth}\centering\strut
  \strut
  \end{minipage} & \begin{minipage}[t]{0.13\columnwidth}\centering\strut
  -0.118**\strut
  \end{minipage} & \begin{minipage}[t]{0.14\columnwidth}\centering\strut
  -0.117**\strut
  \end{minipage} & \begin{minipage}[t]{0.14\columnwidth}\centering\strut
  -0.119**\strut
  \end{minipage}\tabularnewline
  \begin{minipage}[t]{0.26\columnwidth}\raggedright\strut
  \strut
  \end{minipage} & \begin{minipage}[t]{0.12\columnwidth}\centering\strut
  \strut
  \end{minipage} & \begin{minipage}[t]{0.13\columnwidth}\centering\strut
  (0.022)\strut
  \end{minipage} & \begin{minipage}[t]{0.14\columnwidth}\centering\strut
  (0.021)\strut
  \end{minipage} & \begin{minipage}[t]{0.14\columnwidth}\centering\strut
  (0.021)\strut
  \end{minipage}\tabularnewline
  \begin{minipage}[t]{0.26\columnwidth}\raggedright\strut
  typeGeneral\strut
  \end{minipage} & \begin{minipage}[t]{0.12\columnwidth}\centering\strut
  \strut
  \end{minipage} & \begin{minipage}[t]{0.13\columnwidth}\centering\strut
  \strut
  \end{minipage} & \begin{minipage}[t]{0.14\columnwidth}\centering\strut
  0.190**\strut
  \end{minipage} & \begin{minipage}[t]{0.14\columnwidth}\centering\strut
  0.254**\strut
  \end{minipage}\tabularnewline
  \begin{minipage}[t]{0.26\columnwidth}\raggedright\strut
  \strut
  \end{minipage} & \begin{minipage}[t]{0.12\columnwidth}\centering\strut
  \strut
  \end{minipage} & \begin{minipage}[t]{0.13\columnwidth}\centering\strut
  \strut
  \end{minipage} & \begin{minipage}[t]{0.14\columnwidth}\centering\strut
  (0.070)\strut
  \end{minipage} & \begin{minipage}[t]{0.14\columnwidth}\centering\strut
  (0.065)\strut
  \end{minipage}\tabularnewline
  \begin{minipage}[t]{0.26\columnwidth}\raggedright\strut
  typeMidterm\strut
  \end{minipage} & \begin{minipage}[t]{0.12\columnwidth}\centering\strut
  \strut
  \end{minipage} & \begin{minipage}[t]{0.13\columnwidth}\centering\strut
  \strut
  \end{minipage} & \begin{minipage}[t]{0.14\columnwidth}\centering\strut
  0.252**\strut
  \end{minipage} & \begin{minipage}[t]{0.14\columnwidth}\centering\strut
  0.070\strut
  \end{minipage}\tabularnewline
  \begin{minipage}[t]{0.26\columnwidth}\raggedright\strut
  \strut
  \end{minipage} & \begin{minipage}[t]{0.12\columnwidth}\centering\strut
  \strut
  \end{minipage} & \begin{minipage}[t]{0.13\columnwidth}\centering\strut
  \strut
  \end{minipage} & \begin{minipage}[t]{0.14\columnwidth}\centering\strut
  (0.068)\strut
  \end{minipage} & \begin{minipage}[t]{0.14\columnwidth}\centering\strut
  (0.063)\strut
  \end{minipage}\tabularnewline
  \begin{minipage}[t]{0.26\columnwidth}\raggedright\strut
  typePrimary\strut
  \end{minipage} & \begin{minipage}[t]{0.12\columnwidth}\centering\strut
  \strut
  \end{minipage} & \begin{minipage}[t]{0.13\columnwidth}\centering\strut
  \strut
  \end{minipage} & \begin{minipage}[t]{0.14\columnwidth}\centering\strut
  -0.071\strut
  \end{minipage} & \begin{minipage}[t]{0.14\columnwidth}\centering\strut
  -0.170**\strut
  \end{minipage}\tabularnewline
  \begin{minipage}[t]{0.26\columnwidth}\raggedright\strut
  \strut
  \end{minipage} & \begin{minipage}[t]{0.12\columnwidth}\centering\strut
  \strut
  \end{minipage} & \begin{minipage}[t]{0.13\columnwidth}\centering\strut
  \strut
  \end{minipage} & \begin{minipage}[t]{0.14\columnwidth}\centering\strut
  (0.069)\strut
  \end{minipage} & \begin{minipage}[t]{0.14\columnwidth}\centering\strut
  (0.062)\strut
  \end{minipage}\tabularnewline
  \begin{minipage}[t]{0.26\columnwidth}\raggedright\strut
  typeCoordinated*VBM\strut
  \end{minipage} & \begin{minipage}[t]{0.12\columnwidth}\centering\strut
  \strut
  \end{minipage} & \begin{minipage}[t]{0.13\columnwidth}\centering\strut
  \strut
  \end{minipage} & \begin{minipage}[t]{0.14\columnwidth}\centering\strut
  -0.001\strut
  \end{minipage} & \begin{minipage}[t]{0.14\columnwidth}\centering\strut
  0.002\strut
  \end{minipage}\tabularnewline
  \begin{minipage}[t]{0.26\columnwidth}\raggedright\strut
  \strut
  \end{minipage} & \begin{minipage}[t]{0.12\columnwidth}\centering\strut
  \strut
  \end{minipage} & \begin{minipage}[t]{0.13\columnwidth}\centering\strut
  \strut
  \end{minipage} & \begin{minipage}[t]{0.14\columnwidth}\centering\strut
  (0.067)\strut
  \end{minipage} & \begin{minipage}[t]{0.14\columnwidth}\centering\strut
  (0.058)\strut
  \end{minipage}\tabularnewline
  \begin{minipage}[t]{0.26\columnwidth}\raggedright\strut
  typeGeneral*VBM\strut
  \end{minipage} & \begin{minipage}[t]{0.12\columnwidth}\centering\strut
  \strut
  \end{minipage} & \begin{minipage}[t]{0.13\columnwidth}\centering\strut
  \strut
  \end{minipage} & \begin{minipage}[t]{0.14\columnwidth}\centering\strut
  0.151*\strut
  \end{minipage} & \begin{minipage}[t]{0.14\columnwidth}\centering\strut
  0.087*\strut
  \end{minipage}\tabularnewline
  \begin{minipage}[t]{0.26\columnwidth}\raggedright\strut
  \strut
  \end{minipage} & \begin{minipage}[t]{0.12\columnwidth}\centering\strut
  \strut
  \end{minipage} & \begin{minipage}[t]{0.13\columnwidth}\centering\strut
  \strut
  \end{minipage} & \begin{minipage}[t]{0.14\columnwidth}\centering\strut
  (0.073)\strut
  \end{minipage} & \begin{minipage}[t]{0.14\columnwidth}\centering\strut
  (0.037)\strut
  \end{minipage}\tabularnewline
  \begin{minipage}[t]{0.26\columnwidth}\raggedright\strut
  typeMidterm*VBM\strut
  \end{minipage} & \begin{minipage}[t]{0.12\columnwidth}\centering\strut
  \strut
  \end{minipage} & \begin{minipage}[t]{0.13\columnwidth}\centering\strut
  \strut
  \end{minipage} & \begin{minipage}[t]{0.14\columnwidth}\centering\strut
  -0.058\strut
  \end{minipage} & \begin{minipage}[t]{0.14\columnwidth}\centering\strut
  0.109*\strut
  \end{minipage}\tabularnewline
  \begin{minipage}[t]{0.26\columnwidth}\raggedright\strut
  \strut
  \end{minipage} & \begin{minipage}[t]{0.12\columnwidth}\centering\strut
  \strut
  \end{minipage} & \begin{minipage}[t]{0.13\columnwidth}\centering\strut
  \strut
  \end{minipage} & \begin{minipage}[t]{0.14\columnwidth}\centering\strut
  (0.026)\strut
  \end{minipage} & \begin{minipage}[t]{0.14\columnwidth}\centering\strut
  (0.030)\strut
  \end{minipage}\tabularnewline
  \begin{minipage}[t]{0.26\columnwidth}\raggedright\strut
  typePrimary*VBM\strut
  \end{minipage} & \begin{minipage}[t]{0.12\columnwidth}\centering\strut
  \strut
  \end{minipage} & \begin{minipage}[t]{0.13\columnwidth}\centering\strut
  \strut
  \end{minipage} & \begin{minipage}[t]{0.14\columnwidth}\centering\strut
  -0.089\strut
  \end{minipage} & \begin{minipage}[t]{0.14\columnwidth}\centering\strut
  -0.003\strut
  \end{minipage}\tabularnewline
  \begin{minipage}[t]{0.26\columnwidth}\raggedright\strut
  \strut
  \end{minipage} & \begin{minipage}[t]{0.12\columnwidth}\centering\strut
  \strut
  \end{minipage} & \begin{minipage}[t]{0.13\columnwidth}\centering\strut
  \strut
  \end{minipage} & \begin{minipage}[t]{0.14\columnwidth}\centering\strut
  (0.028)\strut
  \end{minipage} & \begin{minipage}[t]{0.14\columnwidth}\centering\strut
  (0.027)\strut
  \end{minipage}\tabularnewline
  \begin{minipage}[t]{0.26\columnwidth}\raggedright\strut
  CV MSE\strut
  \end{minipage} & \begin{minipage}[t]{0.12\columnwidth}\centering\strut
  0.041\strut
  \end{minipage} & \begin{minipage}[t]{0.13\columnwidth}\centering\strut
  0.040\strut
  \end{minipage} & \begin{minipage}[t]{0.14\columnwidth}\centering\strut
  0.004\strut
  \end{minipage} & \begin{minipage}[t]{0.14\columnwidth}\centering\strut
  0.006\strut
  \end{minipage}\tabularnewline
  \begin{minipage}[t]{0.26\columnwidth}\raggedright\strut
  Obs\strut
  \end{minipage} & \begin{minipage}[t]{0.12\columnwidth}\centering\strut
  704\strut
  \end{minipage} & \begin{minipage}[t]{0.13\columnwidth}\centering\strut
  704\strut
  \end{minipage} & \begin{minipage}[t]{0.14\columnwidth}\centering\strut
  704\strut
  \end{minipage} & \begin{minipage}[t]{0.14\columnwidth}\centering\strut
  704\strut
  \end{minipage}\tabularnewline
  \begin{minipage}[t]{0.26\columnwidth}\raggedright\strut
  Groups\strut
  \end{minipage} & \begin{minipage}[t]{0.12\columnwidth}\centering\strut
  64\strut
  \end{minipage} & \begin{minipage}[t]{0.13\columnwidth}\centering\strut
  64\strut
  \end{minipage} & \begin{minipage}[t]{0.14\columnwidth}\centering\strut
  64\strut
  \end{minipage} & \begin{minipage}[t]{0.14\columnwidth}\centering\strut
  64\strut
  \end{minipage}\tabularnewline
  \bottomrule
  \end{longtable}
  
  Given that, the first observable result is that the percentage of white
  population and the percentage of urban population are fairly stable
  indicators of slightly higher and lower levels of turnout respectively,
  although only urban population reaches statistical significance at the
  .05 level. The lack of variability between models is not surprising;
  these represent a county-level, time-independent demographic statistic,
  and there would be no reason to assume that part of their effect would
  be subsumed by other variables in Models 3 and 4.
  
  Moving on to election type, the coefficients for the different election
  types should be read as differences from the ``baseline'' that is
  typeCoordinated. First surprising result here is that the coefficient
  for general presidential elections is substantially lower than that of
  midterms. Rather, this would be surprising if we did not notice the
  interaction terms with VBM, which indicate that after allowing for VBM
  effects, presidential elections do actually have higher turnout in my
  model than midterms do\footnote{Remember here that due to Figure 4.1
    most counties will have a proportion of mail ballots close to .9}.
  Other than this, coefficients in Model 3 and Model 4 make sense, in the
  assumed ordering of turnout in such elections: presidential, then
  midterm, then coordinated and primary.
  
  Next, taking election type and all interaction terms into consideration,
  let's examine what happens when the spline function for time is
  introduced between Models 3 and 4. Most coefficients shift dramatically,
  with the exception of the interaction between coordinated elections and
  VBM. This dramatic shift--between 5 and 15(!) percentage
  points--indicates that several of the effects that the third model
  estimated are actually time-specific trends, and that there is a
  significant difference if we account for them. In the fourth model, the
  coefficients for election type on their own are still indicative of a
  common assumption for turnout in such elections\footnote{Also see Figure
    3.4}. As for interaction terms with VBM, the effect of VBM on primary
  election turnout is almost wiped out entirely, the interaction with
  general election turnout is depleted but still present at around 8\%,
  and coordinated election VBM effects remain statistically insignificant.
  Interestingly, the effect of VBM on midterm turnout switches sign from a
  negative effect of 5\% to a positive effect of around 11\%.
  
  Taking my hypotheses one by one, these models present evidence in favor
  of H1. Mail voting does seem to affect turnout in a way consistent
  across time--see the coefficients for VBM effects on general
  elections--but this effect is not particularly more strong than the
  percentage of urban population in each county. Conversely, my second and
  third hypotheses can be convincingly rejected at the county level. After
  controlling for time, the effect that VBM has on coordinated or primary
  elections is not statistically significant, compared to significant,
  consistent effects on midterm and general elections. The one point in
  favor of H3 here is that the effect of VBM on midterm elections is
  slightly higher--about 2\%--than the effect on presidential elections in
  model 4. However, this difference is not enough to rule in favor of H3;
  if this difference was caused by the lack of presence of national
  effects, it would be more pronounced in primary and coordinated
  elections as well.
  
  \section{Individual Level Models}\label{individual-level-models}
  
  \subsection{Specifications}\label{specifications-1}
  
  For the rest of this section, assume the following:
  
  \[y_i \sim \text{Bernoulli}(p)\]
  
  Where \(y_i \in \{0,1\}\) is the probability that the i-th ballot was
  completed. The goal of such an individual level model is to estimate
  \(p\) as a function of variables measured at four different level of
  observation:
  
  \begin{enumerate}
     \item Ballot
     \item Individual
     \item County
     \item Election
  \end{enumerate}
  
  For this section models are built based on the following concept: assume
  that an election worker is trying to asses whether a ballot in their
  hands is completed or not, without opening the envelope it is included
  in. The possible outcomes are either that the person it corresponds to
  decided not to complete it (an outcome that includes the individual not
  going to a polling place at all), or that they decided to fill it in and
  vote. Assume that the ballot has some information written on the cover,
  such as the county it is from, or the election date, or whether it is a
  mail ballot. The models in this section are built for different
  combinations of such information.
  
  As a preliminary baseline model I would predict the probability that an
  individual voted in a particular election would be equal to turnout, as
  calculated through all other ballots. Therefore:
  
  \[\hat{\mathbb{P}}(y_i = 1) = \frac{\# \text{votes cast}}{\# \text{ballots}}\]
  
  \subsection{Estimation with only one type of
  data}\label{estimation-with-only-one-type-of-data}
  
  \subsubsection{County Level}\label{county-level}
  
  Given that the ballot I am assessing has county of origin written on it,
  there are two ways to predict \(\mathbb{P}(y_i = 1)\). First, assume
  that each different county has a different, independent
  \(\mathbb{P}(y_i = 1)\), then:
  
  \[\hat{\mathbb{P}}(y_i = 1) \sim \text{logit}^{-1}(\sum_{k = 1}^{64}x_{k[i]}\beta_{k})\]
  
  Where \(k\) counts over the 64 counties of Colorado, and \(x_{k}\) is an
  indicator variable for each county. Without assuming independence
  between counties, I could also fit a mixed effects model. This is a
  reasonable step since these counties are in the same state and country,
  and also often share borders.
  
  \begin{equation} \tag{Model 1}
  \hat{\mathbb{P}}(y_i = 1) \sim \text{logit}^{-1}(a_{k[i]}),
  \end{equation}
  
  \[a_{k} \sim \text{N}(\gamma_0, \sigma_{\alpha}^2)\]
  
  Where \(\alpha_{k[i]}\) varies by county, constrained by its standard
  deviation and \(\gamma_0\), an intercept coefficient. I name this
  \textbf{Model 1}.
  
  Assume that along with the one ballot, I was given a short list of
  \(n^{\text{county vars}}\) other county-level variables, be they
  discrete, continuous, or indicators. With this additional information,
  the models can be updated to the following form:
  
  \[\hat{\mathbb{P}}(y_i = 1) \sim \text{logit}^{-1}(\sum_{k = 1}^{64}x_{k[i]}\beta_{k} + \sum_{v=1}^{n^{\text{county vars}}}x_{k[i], v}\beta_{v+64})\]
  
  Where \(x_{k[i]}\) is the k-th value of the v-th variable. If, as
  before, I do not assume independence, the model can be written as:
  
  \begin{equation} \tag{Model 2}  
  \hat{\mathbb{P}}(y_i = 1) \sim \text{logit}^{-1}(a_{k[i]}),
  \end{equation}
  
  \[a_{k} \sim \text{N}(\gamma_0 + \sum_{v=1}^{n^{\text{county vars}}}x_{k[i], v}\gamma_{v}, \sigma_{\alpha}^2)\]
  
  In the case of my specific data, for the time being I have county-level
  data for white population and urban population, so
  \(n^{\text{county vars}} = 2\). Both of these variables are ratios over
  the total county population, taking values in \([0,1]\). I name this
  \textbf{Model 2}.
  
  \subsubsection{Individual Level}\label{individual-level}
  
  Assuming that I know the voter ID of the individual that cast their
  ballot, I can treat this piece of information in about the same way that
  I did for county as described above. This means that the following is
  mostly an exercise in maintaining consistent notation. For these
  purposes, let \(n^{ID}\) be the number of total unique voter IDs, or
  individuals, that I have data on, and \(j\) an index that sums over all
  individuals. Also let \(x_{j}\) be an indicator variable for each
  individual. Then:
  
  \[\hat{\mathbb{P}}(y_i = 1) \sim \text{logit}^{-1}(\sum_{j = 1}^{n^{ID}}x_{j[i]}\beta_{j})\]
  
  And the second model, not assuming independence, would be:
  
  \[\hat{\mathbb{P}}(y_i = 1) \sim \text{logit}^{-1}(\delta_{j[i]}), \]
  \[\delta_{j} \sim \text{N}(\zeta_0, \sigma_{\delta}^2)\]
  
  Again, in a similar way to county level data, there are variables at an
  individual level, thus making it relatively easy to build further
  models. Let's say now that along with the one ballot, I was given a
  short list of \(n^{\text{indiv vars}}\) other individual-level
  variables, be they discrete, continuous, or indicators. The two models
  would then look like:
  
  \[\hat{\mathbb{P}}(y_i = 1) \sim \text{logit}^{-1}(\sum_{j = 1}^{n^{ID}}x_{j[i]}\beta_{j} + \sum_{v=1}^{n^{\text{indiv vars}}}x_{j[i], v}\beta_{v+n^{ID}})\]
  
  Where \(z_{j[i]}\) is the j-th value of the v-th variable, where \(j\)
  is the individual corresponding to ballot \(i\). If, as before, I do not
  assume independence, the model can be written as:
  
  \[\hat{\mathbb{P}}(y_i = 1) \sim \text{logit}^{-1}(\delta_{j[i]})\]
  
  \[\delta_{j} \sim \text{N}(\zeta_0 + \sum_{v=1}^{n^{\text{indiv vars}}}z_{j[i], v}\delta_{v}, \sigma_{\delta}^2)\]
  
  In the case of my specific data, for the time being I have
  individual-level data for gender, so \(n^{\text{indiv vars}} = 1\). I
  name the combination of this model and Model 2: \textbf{Model 3}. Model
  3 can be written as follows:
  
  \begin{equation} \tag{Model 3}
  \hat{\mathbb{P}}(y_i = 1) \sim \text{logit}^{-1}(\delta_{j[i]} + a_{k[i]}), 
  \end{equation}
  
  \[a_{k} \sim \text{N}(\gamma_0 + \sum_{v=1}^{n^{\text{county vars}}}x_{k[i], v}\gamma_{v}, \sigma_{\alpha}^2)\]
  \[\delta_{j} \sim \text{N}(\zeta_0 + \sum_{v=1}^{n^{\text{indiv vars}}}x_{j[i], v}\delta_{v}, \sigma_{\delta}^2)\]
  
  \subsubsection{Election Level}\label{election-level}
  
  Again as previously, four additional models result from the inclusion of
  election-level data. The first two are assuming I only knew what
  specific election the ballot comes from. Let \(w_{l}\) be an indicator
  variable for each election and \(n^{elect}\) the number of elections.
  The model assuming independence, with \(x_{l}\) being indicator
  variables for each election, is:
  
  \[\hat{\mathbb{P}}(y_i = 1) \sim \text{logit}^{-1}(\sum_{l = 1}^{n^{elect}}x_{l[i]}\beta_{l})\]
  
  Again, as previously, it would be safe to assume that each election is
  not held in a vacuum. Adding mixed effects this model would be:
  
  \[\hat{\mathbb{P}}(y_i = 1) \sim \text{logit}^{-1}(\eta_{l[i]}), \]
  \[\eta_{l} \sim \text{N}(\nu_0, \sigma_{\nu}^2)\]
  
  Again, in a similar way to county- and individual-level data, I add in
  variables at an election-level. Let's say now that along with the one
  ballot, I was given a short list of \(n^{\text{election vars}}\) other
  election-level variables, be they discrete, continuous, or indicators.
  The two models would then look like:
  
  \begin{equation} \tag{Model 4}
  \hat{\mathbb{P}}(y_i = 1) \sim \text{logit}^{-1}(\sum_{l = 1}^{n^{elect}}x_{l[i]}\beta_{l} + \sum_{v=1}^{n^{\text{election vars}}}x_{l[i], v}\beta_{v+n^{elect}} + ns(x^{\text{year}}))
  \end{equation}
  
  Where \(x_{l[i], v}\) is the l-th value of the v-th variable, where
  \(l\) is the election corresponding to ballot \(i\). For the time being
  I have two different variables that describe individual elections: date
  and type. I choose to fit a glm with a natural cubic smoothing spline
  function for year. This would also include four distinct indicators for
  election type. I name this \textbf{Model 4}. Model 4 would not be a
  mixed effects model, since all the variability between elections is
  incorporated in election type and election year--with those two
  variables I can fully describe each election\footnote{It is much safer
    to assume election types to be independent when it comes to turnout,
    than to make the same assumption for individuals or counties. The
    reason is that different election types historically have different
    levels of turnout. Any dependence can safely be estimated by the
    inclusion of a trend over time.}.
  
  \subsubsection{Ballot Level}\label{ballot-level}
  
  In this section I assume that the ballot has some key features written
  on it, like the voting method, age, or party registration of the person
  that filled it out. A mixed effects model here would make no sense,
  since all the data is at the same unit of observation. Therefore, when
  adding ballot level variables, the model would look like:
  
  \begin{equation} \tag{Model 5}
  \hat{\mathbb{P}}(y_i = 1) \sim \text{logit}^{-1}(\beta_0 + \sum_{v = 1}^{n^{\text{ballot vars}}}x_{i,v}\beta_{v})
  \end{equation}
  
  Where \(x_{i,v}\) is the i-th value of the v-th variable, and
  \(n^{\text{ballot vars}}\) is the number of ballot level variables. For
  now, I have data on voting method, age, and party. Voting method is
  coded as a binary variable with value one if the method was a Mail Vote.
  Party includes four distinct indicators for REP, DEM, Other, and
  Unaffiliated. A linear term is used for age. I name this \textbf{Model
  5}.
  
  \subsection{Estimation with the full
  dataset}\label{estimation-with-the-full-dataset}
  
  I now proceed to include variables from all units of observation into
  one model. The first model, assuming independence, is:
  
  \begin{multline*}
  \hat{\mathbb{P}}(y_i = 1) \sim \text{logit}^{-1}(\sum_{k = 1}^{64}x_{k}\beta_{*} + \sum_{v=1}^{n^{\text{county vars}}}x_{k[i], v}\beta_{*} + \sum_{j = 1}^{n^{ID}}z_{j}\beta_{*} + \sum_{v=1}^{n^{\text{indiv vars}}}z_{j[i], v}\beta_{*} + \\
  \sum_{l = 1}^{n^{elect}}w_{l}\beta_{*} + \sum_{v=1}^{n^{\text{election vars}}}w_{l[i], v}\beta_{*} + \sum_{v = 1}^{n^{\text{ballot vars}}}u_{i,v}\beta_{*})
  \end{multline*}
  
  You will notice that I have omitted the subscript for all beta
  coefficients. This is because after two or three parameters the
  subscript becomes increasingly large. For simplicity, assume increasing
  indexes for different beta coefficients from left to right in this
  expression.
  
  The mixed effects model will again operate on two ``levels'' of
  hierarchy, but the second level will now include two distinct
  regressions. Caveats for variables like age and date should be noted
  from previous sections. This, the most complex model, will be
  \textbf{Model 6}
  
  \begin{equation} \tag{Model 6}
  \hat{p\_vote} \sim \text{logit}^{-1}(\sum_{v = 1}^{n^{\text{ballot vars}}}x_{i,v}\beta_{v} +\delta_{j[i]} + \alpha_{k[i]}),
  \end{equation}
  
  \[\alpha_{k} \sim \text{N}(\gamma_0 + \sum_{v=1}^{n^{\text{county vars}}}x_{k[i], v}\gamma_{v}, \sigma_{\alpha}^2)\]
  
  \[\delta_{j} \sim \text{N}(\zeta_0 + \sum_{v=1}^{n^{\text{indiv vars}}}x_{j[i], v}\delta_{v}, \sigma_{\delta}^2)\]
  
  In summary, Table 4.3 includes all noteworthy models from the previous
  section. I add a few models which should be easily understood based on
  the specifications given above.
  
  \begin{longtable}[]{@{}ll@{}}
  \caption{Individual level model descriptions
  \label{tab:model_desc_individual}}\tabularnewline
  \toprule
  \begin{minipage}[b]{0.15\columnwidth}\raggedright\strut
  Model No\strut
  \end{minipage} & \begin{minipage}[b]{0.80\columnwidth}\raggedright\strut
  Model Description\strut
  \end{minipage}\tabularnewline
  \midrule
  \endfirsthead
  \toprule
  \begin{minipage}[b]{0.15\columnwidth}\raggedright\strut
  Model No\strut
  \end{minipage} & \begin{minipage}[b]{0.80\columnwidth}\raggedright\strut
  Model Description\strut
  \end{minipage}\tabularnewline
  \midrule
  \endhead
  \begin{minipage}[t]{0.15\columnwidth}\raggedright\strut
  Model 1\strut
  \end{minipage} & \begin{minipage}[t]{0.80\columnwidth}\raggedright\strut
  Naive model with only county mixed effects\strut
  \end{minipage}\tabularnewline
  \begin{minipage}[t]{0.15\columnwidth}\raggedright\strut
  Model 2\strut
  \end{minipage} & \begin{minipage}[t]{0.80\columnwidth}\raggedright\strut
  Multilevel model; added county level predictors\strut
  \end{minipage}\tabularnewline
  \begin{minipage}[t]{0.15\columnwidth}\raggedright\strut
  Model 3\strut
  \end{minipage} & \begin{minipage}[t]{0.80\columnwidth}\raggedright\strut
  Multilevel model; individual- and county-level mixed effects and
  predictors\strut
  \end{minipage}\tabularnewline
  \begin{minipage}[t]{0.15\columnwidth}\raggedright\strut
  Model 3a\strut
  \end{minipage} & \begin{minipage}[t]{0.80\columnwidth}\raggedright\strut
  Multilevel model; same as 3 without individual-level mixed effects\strut
  \end{minipage}\tabularnewline
  \begin{minipage}[t]{0.15\columnwidth}\raggedright\strut
  Model 4\strut
  \end{minipage} & \begin{minipage}[t]{0.80\columnwidth}\raggedright\strut
  General Additive model; election predictors and time smoothing
  splines\strut
  \end{minipage}\tabularnewline
  \begin{minipage}[t]{0.15\columnwidth}\raggedright\strut
  Model 5\strut
  \end{minipage} & \begin{minipage}[t]{0.80\columnwidth}\raggedright\strut
  Ballot-level predictors fixed effects model\strut
  \end{minipage}\tabularnewline
  \begin{minipage}[t]{0.15\columnwidth}\raggedright\strut
  Model 5a\strut
  \end{minipage} & \begin{minipage}[t]{0.80\columnwidth}\raggedright\strut
  Multilevel model; ballot predictors with county mixed effects\strut
  \end{minipage}\tabularnewline
  \begin{minipage}[t]{0.15\columnwidth}\raggedright\strut
  Model 6\strut
  \end{minipage} & \begin{minipage}[t]{0.80\columnwidth}\raggedright\strut
  Multilevel General Additive model; year splines; individual, county
  mixed effects and all predictors\strut
  \end{minipage}\tabularnewline
  \bottomrule
  \end{longtable}
  
  \subsection{Results}\label{results-1}
  
  While the aforementioned models are sound in their construction, the
  leap from theory to implementation hit a few roadblocks as a direct
  result of the first section in this chapter and the problems outlined
  within. However, there are still valid reasons to include whatever
  results were possible to estimate. The results indicate that the data I
  have on their own \emph{can} be used to build and run an individual
  model of turnout regardless of if that model is useful in responding to
  my hypotheses on VBM.
  
  \begin{figure}
  
  {\centering \includegraphics[width=0.6\linewidth]{/Users/tdounias/Desktop/Reed_Senior_Thesis/plots/roc_curves_indiv} 
  
  }
  
  \caption[ROC Curve for all individual models]{ROC Curve for all individual models}\label{fig:roc the curves}
  \end{figure}
  
  I am confident in the results of models 1, 3a, and 4, somewhat confident
  in model 5, and less so for models 2, 3, and 5a. Models 2, 3 and 5a
  ``failed to converge''; this means that the numeric approximation
  process by which \textit{R} implements maximum likelihood
  estimation\footnote{Estimation of maximum likelihood here uses Adaptive
    Gaussian-Hermitian Quadrature (AGQ) to estimate coefficients
    (Handayani, Notodiputro, Sadik, \& Kurnia, 2017)} for coefficients
  doesn't give stable results, within certain conditions. While model 5
  did converge, it suffers from lack of variance in the predictor for VBM,
  as explained in the beginning of this chapter; this is the reason why
  the coefficient for mail vote is so disproportionately large and
  variable. Model 6 simply did not run, even on a sub-sampled dataset.
  
  While I can't really derive any conclusions from this fact, there is a
  distinct possibility that this either occurred due to a lack of
  processing power, or lack of sufficient data for the model estimation to
  even reach close to convergence. It is also important to point out that
  model non-convergence is not a fatal issue in and of itself, but becomes
  so if outputted coefficients wildly differ between calls of the model.
  Running code to output 5-fold CV AUC involves estimating each model 5
  times on different ``folds'', or sub-samples, of the dataset. During
  this process the coefficients and AUC values remained fairly stable
  between runs of the model, despite non-convergence. This leads me to
  assume that the problem is significant, but not fatal for my models.
  
  In terms of model fit, the models neatly fall into three groups based on
  their cross-validated AUC. The first group, consisting of models 1, 2,
  and 3a has an AUC of around \(.544\), making them only slightly
  distinguishable from a coin-toss. This is fairly reasonable, since I am
  building a model to make predictions at the ballot level while only
  using county data or gender\footnote{Few counties wildly differ in their
    turnout percentages, and that the coefficient for male gender results
    in only around a 2.5\% decrease in voting probability}. The second
  group based on AUC includes model 4, the only non-multilevel individual
  model, with an AUC of around \(.733\), significantly outperforming the
  first three models. Again, this is reasonable considering how wildly
  different turnout is between election types; it is only natural that
  these election-level variables would be so informative. The third group,
  with the highest AUC of around \(.96\) are models 5 and 5a. This is a
  direct result of the lack of variability in my data: Mail Voting is an
  almost perfect predictor of the probability of voting.
  
  There are two conclusions that can be reached from these results. None
  of these conclusions are, sadly, related to my hypotheses on VBM. The
  first is that the lack of variance in the data and a lack of processing
  power are direct causes of my inability to estimate these models. This
  is apparent in how model 6 does not run, other models do not converge,
  and the coefficient for VBM is very large, since it doesn't vary enough
  even after stratified sampling to account for any variance between mail
  vote and conventional ballots. The second conclusion here is that,
  despite these issues, there are some confirmable results on turnout in
  general that are common between individual and county models. For
  example, across models 2, 3, 3a the urban population of a county is a
  substantial, negative factor in probability of voting, while the white
  population is a very small, positive effect\footnote{This did, however,
    fail to reach statistical significance in both county and individual
    level models. This means that the small, positive effect is not that
    distinguishable from no effect at all.}. Similar conclusions can be
  drawn for male gender, which is a very small negative effect in voting
  probability as compared to female gender. These effects being stable
  across several models mean that they are independent of the additions to
  those models; for example, gender, urban population, and white
  population have effects that are not accounted for when adding
  individual level mixed effects. Despite not being able to assess VBM as
  a factor of turnout probability, these models at least show that the
  data does have substantial use for modelling at the individual level.
  
  \begin{longtable}[]{@{}lccccccc@{}}
  \caption{Estimated individual level coefficients (**Significant at the
  .01 level *at the .05 level)
  \label{tab:model_indiv_coef}}\tabularnewline
  \toprule
  \begin{minipage}[b]{0.12\columnwidth}\raggedright\strut
  Predictor\strut
  \end{minipage} & \begin{minipage}[b]{0.09\columnwidth}\centering\strut
  Model 1\strut
  \end{minipage} & \begin{minipage}[b]{0.10\columnwidth}\centering\strut
  Model 2\strut
  \end{minipage} & \begin{minipage}[b]{0.10\columnwidth}\centering\strut
  Model 3\strut
  \end{minipage} & \begin{minipage}[b]{0.10\columnwidth}\centering\strut
  Model 3a\strut
  \end{minipage} & \begin{minipage}[b]{0.10\columnwidth}\centering\strut
  Model 4\strut
  \end{minipage} & \begin{minipage}[b]{0.10\columnwidth}\centering\strut
  Model 5\strut
  \end{minipage} & \begin{minipage}[b]{0.10\columnwidth}\centering\strut
  Model 5a\strut
  \end{minipage}\tabularnewline
  \midrule
  \endfirsthead
  \toprule
  \begin{minipage}[b]{0.12\columnwidth}\raggedright\strut
  Predictor\strut
  \end{minipage} & \begin{minipage}[b]{0.09\columnwidth}\centering\strut
  Model 1\strut
  \end{minipage} & \begin{minipage}[b]{0.10\columnwidth}\centering\strut
  Model 2\strut
  \end{minipage} & \begin{minipage}[b]{0.10\columnwidth}\centering\strut
  Model 3\strut
  \end{minipage} & \begin{minipage}[b]{0.10\columnwidth}\centering\strut
  Model 3a\strut
  \end{minipage} & \begin{minipage}[b]{0.10\columnwidth}\centering\strut
  Model 4\strut
  \end{minipage} & \begin{minipage}[b]{0.10\columnwidth}\centering\strut
  Model 5\strut
  \end{minipage} & \begin{minipage}[b]{0.10\columnwidth}\centering\strut
  Model 5a\strut
  \end{minipage}\tabularnewline
  \midrule
  \endhead
  \begin{minipage}[t]{0.12\columnwidth}\raggedright\strut
  (Intercept)\strut
  \end{minipage} & \begin{minipage}[t]{0.09\columnwidth}\centering\strut
  -0.175\strut
  \end{minipage} & \begin{minipage}[t]{0.10\columnwidth}\centering\strut
  -0.042\strut
  \end{minipage} & \begin{minipage}[t]{0.10\columnwidth}\centering\strut
  0.001\strut
  \end{minipage} & \begin{minipage}[t]{0.10\columnwidth}\centering\strut
  0.001\strut
  \end{minipage} & \begin{minipage}[t]{0.10\columnwidth}\centering\strut
  -0.541**\strut
  \end{minipage} & \begin{minipage}[t]{0.10\columnwidth}\centering\strut
  -2.478**\strut
  \end{minipage} & \begin{minipage}[t]{0.10\columnwidth}\centering\strut
  -1.888**\strut
  \end{minipage}\tabularnewline
  \begin{minipage}[t]{0.12\columnwidth}\raggedright\strut
  \strut
  \end{minipage} & \begin{minipage}[t]{0.09\columnwidth}\centering\strut
  (0.030)\strut
  \end{minipage} & \begin{minipage}[t]{0.10\columnwidth}\centering\strut
  (0.083)\strut
  \end{minipage} & \begin{minipage}[t]{0.10\columnwidth}\centering\strut
  (0.060)\strut
  \end{minipage} & \begin{minipage}[t]{0.10\columnwidth}\centering\strut
  (0.076)\strut
  \end{minipage} & \begin{minipage}[t]{0.10\columnwidth}\centering\strut
  (0.009)\strut
  \end{minipage} & \begin{minipage}[t]{0.10\columnwidth}\centering\strut
  (0.015)\strut
  \end{minipage} & \begin{minipage}[t]{0.10\columnwidth}\centering\strut
  (0.238)\strut
  \end{minipage}\tabularnewline
  \begin{minipage}[t]{0.12\columnwidth}\raggedright\strut
  Pct\_urban\strut
  \end{minipage} & \begin{minipage}[t]{0.09\columnwidth}\centering\strut
  \strut
  \end{minipage} & \begin{minipage}[t]{0.10\columnwidth}\centering\strut
  -0.423**\strut
  \end{minipage} & \begin{minipage}[t]{0.10\columnwidth}\centering\strut
  -0.436**\strut
  \end{minipage} & \begin{minipage}[t]{0.10\columnwidth}\centering\strut
  -0.424**\strut
  \end{minipage} & \begin{minipage}[t]{0.10\columnwidth}\centering\strut
  \strut
  \end{minipage} & \begin{minipage}[t]{0.10\columnwidth}\centering\strut
  \strut
  \end{minipage} & \begin{minipage}[t]{0.10\columnwidth}\centering\strut
  -0.538**\strut
  \end{minipage}\tabularnewline
  \begin{minipage}[t]{0.12\columnwidth}\raggedright\strut
  \strut
  \end{minipage} & \begin{minipage}[t]{0.09\columnwidth}\centering\strut
  \strut
  \end{minipage} & \begin{minipage}[t]{0.10\columnwidth}\centering\strut
  (0.055)\strut
  \end{minipage} & \begin{minipage}[t]{0.10\columnwidth}\centering\strut
  (0.059)\strut
  \end{minipage} & \begin{minipage}[t]{0.10\columnwidth}\centering\strut
  (0.062)\strut
  \end{minipage} & \begin{minipage}[t]{0.10\columnwidth}\centering\strut
  \strut
  \end{minipage} & \begin{minipage}[t]{0.10\columnwidth}\centering\strut
  \strut
  \end{minipage} & \begin{minipage}[t]{0.10\columnwidth}\centering\strut
  (0.114)\strut
  \end{minipage}\tabularnewline
  \begin{minipage}[t]{0.12\columnwidth}\raggedright\strut
  Pct\_white\strut
  \end{minipage} & \begin{minipage}[t]{0.09\columnwidth}\centering\strut
  \strut
  \end{minipage} & \begin{minipage}[t]{0.10\columnwidth}\centering\strut
  0.067\strut
  \end{minipage} & \begin{minipage}[t]{0.10\columnwidth}\centering\strut
  0.075\strut
  \end{minipage} & \begin{minipage}[t]{0.10\columnwidth}\centering\strut
  0.073\strut
  \end{minipage} & \begin{minipage}[t]{0.10\columnwidth}\centering\strut
  \strut
  \end{minipage} & \begin{minipage}[t]{0.10\columnwidth}\centering\strut
  \strut
  \end{minipage} & \begin{minipage}[t]{0.10\columnwidth}\centering\strut
  -0.151\strut
  \end{minipage}\tabularnewline
  \begin{minipage}[t]{0.12\columnwidth}\raggedright\strut
  \strut
  \end{minipage} & \begin{minipage}[t]{0.09\columnwidth}\centering\strut
  \strut
  \end{minipage} & \begin{minipage}[t]{0.10\columnwidth}\centering\strut
  (0.102)\strut
  \end{minipage} & \begin{minipage}[t]{0.10\columnwidth}\centering\strut
  (0.073)\strut
  \end{minipage} & \begin{minipage}[t]{0.10\columnwidth}\centering\strut
  (0.094)\strut
  \end{minipage} & \begin{minipage}[t]{0.10\columnwidth}\centering\strut
  \strut
  \end{minipage} & \begin{minipage}[t]{0.10\columnwidth}\centering\strut
  \strut
  \end{minipage} & \begin{minipage}[t]{0.10\columnwidth}\centering\strut
  (0.281)\strut
  \end{minipage}\tabularnewline
  \begin{minipage}[t]{0.12\columnwidth}\raggedright\strut
  genderMale\strut
  \end{minipage} & \begin{minipage}[t]{0.09\columnwidth}\centering\strut
  \strut
  \end{minipage} & \begin{minipage}[t]{0.10\columnwidth}\centering\strut
  \strut
  \end{minipage} & \begin{minipage}[t]{0.10\columnwidth}\centering\strut
  -0.097**\strut
  \end{minipage} & \begin{minipage}[t]{0.10\columnwidth}\centering\strut
  -0.094**\strut
  \end{minipage} & \begin{minipage}[t]{0.10\columnwidth}\centering\strut
  \strut
  \end{minipage} & \begin{minipage}[t]{0.10\columnwidth}\centering\strut
  \strut
  \end{minipage} & \begin{minipage}[t]{0.10\columnwidth}\centering\strut
  0.094**\strut
  \end{minipage}\tabularnewline
  \begin{minipage}[t]{0.12\columnwidth}\raggedright\strut
  \strut
  \end{minipage} & \begin{minipage}[t]{0.09\columnwidth}\centering\strut
  \strut
  \end{minipage} & \begin{minipage}[t]{0.10\columnwidth}\centering\strut
  \strut
  \end{minipage} & \begin{minipage}[t]{0.10\columnwidth}\centering\strut
  (0.007)\strut
  \end{minipage} & \begin{minipage}[t]{0.10\columnwidth}\centering\strut
  (0.007)\strut
  \end{minipage} & \begin{minipage}[t]{0.10\columnwidth}\centering\strut
  \strut
  \end{minipage} & \begin{minipage}[t]{0.10\columnwidth}\centering\strut
  \strut
  \end{minipage} & \begin{minipage}[t]{0.10\columnwidth}\centering\strut
  (0.017)\strut
  \end{minipage}\tabularnewline
  \begin{minipage}[t]{0.12\columnwidth}\raggedright\strut
  Republican\strut
  \end{minipage} & \begin{minipage}[t]{0.09\columnwidth}\centering\strut
  \strut
  \end{minipage} & \begin{minipage}[t]{0.10\columnwidth}\centering\strut
  \strut
  \end{minipage} & \begin{minipage}[t]{0.10\columnwidth}\centering\strut
  \strut
  \end{minipage} & \begin{minipage}[t]{0.10\columnwidth}\centering\strut
  \strut
  \end{minipage} & \begin{minipage}[t]{0.10\columnwidth}\centering\strut
  \strut
  \end{minipage} & \begin{minipage}[t]{0.10\columnwidth}\centering\strut
  0.233**\strut
  \end{minipage} & \begin{minipage}[t]{0.10\columnwidth}\centering\strut
  0.208**\strut
  \end{minipage}\tabularnewline
  \begin{minipage}[t]{0.12\columnwidth}\raggedright\strut
  \strut
  \end{minipage} & \begin{minipage}[t]{0.09\columnwidth}\centering\strut
  \strut
  \end{minipage} & \begin{minipage}[t]{0.10\columnwidth}\centering\strut
  \strut
  \end{minipage} & \begin{minipage}[t]{0.10\columnwidth}\centering\strut
  \strut
  \end{minipage} & \begin{minipage}[t]{0.10\columnwidth}\centering\strut
  \strut
  \end{minipage} & \begin{minipage}[t]{0.10\columnwidth}\centering\strut
  \strut
  \end{minipage} & \begin{minipage}[t]{0.10\columnwidth}\centering\strut
  (0.021)\strut
  \end{minipage} & \begin{minipage}[t]{0.10\columnwidth}\centering\strut
  (0.073)\strut
  \end{minipage}\tabularnewline
  \begin{minipage}[t]{0.12\columnwidth}\raggedright\strut
  Other\strut
  \end{minipage} & \begin{minipage}[t]{0.09\columnwidth}\centering\strut
  \strut
  \end{minipage} & \begin{minipage}[t]{0.10\columnwidth}\centering\strut
  \strut
  \end{minipage} & \begin{minipage}[t]{0.10\columnwidth}\centering\strut
  \strut
  \end{minipage} & \begin{minipage}[t]{0.10\columnwidth}\centering\strut
  \strut
  \end{minipage} & \begin{minipage}[t]{0.10\columnwidth}\centering\strut
  \strut
  \end{minipage} & \begin{minipage}[t]{0.10\columnwidth}\centering\strut
  -0.085\strut
  \end{minipage} & \begin{minipage}[t]{0.10\columnwidth}\centering\strut
  -0.124\strut
  \end{minipage}\tabularnewline
  \begin{minipage}[t]{0.12\columnwidth}\raggedright\strut
  \strut
  \end{minipage} & \begin{minipage}[t]{0.09\columnwidth}\centering\strut
  \strut
  \end{minipage} & \begin{minipage}[t]{0.10\columnwidth}\centering\strut
  \strut
  \end{minipage} & \begin{minipage}[t]{0.10\columnwidth}\centering\strut
  \strut
  \end{minipage} & \begin{minipage}[t]{0.10\columnwidth}\centering\strut
  \strut
  \end{minipage} & \begin{minipage}[t]{0.10\columnwidth}\centering\strut
  \strut
  \end{minipage} & \begin{minipage}[t]{0.10\columnwidth}\centering\strut
  (0.073)\strut
  \end{minipage} & \begin{minipage}[t]{0.10\columnwidth}\centering\strut
  (0.073)\strut
  \end{minipage}\tabularnewline
  \begin{minipage}[t]{0.12\columnwidth}\raggedright\strut
  UAF\strut
  \end{minipage} & \begin{minipage}[t]{0.09\columnwidth}\centering\strut
  \strut
  \end{minipage} & \begin{minipage}[t]{0.10\columnwidth}\centering\strut
  \strut
  \end{minipage} & \begin{minipage}[t]{0.10\columnwidth}\centering\strut
  \strut
  \end{minipage} & \begin{minipage}[t]{0.10\columnwidth}\centering\strut
  \strut
  \end{minipage} & \begin{minipage}[t]{0.10\columnwidth}\centering\strut
  \strut
  \end{minipage} & \begin{minipage}[t]{0.10\columnwidth}\centering\strut
  -0.308**\strut
  \end{minipage} & \begin{minipage}[t]{0.10\columnwidth}\centering\strut
  0.325**\strut
  \end{minipage}\tabularnewline
  \begin{minipage}[t]{0.12\columnwidth}\raggedright\strut
  \strut
  \end{minipage} & \begin{minipage}[t]{0.09\columnwidth}\centering\strut
  \strut
  \end{minipage} & \begin{minipage}[t]{0.10\columnwidth}\centering\strut
  \strut
  \end{minipage} & \begin{minipage}[t]{0.10\columnwidth}\centering\strut
  \strut
  \end{minipage} & \begin{minipage}[t]{0.10\columnwidth}\centering\strut
  \strut
  \end{minipage} & \begin{minipage}[t]{0.10\columnwidth}\centering\strut
  \strut
  \end{minipage} & \begin{minipage}[t]{0.10\columnwidth}\centering\strut
  (0.021)\strut
  \end{minipage} & \begin{minipage}[t]{0.10\columnwidth}\centering\strut
  (0.021)\strut
  \end{minipage}\tabularnewline
  \begin{minipage}[t]{0.12\columnwidth}\raggedright\strut
  VBM\strut
  \end{minipage} & \begin{minipage}[t]{0.09\columnwidth}\centering\strut
  \strut
  \end{minipage} & \begin{minipage}[t]{0.10\columnwidth}\centering\strut
  \strut
  \end{minipage} & \begin{minipage}[t]{0.10\columnwidth}\centering\strut
  \strut
  \end{minipage} & \begin{minipage}[t]{0.10\columnwidth}\centering\strut
  \strut
  \end{minipage} & \begin{minipage}[t]{0.10\columnwidth}\centering\strut
  \strut
  \end{minipage} & \begin{minipage}[t]{0.10\columnwidth}\centering\strut
  23.764**\strut
  \end{minipage} & \begin{minipage}[t]{0.10\columnwidth}\centering\strut
  26.502**\strut
  \end{minipage}\tabularnewline
  \begin{minipage}[t]{0.12\columnwidth}\raggedright\strut
  \strut
  \end{minipage} & \begin{minipage}[t]{0.09\columnwidth}\centering\strut
  \strut
  \end{minipage} & \begin{minipage}[t]{0.10\columnwidth}\centering\strut
  \strut
  \end{minipage} & \begin{minipage}[t]{0.10\columnwidth}\centering\strut
  \strut
  \end{minipage} & \begin{minipage}[t]{0.10\columnwidth}\centering\strut
  \strut
  \end{minipage} & \begin{minipage}[t]{0.10\columnwidth}\centering\strut
  \strut
  \end{minipage} & \begin{minipage}[t]{0.10\columnwidth}\centering\strut
  (45.255)\strut
  \end{minipage} & \begin{minipage}[t]{0.10\columnwidth}\centering\strut
  (285.774)\strut
  \end{minipage}\tabularnewline
  \begin{minipage}[t]{0.12\columnwidth}\raggedright\strut
  Age\strut
  \end{minipage} & \begin{minipage}[t]{0.09\columnwidth}\centering\strut
  \strut
  \end{minipage} & \begin{minipage}[t]{0.10\columnwidth}\centering\strut
  \strut
  \end{minipage} & \begin{minipage}[t]{0.10\columnwidth}\centering\strut
  \strut
  \end{minipage} & \begin{minipage}[t]{0.10\columnwidth}\centering\strut
  \strut
  \end{minipage} & \begin{minipage}[t]{0.10\columnwidth}\centering\strut
  \strut
  \end{minipage} & \begin{minipage}[t]{0.10\columnwidth}\centering\strut
  0.093**\strut
  \end{minipage} & \begin{minipage}[t]{0.10\columnwidth}\centering\strut
  0.086**\strut
  \end{minipage}\tabularnewline
  \begin{minipage}[t]{0.12\columnwidth}\raggedright\strut
  \strut
  \end{minipage} & \begin{minipage}[t]{0.09\columnwidth}\centering\strut
  \strut
  \end{minipage} & \begin{minipage}[t]{0.10\columnwidth}\centering\strut
  \strut
  \end{minipage} & \begin{minipage}[t]{0.10\columnwidth}\centering\strut
  \strut
  \end{minipage} & \begin{minipage}[t]{0.10\columnwidth}\centering\strut
  \strut
  \end{minipage} & \begin{minipage}[t]{0.10\columnwidth}\centering\strut
  \strut
  \end{minipage} & \begin{minipage}[t]{0.10\columnwidth}\centering\strut
  (0.009)\strut
  \end{minipage} & \begin{minipage}[t]{0.10\columnwidth}\centering\strut
  (0.009)\strut
  \end{minipage}\tabularnewline
  \begin{minipage}[t]{0.12\columnwidth}\raggedright\strut
  typeGeneral\strut
  \end{minipage} & \begin{minipage}[t]{0.09\columnwidth}\centering\strut
  \strut
  \end{minipage} & \begin{minipage}[t]{0.10\columnwidth}\centering\strut
  \strut
  \end{minipage} & \begin{minipage}[t]{0.10\columnwidth}\centering\strut
  \strut
  \end{minipage} & \begin{minipage}[t]{0.10\columnwidth}\centering\strut
  \strut
  \end{minipage} & \begin{minipage}[t]{0.10\columnwidth}\centering\strut
  1.537**\strut
  \end{minipage} & \begin{minipage}[t]{0.10\columnwidth}\centering\strut
  \strut
  \end{minipage} & \begin{minipage}[t]{0.10\columnwidth}\centering\strut
  \strut
  \end{minipage}\tabularnewline
  \begin{minipage}[t]{0.12\columnwidth}\raggedright\strut
  \strut
  \end{minipage} & \begin{minipage}[t]{0.09\columnwidth}\centering\strut
  \strut
  \end{minipage} & \begin{minipage}[t]{0.10\columnwidth}\centering\strut
  \strut
  \end{minipage} & \begin{minipage}[t]{0.10\columnwidth}\centering\strut
  \strut
  \end{minipage} & \begin{minipage}[t]{0.10\columnwidth}\centering\strut
  \strut
  \end{minipage} & \begin{minipage}[t]{0.10\columnwidth}\centering\strut
  (0.011)\strut
  \end{minipage} & \begin{minipage}[t]{0.10\columnwidth}\centering\strut
  \strut
  \end{minipage} & \begin{minipage}[t]{0.10\columnwidth}\centering\strut
  \strut
  \end{minipage}\tabularnewline
  \begin{minipage}[t]{0.12\columnwidth}\raggedright\strut
  typeMidterm\strut
  \end{minipage} & \begin{minipage}[t]{0.09\columnwidth}\centering\strut
  \strut
  \end{minipage} & \begin{minipage}[t]{0.10\columnwidth}\centering\strut
  \strut
  \end{minipage} & \begin{minipage}[t]{0.10\columnwidth}\centering\strut
  \strut
  \end{minipage} & \begin{minipage}[t]{0.10\columnwidth}\centering\strut
  \strut
  \end{minipage} & \begin{minipage}[t]{0.10\columnwidth}\centering\strut
  0.829**\strut
  \end{minipage} & \begin{minipage}[t]{0.10\columnwidth}\centering\strut
  \strut
  \end{minipage} & \begin{minipage}[t]{0.10\columnwidth}\centering\strut
  \strut
  \end{minipage}\tabularnewline
  \begin{minipage}[t]{0.12\columnwidth}\raggedright\strut
  \strut
  \end{minipage} & \begin{minipage}[t]{0.09\columnwidth}\centering\strut
  \strut
  \end{minipage} & \begin{minipage}[t]{0.10\columnwidth}\centering\strut
  \strut
  \end{minipage} & \begin{minipage}[t]{0.10\columnwidth}\centering\strut
  \strut
  \end{minipage} & \begin{minipage}[t]{0.10\columnwidth}\centering\strut
  \strut
  \end{minipage} & \begin{minipage}[t]{0.10\columnwidth}\centering\strut
  (0.011)\strut
  \end{minipage} & \begin{minipage}[t]{0.10\columnwidth}\centering\strut
  \strut
  \end{minipage} & \begin{minipage}[t]{0.10\columnwidth}\centering\strut
  \strut
  \end{minipage}\tabularnewline
  \begin{minipage}[t]{0.12\columnwidth}\raggedright\strut
  typePrimary\strut
  \end{minipage} & \begin{minipage}[t]{0.09\columnwidth}\centering\strut
  \strut
  \end{minipage} & \begin{minipage}[t]{0.10\columnwidth}\centering\strut
  \strut
  \end{minipage} & \begin{minipage}[t]{0.10\columnwidth}\centering\strut
  \strut
  \end{minipage} & \begin{minipage}[t]{0.10\columnwidth}\centering\strut
  \strut
  \end{minipage} & \begin{minipage}[t]{0.10\columnwidth}\centering\strut
  -0.880**\strut
  \end{minipage} & \begin{minipage}[t]{0.10\columnwidth}\centering\strut
  \strut
  \end{minipage} & \begin{minipage}[t]{0.10\columnwidth}\centering\strut
  \strut
  \end{minipage}\tabularnewline
  \begin{minipage}[t]{0.12\columnwidth}\raggedright\strut
  \strut
  \end{minipage} & \begin{minipage}[t]{0.09\columnwidth}\centering\strut
  \strut
  \end{minipage} & \begin{minipage}[t]{0.10\columnwidth}\centering\strut
  \strut
  \end{minipage} & \begin{minipage}[t]{0.10\columnwidth}\centering\strut
  \strut
  \end{minipage} & \begin{minipage}[t]{0.10\columnwidth}\centering\strut
  \strut
  \end{minipage} & \begin{minipage}[t]{0.10\columnwidth}\centering\strut
  (0.010)\strut
  \end{minipage} & \begin{minipage}[t]{0.10\columnwidth}\centering\strut
  \strut
  \end{minipage} & \begin{minipage}[t]{0.10\columnwidth}\centering\strut
  \strut
  \end{minipage}\tabularnewline
  \begin{minipage}[t]{0.12\columnwidth}\raggedright\strut
  CV AUC\strut
  \end{minipage} & \begin{minipage}[t]{0.09\columnwidth}\centering\strut
  0.543\strut
  \end{minipage} & \begin{minipage}[t]{0.10\columnwidth}\centering\strut
  0.543\strut
  \end{minipage} & \begin{minipage}[t]{0.10\columnwidth}\centering\strut
  \strut
  \end{minipage} & \begin{minipage}[t]{0.10\columnwidth}\centering\strut
  0.545\strut
  \end{minipage} & \begin{minipage}[t]{0.10\columnwidth}\centering\strut
  0.733\strut
  \end{minipage} & \begin{minipage}[t]{0.10\columnwidth}\centering\strut
  0.961\strut
  \end{minipage} & \begin{minipage}[t]{0.10\columnwidth}\centering\strut
  0.963\strut
  \end{minipage}\tabularnewline
  \begin{minipage}[t]{0.12\columnwidth}\raggedright\strut
  Baseline\strut
  \end{minipage} & \begin{minipage}[t]{0.09\columnwidth}\centering\strut
  0.178\strut
  \end{minipage} & \begin{minipage}[t]{0.10\columnwidth}\centering\strut
  0.178\strut
  \end{minipage} & \begin{minipage}[t]{0.10\columnwidth}\centering\strut
  \strut
  \end{minipage} & \begin{minipage}[t]{0.10\columnwidth}\centering\strut
  0.179\strut
  \end{minipage} & \begin{minipage}[t]{0.10\columnwidth}\centering\strut
  0.291\strut
  \end{minipage} & \begin{minipage}[t]{0.10\columnwidth}\centering\strut
  0.547\strut
  \end{minipage} & \begin{minipage}[t]{0.10\columnwidth}\centering\strut
  0.548\strut
  \end{minipage}\tabularnewline
  \begin{minipage}[t]{0.12\columnwidth}\raggedright\strut
  Obs\strut
  \end{minipage} & \begin{minipage}[t]{0.09\columnwidth}\centering\strut
  370,586\strut
  \end{minipage} & \begin{minipage}[t]{0.10\columnwidth}\centering\strut
  370,586\strut
  \end{minipage} & \begin{minipage}[t]{0.10\columnwidth}\centering\strut
  370,586\strut
  \end{minipage} & \begin{minipage}[t]{0.10\columnwidth}\centering\strut
  370,586\strut
  \end{minipage} & \begin{minipage}[t]{0.10\columnwidth}\centering\strut
  370,586\strut
  \end{minipage} & \begin{minipage}[t]{0.10\columnwidth}\centering\strut
  370,586\strut
  \end{minipage} & \begin{minipage}[t]{0.10\columnwidth}\centering\strut
  370,586\strut
  \end{minipage}\tabularnewline
  \begin{minipage}[t]{0.12\columnwidth}\raggedright\strut
  Groups\strut
  \end{minipage} & \begin{minipage}[t]{0.09\columnwidth}\centering\strut
  64\strut
  \end{minipage} & \begin{minipage}[t]{0.10\columnwidth}\centering\strut
  64\strut
  \end{minipage} & \begin{minipage}[t]{0.10\columnwidth}\centering\strut
  64\strut
  \end{minipage} & \begin{minipage}[t]{0.10\columnwidth}\centering\strut
  64\strut
  \end{minipage} & \begin{minipage}[t]{0.10\columnwidth}\centering\strut
  64\strut
  \end{minipage} & \begin{minipage}[t]{0.10\columnwidth}\centering\strut
  64\strut
  \end{minipage} & \begin{minipage}[t]{0.10\columnwidth}\centering\strut
  64\strut
  \end{minipage}\tabularnewline
  \bottomrule
  \end{longtable}
  
  \chapter*{Conclusion}\label{conclusion}
  \addcontentsline{toc}{chapter}{Conclusion}
  
  \setcounter{chapter}{5} \setcounter{section}{0}
  
  This thesis started off with the prospect of testing three hypotheses,
  on the impact of VBM on turnout (H1) and how that impact may differ
  based on how prominent national effects are during election time (H2),
  specifically using the metric of election type between Midterm, General,
  Primary, and Coordinated elections (H3). These hypotheses derived from a
  theory of voting ``at the margins'', first developed by Aldrich (1993),
  which stated that the decision on whether to turn out or not is based on
  a rational calculus made by individual voters. This calculus is affected
  by a wide range of effects, from small local ones to large, national
  variables. In my first Chapter I identified how this theory fits in the
  general literature on voter behavior, and made a series of observations
  on how each theory views the impact of VBM. The hypotheses were then
  tested on Voter Files from Colorado, which described registration and
  voter history for elections ranging from 2010-2016. The process of data
  wrangling, modeling, and estimations revealed that the data I used was
  not ideal for the purposes I intended, particularly for fitting
  individual-level models.
  
  At the end of my thesis I am able to confirm H1, but reject H2 and H3.
  These results come from my County level models. Despite non-convergence
  my individual level models have some inferential potential, but they are
  still not particularly useful at testing my hypotheses on VBM. This is
  true because of the variability concerns outlined early in Chapter 4. I
  confirm H1 because mail voting seems to be an incremental, comparably
  small effect on turnout. I reject H2/H3 because that effect is not more
  pronounced for elections that have less strong national influences. In
  fact, my county-level models showed the exact opposite: VBM had a
  strong, significant positive effect for presidential elections, but no
  discernible effect for primary, coordinated races, or midterm races.
  
  This is evidence against voting ``at the margins'', since my hypotheses
  that were most connected to Aldrich's model (H2, H3) were convincingly
  rejected. I find evidence against a hypothesis of habitual voting as
  well, since in my county models Mail Voting has a significant, positive
  effect. This means that the absence of election-day effects or the
  strong presence of habitual voting did not override the effect on
  turnout that mail voting has. The same conclusion can be drawn for a
  social theory of voting, which predicts no effect.
  
  The best way to interpret my results is through a resources and
  organizational lens, since this theory predicts a consistent, positive
  effect of VBM on turnout because of the increase in capacity it gives to
  individuals. In the trade-off between voting and not voting, whether
  casting a ballot consumes time and effort seems to have had the largest
  impact, which is why my models show such an effect. However, it should
  be noted that this effect should be present in \emph{all} elections to
  be convincing evidence for the resources electoral participation
  paradigm. This is not the case. One explanation may be that my models
  were sensitive to the vastly greater turnout that occurs during
  presidential and midterm years, particularly since the differences in
  mail voting percentage between counties was at most around 20\%. This,
  again, leads to the conclusion that the lack of variability in my data
  does not permit any more precise analysis.
  
  The first step towards future research, directly resulting from this
  problem, would be to get more data. This is possible, but a task I was
  unable to accomplish due to time constraints related to this project.
  More data would allow for a comparative study of Colorado elections
  before and after all major legislative electoral changes of this century
  (2008, 2013). Another way to expand on my research would be to look at
  even lower level elections, like municipal elections, school board
  elections, or recall elections; voter history files contain data on all
  of these races. A study with more data could use such races as well for
  testing the hypotheses I set up in this thesis. Third, it is possible to
  replicate my research here for other states like Oregon and Washington,
  or for the specific counties in Arizona, Utah, or California that have
  all-mail elections. Fourth, elections thankfully do not stop happening;
  my research here can always be updated with data from the 2017
  Coordinated Colorado election, or the 2018 Midterm that just took place.
  I would caution that the same issue with data variability would still
  exist in this case; there is a much greater need for data going back,
  rather than new all-mail elections.
  
  Apart from expanding my research in terms of data, another path would be
  to implement some of the methods I outline in the beginning of Chapter
  4, like the Synthetic Control Group. Such methods would allow for
  inferences to be made despite the data issues I encountered. Lastly, it
  should be noted that VBM is just one of a series of electoral reforms
  like Automatic Voter Registration (AVR), Early Voting, or Voter ID
  Restrictions, all of which can potentially be tested using multilevel
  models and data wrangling methods that I have applied in this thesis.
  
  Another contribution of my thesis is my analysis of data wrangling, the
  construction of multilevel general additive models for turnout, and the
  accompanying \textit{R} package. To take these one by one, I have
  provided arguments in favor of preferring multiple snapshots of
  registration files rather than just the latest iteration of the record.
  I have analyzed the pitfalls that exist in such documents, and given
  specific examples on how this can be dealt with for the Colorado data
  files. I have also provided a set of variable specification that can be
  useful as indicators of the content of these data, or the potential uses
  of voter registration files in other future studies. Finally, I have
  presented potential future solutions to issues these data have with
  variability, and ways to circumvent processing power limitations.
  
  Additionally, I have meticulously gone through the creation of
  multilevel general additive models of individual and county level
  turnout. While due to data and processing power limitations I am unable
  to run all these models to typical standards of convergence. This does
  not mean that they present no value to future research. Quite the
  contrary, future researchers just have to go through the data clean-up
  stage, and then implement my models without having to construct them
  from scratch. In particular, mixed effects and general additive models
  are not widely used in such studies, making their presentation and
  specification rather unique regardless of their application in this
  piece of research.
  
  Lastly, I provide an extensive library of code used to create this
  document and the research I conduct. I have made an \textit{R}
  package--which I named \texttt{riggd}--that includes more than a dozen
  different functions that serve data wrangling and presentation purposes.
  These functions are made for use on Colorado files, but require
  relatively small amounts of changes to be applied to voter files from
  around the US. I also provide code for all tables and graphics that are
  included in this thesis on gitHub, which is a testament to the
  reproducibility and future value of the research I conducted.
  
  I recognize that, despite many obstacles in terms of data or computing
  power, the outcome of this thesis being more constructive rather than
  conclusive is to some extent my fault. There were many problems in this
  thesis that I should have been aware of earlier in the process, which
  may have allowed me to present more concrete results rather than a
  series of tools and methods. However, in the combination of my existing
  conclusions and the materials I have created through this process, it is
  my belief that this thesis does in fact present a step forward in the
  literature, and that it adds to existing quantitative elections studies
  works. This is a small step, but it helps in our understanding of how
  voters behave, what the actual results of election policy are, and how
  to expand participation.
  
  \appendix
  
  \chapter{MCMC Estimation Processes for Multilevel
  Models}\label{mcmc-estimation-processes-for-multilevel-models}
  
  In statistical science a Markov Chain is a sequence of random variables
  whose value depends on the value of the exact previous random variable.
  In mathematical terms, this would be a sequence
  \(\theta^{(1)}, \theta^{(2)}, \theta^{(3)}, ..., \theta^{(t)}\) where
  \(\mathbb{P}(\theta^{(t)} = y|\theta^{(n)}, n<t) = \mathbb{P}(\theta^{(t)} = y|\theta^{(t-1)})\).
  A Markov Chain Monte Carlo simulation uses Bayesian estimation to update
  each sequential estimate of \(\theta\), leading it to converge to the
  true value being estimated (Gelman \& Hill, 2006; Jackman, 2009).
  
  Multilevel models can be estimated using MCMC sampling. Indicatively,
  this appendix presents the construction and coding of two types of MCMC
  samplers based on the Gibbs algorithm. The code and mathematical
  derivations are adapted to my models from Gelman and Hill (2006).
  
  \section{Gibbs Sampler for the County
  Models}\label{gibbs-sampler-for-the-county-models}
  
  The Gibbs algorithm works as follows:
  
  \begin{enumerate}
    \item Choose a number of parallel simulation runs (chains). This number should be relatively low. In this example it is set to 3.
    \item For each chain do the following:
    \begin{enumerate}
      \item Initialize vector of parameters $\Theta^{(0)} = \{\theta^{(0)}_1\, \theta^{(0)}_2\, ..., \theta^{(0)}_n\}$
      \item Choose a number of iterations. For each iteration update every parameter in vector $\Theta^{(n_{iteration})}$, based on the values of vector $\Theta^{(n_{iteration} - 1)}$.
    \end{enumerate}
    \item Evaluate convergence between the chains.
  \end{enumerate}
  
  If convergence is poor, repeat for more iterations, or follow diagnostic
  procedures. These are not specified here, but Gelman and Hill provide a
  good overview (Gelman \& Hill, 2006; Gelman, Carlin, Stern, \& Rubin,
  2003).
  
  \subsection{County Model 1 (Only Random County
  Effects)}\label{county-model-1-only-random-county-effects}
  
  A basic multilevel model with only group-level intercept mixed effects
  can be written as follows:
  
  \[y_i \ \sim \ N(a_{j[i]}, \sigma^2_y),\ i \in [1, n] \\ a_j \ \sim \ N(\mu_{\alpha}, \sigma^2_{\alpha}), \ j \in [1,J]\]
  This specification is slightly different from that presented in Chapter
  2. Here \(\alpha_{j[i]}\) is the coefficient for the group \(j\) that
  individual \(i\) belongs to, \(\sigma_y, \sigma_{\alpha}\) the variances
  of the individual and group level distributions respectively, and
  \(\mu_{\alpha}\) the mean of the group-level distribution. In the case
  of the most basic county-level model estimated in my thesis (County
  Model 1), \(n = 704\) and \(J = 64\). Using Maximum Likelihood
  Estimation, and given that:
  
  \begin{equation}
    \alpha_j|y, \mu_{\alpha}, \sigma_y, \sigma_{\alpha} \ \sim \ N(\hat{\alpha_j}, V_j)
  \end{equation}
  
  we can obtain estimates:
  
  \begin{equation}
  \hat{\alpha_j} = \frac{\frac{n_{[j]}}{\sigma^2_y}\bar{y}_{[j]} + \frac{1}{\sigma^2{\alpha}}}{\frac{n_{[j]}}{\sigma^2_y} + \frac{1}{\sigma^2{\alpha}}},\ \ \ \ \  V_j = \frac{1}{\frac{n_{[j]}}{\sigma^2_y} + \frac{1}{\sigma^2{\alpha}}},
  \end{equation}
  
  where \(n_{[j]}\) is the number of observations for group j, and
  \(\bar{y}_{[j]}\) is the mean response for group j. Using these
  estimates and the common MLE estimates for variance and mean in a normal
  distribution, it is possible to construct a Gibbs sampler for model
  coefficients and errors. Step 2(b) in the Gibbs sampler would then be:
  
  \begin{enumerate}
    \item Estimate $a_j, \ j\in[1,J]$ using equations (1), (2).
    \item Estimate $\mu_{\alpha}$ by drawing from $N(\frac{1}{J}\sum_{1}^{J}\alpha_j, \sigma_{\alpha}^2/J)$ using the previous values estimated in step 1.
    \item Estimate $\sigma_y^2$ as $\frac{\frac{1}{n}\sum_{1}^{n}(y_i - \alpha_{j[i]})^2}{X_{n-1}^2}$ where $X_{n-1}^2$ is a draw from a $\chi^2$ distribution with $n-1$ degrees of freedom.
    \item Estimate $\sigma_{\alpha}^2$ as $\frac{\frac{1}{J}\sum_{1}^{J}(\alpha_j - \mu_{\alpha})^2}{X_{J-1}^2}$ where $X_{n-1}^2$ is a draw from a $\chi^2$ distribution with $J-1$ degrees of freedom.
  \end{enumerate}
  
  While each step here seems relatively intuitive, the derivations behind
  some of the details (like the chi-squared distribution) are complex MLE
  processes and beyond the scope of this thesis. The R code for this
  algorithm is as follows:
  
  \begin{Shaded}
  \begin{Highlighting}[]
  \NormalTok{## Gibbs sampler in R}
  \NormalTok{a.update <-}\StringTok{ }\ControlFlowTok{function}\NormalTok{()\{}
  \NormalTok{  a.new <-}\StringTok{ }\KeywordTok{rep}\NormalTok{ (}\OtherTok{NA}\NormalTok{, J)}
    \ControlFlowTok{for}\NormalTok{ (j }\ControlFlowTok{in} \DecValTok{1}\OperatorTok{:}\NormalTok{J)\{}
  \NormalTok{    n.j <-}\StringTok{ }\KeywordTok{sum}\NormalTok{ (model_dt}\OperatorTok{$}\NormalTok{county}\OperatorTok{==}\NormalTok{cnt_vec[j])}
  \NormalTok{    y.bar.j <-}\StringTok{ }\KeywordTok{mean}\NormalTok{ (model_dt}\OperatorTok{$}\NormalTok{turnout[model_dt}\OperatorTok{$}\NormalTok{county}\OperatorTok{==}\NormalTok{cnt_vec[j]])}
  \NormalTok{    a.hat.j <-}\StringTok{ }\NormalTok{((n.j}\OperatorTok{/}\NormalTok{sigma.y}\OperatorTok{^}\DecValTok{2}\NormalTok{)}\OperatorTok{*}\NormalTok{y.bar.j }\OperatorTok{+}\StringTok{ }\NormalTok{(}\DecValTok{1}\OperatorTok{/}\NormalTok{sigma.a}\OperatorTok{^}\DecValTok{2}\NormalTok{)}\OperatorTok{*}\NormalTok{mu.a)}\OperatorTok{/}
  \StringTok{               }\NormalTok{(n.j}\OperatorTok{/}\NormalTok{sigma.y}\OperatorTok{^}\DecValTok{2} \OperatorTok{+}\StringTok{ }\DecValTok{1}\OperatorTok{/}\NormalTok{sigma.a}\OperatorTok{^}\DecValTok{2}\NormalTok{)}
  \NormalTok{    V.a.j <-}\StringTok{ }\DecValTok{1}\OperatorTok{/}\NormalTok{(n.j}\OperatorTok{/}\NormalTok{sigma.y}\OperatorTok{^}\DecValTok{2} \OperatorTok{+}\StringTok{ }\DecValTok{1}\OperatorTok{/}\NormalTok{sigma.a}\OperatorTok{^}\DecValTok{2}\NormalTok{)}
  \NormalTok{    a.new[j] <-}\StringTok{ }\KeywordTok{rnorm}\NormalTok{ (}\DecValTok{1}\NormalTok{, a.hat.j, }\KeywordTok{sqrt}\NormalTok{(V.a.j))}
  \NormalTok{  \}}
    \KeywordTok{return}\NormalTok{ (a.new)}
  \NormalTok{\}}
  \NormalTok{mu.a.update <-}\StringTok{ }\ControlFlowTok{function}\NormalTok{()\{}
  \NormalTok{  mu.a.new <-}\StringTok{ }\KeywordTok{rnorm}\NormalTok{ (}\DecValTok{1}\NormalTok{, }\KeywordTok{mean}\NormalTok{(a), sigma.a}\OperatorTok{/}\KeywordTok{sqrt}\NormalTok{(J))}
    \KeywordTok{return}\NormalTok{ (mu.a.new)}
  \NormalTok{\}}
  \NormalTok{sigma.y.update <-}\StringTok{ }\ControlFlowTok{function}\NormalTok{()\{}
  \NormalTok{  sigma.y.new <-}\StringTok{ }\KeywordTok{sqrt}\NormalTok{(}\KeywordTok{sum}\NormalTok{((model_dt}\OperatorTok{$}\NormalTok{turnout}\OperatorTok{-}
  \StringTok{                             }\NormalTok{a[model_dt}\OperatorTok{$}\NormalTok{county])}\OperatorTok{^}\DecValTok{2}\NormalTok{)}\OperatorTok{/}\KeywordTok{rchisq}\NormalTok{(}\DecValTok{1}\NormalTok{,}\DecValTok{703}\NormalTok{))}
    \KeywordTok{return}\NormalTok{ (sigma.y.new)}
  \NormalTok{\}}
  \NormalTok{sigma.a.update <-}\StringTok{ }\ControlFlowTok{function}\NormalTok{()\{}
  \NormalTok{  sigma.a.new <-}\StringTok{ }\KeywordTok{sqrt}\NormalTok{(}\KeywordTok{sum}\NormalTok{((a}\OperatorTok{-}\NormalTok{mu.a)}\OperatorTok{^}\DecValTok{2}\NormalTok{)}\OperatorTok{/}\KeywordTok{rchisq}\NormalTok{(}\DecValTok{1}\NormalTok{,J}\OperatorTok{-}\DecValTok{1}\NormalTok{))}
    \KeywordTok{return}\NormalTok{ (sigma.a.new)}
  \NormalTok{\}}
  
  \NormalTok{J <-}\StringTok{ }\DecValTok{64}
  \NormalTok{n.chains <-}\StringTok{ }\DecValTok{3}
  \NormalTok{n.iter <-}\StringTok{ }\DecValTok{1000}
  \NormalTok{sims <-}\StringTok{ }\KeywordTok{array}\NormalTok{ (}\OtherTok{NA}\NormalTok{, }\KeywordTok{c}\NormalTok{(n.iter, n.chains, J}\OperatorTok{+}\DecValTok{3}\NormalTok{))}
  \KeywordTok{dimnames}\NormalTok{ (sims) <-}\StringTok{ }\KeywordTok{list}\NormalTok{ (}\OtherTok{NULL}\NormalTok{, }\OtherTok{NULL}\NormalTok{, }
                           \KeywordTok{c}\NormalTok{ (}\KeywordTok{paste}\NormalTok{ (}\StringTok{"a["}\NormalTok{, }\DecValTok{1}\OperatorTok{:}\NormalTok{J, }\StringTok{"]"}\NormalTok{, }\DataTypeTok{sep=}\StringTok{""}\NormalTok{), }\StringTok{"mu.a"}\NormalTok{,}
     \StringTok{"sigma.y"}\NormalTok{, }\StringTok{"sigma.a"}\NormalTok{))}
  
  \ControlFlowTok{for}\NormalTok{ (m }\ControlFlowTok{in} \DecValTok{1}\OperatorTok{:}\NormalTok{n.chains)\{}
  \NormalTok{  mu.a <-}\StringTok{ }\KeywordTok{rnorm}\NormalTok{ (}\DecValTok{1}\NormalTok{, }\KeywordTok{mean}\NormalTok{(model_dt}\OperatorTok{$}\NormalTok{turnout), }\KeywordTok{sd}\NormalTok{(model_dt}\OperatorTok{$}\NormalTok{turnout))}
  \NormalTok{  sigma.y <-}\StringTok{ }\KeywordTok{runif}\NormalTok{ (}\DecValTok{1}\NormalTok{, }\DecValTok{0}\NormalTok{, }\KeywordTok{sd}\NormalTok{(model_dt}\OperatorTok{$}\NormalTok{turnout))}
  \NormalTok{  sigma.a <-}\StringTok{ }\KeywordTok{runif}\NormalTok{ (}\DecValTok{1}\NormalTok{, }\DecValTok{0}\NormalTok{, }\KeywordTok{sd}\NormalTok{(model_dt}\OperatorTok{$}\NormalTok{turnout))}
    \ControlFlowTok{for}\NormalTok{ (t }\ControlFlowTok{in} \DecValTok{1}\OperatorTok{:}\NormalTok{n.iter)\{}
  \NormalTok{    a <-}\StringTok{ }\KeywordTok{a.update}\NormalTok{ ()}
  \NormalTok{    mu.a <-}\StringTok{ }\KeywordTok{mu.a.update}\NormalTok{ ()}
  \NormalTok{    sigma.y <-}\StringTok{ }\KeywordTok{sigma.y.update}\NormalTok{ ()}
  \NormalTok{    sigma.a <-}\StringTok{ }\KeywordTok{sigma.a.update}\NormalTok{ ()}
  \NormalTok{    sims[t,m,] <-}\StringTok{ }\KeywordTok{c}\NormalTok{ (a, mu.a, sigma.y, sigma.a)}
  \NormalTok{  \}}
  \NormalTok{\}}
  \end{Highlighting}
  \end{Shaded}
  
  \begin{longtable}[]{@{}lccc@{}}
  \caption{Gibbs sampler results for County Model 1
  \label{tab:gibbs_1}}\tabularnewline
  \toprule
  \begin{minipage}[b]{0.26\columnwidth}\raggedright\strut
  Calculated from\ldots{}\strut
  \end{minipage} & \begin{minipage}[b]{0.11\columnwidth}\centering\strut
  mu.a\strut
  \end{minipage} & \begin{minipage}[b]{0.12\columnwidth}\centering\strut
  sigma.y\strut
  \end{minipage} & \begin{minipage}[b]{0.12\columnwidth}\centering\strut
  sigma.a\strut
  \end{minipage}\tabularnewline
  \midrule
  \endfirsthead
  \toprule
  \begin{minipage}[b]{0.26\columnwidth}\raggedright\strut
  Calculated from\ldots{}\strut
  \end{minipage} & \begin{minipage}[b]{0.11\columnwidth}\centering\strut
  mu.a\strut
  \end{minipage} & \begin{minipage}[b]{0.12\columnwidth}\centering\strut
  sigma.y\strut
  \end{minipage} & \begin{minipage}[b]{0.12\columnwidth}\centering\strut
  sigma.a\strut
  \end{minipage}\tabularnewline
  \midrule
  \endhead
  \begin{minipage}[t]{0.26\columnwidth}\raggedright\strut
  Sampler\strut
  \end{minipage} & \begin{minipage}[t]{0.11\columnwidth}\centering\strut
  0.4687\strut
  \end{minipage} & \begin{minipage}[t]{0.12\columnwidth}\centering\strut
  0.2\strut
  \end{minipage} & \begin{minipage}[t]{0.12\columnwidth}\centering\strut
  0.03929\strut
  \end{minipage}\tabularnewline
  \begin{minipage}[t]{0.26\columnwidth}\raggedright\strut
  Model\strut
  \end{minipage} & \begin{minipage}[t]{0.11\columnwidth}\centering\strut
  0.469\strut
  \end{minipage} & \begin{minipage}[t]{0.12\columnwidth}\centering\strut
  0.199\strut
  \end{minipage} & \begin{minipage}[t]{0.12\columnwidth}\centering\strut
  0.039\strut
  \end{minipage}\tabularnewline
  \bottomrule
  \end{longtable}
  
  As is obvious from Table, the Gibbs sampler produces values very similar
  to the ones given by an \textit{R} call of Model 1.
  
  \subsection{County Model 2 (Random County Effects and County-Level
  Predictors)}\label{county-model-2-random-county-effects-and-county-level-predictors}
  
  With slight changes from the previous model the following is the
  mathematical expression for a mixed effects model with group-level
  predictors:
  
  \[y_i \ \sim \ N(a_{j[i]}, \sigma^2_y),\ i \in [1, n] \\ a_j \ \sim \ N(U_j\gamma, \sigma^2_{\alpha}), \ j \in [1,J], \]
  
  where \(U_j\) is a vector of predictor values for group \(j\), and
  \(\gamma\) a vector of group-level coefficients, with the rest of the
  parameters having the same designation as previously. Bear in mind that
  the second of the previous expressions can also be written as:
  
  \begin{equation}
  \alpha_j = U_j\gamma + \eta_j, \ \ \eta_j \ \sim \ N(0, \sigma_{\alpha}^2)
  \end{equation}
  
  Updating the estimates used previously, it is again possible to
  construct a Gibbs sampler for model coefficients and errors. Step 2(b)
  in the Gibbs sampler in this case is:
  
  \begin{enumerate}
    \item Estimate $a_j, \ j\in[1,J]$. Start by calculating $y_i^{temp} = y_i - U_{j[i]}\gamma$. Then calculate an estimate $\hat\eta_j$ and variance matrix $V_j$ from equations (1), (2), by replacing $\hat\alpha_j$ with $\hat\eta_j$ and $y$ with $y^{temp}$. Use $\eta_j \ \sim \ N(\hat\eta_j, V_j)$ to draw errors $\eta_j$ and then use (3) to estimate $\alpha_j$ for $j \in [1,J]$.
    \item Estimate $\gamma$ by first regressing $\alpha$ by predictor matrix $U$ to obtain $\hat\gamma$ and variance matrix $V_{\gamma}$. Then use distribution $\gamma_j \ \sim \ N(\hat\gamma_j, V_j)$ to obtain estimates for vector $\gamma$.
    \item Estimate $\sigma_y^2$ as $\frac{\frac{1}{n}\sum_{1}^{n}(y_i - \alpha_{j[i]})^2}{X_{n-1}^2}$ where $X_{n-1}^2$ is a draw from a $\chi^2$ distribution with $n-1$ degrees of freedom.
    \item Estimate $\sigma_{\alpha}^2$ as $\frac{\frac{1}{J}\sum_{1}^{J}(\alpha_j - U_j\gamma)^2}{X_{J-1}^2}$ where $X_{n-1}^2$ is a draw from a $\chi^2$ distribution with $J-1$ degrees of freedom.
  \end{enumerate}
  
  County Model 2, as presented in Chapter 4, includes two county-level
  predictors: percentage of white residents and percentage of urban
  population; this means that \(U = \{x^{\%white}, x^{\%urban}\}\).
  Keeping this in mind the following code estimates the coefficients and
  standard errors for Model 2:
  
  \begin{Shaded}
  \begin{Highlighting}[]
  \NormalTok{## Gibbs sampler for a multilevel model with county predictors}
    
  
  \NormalTok{a.update <-}\StringTok{ }\ControlFlowTok{function}\NormalTok{()\{}
  \NormalTok{  y.temp <-}\StringTok{ }\NormalTok{model_dt}\OperatorTok{$}\NormalTok{turnout }\OperatorTok{-}\StringTok{ }
  \StringTok{    }\NormalTok{(model_dt}\OperatorTok{$}\NormalTok{pct_urban}\OperatorTok{*}\NormalTok{g[}\DecValTok{1}\NormalTok{] }\OperatorTok{+}\StringTok{ }\NormalTok{model_dt}\OperatorTok{$}\NormalTok{pct_white}\OperatorTok{*}\NormalTok{g[}\DecValTok{2}\NormalTok{])}
  \NormalTok{  eta.new <-}\StringTok{ }\KeywordTok{rep}\NormalTok{ (}\OtherTok{NA}\NormalTok{, J)}
    \ControlFlowTok{for}\NormalTok{ (j }\ControlFlowTok{in} \DecValTok{1}\OperatorTok{:}\NormalTok{J)\{}
  \NormalTok{    n.j <-}\StringTok{ }\KeywordTok{sum}\NormalTok{ (model_dt}\OperatorTok{$}\NormalTok{county}\OperatorTok{==}\NormalTok{cnt_vec[j])}
  \NormalTok{    y.bar.j <-}\StringTok{ }\KeywordTok{mean}\NormalTok{ (y.temp[model_dt}\OperatorTok{$}\NormalTok{county}\OperatorTok{==}\NormalTok{cnt_vec[j]])}
  \NormalTok{    eta.hat.j <-}\StringTok{ }\NormalTok{((n.j}\OperatorTok{/}\NormalTok{sigma.y}\OperatorTok{^}\DecValTok{2}\NormalTok{)}\OperatorTok{*}\NormalTok{y.bar.j}\OperatorTok{/}
  \StringTok{                 }\NormalTok{(n.j}\OperatorTok{/}\NormalTok{sigma.y}\OperatorTok{^}\DecValTok{2} \OperatorTok{+}\StringTok{ }\DecValTok{1}\OperatorTok{/}\NormalTok{sigma.a}\OperatorTok{^}\DecValTok{2}\NormalTok{))}
  \NormalTok{    V.eta.j <-}\StringTok{ }\DecValTok{1}\OperatorTok{/}\NormalTok{(n.j}\OperatorTok{/}\NormalTok{sigma.y}\OperatorTok{^}\DecValTok{2} \OperatorTok{+}\StringTok{ }\DecValTok{1}\OperatorTok{/}\NormalTok{sigma.a}\OperatorTok{^}\DecValTok{2}\NormalTok{)}
  \NormalTok{    eta.new[j] <-}\StringTok{ }\KeywordTok{rnorm}\NormalTok{ (}\DecValTok{1}\NormalTok{, eta.hat.j, }\KeywordTok{sqrt}\NormalTok{(V.eta.j))}
  \NormalTok{  \}}
  \NormalTok{  a.new <-}\StringTok{ }\NormalTok{(U}\OperatorTok{$}\NormalTok{urban}\OperatorTok{*}\NormalTok{g[}\DecValTok{1}\NormalTok{] }\OperatorTok{+}\StringTok{ }\NormalTok{U}\OperatorTok{$}\NormalTok{white}\OperatorTok{*}\NormalTok{g[}\DecValTok{2}\NormalTok{]) }\OperatorTok{+}\StringTok{ }\NormalTok{eta.new}
    \KeywordTok{return}\NormalTok{ (a.new)}
  \NormalTok{\}}
  
  \NormalTok{g.update <-}\StringTok{ }\ControlFlowTok{function}\NormalTok{()\{}
  \NormalTok{  lm.}\DecValTok{0}\NormalTok{ <-}\StringTok{ }\KeywordTok{lm}\NormalTok{ (a }\OperatorTok{~}\StringTok{ }\NormalTok{U}\OperatorTok{$}\NormalTok{urban }\OperatorTok{+}\StringTok{ }\NormalTok{U}\OperatorTok{$}\NormalTok{white)}
  \NormalTok{  g.new <-}\StringTok{ }\KeywordTok{coef}\NormalTok{ (lm.}\DecValTok{0}\NormalTok{)[}\DecValTok{2}\OperatorTok{:}\DecValTok{3}\NormalTok{]}
    \KeywordTok{return}\NormalTok{ (g.new)}
  \NormalTok{\}}
  
  \NormalTok{sigma.y.update <-}\StringTok{ }\ControlFlowTok{function}\NormalTok{()\{}
  \NormalTok{  sigma.y.new <-}\StringTok{ }\KeywordTok{sqrt}\NormalTok{(}\KeywordTok{sum}\NormalTok{((model_dt}\OperatorTok{$}\NormalTok{turnout}\OperatorTok{-}
  \StringTok{                }\NormalTok{a[model_dt}\OperatorTok{$}\NormalTok{county])}\OperatorTok{^}\DecValTok{2}\NormalTok{)}\OperatorTok{/}\KeywordTok{rchisq}\NormalTok{(}\DecValTok{1}\NormalTok{,}\DecValTok{703}\NormalTok{))}
    \KeywordTok{return}\NormalTok{ (sigma.y.new)}
  \NormalTok{\}}
  
  \NormalTok{sigma.a.update <-}\StringTok{ }\ControlFlowTok{function}\NormalTok{()\{}
  \NormalTok{  sigma.a.new <-}\StringTok{ }\KeywordTok{sqrt}\NormalTok{(}\KeywordTok{sum}\NormalTok{((a}\OperatorTok{-}\NormalTok{(model_dt}\OperatorTok{$}\NormalTok{pct_urban}\OperatorTok{*}\NormalTok{g[}\DecValTok{1}\NormalTok{] }\OperatorTok{+}\StringTok{ }
  \StringTok{                }\NormalTok{model_dt}\OperatorTok{$}\NormalTok{pct_white}\OperatorTok{*}\NormalTok{g[}\DecValTok{2}\NormalTok{]))}\OperatorTok{^}\DecValTok{2}\NormalTok{)}\OperatorTok{/}\KeywordTok{rchisq}\NormalTok{(}\DecValTok{1}\NormalTok{,J}\OperatorTok{-}\DecValTok{1}\NormalTok{))}
    \KeywordTok{return}\NormalTok{ (sigma.a.new)}
  \NormalTok{\}}
  
  \NormalTok{J <-}\StringTok{ }\DecValTok{64}
  \NormalTok{n.chains <-}\StringTok{ }\DecValTok{3}
  \NormalTok{n.iter <-}\StringTok{ }\DecValTok{2000}
  \NormalTok{sims <-}\StringTok{ }\KeywordTok{array}\NormalTok{ (}\OtherTok{NA}\NormalTok{, }\KeywordTok{c}\NormalTok{(n.iter, n.chains, J}\OperatorTok{+}\DecValTok{4}\NormalTok{))}
  \KeywordTok{dimnames}\NormalTok{ (sims) <-}\StringTok{ }\KeywordTok{list}\NormalTok{ (}\OtherTok{NULL}\NormalTok{, }\OtherTok{NULL}\NormalTok{, }\KeywordTok{c}\NormalTok{ (}\KeywordTok{paste}\NormalTok{ (}\StringTok{"a["}\NormalTok{, }\DecValTok{1}\OperatorTok{:}\NormalTok{J, }\StringTok{"]"}\NormalTok{, }\DataTypeTok{sep=}\StringTok{""}\NormalTok{), }
     \KeywordTok{c}\NormalTok{(}\StringTok{"coef.urban"}\NormalTok{, }\StringTok{"coef.white"}\NormalTok{),}
     \StringTok{"sigma.y"}\NormalTok{, }\StringTok{"sigma.a"}\NormalTok{))}
  
  \ControlFlowTok{for}\NormalTok{ (m }\ControlFlowTok{in} \DecValTok{1}\OperatorTok{:}\NormalTok{n.chains)\{}
  \NormalTok{  g <-}\StringTok{ }\KeywordTok{rnorm}\NormalTok{ (}\DecValTok{2}\NormalTok{)}
  \NormalTok{  sigma.y <-}\StringTok{ }\KeywordTok{runif}\NormalTok{ (}\DecValTok{1}\NormalTok{, }\DecValTok{0}\NormalTok{, }\KeywordTok{sd}\NormalTok{(model_dt}\OperatorTok{$}\NormalTok{turnout))}
  \NormalTok{  sigma.a <-}\StringTok{ }\KeywordTok{runif}\NormalTok{ (}\DecValTok{1}\NormalTok{, }\DecValTok{0}\NormalTok{, }\KeywordTok{sd}\NormalTok{(model_dt}\OperatorTok{$}\NormalTok{turnout))}
    \ControlFlowTok{for}\NormalTok{ (t }\ControlFlowTok{in} \DecValTok{1}\OperatorTok{:}\NormalTok{n.iter)\{}
  \NormalTok{    a <-}\StringTok{ }\KeywordTok{a.update}\NormalTok{ ()}
  \NormalTok{    g <-}\StringTok{ }\KeywordTok{g.update}\NormalTok{ ()}
  \NormalTok{    sigma.y <-}\StringTok{ }\KeywordTok{sigma.y.update}\NormalTok{ ()}
  \NormalTok{    sigma.a <-}\StringTok{ }\KeywordTok{sigma.a.update}\NormalTok{ ()}
  \NormalTok{    sims[t,m,] <-}\StringTok{ }\KeywordTok{c}\NormalTok{ (a, g, sigma.y, sigma.a)}
  \NormalTok{  \}}
  \NormalTok{\}}
  \end{Highlighting}
  \end{Shaded}
  
  \begin{longtable}[]{@{}lcccc@{}}
  \caption{Gibbs sampler results for County Model 2
  \label{tab:gibbs_2}}\tabularnewline
  \toprule
  \begin{minipage}[b]{0.25\columnwidth}\raggedright\strut
  Calculated from\ldots{}\strut
  \end{minipage} & \begin{minipage}[b]{0.16\columnwidth}\centering\strut
  coef\_urban\strut
  \end{minipage} & \begin{minipage}[b]{0.16\columnwidth}\centering\strut
  coef\_white\strut
  \end{minipage} & \begin{minipage}[b]{0.12\columnwidth}\centering\strut
  sigma.y\strut
  \end{minipage} & \begin{minipage}[b]{0.12\columnwidth}\centering\strut
  sigma.a\strut
  \end{minipage}\tabularnewline
  \midrule
  \endfirsthead
  \toprule
  \begin{minipage}[b]{0.25\columnwidth}\raggedright\strut
  Calculated from\ldots{}\strut
  \end{minipage} & \begin{minipage}[b]{0.16\columnwidth}\centering\strut
  coef\_urban\strut
  \end{minipage} & \begin{minipage}[b]{0.16\columnwidth}\centering\strut
  coef\_white\strut
  \end{minipage} & \begin{minipage}[b]{0.12\columnwidth}\centering\strut
  sigma.y\strut
  \end{minipage} & \begin{minipage}[b]{0.12\columnwidth}\centering\strut
  sigma.a\strut
  \end{minipage}\tabularnewline
  \midrule
  \endhead
  \begin{minipage}[t]{0.25\columnwidth}\raggedright\strut
  Sampler\strut
  \end{minipage} & \begin{minipage}[t]{0.16\columnwidth}\centering\strut
  -0.1184\strut
  \end{minipage} & \begin{minipage}[t]{0.16\columnwidth}\centering\strut
  0.03217\strut
  \end{minipage} & \begin{minipage}[t]{0.12\columnwidth}\centering\strut
  0.1997\strut
  \end{minipage} & \begin{minipage}[t]{0.12\columnwidth}\centering\strut
  1.686\strut
  \end{minipage}\tabularnewline
  \begin{minipage}[t]{0.25\columnwidth}\raggedright\strut
  Model\strut
  \end{minipage} & \begin{minipage}[t]{0.16\columnwidth}\centering\strut
  -0.118\strut
  \end{minipage} & \begin{minipage}[t]{0.16\columnwidth}\centering\strut
  0.034\strut
  \end{minipage} & \begin{minipage}[t]{0.12\columnwidth}\centering\strut
  0.199\strut
  \end{minipage} & \begin{minipage}[t]{0.12\columnwidth}\centering\strut
  2.631\strut
  \end{minipage}\tabularnewline
  \bottomrule
  \end{longtable}
  
  As previously the values outputted by the Gibbs sampler are very close
  to those estimated by the model, apart from the group level standard
  deviation. Some variability in how closely the sampler approximates the
  model call is to be expected, due to the difference in how the model is
  estimated in R (much more precise Bayesian processes).
  
  \backmatter
  
  \chapter{References}\label{references}
  
  \noindent
  
  \setlength{\parindent}{-0.20in} \setlength{\leftskip}{0.20in}
  \setlength{\parskip}{8pt}
  
  \hypertarget{refs}{}
  \hypertarget{ref-national_council_of_state_legislatures_absentee_2018}{}
  Absentee and Early Voting. (2018, October). \emph{National Council of
  State Legislatures}. Retrieved from
  \url{http://www.ncsl.org/research/elections-and-campaigns/absentee-and-early-voting.aspx\#a}
  
  \hypertarget{ref-aldrich_rational_1993}{}
  Aldrich, J. H. (1993). Rational Choice and Turnout. \emph{American
  Journal of Political Science}, \emph{37}(1), 246--278.
  \url{http://doi.org/10.2307/2111531}
  
  \hypertarget{ref-ansolabehere_quality_2010}{}
  Ansolabehere, S., \& Hersh, E. (2010). The Quality of Voter Registration
  Records: A State-by-State Analysis. \emph{Institute for Quantitative
  Social Science and Caltech/MIT Voting Technology Project Working Paper}.
  Retrieved from
  \url{https://dataverse.harvard.edu/dataset.xhtml?persistentId=hdl:1902.1/18550}
  
  \hypertarget{ref-ansolabehere_adgn:_2017}{}
  Ansolabehere, S., \& Hersh, E. D. (2017). ADGN: An Algorithm for Record
  Linkage Using Address, Date of Birth, Gender, and Name. \emph{Statistics
  and Public Policy}, \emph{4}(1), 1--10.
  \url{http://doi.org/10.1080/2330443X.2017.1389620}
  
  \hypertarget{ref-barr_comprehensive_2012}{}
  Barr, C. D., Diez, D. M., Wang, Y., Dominici, F., \& Samet, J. M.
  (2012). Comprehensive Smoking Bans and Acute Myocardial Infarction Among
  Medicare Enrollees in 387 US Counties: 1999--2008. \emph{American
  Journal of Epidemiology}, \emph{176}(7), 642--648.
  \url{http://doi.org/10.1093/aje/kws267}
  
  \hypertarget{ref-bates_fitting_2015}{}
  Bates, D., Mächler, M., Bolker, B., \& Walker, S. (2015). Fitting Linear
  Mixed-Effects Models using lme4. \emph{Journal of Statistical Software},
  \emph{67}(1). Retrieved from \url{http://arxiv.org/abs/1406.5823}
  
  \hypertarget{ref-bergman_changing_2011}{}
  Bergman, E., \& Yates, P. A. (2011). Changing Election Methods: How Does
  Mandated Vote-By-Mail Affect Individual Registrants? \emph{Election Law
  Journal: Rules, Politics, and Policy}, \emph{10}(2), 115--127.
  \url{http://doi.org/10.1089/elj.2010.0079}
  
  \hypertarget{ref-berinsky_perverse_2005}{}
  Berinsky, A. J. (2005). The Perverse Consequences of Electoral Reform in
  the United States. \emph{American Politics Research}, \emph{33}(4),
  471--491. \url{http://doi.org/10.1177/1532673X04269419}
  
  \hypertarget{ref-berinsky_making_2016}{}
  Berinsky, A. J. (2016, February). Making Voting Easier Doesn't Increase
  Turnout. \emph{Stanford Social Innovation Review}. Retrieved from
  \url{https://ssir.org/articles/entry/making_voting_easier_doesnt_increase_turnout}
  
  \hypertarget{ref-burden_election_2013}{}
  Burden, B. C., \& Neiheisel, J. R. (2013). Election Administration and
  the Pure Effect of Voter Registration on Turnout. \emph{Political
  Research Quarterly}, \emph{66}(1), 77--90.
  \url{http://doi.org/10.1177/1065912911430671}
  
  \hypertarget{ref-burden_election_2014}{}
  Burden, B. C., Canon, D. T., Mayer, K. R., \& Moynihan, D. P. (2014).
  Election Laws, Mobilization, and Turnout: The Unanticipated Consequences
  of Election Reform. \emph{American Journal of Political Science},
  \emph{58}(1), 95--109. \url{http://doi.org/10.1111/ajps.12063}
  
  \hypertarget{ref-campbell_self-interest_2002}{}
  Campbell, A. L. (2002). Self-Interest, Social Security, and the
  Distinctive Participation Patterns of Senior Citizens. \emph{American
  Political Science Review}, \emph{96}(3), 565--574.
  \url{http://doi.org/10.1017/S0003055402000333}
  
  \hypertarget{ref-chen_voter_2013}{}
  Chen, J. (2013). Voter Partisanship and the Effect of Distributive
  Spending on Political Participation. \emph{American Journal of Political
  Science}, \emph{57}(1), 200--217.
  \url{http://doi.org/10.1111/j.1540-5907.2012.00613.x}
  
  \hypertarget{ref-chihara_mathematical_2011}{}
  Chihara, L. M., \& Hesterberg, T. C. (2011). \emph{Mathematical
  Statistics with Resampling and R} (1 edition). Hoboken, N.J: Wiley.
  
  \hypertarget{ref-cronin_colorado_2012}{}
  Cronin, T. E., \& Loevy, R. D. (2012). \emph{Colorado Politics and
  Policy: Governing a Purple State}. Lincoln: University of Nebraska
  Press. Retrieved from
  \url{http://ebookcentral.proquest.com/lib/reed/detail.action?docID=1034959}
  
  \hypertarget{ref-edlin_voting_2007}{}
  Edlin, A., Gelman, A., \& Kaplan, N. (2007). Voting as a Rational
  Choice: Why and How People Vote To Improve the Well-Being of Others.
  \emph{Rationality and Society}, \emph{19}(3), 293--314.
  \url{http://doi.org/10.1177/1043463107077384}
  
  \hypertarget{ref-ewald_way_2009}{}
  Ewald, A. C. (2009). \emph{The Way We Vote: The Local Dimension of
  American Suffrage}. Nashville: Vanderbilt University Press.
  
  \hypertarget{ref-fortier_absentee_2006}{}
  Fortier, J. C. (2006). \emph{Absentee and early voting: Trends,
  promises, and perils}. Washington, DC: AEI Press.
  
  \hypertarget{ref-fowler_habitual_2006}{}
  Fowler, J. H. (2006). Habitual Voting and Behavioral Turnout.
  \emph{Journal of Politics}, \emph{68}(2), 335--344.
  \url{http://doi.org/10.1111/j.1468-2508.2006.00410.x}
  
  \hypertarget{ref-gelman_data_2006}{}
  Gelman, A., \& Hill, J. (2006). \emph{Data Analysis Using Regression and
  Multilevel/Hierarchical Models} (1st Edition). Cambridge, NY: Cambridge
  University Press.
  
  \hypertarget{ref-gelman_bayesian_2003}{}
  Gelman, A., Carlin, J. B., Stern, H. S., \& Rubin, D. B. (2003).
  \emph{Bayesian Data Analysis, Second Edition} (2nd Edition). Boca Raton,
  Fla: Chapman Hall.
  
  \hypertarget{ref-gerber_identifying_2013}{}
  Gerber, A. S., Huber, G. A., \& Hill, S. J. (2013). Identifying the
  Effect of All-Mail Elections on Turnout: Staggered Reform in the
  Evergreen State. \emph{Political Science Research and Methods},
  \emph{1}(1), 91--116. \url{http://doi.org/10.1017/psrm.2013.5}
  
  \hypertarget{ref-geys_explaining_2006}{}
  Geys, B. (2006). Explaining voter turnout: A review of aggregate-level
  research. \emph{Electoral Studies}, \emph{25}(4), 637--663.
  \url{http://doi.org/10.1016/j.electstud.2005.09.002}
  
  \hypertarget{ref-gronke_voting_2012}{}
  Gronke, P., \& Miller, P. (2012). Voting by Mail and Turnout in Oregon:
  Revisiting Southwell and Burchett. \emph{American Politics Research},
  \emph{40}(6), 976--997. \url{http://doi.org/10.1177/1532673X12457809}
  
  \hypertarget{ref-gronke_convenience_2008}{}
  Gronke, P., Galanes-Rosenbaum, E., Miller, P. A., \& Toffey, D. (2008).
  Convenience Voting. \emph{Annual Review of Political Science},
  \emph{11}(1), 437--455.
  \url{http://doi.org/10.1146/annurev.polisci.11.053006.190912}
  
  \hypertarget{ref-gronke_voter_2017}{}
  Gronke, P., McGhee, E., Romero, M., \& Griffin, R. (2017).
  Voter~~Registration~~and~~Turnout~~under
  ``Oregon~~Motor~~Voter'':~~A~~Second~~Look. In. Portland. Retrieved from
  \url{https://dl.dropboxusercontent.com/nativeprint?file=https\%3A\%2F\%2Fwww.dropbox.com\%2Fs\%2Fo63a0zi7j8plhgl\%2FOMV_and_Turnout_McGhee_Gronke_Romero_Griffin_July2017.pdf\%3Fdisable_range\%3D1\%26from_native_print\%3D1\%26preview\%3D1}
  
  \hypertarget{ref-hamm_how_2017}{}
  Hamm, K. (2017). How Colorado has voted in presidential elections (and
  how its politics have changed) since 1980. \emph{The Denver Post}.
  Retrieved from
  \url{https://www.denverpost.com/2017/12/22/how-colorado-votes/}
  
  \hypertarget{ref-handayani_comparative_2017}{}
  Handayani, D., Notodiputro, K. A., Sadik, K., \& Kurnia, A. (2017). A
  comparative study of approximation methods for maximum likelihood
  estimation in generalized linear mixed models (GLMM). In. Jawa Barat,
  Indonesia. \url{http://doi.org/10.1063/1.4979449}
  
  \hypertarget{ref-hersh_hacking_2015}{}
  Hersh, E. D. (2015). \emph{Hacking the Electorate: How Campaigns
  Perceive Voters}. New York, NY: Cambridge University Press.
  
  \hypertarget{ref-hullinghorst_voter_2013}{}
  Hullinghorst, D. L., \& Pabon, D. (2013, May). Voter Access and
  Moderninzed Elections Act.
  
  \hypertarget{ref-jackman_bayesian_2009}{}
  Jackman, S. (2009). \emph{Bayesian Analysis for the Social Sciences}
  (1st edition). Chichester, U.K: Wiley.
  
  \hypertarget{ref-james_introduction_2017}{}
  James, G., Witten, D., Hastie, T., \& Tibshirani, R. (2017). \emph{An
  Introduction to Statistical Learning: With Applications in R} (1st ed.
  2013, Corr. 7th printing 2017 edition). New York: Springer.
  
  \hypertarget{ref-keele_geographic_2017}{}
  Keele, L., \& Titiunik, R. (2017). Geographic Natural Experiments with
  Interference: The Effect of All-Mail Voting on Turnout in Colorado.
  
  \hypertarget{ref-martin_colorado_1962}{}
  Martin, C. (1962). \emph{Colorado politics} (2nd ed.). Denver, Colorado:
  Big Mountain Press. Retrieved from
  \url{http://hdl.handle.net/2027/mdp.39015024371158}
  
  \hypertarget{ref-matsusaka_voter_1999}{}
  Matsusaka, J. G., \& Palda, F. (1999). Voter turnout: How much can we
  explain? \emph{Public Choice}, \emph{98}(3-4), 431--446.
  \url{http://doi.org/10.1023/A:1018328621580}
  
  \hypertarget{ref-mcclelland_synthetic_2017}{}
  McClelland, R., \& Gault, S. (2017). The Synthetic Control Method as a
  Tool to Understand State Policy. \emph{Washington, DC: Urban-Brookings
  Tax Policy Center}.
  
  \hypertarget{ref-mcdonald_true_2007}{}
  McDonald, M. P. (2007). The True Electorate: A Cross-Validation of Voter
  Registration Files and Election Survey Demographics. \emph{Public
  Opinion Quarterly}, \emph{71}(4), 588--602.
  \url{http://doi.org/10.1093/poq/nfm046}
  
  \hypertarget{ref-mettler_government_2008}{}
  Mettler, S., \& Stonecash, J. M. (2008). Government Program Usage and
  Political Voice. \emph{Social Science Quarterly}, \emph{89}(2),
  273--293. \url{http://doi.org/10.1111/j.1540-6237.2008.00532.x}
  
  \hypertarget{ref-neiheisel_impact_2012}{}
  Neiheisel, J. R., \& Burden, B. C. (2012). The Impact of Election Day
  Registration on Voter Turnout and Election Outcomes. \emph{American
  Politics Research}, \emph{40}(4), 636--664.
  \url{http://doi.org/10.1177/1532673X11432470}
  
  \hypertarget{ref-plutzer_becoming_2002}{}
  Plutzer, E. (2002). Becoming a Habitual Voter: Inertia, Resources, and
  Growth in Young Adulthood. \emph{The American Political Science Review},
  \emph{96}(1), 41--56. Retrieved from
  \url{https://www.jstor.org/stable/3117809}
  
  \hypertarget{ref-richey_sean_voting_2008}{}
  Richey Sean. (2008). Voting by Mail: Turnout and Institutional Reform in
  Oregon. \emph{Social Science Quarterly}, \emph{89}(4), 902--915.
  \url{http://doi.org/10.1111/j.1540-6237.2008.00590.x}
  
  \hypertarget{ref-rosenstone_mobilization_2003}{}
  Rosenstone, S. J. (2003). \emph{Mobilization, participation, and
  democracy in America}. New York: Longman.
  
  \hypertarget{ref-saltman_history_2009}{}
  Saltman, R. (2009). \emph{The History and Politics of Voting Technology:
  In Quest of Integrity and Public Confidence} (2006 edition).
  Gordonsville: Palgrave Macmillan.
  
  \hypertarget{ref-schneider_behavioral_1990}{}
  Schneider, A., \& Ingram, H. (1990). Behavioral Assumptions of Policy
  Tools. \emph{The Journal of Politics}, \emph{52}(2), 510--529.
  \url{http://doi.org/10.2307/2131904}
  
  \hypertarget{ref-smets_embarrassment_2013}{}
  Smets, K., \& Ham, C. van. (2013). The embarrassment of riches? A
  meta-analysis of individual-level research on voter turnout.
  \emph{Electoral Studies}, \emph{32}(2), 344--359.
  \url{http://doi.org/10.1016/j.electstud.2012.12.006}
  
  \hypertarget{ref-legislative_council_staff_tabor_2009}{}
  Staff, L. C. (2009, July). TABOR and Referendum C. \emph{Colorado
  General Assembly}. Retrieved from
  \url{https://leg.colorado.gov/publications/tabor-and-referendum-c-2009}
  
  \hypertarget{ref-stein_engaging_2008}{}
  Stein, R. M., \& Vonnahme, G. (2008). Engaging the Unengaged Voter: Vote
  Centers and Voter Turnout. \emph{The Journal of Politics}, \emph{70}(2),
  487--497. \url{http://doi.org/10.1017/S0022381608080456}
  
  \hypertarget{ref-thompson_first_2016}{}
  Thompson, J. (2016). The first Sagebrush Rebellion: What sparked it and
  how it ended. Retrieved from
  \url{https://www.hcn.org/articles/a-look-back-at-the-first-sagebrush-rebellion}
  
  \hypertarget{ref-us_census_bureau_us_2010}{}
  US Census Bureau QuickFacts: Colorado. (2010). \emph{US Census Bureau
  Quickfacts}. Retrieved from \url{https://www.census.gov/quickfacts/co}
  
  \hypertarget{ref-verba_participation_1972}{}
  Verba, S., \& Nie, N. H. (1972). \emph{Participation in America:
  Political democracy and social equality}. New York: Harper \& Row.
  
  \hypertarget{ref-wand_butterfly_2001}{}
  Wand, J. N., Shotts, K. W., Sekhon, J. S., Mebane, W. R., Herron, M. C.,
  \& Brady, H. E. (2001). The Butterfly Did It: The Aberrant Vote for
  Buchanan in Palm Beach County, Florida. \emph{American Political Science
  Review}, \emph{95}(4), 793--810.
  \url{http://doi.org/10.1017/S000305540040002X}
  
  \hypertarget{ref-wood_generalized_2006}{}
  Wood, S. N. (2006). \emph{Generalized Additive Models: An Introduction
  with R} (1st edition). Boca Raton, FL: Chapman; Hall/CRC.
  
  \hypertarget{ref-wood_gamm4:_2017}{}
  Wood, S., \& Scheipl, F. (2017, July). Gamm4: Generalized Additive Mixed
  Models using 'mgcv' and 'lme4'. Retrieved from
  \url{https://CRAN.R-project.org/package=gamm4}


  % Index?

\end{document}

