% This is the Reed College LaTeX thesis template. Most of the work
% for the document class was done by Sam Noble (SN), as well as this
% template. Later comments etc. by Ben Salzberg (BTS). Additional
% restructuring and APA support by Jess Youngberg (JY).
% Your comments and suggestions are more than welcome; please email
% them to cus@reed.edu
%
% See http://web.reed.edu/cis/help/latex.html for help. There are a
% great bunch of help pages there, with notes on
% getting started, bibtex, etc. Go there and read it if you're not
% already familiar with LaTeX.
%
% Any line that starts with a percent symbol is a comment.
% They won't show up in the document, and are useful for notes
% to yourself and explaining commands.
% Commenting also removes a line from the document;
% very handy for troubleshooting problems. -BTS

% As far as I know, this follows the requirements laid out in
% the 2002-2003 Senior Handbook. Ask a librarian to check the
% document before binding. -SN

%%
%% Preamble
%%
% \documentclass{<something>} must begin each LaTeX document
\documentclass[12pt,twoside]{reedthesis}
% Packages are extensions to the basic LaTeX functions. Whatever you
% want to typeset, there is probably a package out there for it.
% Chemistry (chemtex), screenplays, you name it.
% Check out CTAN to see: http://www.ctan.org/
%%
\usepackage{graphicx,latexsym}
\usepackage{amsmath}
\usepackage{amssymb,amsthm}
\usepackage{longtable,booktabs,setspace}
\usepackage{chemarr} %% Useful for one reaction arrow, useless if you're not a chem major
\usepackage[hyphens]{url}
% Added by CII
\usepackage{hyperref}
\usepackage{lmodern}
% End of CII addition
\usepackage{rotating}

% Next line commented out by CII
%%% \usepackage{natbib}
% Comment out the natbib line above and uncomment the following two lines to use the new 
% biblatex-chicago style, for Chicago A. Also make some changes at the end where the 
% bibliography is included. 
%\usepackage{biblatex-chicago}
%\bibliography{thesis}


% Added by CII (Thanks, Hadley!)
% Use ref for internal links
\renewcommand{\hyperref}[2][???]{\autoref{#1}}
\def\chapterautorefname{Chapter}
\def\sectionautorefname{Section}
\def\subsectionautorefname{Subsection}
% End of CII addition

% Added by CII 
\usepackage{caption}
\captionsetup{width=5in}
% End of CII addition

% \usepackage{times} % other fonts are available like times, bookman, charter, palatino


% To pass between YAML and LaTeX the dollar signs are added by CII
\title{Turnout and Mail Voting in Colorado; or How I Learned to Stop Worrying
and Love Voter Registration Files}
\author{Theodore Dounias}
% The month and year that you submit your FINAL draft TO THE LIBRARY (May or December)
\date{December 2018}
\division{Mathematics and Natural Sciences and History and Social Sciences}
\advisor{Paul Gronke}
%If you have two advisors for some reason, you can use the following
% Uncommented out by CII
\altadvisor{Andrew Bray} 
% End of CII addition

%%% Remember to use the correct department!
\department{Mathematics and Political Science}
% if you're writing a thesis in an interdisciplinary major,
% uncomment the line below and change the text as appropriate.
% check the Senior Handbook if unsure.
%\thedivisionof{The Established Interdisciplinary Committee for}
% if you want the approval page to say "Approved for the Committee",
% uncomment the next line
%\approvedforthe{Committee}

% Added by CII
%%% Copied from knitr
%% maxwidth is the original width if it's less than linewidth
%% otherwise use linewidth (to make sure the graphics do not exceed the margin)
\makeatletter
\def\maxwidth{ %
  \ifdim\Gin@nat@width>\linewidth
    \linewidth
  \else
    \Gin@nat@width
  \fi
}
\makeatother

\renewcommand{\contentsname}{Table of Contents}
% End of CII addition

\setlength{\parskip}{0pt}

% Added by CII

\providecommand{\tightlist}{%
  \setlength{\itemsep}{0pt}\setlength{\parskip}{0pt}}

\Acknowledgements{

}

\Dedication{

}

\Preface{
Coming to Reed College from the National Technical University of
Athens--where I was pursuing an electrical and computer engineering
degree--I tentatively declared my major interest as being Political
Science, only to be instantly swept up by Professor Paul Gronke and
enlisted in his ongoing campaign to conquer elections data. I developed
into an interdisciplinary Mathematics and Political Science major, which
also led me to gain the support of Professor Andrew Bray as my second
thesis adviser. My topic here is mail voting, which is very close to
Professor Gronke's scholarly interests; his enthusiasm is, to some
extent infectious. This topic is of great interest to me because it
involves the often overlooked administrative workings of democracy--the
cogs which support the process of getting the people's will translated
into action. This is of particular interest in Oregon politics as well,
as our representatives and local officials are all strong proponents of
VBM. \par There seems to be a nationwide abundance of individuals that
are committed to studying how the franchise of democracy measures the
people's will, and I am eternally grateful to several of them for their
help with procuring data, answering quantitative and qualitative
questions, and enthusiastically engaging with me when we conversed
directly. \par I would also like to thank my advisers, Paul Gronke and
Andrew Bray, who have put in an incredibly large amount of work for me.
Paul Gronke was present and quickly gave solutions when my data seemed
to be unfit for the conclusions I wished to draw; through this process
he projected calm when I was not, and injected my work with a sense of
pragmatism. Paul Gronke made me feel that my thesis would be completed
and ready, even when I was doubting that fact myself. Andrew Bray, I
thank especially for still working with me despite being on sabbatical
this semester; he sacrificed a lot of his independent research time on
my thesis of his own volition, and showed the same enthusiasm about
statistical applications and questions that made him such a great
instructor in class. \par Reed College is hard, and I could not have
completed my time here without so many different people supporting me.
Although my space here is too limited to describe my gratitude, I will
try giving a woefully incomplete list. (After my first draft)
}

\Abstract{
Mail voting in the United States was conceived and first implemented to
serve absentee voters during the Civil War (Fortier, 2006) and has
persisted until the present day, becoming one of the key reforms
associated with ``convenience voting'' and the expansion of the
democratic franchise (Gronke, Galanes-Rosenbaum, Miller, \& Toffey,
2008). In 2013, Colorado implemented the latest in a series of in-state
election reforms and became the third state in the nation with universal
mail voting for all elections, after Oregon and Washington. Despite
claims by policymakers that mail voting should have a strong, positive
effect on voter turnout, a recent series of studies on Oregon,
Washington, parts of California, and Colorado have failed to show
consistent results, disagreeing both on the scale and the direction
(positive or negative) that this effect has. This thesis aims at
following this series of studies by examining Colorado voter
registration files for recent elections (2010-2016) and describing,
fitting, and interpreting multilevel general additive regression models
of voter turnout based on these data. I show that there is a small
positive effect of mail voting on turnout in national elections at the
county level. This thesis also contributes to the literature by
presenting a description of modeling and data wrangling difficulties
associated with voter registration files, and giving a series of
potential solutions, as well as an extensive coding library to aid
future research on the subject.
}

% End of CII addition
%%
%% End Preamble
%%
%

\begin{document}

% Everything below added by CII
      \maketitle
  
  \frontmatter % this stuff will be roman-numbered
  \pagestyle{empty} % this removes page numbers from the frontmatter

  
      \begin{preface}
      Coming to Reed College from the National Technical University of
      Athens--where I was pursuing an electrical and computer engineering
      degree--I tentatively declared my major interest as being Political
      Science, only to be instantly swept up by Professor Paul Gronke and
      enlisted in his ongoing campaign to conquer elections data. I developed
      into an interdisciplinary Mathematics and Political Science major, which
      also led me to gain the support of Professor Andrew Bray as my second
      thesis adviser. My topic here is mail voting, which is very close to
      Professor Gronke's scholarly interests; his enthusiasm is, to some
      extent infectious. This topic is of great interest to me because it
      involves the often overlooked administrative workings of democracy--the
      cogs which support the process of getting the people's will translated
      into action. This is of particular interest in Oregon politics as well,
      as our representatives and local officials are all strong proponents of
      VBM. \par There seems to be a nationwide abundance of individuals that
      are committed to studying how the franchise of democracy measures the
      people's will, and I am eternally grateful to several of them for their
      help with procuring data, answering quantitative and qualitative
      questions, and enthusiastically engaging with me when we conversed
      directly. \par I would also like to thank my advisers, Paul Gronke and
      Andrew Bray, who have put in an incredibly large amount of work for me.
      Paul Gronke was present and quickly gave solutions when my data seemed
      to be unfit for the conclusions I wished to draw; through this process
      he projected calm when I was not, and injected my work with a sense of
      pragmatism. Paul Gronke made me feel that my thesis would be completed
      and ready, even when I was doubting that fact myself. Andrew Bray, I
      thank especially for still working with me despite being on sabbatical
      this semester; he sacrificed a lot of his independent research time on
      my thesis of his own volition, and showed the same enthusiasm about
      statistical applications and questions that made him such a great
      instructor in class. \par Reed College is hard, and I could not have
      completed my time here without so many different people supporting me.
      Although my space here is too limited to describe my gratitude, I will
      try giving a woefully incomplete list. (After my first draft)
    \end{preface}
  
      \hypersetup{linkcolor=black}
    \setcounter{tocdepth}{2}
    \tableofcontents
  
      \listoftables
  
      \listoffigures
  
      \begin{abstract}
      Mail voting in the United States was conceived and first implemented to
      serve absentee voters during the Civil War (Fortier, 2006) and has
      persisted until the present day, becoming one of the key reforms
      associated with ``convenience voting'' and the expansion of the
      democratic franchise (Gronke, Galanes-Rosenbaum, Miller, \& Toffey,
      2008). In 2013, Colorado implemented the latest in a series of in-state
      election reforms and became the third state in the nation with universal
      mail voting for all elections, after Oregon and Washington. Despite
      claims by policymakers that mail voting should have a strong, positive
      effect on voter turnout, a recent series of studies on Oregon,
      Washington, parts of California, and Colorado have failed to show
      consistent results, disagreeing both on the scale and the direction
      (positive or negative) that this effect has. This thesis aims at
      following this series of studies by examining Colorado voter
      registration files for recent elections (2010-2016) and describing,
      fitting, and interpreting multilevel general additive regression models
      of voter turnout based on these data. I show that there is a small
      positive effect of mail voting on turnout in national elections at the
      county level. This thesis also contributes to the literature by
      presenting a description of modeling and data wrangling difficulties
      associated with voter registration files, and giving a series of
      potential solutions, as well as an extensive coding library to aid
      future research on the subject.
    \end{abstract}
  
  
  \mainmatter % here the regular arabic numbering starts
  \pagestyle{fancyplain} % turns page numbering back on

  \chapter*{Introduction}\label{introduction}
  \addcontentsline{toc}{chapter}{Introduction}
  
  The democratic system is based on procedures as much as principles. The
  way that democracies choose to tally the will of the people is always a
  messy, controversial process. Thus the design and implementation of
  voting systems is far from being neutral; the decisions made on who
  votes, and how, when, and where they do so is inherently coupled with
  the outcome. Underlying those decisions is a nebulous, inconclusively
  answered question: are elections fair, and how can we make them more so?
  
  The passage of the Help America Vote Act---or HAVA--(Robert Nay, 2002)
  innovated the way we use data based approaches to answer this question,
  by mandating states to update and consolidate public voter registration
  files, and creating the US Elections Assistance Commission that makes
  county level data available. Public voter files were initially used by
  private corporations like TargetSmart or Catalyst, which cleaned up the
  files for use by political campaigns trying to tailor their message as
  closely as possible to individual voters. Researchers quickly realized
  the massive potential that such data has, and started partnering with
  such firms or conducting independent cleaning and structuring of such
  data themselves. These data provided information at the individual
  level, with geocoding usually at the precinct level. This in turn meant
  that, though very complex as a task, it became possible to use such
  files along with census data at the block level to estimate individual
  characteristics like race, education, or income at incredible levels of
  accuracy. Even without the resources of a multi-million dollar private
  market, such methods along with voter files alowed researchers to make
  concrete inferences of individual characteristics (E. D. Hersh, 2015).
  Voting related theories derived from political science are now commonly
  tested using advanced statistical methods and huge amounts of data; both
  disciplines tackle these data to face joint problems such as quantifying
  the quality of voter registration files (Ansolabehere \& Hersh, 2010),
  or linking disparate voter records (Ansolabehere \& Hersh, 2017).
  
  \chapter{The State of the Literature}\label{rmd-basics}
  
  \section{Deciding to Vote}\label{deciding-to-vote}
  
  \subsection{Why Turnout Matters}\label{why-turnout-matters}
  
  Turnout is the most commonly used measure for participation. It is
  important because it signifies the level of engagement of the population
  with the state, the level of incorporation of different subgroups of the
  population into democratic processes, and the legitimacy of elected
  officials. Turnout is a metric for how widespread democratic
  participation is; it is one of the best quantitative measures of the
  strength of the democratic franchise, alongside qualitative metrics such
  as voter education and information. Turnout for an election can be
  calculated or predicted, the difference being that in the former case we
  use data post-election that is reflective of the actual number of
  voters, while in the latter we use a series of individual and community
  covariates to infer the levels of turnout.
  
  To calculate turnout, we simply divide the number of ballots cast by the
  potential voting population, as in the following equation:\\
  \[ \% ~Turnout = \frac{Total~Ballots~Cast}{Measure~of~Total~Voting~Population}\times100\%\]
  
  The choice of numerator is fairly obvious and universal; the
  denominator, however, is a different story. The three main statistics
  used are the total voting age population, voting eligible population,
  and the number of registered voters in a certain geographical location.
  The total voting age population (all individuals over 18 years of age)
  can be measured using data from the US Census. However, such an
  interpretation of voting age population positively counts individuals of
  age that are not allowed to vote, like people with severe mental
  illnesses or felons, and does not count oversees voters or military
  personnel. Michael McDonald offers an alternative to voting age
  population he calls ``voting eligible population'', which corrects for
  such individuals (McDonald, n.d.).
  
  Counts of registered voters are also a useful tool for calculation of
  turnout, as they usually require no estimation. These counts can simply
  be extracted from voter registration files. Using registered voters,
  however, also brings with it two problems. First, voter registration
  files many times can include discrepancies like deceased voters, voters
  included in multiple counties, or individual voters included multiple
  times. Furthermore, the total amount of actual voters among registered
  voters can be misrepresentative of democratic participation; consider
  that if a certain minority community has historically low registration
  rates they are not included at all in calculations of the turnout
  statistic.
  
  The punch line here is that how the turnout statistic is calculated is
  not a clear choice, and will have an impact on how studies are set up.
  To give one example, consider Oregon's Motor Voter program, that
  automatically registers voters when they interact with the DMV. It is
  conceivable that this reform will \emph{decrease} turnout when measured
  as a percentage of the total registered voter count, but \emph{increase}
  turnout when measured against total population. This happens if more
  people register to vote, but do not actually do so--in other words, both
  number of registrants and number of ballots cast are increasing, but the
  former increases at a larger rate than the latter. I will specify how I
  calculate turnout in the next chapter.
  
  Statistical models of turnout can be constructed at either the
  individual or community level. At the individual level, a model is built
  to predict the probability of voting for every member of a group, and
  then sum over the members to create an estimate for turnout. Probit or
  Logit models are preferred. At the community level, researchers first
  choose a geographical level at which to calculate, which then
  constitutes the individual observation in the data that is used to
  create the model.
  
  Both these models include a standard set of societal variables at the
  individual and aggregate level, policy variables (whether the district
  does Postal Voting, whether Voter ID requirements are particularly
  strict), election-specific variables (closeness of election or campaign
  expenditure) and sometimes time-series data, like previous levels of
  turnout. This type of analysis is not exclusively used to predict
  turnout but also to, as will be later shown, draw inferences on the
  effects that certain explanatory variables have on electoral
  participation.
  
  Through meta-analyses on studies of turnout, it is possible to get a
  clear picture on what variables effect individual and collective choices
  to turn out. Three such studies are conducted by Geys (2006), Geys and
  Cancela (2016), and Smets (2013). Geys includes 83 studies of national
  US elections in his initial meta-analysis (Geys, 2006), later increasing
  that number to 185 (Geys and Cancela, 2016) and adding local elections.
  On aggregate-level models for national elections they conclude that
  competitiveness, campaign financing, and registration policy have the
  most pronounced effects, while on the sub-national level there are more
  pronounced effects for societal variables and characteristics of
  election administration (spending, voting policy, etc.). Smets and Van
  Ham (2013) examine individual-level predictors for turnout in a similar
  meta-analysis, and conclude that ``age and age squared, education,
  residential mobility, region, media exposure, mobilization (partisan and
  nonpartisan), vote in previous election, party identification, political
  interest, and political knowledge'' (Smets \& Ham, 2013) are the most
  significant explanatory variables for turnout, along with income and
  race. I will specify the model I will use for turnout in the second
  chapter.
  
  \subsection{Theories of Voting}\label{theories-of-voting}
  
  Here I take one step back from turnout, and examine the theories
  surrounding individual choices to vote or abstain. There are three main
  theories outlined in the literature on why individuals chose to vote.
  While there is some overlap, the following are mostly distinct:
  
  \begin{itemize}
  \item
    \emph{Decision ``at the margins''}: In his 1993 study, Aldrich posits
    that voting is a low cost-low benefit behavior. Therefore, he
    continues, voting is a decision that individuals make ``at the
    margins''; in most people, the urge to vote is not overwhelmingly
    strong, and therefore individuals will vote when it is convenient to
    them, when they are motivated by a competitive race, when policies are
    put in place to help them, and when they are subjected to GOTV (Get
    Out the Vote) efforts. For Aldrich, this is corroborated by the fact
    that most turnout models present consistent, yet weak, relational
    variables; if decisions are made ``at the margins'', then no single
    predictor would have an overwhelming result. This is also supported by
    Matsusaka (1997), and Burden \& Neiheisel (2012). Matsusaka expresses
    support for a more ``random'' process of voting, where turnout models
    are ambiguous because of the difficulty that predicting ``at the
    margins'' entails (Matsusaka \& Palda, 1999). Burden \& Neiheisel
    (2013) also demonstrate support for Aldrich's thesis by using data
    from Wisconsin to calculate a net \emph{negative} effect of 2\% on
    turnout following the expansion of early voting access in the state.
    (Aldrich, 1993; Neiheisel \& Burden, 2012)
  \item
    \emph{Habitual Voting}: While Aldrich supports that there is no single
    overwhelming predictor of turnout, Fowler (2006) posits that future
    voting behavior can be strongly predicted using individual voting
    history. This leads to the conclusion that individuals are set to
    either be habitual voters, or habitual non-voters (Plutzer, 2002) by
    their upbringing and social circumstances, locking them into distinct
    groups. (Fowler, 2006)
  \item
    \emph{Social/Structural Voting}: Close to habitual voting are those
    that support a model of social and structural voting; these
    researchers claim that the decision to vote or not is deeply rooted in
    socioeconomic factors, which means that the divide between
    traditionally voting and non-voting groups can only be bridged by
    directly dealing with the socioeconomic divide between them (Berinsky,
    2005; Edlin, Gelman, \& Kaplan, 2007 ). Their reasoning is that ``at
    the margins'' voting only addresses groups that do not face
    significant burdens against voting (like the working poor, or
    marginalized racial groups) and are usually already registered.
    Similarly, they address habitual voting claims by arguing that they
    are too short-sighted; individuals themselves might be habitually
    voting, but their decision to do so is rooted in strong societal and
    policy factors.
  \item
    \emph{Resources and Organization}: To some extent growing from
    structural theories of voting, resources and organizations theory
    emphasizes the interaction of personal political and societal
    characteristics of voters, and actions taken by politicians to
    mobilize participation. This theory is very broad in the inputs it
    assesses for voter participation, ranging from practical issues of
    access and resources (how easy it is for someone to vote if they so
    choose), to public policy feedback effects and signaling (how the
    government's policies affect the people and how they react), to how
    political parties and groups choose to mobilize and approach voters
    (Rosenstone, 2003). Apart from Rosenstone and Hansen's work (2013),
    there have been several studies examining voter participation based on
    resources and organizations theory, a lot of which come from the
    public policy side of political science. Some examples are Chen's
    study of how distributive benefits like federal emergency aid affect
    participation among recipients, after controlling for partisan
    characteristics (2012), or Mettler and Stonecash's examination of
    correlation between welfare program participation and political
    mobilization (2008), or Campbell's analysis of social security
    recipients and their voting patterns (2002). The punchline in all
    these studies is that public policy is correlated with trends in
    participation, either because recipients of benefits wish to protect
    such programs, or because of the interaction between partisanship and
    government support, or because of access related to resources and
    voting laws (Campbell, 2002; Chen, 2013; Mettler \& Stonecash, 2008).
  \end{itemize}
  
  \section{From Theory to Policy}\label{from-theory-to-policy}
  
  \subsection{Voting Methods}\label{voting-methods}
  
  I have already flagged in my introduction the reason why theories behind
  voting choice matter: each construct an image of the electorate that
  reacts differently to policy change around voting. They are all an
  answer to the fact that elections policy, and how we conduct elections,
  is not value neutral but has implications for turnout, which in turn has
  implications on the franchise of democracy.
  
  In trying to respond to the issues set up by theoretical paradigms,
  different states--both in the global and US contexts--have adapted to
  different ways of conducting elections. In the US, voting styles can be
  simplified into three categories:
  
  \begin{itemize}
  \item
    \emph{In-Person Election Day}, for which all individuals are required
    to vote at a polling place, on a single election day. There can be
    some leeway for overseas voters, or excused absentee voters, but the
    vast majority of people will have to be present to vote in a
    particular time frame.
  \item
    \emph{In-Person Early Voting}, for which all individuals must vote in
    person at a polling place or vote center, but the timeframe for voting
    extends for around two weeks, not a single day.
  \item
    \emph{Vote-By-Mail, Absentee Early Voting}, for which individuals have
    a clear, no-excuse-necessary option for not being present when they
    vote, or for filling in a mailed ballot and dropping it off at
    designated locations.
  \end{itemize}
  
  For the purposes of this thesis I will examine the latter category, and
  specifically Vote-By-Mail. The reason behind this is that the model of
  in-person, election day voting is usually seen as the baseline, the
  ``vanilla'' way of conducting elections if you will. Therefore it has
  been of interest for researchers to examine if other systems can
  outperform that baseline. Specifically, it is most interesting to
  examine voting styles that are heralded for their expansion of turnout,
  to see whether popular beliefs on their benefits and drawbacks hold; if
  they are different from the base model of conducting American elections,
  or if they present new challenges and unique selling points.
  Vote-By-Mail is particularly interesting because it is quickly taking
  the form of a trend in state elections, as more and more states are
  enforcing more open models of VBM. In the next section, I will more
  closely examine the particulars of Vote-By-Mail.
  
  \subsection{What is VBM?}\label{what-is-vbm}
  
  Vote-By-Mail is a process by which voters receive a ballot delivered by
  mail to their homes. Voters then have a variety of options on how to
  return these ballots, ranging from dropping them off at pre-designated
  locations, to mailing them in, to bringing them to a polling place. The
  two first options are most commonly implemented, with a very small
  number of states still operating polling places for mail ballots. This
  varies across states that have implemented VBM. Some common forms of the
  VBM policy are:
  
  \begin{itemize}
  \item
    \emph{Postal Voting}: All voters receive a ballot by mail, which can
    then be returned to a pre-designated location or mailed in to be
    counted. All-mail elections currently occur in Oregon, Washington, and
    Colorado.
  \item
    \emph{No-Excuse Absentee}: Voters can choose to register as absentee
    voters without giving any reason related to disability, health,
    distance to polling place etc. This is the case in 27 states and the
    District of Columbia.
  \item
    \emph{Permanent No-Excuse Absentee}: This is similar to the previous
    system, but allows voters to register as absentees indefinitely,
    without having to renew their registration each year; they become de
    facto all-mail voters. This is in place in a very large number of the
    no-excuse absentee voting states like Utah, California, Montana,
    Arizona, New Jersey, and others.
  \item
    \emph{Hybrid or Transitionary Systems}: In hybrid systems, voters
    receive a mail ballot but can choose to disregard it and vote
    conventionally. This is the case in Colorado. Transitional systems
    exist in states that have chosen to eventually conduct all elections
    by postal voting, but have given counties an adjustment period during
    which this shift is not mandatory, or mandatory only for certain
    elections. This is the case in California, Utah, and Montana.
  \end{itemize}
  
  Vote-By-Mail is also commonly considered a type of early voting, since
  voters receive their ballots around two weeks in advance of election
  day; they are also able to return that ballot whenever they wish within
  that time-frame. This means that Vote-By-Mail can be counted as a
  ``convenience voting'' reform (Gronke et al., 2008). These are usually
  implemented by state and local governments with the argument that they
  either expand the democratic franchise by bringing in new voters, or by
  making it more likely that current registered voters participate in the
  electoral process (State Legislatures, n.d.).
  
  \subsection{How Theories Apply to VBM}\label{how-theories-apply-to-vbm}
  
  Under Aldrich's paradigm, vote by mail would not effect significant
  change in voting behavior. The whole concept of a decision ``at the
  margins'' is that the forces at play when an individual decides to vote
  are overwhelmingly strong both ways, so any effect that policy can have
  will minimaly shift these margins. If, for example, we take a
  presidential election the forces at play include the media, national
  committees, social effects etc. In this environment, some added
  convenience does not significantly add to an individual's decision to
  turn out. However, this would indicate that at a local level, where
  national and media effects are less strong, the effect of VBM on turnout
  might be more significant. The effect would be present for all groups,
  not only those currently registered, since voting would be easier
  uniformly.
  
  If we asume habitual voting, the conclusion on VBM would differ
  significantly. In this case, the effect to be considered is how VBM
  impacts already formed habits around voting. It could be argued that VBM
  has no effect, which follows if we assume that voting habits formed do
  not shift if the mode of voting changes. It could also be argued that
  VBM might have a negative effect on turnout in the short term, because
  it disrupts the habit of election day for a readjustment period, before
  people settle into new groups of habitual voters and non-voters, adapted
  to the new policy context.
  
  Under social and structural voting contexts, VBM retains rather than
  stimulates new voters (Berinsky, 2005). This means that already
  registered and semi-active voters are more likely to participate, but
  there is no significant change in the amount of new voters entering the
  franchise. This would mean that traditional forms of voting policy that
  emphasize access to the polls will do nothing to bring in
  disenfranchised people, and potentially hide the problem under an
  inflated turnout statistic calculated on registered voters. Berinsky in
  particular emphasizes the need for a shift towards voter education,
  rather than early voting or VBM policies (Berinsky, 2005).
  
  Vote-by-Mail is obviously not a welfare or spending program, but it does
  expand individual resources in terms of voting capacity. A ballot
  delivered to your home means that less resources need to be expended in
  the act of voting, which in turn has both a practical effect--building
  capacity--and a more behavioral effect--a feeling of inclusion, an
  interaction with the process of voting that comes through a ballot at
  your doorstep that would not exist if you had not gone to a polling
  place (Schneider \& Ingram, 1990). Under a resources and organizations
  framework, both these effects are most likely to be net positive to
  political participation, and as such would predict a strong, positive
  effect of VBM on turnout.
  
  \subsection{General Results on VBM}\label{general-results-on-vbm}
  
  I will start with studies that show a negative effect on turnout.
  Bergman (2011) uses a series of logit models of individual voting
  probability in California, during a period where part of the state
  conducted VBM elections, while others maintained traditional voting.
  This is called a ``quasi-experiment'', and it is common throughout the
  literature. Bergman's results show a statistically significant drop in
  voting probability in VBM counties (Bergman \& Yates, 2011). Using a
  similar method, Keele (2018) takes a single city in Colorado, Basalt
  City, which is divided into two different voting districts using
  different voting systems. The conclusion is, again, a 2-4\% drop in
  turnout along the VBM part of the city (Keele \& Titiunik, 2017). Burden
  et al. (2014) takes a different approach, using country-wide election
  data from 2004 and 2008 presidential elections, and compares districts
  based on early voting practices. Their results show a significant drop
  in turnout, which can be associated to VBM as well due to its closeness
  to EV (Burden, Canon, Mayer, \& Moynihan, 2014).
  
  In contrast, Gerber et al. (2013), applying both individual and
  county-level models for the state of Washington, reach the conclusion
  that VBM increases turnout by around 2-4\%; they use the same
  quasi-experimental model that offers itself to researchers in states
  that are under transitional systems (Gerber, Huber, \& Hill, 2013). R.M.
  Stein also reaches a similar conclusion when examining Colorado's
  practice of ``vote centers'', which are non-precinct attached polling
  places, which can service multiple counties (Stein \& Vonnahme, 2008). I
  include this paper here due to the link that voting centers have with
  VBM, as they serve as drop-off points for mail-in ballots. Richey (2008)
  examines the effects that Oregon's VBM program has on turnout by using
  past elections data, concluding a 10\% positive trend associated with
  the policy (Richey Sean, 2008). This effect is studied again by Gronke
  et al.(2012) who find a similar positive effect with much lower
  magnitude, which might point to a novelty effect: the existence of
  diminishing returns in turnout after the implementation of this policy
  (Gronke \& Miller, 2012). Gronke et al. (2017), again studying Oregon
  but focusing on Oregon's Motor Voter program, find evidence of positive
  association to turnout. I include these effects due to Oregon being an
  exclusively VBM state, and because this paper uses a ``synthetic control
  group'' model, a particularly interesting statistical technique. Lastly,
  I include a study conducted by Pantheon Analytics on Colorado, which
  compares actual turnout to predicted levels for VBM counties in
  Colorado. The results show a positive effect of approximately 3.3\% due
  to VBM (Edelman \& Glastris, 2018).
  
  The conclusion to be drawn from this section is that results on VBM vary
  significantly. There are multiple studies, using multiple methods, on
  multiple states, with multiple results. This only adds to the importance
  of being careful when constructing models and hypotheses to test VBM's
  effects on turnout, as assumptions made in the process can critically
  impact the results.
  
  \section{Voter Registration Files as Data
  Sources}\label{voter-registration-files-as-data-sources}
  
  Before concluding this chapter, I want to briefly discuss some
  background research into the use of voter registration files for the
  purpose of elections research. This may seem like an abrupt shift from
  the previous section but, as I mentioned in my introduction, access to
  such files is what motivated and facilitated a lot of the aforementioned
  studies in the first place. I will not directly go into the structure of
  such files; such a section will be included later on in this thesis.
  
  \subsection{Inaccuracy of Survey Data}\label{inaccuracy-of-survey-data}
  
  Apart from Voter Registration Files, the main source of data on the
  American electorate is national surveys, like the American National
  Election Studies' survey (ANES), or the Cooperative Congressional
  Elections Study (CCES). These are post-election surveys, distributed to
  voters, which include fields associated directly with
  voting--participation, precinct, which party you voted for--and
  indirectly, through questions on individual characteristics like race,
  income, or gender. On the surface these seem like a better source of
  data, since no record linkage or ecological inference need be made to
  connect individual voters with an extensive list of covariates. There
  is, however, a significant problem with these data: survey misreporting
  (Burden, 2000).
  
  A cursory glance at the CCES and ANES estimates of turnout reveals the
  existence of a problem right off the bat: turnout calculated through
  surveys is usually higher than reported figures. When looking at surveys
  a bit closer, using either private, extensive data files like Catalyst
  (Ansolabehere \& Hersh, 2012) or validated voter files from the late
  20th century (Deufel \& Kedar, 2010), the results show consistent
  misreporting among certain groups that tend to either be politically
  engaged non-voters or minorities and low socioeconomic status
  individuals. This gap, according to Deufel et al. (2010), has served to
  propagate societal stereotypes and class entrenchment into studies on
  turnout, which in turn negatively effect policy, since research using
  the ANES and CCES are widely used to study turnout among the groups that
  are consistently misreporting. Admittedly, the fact that misreporting
  happens among specific groups does open the way for statistical methods
  to compensate for the bias introduced, but for the purpose of my thesis
  I will prefer the use of Voter Registration Files.
  
  \subsection{The Importance of VRF}\label{the-importance-of-vrf}
  
  As mentioned in my introduction, access to voter registration files has
  provided researchers with unique insight into the voting process.
  Quantitative research has expanded significantly, for three key reasons.
  First, VRF data exists in a consolidated, state-wide format at least for
  national elections. This means that the process of data collection
  involves interaction with significantly fewer government agencies, and a
  data wrangling process that can be quickly adapted to a set format. This
  is, of course, not to say that the process of data collection and
  handling doesn't still pose a significant challenge, as will become
  apparent in my second chapter. Second, there is a huge benefit attached
  to the fact that VRF data describes the whole population, rather than a
  sample. As mentioned in the previous section, survey data might give
  more insight into variables not included in VRF, but that comes at a
  steep cost for accuracy. Using VRF, the problem of self-reporting bias
  is eliminated for some studies, and transformed into a problem of record
  linkage and ecological inference for others (Ansolabehere \& Hersh,
  2017, Burden \& Kimball (1998)). Third, wide public access means
  reproducible and accessibility, which translates into greater
  accountability for researchers. This effect is important, even if
  mitigated somewhat by private data companies and access fees.
  
  \chapter{Hypotheses and Methods}\label{hypotheses-and-methods}
  
  \section{Hypotheses}\label{hypotheses}
  
  \subsection{Questions}\label{questions}
  
  Before moving in to outlining hypotheses, the first step necessary is to
  frame a series of questions, which the hypotheses will flow from. Based
  on relevant research, the most obvious first question to ask would be:
  
  \begin{quotation}
  Q1: \textit{What is the effect of mail voting on turnout?}
  \end{quotation}
  
  I went through this question substantially in the previous chapter; it
  should be clear that depending on which paradigm of participation choice
  is present, the answer here can be radically different. In order to best
  answer the previous question, it is necessary to establish some
  conditions on importance of effect. Therefore it is also necessary to
  ask the following question:
  
  \begin{quotation}   
  Q2: \textit{Does this effect persist when accounting for other relevant predictors of turnout?}
  \end{quotation}
  
  The last question asked in this thesis is more specific to a particular
  formulation of Aldrich's hypothesis on voting ``at the margins''. I
  mentioned in the previous section that VBM could be theorized to have a
  more significant effect when discussing elections at the local level, or
  the regional level, rather than national general elections. Therefore a
  third question is:
  
  \begin{quotation}   
  Q3: \textit{Is  the  effect  of  VBM on turnout more  pronounced as significant, national determinants become less strong?}
  \end{quotation}
  
  \subsection{Hypotheses}\label{hypotheses-1}
  
  Using the above questions I can now move on to formulate more clear
  hypotheses. The hypotheses in this section are meant to test theories of
  voter choice from the perspective of the theory of voting ``at the
  margins'' as introduced by Aldrich.
  
  In response to Q1, Q2, a first hypothesis is:
  
  \begin{quotation}  
  H1: \textit{Mail voting is another incremental effect on voting decisions, and therefore
  does not significantly affect turnout}
  \end{quotation}
  
  The alternative hypothesis would be:
  
  \begin{quotation}  
  H1$'$: \textit{Mail  voting  significantly  affects  turnout,  even  compared  to  other  metrics}
  \end{quotation}
  
  Similarly, for the third question, a corresponding hypothesis derived
  from Aldrich's paradigm is:
  
  \begin{quotation}  
  H2: \textit{The  effect  of  VBM  on  turnout  is  larger  as  national  effects  become less pronounced}
  \end{quotation}
  
  The alternative hypothesis is:
  
  \begin{quotation}  
  H2$'$: \textit{The  effect  of  VBM  on  turnout  is either independent of or proportional to how pronounced national effects are}
  \end{quotation}
  
  \subsection{Criteria}\label{criteria}
  
  A first, glaring issue that needs to be clarified is the apparent
  contradictions between my two hypothesized results. This becomes clear,
  however, if I define ``significant effect'' in the context of my first
  hypothesis. Aldrich's paradigm does state that ``conveniences'' like
  mail voting should not have significant effects, but those effects are
  defined in the context of huge, clashing forces that vastly outweigh
  them. This does not necessarily mean that they are literally
  non-existent, but that they are poor indicators of consistently
  increased turnout. Therefore, I will confirm my first hypothesis not
  only if the effect of mail voting on turnout is vanishingly small, but
  also if it is relatively small in comparison to the effects of other
  variables I include. I will confirm the alternative hypothesis if,
  across multiple of the models I will parametrize and fit, VBM retains a
  consistent, significant effect on turnout. If the effect is negative,
  this may point to a habitual or structural voting paradigm being
  present. If the effect is positive, this may be a signifier that issues
  of convenience in voting--having a mail delivered ballot, voting from
  your kitchen table etc.--have a particularly strong effect in the
  examined elections.
  
  Moving on to the second hypothesis. It is extremely hard to correctly
  operationalize and account for all variables going into turnout.
  Therefore, instead of trying to include all possibly relevant effects
  into a model and try to see how they interact with VBM, I will test my
  hypothesis on different levels of elections: local, midterm,
  presidential, and primary. National effects on turnout should be
  especially present in presidential races, since a specific set of
  candidates is running across the whole nation. These effects should also
  be present in midterm and primary elections to some extent, as the
  results of local races are aggregated in control of congress or
  high-profile governorship. Apart from the off-chance of a nationally
  reported-on ballot measure, or a singularly high-profile race, local
  off-year election turnout should have a negligible relation to national
  effects. Therefore I will use election level as a stand-in for the
  prominence of national turnout effects. A potential re-formulation of
  the second hypothesis that makes it more specific to the criteria I have
  set is:
  
  \begin{center}    
  H3: \textit{The  effect  of  VBM  on  turnout  is  more  pronounced  in  local  or  off year elections}
  \end{center}
  
  I will confirm this hypothesis if mail voting has substantially larger
  positive effects on turnout in smaller, local elections.
  
  \subsection{Importance of Hypotheses}\label{importance-of-hypotheses}
  
  The importance of these hypotheses is intrinsically tied to the
  importance of different theories of electoral participation. Confirming
  or rejecting each hypothesis--even when only applied to a single
  state--serves as an argument for or against one of the aforementioned
  theories. The theories in and of themselves are important, since they
  form a part of a broader literature on elections, democracy, and
  electoral processes, that can be said to be foundational to political
  science as a whole. Elections are the root from which all democratic
  governing springs; understanding why people participate in them is
  understanding how they choose to be included or excluded from the
  process of policy-building, and how they interact with the state.
  
  Additionally, from a public policy perspective, these hypotheses are
  significant since they serve as metrics for the effectiveness of mail
  voting as an electoral reform. Whether, in general, mail voting
  increases turnout is directly connected to whether it is successful in
  expanding the democratic franchise. If it is not, questions can be
  raised as to the effectiveness of expanding voter access through
  elections administration, rather than education, or even measures like
  voting-day-holidays or local transportation to poling places. In local
  elections in particular, significant effects of mail voting could be
  precursors to more general involvement of individuals in their local
  politics. This may open the way to numerous comparative studies on local
  politics between states that apply VBM and states that do not.
  
  Lastly, from a narrower perspective specific to the study of early and
  mail voting, my first hypothesis can still be said to be quite
  important, yet mundane. It does its job according to the particular
  state I chose to look at--in this case Colorado--to add to existing
  literature on mail voting effects in different parts of the country.
  However, my second and third hypotheses are much more unique in their
  scope. There have not been many studies that look at VBM at a more
  localized level, and any addition to the literature on this
  front--however limited--could be significant.
  
  \section{Methodology}\label{methodology}
  
  Before directly defining all parameters of the models I will later use
  in writing this thesis, I will go through each type of method to provide
  some background on the statistics behind the models. In the next
  chapter, I will introduce the data and fully outline my models. This
  section should serve as a general introduction to the methods. I will
  not extensively go through the statistics behind linear or multiple
  regression, but will assume that it is common knowledge. For an
  extensive introduction to such methods, James et al.(2017) or Chihara
  and Hesterberg (2011) are particularly useful.
  
  \subsection{Logistic Regression}\label{logistic-regression}
  
  Let function \(f : [0, 1] \to \mathbb{R}\) be defined as:
  \[f(p) = \text{logit}(p) = \text{log}\left( \frac{p}{1-p} \right)\] This
  is called the logit function or, when \(p\) refers to a probability, the
  log-odds function. When modelling a binary response Y, which follows a
  Bernoulli distribution: \[Y \sim \text{Bernoulli}(p),\] the logit
  function can be used as a link function to model Y in a generalized
  linear model. The generic form of a generalized linear model looks
  like:\\
  \[f(Y) = X\beta ,\] where Y is a vector of response variable values, X
  is a matrix of predictors, and B is a vector of coefficients to be
  estimated. The function \(f\) is called a link function, because it
  ``links'' the response variable with the set of predictors included in
  the model. This is typically done to ensure that the range of values
  outputted by the model are consistent with the range of the response
  variable.\footnote{Or, in this case, the range of the parameter defining
    the distribution of the response, which is p for the Bernoulli
    distribution} When wanting to compute a model on a binary response
  through its corresponding Bernoulli distribution probability parameter,
  the inverse logit function should be a perfect fit for a link function,
  since it maps values from all real numbers to a range between 0 and 1.
  Using the inverse logit function, we arrive at the final form of
  logistic regression, which is:\\
  \[\mathbb{P} (Y = 1) = \text{logit}^{-1} (XB)\]
  
  Conveniently, despite the use of a link function, there is an easy way
  to interpret the coefficients of such a regression. While obviously
  individual values from the \(B\) vector will not be particularly
  helpful, \(e^B\) can be used as a vector of multiplicative, one-unit
  shifts in the value of the probability that \(Y = 1\). This means that a
  one unit increase in any predictor will cause an effect equal to
  multiplying p by the exponent of the corresponding coefficient\footnote{This
    can be simplified even more, if one approximates the process of
    exponentiation with just dividing the coefficient by 4. Crude, yet
    effective for a quick scan of the results}. (James, Witten, Hastie, \&
  Tibshirani, 2017)
  
  \subsection{Generalized Additive
  Models}\label{generalized-additive-models}
  
  In simple logistic or linear regression, there is an assumption made on
  the functional form of the relationship between predictors and response
  variable. These are called parametric models, where the data is
  exclusively used to estimate values for coefficients. Non-parametric
  models, on the other hand, use the data to estimate both coefficients
  and the function that serves to connect response to predictors. While on
  the surface this seems like a great idea (more reliance on your data and
  fewer assumptions!), such an exclusively non-parametric model would
  suffer greatly from the curse of dimensionality--where the addition of
  multiple predictors or over-reliance on data leads to substantial
  over-fitting.
  
  One solution is the Generalized Additive Model, or GAM. This model lets
  us fit a different functional form to each predictor, allowing for
  assumptions to be made on the data where it is safe to do so, and for
  non-parametric fitting when it is necessary. This model looks like:
  
  \[y_i = \alpha + \sum_{j = 1}^p \beta_j f_j(x_{ij}), ~~j \in \{1,2,...p\}, i \in \{1,2,...n\} \]
  
  where \(y_i\) the i-th response variable, \(\alpha\) is the intercept
  term, \(f_j, \beta_j\) a series of \(p\) functions and coefficients, and
  \(x_{ij}\) the i-th observation for the j-th predictor. Note that for
  \(f(x_{ij}) = x_{ij}\), this is a multilinear regression! (James et al.,
  2017)
  
  A type of most commonly fit functions--and the type I will make use
  of--are smoothing splines. These are functions connected at specific
  points called ``knots'', with the limitation that the full function must
  be continuous and smooth, and have a continuous first and second
  derivative. Between knots, different functional forms are fit to the
  data, within some constraints; they may, for example, all have to be
  cubic polynomials. These are particularly useful when modeling time
  variables, as they can be fitted to variables like years or months in
  order to distinguish a secular trend from a general trend over time
  (Barr, Diez, Wang, Dominici, \& Samet, 2012). In terms of this thesis,
  this will help when responding to Q2 as it was framed earlier in this
  chapter.
  
  \subsection{Multilevel Models}\label{multilevel-models}
  
  Multilevel models--otherwise known as hierarchical or ``mixed effects''
  models--can be intuitively pictured in two ways: either as a set of
  models working on different ``levels'', where one is calculated first,
  with its effects having implications for the second, or as a model where
  some of the parameters are estimated under a particular series of
  constraints. Multilevel models are, in essence, a compromise between
  levels of ``pooling'' data. If the dataset on which parameters are being
  estimate operates in different units of observation--say on the
  individual and county level--you could run a model that treats all
  individuals as coming from the same larger group; this would be a
  complete pooling model. You could also add indicator variables for each
  and every group, de facto estimating \(n\) different models for \(n\)
  groups; this would be a no pooling model. Multilevel modelling offers
  partial pooling (Gelman \& Hill, 2006).
  
  To consider what this model looks like, let's assume a dataset
  comprising of a vector of values for the response variable \(Y\), a
  matrix of \(i\) individual level predictors \(X\), a matrix of \(j\)
  group level predictors \(U\), intercept terms \(\alpha\), individual
  level coefficients \(B\), and group level coefficients \(\Gamma\). Based
  on this, a multilevel model with intercept terms varying by group looks
  like:
  
  \[Y_i = \alpha_{[i], j} + X_iB~,~~~~\alpha_{[i], j} \sim N(U_{j[i]}\Gamma, \sigma_{\alpha}^2)\]
  
  Multilevel models can be fit using the \texttt{lme4} \textit{R} package
  that uses restricted maximum likelihood calculations for estimating
  coefficients (Bates, Mächler, Bolker, \& Walker, 2014). Multilevel
  modelling also works perfectly well with general additive models! In
  \textit{R} this can be accomplished with the \texttt{gamm4} package (S.
  Wood \& Scheipl, 2017).
  
  \subsection{Model Accurracy and Quality of
  Fit}\label{model-accurracy-and-quality-of-fit}
  
  \subsubsection{Mean Squared Error (MSE)}\label{mean-squared-error-mse}
  
  For all generalized linear regression models (including GAMs, mixed and
  fixed effects models) I use Mean Squared Error to asses the accuracy of
  the fit. Assuming a dataset
  \(\{(y_0, x_0^1, x_0^2, ..., x_0^m),...,(y_n, x_n^1, x_n^2, ..., x_n^m)\}\)
  of n observations and m predictors, with \(X_i\) a vector of the
  predictors for the i-th observation, and \(f:R^m \to R\) the true
  multivariate function connecting the predictors and response, mean
  squared error is calculated as follows:
  \[\text{MSE} = \frac{1}{n}\sum_{i=1}^{n}(y_i - \hat{f}(X_i))^2\]
  
  MSE can be calculated either using the same dataset used in estimating
  the model coefficients, or on a new dataset. In the later case it is
  called predictive or test MSE. Despite prediction not being the purpose
  of the models presented in this thesis, I use test MSE because of the
  independence such a calculation method brings from the data used for the
  fit, compensating in a way for over-fitting(James et al., 2017). To
  calculate test MSE I use five-fold cross-validation, which will be
  analyzed shortly.
  
  \subsubsection{Area Under the Curve
  (AUC)}\label{area-under-the-curve-auc}
  
  Logistic regression models estimate the probability of a binary variable
  taking a TRUE value. The predictive output of such a model will be a
  series of probabilities. These probabilities can then be used to
  approximate a dataset of positive and negative values for the response
  variable (in my case, voting). Based on the true values of the response,
  one can calculate the counts of true positive, true negative, false
  positive, and false negative predictions. To make this calculation, a
  probability threshold is set over which the prediction for the response
  is positive. Positive predictive values of the response are assigned
  based on the following statement:
  
  \[\text{P}(y_i = 1|X_i) > p\]
  
  where \(y_i, X_i\) can be assumed to be the same as in the previous
  section, and \(p\) is the threshold. A common and intuitive threshold is
  \(0.5\), but any number in \((0,1)\) can be used. After getting counts
  for true/false negative/positive values, one can then calculate
  \emph{specificity} and \emph{sensitivity} for the model. These are:
  
  \[\text{specificity} = \frac{\text{True Positive}}{\text{False Negative + True Positive}}\]
  
  \[\text{sensitivity} = \frac{\text{True Negative}}{\text{False Positive + True Negative}}\]
  
  Using these three metrics--sensitivity, specificity, and probability
  threshold--it's possible to create an ROC curve, which is one of the
  most widely used diagnostic plots for classification models\footnote{The
    ROC curve takes its name from a term in communications science, the
    \emph{receiver operating characteristics curve}. The name is historic,
    and not relevant to its statistical application.}. The ROC curve has
  \(1-\text{specificity}\) on the x-axis, \(\text{sensitivity}\) on the
  y-axis, and each point describes a pair of x-y values for each value of
  the probability threshold. Using this plot, it's possible to measure the
  \emph{area under the (ROC) curve}, or AUC, which serves as a
  goodness-of-fit measure for classification models. The AUC is a number
  in the \([0,1]\) range and should be maximized; a \(.5\) AUC is
  representative of an ROC curve on the \(y = x\) line, which is a
  coin-toss no-information classifier(James et al., 2017). Plot 2.1 is an
  example of an ROC curve.
  
  \begin{figure}
  
  {\centering \includegraphics[width=0.5\linewidth]{/Users/tdounias/Desktop/Reed_Senior_Thesis/plots/roc_example} 
  
  }
  
  \caption[Example of an ROC curve]{Example of an ROC curve}\label{fig:roc example}
  \end{figure}
  
  Similarly to MSE, there is value in calculating AUC from a test dataset,
  rather than the dataset used to train the model. Therefore I also use
  5-fold cross-validation for AUC as well\footnote{This also compensates
    for models not converging, as some of mine do.}.
  
  \subsubsection{k-Folds Cross Validation}\label{k-folds-cross-validation}
  
  The goal of statistical modeling is to approximate the true function
  that links predictors to response. While the final model's coefficients
  should be estimated using as much data as possible, when assessing how
  good a fit that model is there can be better uses of the power that
  large amounts of data give us. k-Folds cross validation allows for
  better approximations of goodness-of-of-fit metrics, by partitioning the
  data into training datasets and test datasets. The fundamental idea is
  that the data is split into k different subsets, which are then
  sequentially used to fit the model and calculate the value of some
  metric(James et al., 2017). The algorithm goes as follows:
  
  \begin{enumerate}
  \item Partition data into k folds
  \item Fit model on all but the i-th fold
  \item Calculate goodness-of-fit metric on the i-th fold
  \item Repeat 2 and 3 for i$\in [0,k]$
  \item Calculate the average of all obtained goodness-of-fit measurements
  \end{enumerate}
  
  I perform 5-fold cross validation to calculate MSE and AUC for all
  models which I can fit in R.
  
  \chapter{Case Selection, Data, Model
  Parametrization}\label{case-selection-data-model-parametrization}
  
  \section{The Centennial State and Its
  Voters}\label{the-centennial-state-and-its-voters}
  
  \subsection{Demographics}\label{demographics}
  
  Colorado--named the Centennial State due to assuming statehood on the
  centennial of the Union--lies in the Southwestern United States, with
  its Western half squarely atop the Rocky Mountains. Based on its
  estimated population of just over 5.5 million, Colorado is the 21st most
  populous state, and ranks 37th in population density. The vast majority
  of that population is gathered in a series of urban areas that comprise
  a North-to-South strip in the middle of the state, containing the
  Denver-Aurora-Lakewood Metro Area, Colorado Springs, Pueblo, and Fort
  Collins. Apart from the Western town of Grand Junction, the rest of the
  population resides in vast rural areas.
  
  \begin{figure}
  
  {\centering \includegraphics[width=0.5\linewidth]{/Users/tdounias/Desktop/Reed_Senior_Thesis/maps/pct_white_county_map} 
  
  }
  
  \caption[White voters per Colorado county]{White voters per Colorado county}\label{fig:white pct map}
  \end{figure}
  
  Colorado is landlocked, and far from any coastal town; in place of
  seaside resorts, Colorado attracts a substantial amount of tourists to
  its mountains every year. Therefore the more mountainous regions have
  developed skiing and mountaineering resorts. They also heavily depend on
  federal money and protection for national parks. These importance of
  these characteristics will become apparent in the following sections.
  
  Colorado has a median age of 34.3 and median household income of
  \$65,685. Colorado's population is mostly white, with a higher minority
  group population density in its Southern regions, as shown in figure
  3.1. (Bureau, 2010) The conclusion here is that Colorado is a relatively
  young, mostly white, and fairly well-off state that is increasingly
  getting more diverse, particularly in the South. These factors are
  important as they serve to associate Colorado with other states; such
  associations are useful for the replication of this study or the
  generalization of my results.
  
  The State Capital is Denver; Colorado is split into 64 Counties, of
  which the most populous are, in no particular order, the following: El
  Paso, Denver, Arapahoe, Jefferson, Adams, Larimer, Boulder, and Douglas.
  These counties comprise 73\% of the total population of Colorado.
  
  \begin{longtable}[]{@{}lccl@{}}
  \caption{Colorado population for largest counties
  \label{tab:pop_table}}\tabularnewline
  \toprule
  \begin{minipage}[b]{0.13\columnwidth}\raggedright\strut
  County\strut
  \end{minipage} & \begin{minipage}[b]{0.21\columnwidth}\centering\strut
  Total Population\strut
  \end{minipage} & \begin{minipage}[b]{0.20\columnwidth}\centering\strut
  CO Population \%\strut
  \end{minipage} & \begin{minipage}[b]{0.34\columnwidth}\raggedright\strut
  Largest Metro Area\strut
  \end{minipage}\tabularnewline
  \midrule
  \endfirsthead
  \toprule
  \begin{minipage}[b]{0.13\columnwidth}\raggedright\strut
  County\strut
  \end{minipage} & \begin{minipage}[b]{0.21\columnwidth}\centering\strut
  Total Population\strut
  \end{minipage} & \begin{minipage}[b]{0.20\columnwidth}\centering\strut
  CO Population \%\strut
  \end{minipage} & \begin{minipage}[b]{0.34\columnwidth}\raggedright\strut
  Largest Metro Area\strut
  \end{minipage}\tabularnewline
  \midrule
  \endhead
  \begin{minipage}[t]{0.13\columnwidth}\raggedright\strut
  Adams\strut
  \end{minipage} & \begin{minipage}[t]{0.21\columnwidth}\centering\strut
  441603\strut
  \end{minipage} & \begin{minipage}[t]{0.20\columnwidth}\centering\strut
  0.08781\strut
  \end{minipage} & \begin{minipage}[t]{0.34\columnwidth}\raggedright\strut
  Denver-Aurora-Lakewood\strut
  \end{minipage}\tabularnewline
  \begin{minipage}[t]{0.13\columnwidth}\raggedright\strut
  Arapahoe\strut
  \end{minipage} & \begin{minipage}[t]{0.21\columnwidth}\centering\strut
  572003\strut
  \end{minipage} & \begin{minipage}[t]{0.20\columnwidth}\centering\strut
  0.1137\strut
  \end{minipage} & \begin{minipage}[t]{0.34\columnwidth}\raggedright\strut
  Denver-Aurora-Lakewood\strut
  \end{minipage}\tabularnewline
  \begin{minipage}[t]{0.13\columnwidth}\raggedright\strut
  Boulder\strut
  \end{minipage} & \begin{minipage}[t]{0.21\columnwidth}\centering\strut
  294567\strut
  \end{minipage} & \begin{minipage}[t]{0.20\columnwidth}\centering\strut
  0.05857\strut
  \end{minipage} & \begin{minipage}[t]{0.34\columnwidth}\raggedright\strut
  Boulder\strut
  \end{minipage}\tabularnewline
  \begin{minipage}[t]{0.13\columnwidth}\raggedright\strut
  Denver\strut
  \end{minipage} & \begin{minipage}[t]{0.21\columnwidth}\centering\strut
  600158\strut
  \end{minipage} & \begin{minipage}[t]{0.20\columnwidth}\centering\strut
  0.1193\strut
  \end{minipage} & \begin{minipage}[t]{0.34\columnwidth}\raggedright\strut
  Denver\strut
  \end{minipage}\tabularnewline
  \begin{minipage}[t]{0.13\columnwidth}\raggedright\strut
  Douglas\strut
  \end{minipage} & \begin{minipage}[t]{0.21\columnwidth}\centering\strut
  285465\strut
  \end{minipage} & \begin{minipage}[t]{0.20\columnwidth}\centering\strut
  0.05676\strut
  \end{minipage} & \begin{minipage}[t]{0.34\columnwidth}\raggedright\strut
  Denver-Aurora-Lakewood\strut
  \end{minipage}\tabularnewline
  \begin{minipage}[t]{0.13\columnwidth}\raggedright\strut
  El Paso\strut
  \end{minipage} & \begin{minipage}[t]{0.21\columnwidth}\centering\strut
  622263\strut
  \end{minipage} & \begin{minipage}[t]{0.20\columnwidth}\centering\strut
  0.1237\strut
  \end{minipage} & \begin{minipage}[t]{0.34\columnwidth}\raggedright\strut
  Colorado Springs\strut
  \end{minipage}\tabularnewline
  \begin{minipage}[t]{0.13\columnwidth}\raggedright\strut
  Jefferson\strut
  \end{minipage} & \begin{minipage}[t]{0.21\columnwidth}\centering\strut
  534543\strut
  \end{minipage} & \begin{minipage}[t]{0.20\columnwidth}\centering\strut
  0.1063\strut
  \end{minipage} & \begin{minipage}[t]{0.34\columnwidth}\raggedright\strut
  Denver-Aurora-Lakewood\strut
  \end{minipage}\tabularnewline
  \begin{minipage}[t]{0.13\columnwidth}\raggedright\strut
  Larimer\strut
  \end{minipage} & \begin{minipage}[t]{0.21\columnwidth}\centering\strut
  299630\strut
  \end{minipage} & \begin{minipage}[t]{0.20\columnwidth}\centering\strut
  0.05958\strut
  \end{minipage} & \begin{minipage}[t]{0.34\columnwidth}\raggedright\strut
  Fort Collins\strut
  \end{minipage}\tabularnewline
  \begin{minipage}[t]{0.13\columnwidth}\raggedright\strut
  Other\strut
  \end{minipage} & \begin{minipage}[t]{0.21\columnwidth}\centering\strut
  1378964\strut
  \end{minipage} & \begin{minipage}[t]{0.20\columnwidth}\centering\strut
  0.2742\strut
  \end{minipage} & \begin{minipage}[t]{0.34\columnwidth}\raggedright\strut
  \strut
  \end{minipage}\tabularnewline
  \begin{minipage}[t]{0.13\columnwidth}\raggedright\strut
  Colorado\strut
  \end{minipage} & \begin{minipage}[t]{0.21\columnwidth}\centering\strut
  5029196\strut
  \end{minipage} & \begin{minipage}[t]{0.20\columnwidth}\centering\strut
  100\strut
  \end{minipage} & \begin{minipage}[t]{0.34\columnwidth}\raggedright\strut
  \strut
  \end{minipage}\tabularnewline
  \bottomrule
  \end{longtable}
  
  \clearpage
  
  \subsection{The Politics of Colorado}\label{the-politics-of-colorado}
  
  Curtis Martin (1962) notes that Colorado, due to its status as a
  frontier state, has always been fiercely democratic and independent. He
  connects this fact with Colorado's past, by pointing out that its
  political institutions were deeply rooted in mining culture, ordinary
  citizens' participation,a strong feeling of being ``far away'' from
  sources of centralized power in the coasts, and a wish for the
  protection and preservation of Colorado's natural environment. As such,
  Colorado can be described as a populist state with a strong libertarian
  streak, that highly values democratic processes when they serve the
  people or protect and fund national parks, but staunchly opposes state
  intervention when it is unwarranted. (Martin, 1962)
  
  This 1964 study of Colorado politics rings true to this day. One needs
  not search for long to see instances when Colorado honored this
  description. One example is TABOR, or the Taxpayer's Bill of Rights; a
  strongly libertarian, small-government, populist series of regulations
  that mandated a referendum for any measure that would increase state
  taxes, and caped government spending. TABOR was passed by referendum in
  1992, and later amended in 2005 after the dot com economic crisis
  exposed the fact that inability to spend is very bad for a state
  government trying to jump start its economy. (Assembly, n.d.)
  
  Similarly, Amendment 64 passed in 2012 made Colorado one of the first
  states to legalize the selling, possession, and consumption of
  recreational marijuana--a policy advocated by progressives and
  libertarians alike. Colorado was also the staging ground for what has
  been coined the ``Sagebrush Rebellion'': a movement primarily consisting
  on ranchers in dispute with the federal government over land use laws
  and wildlife protection. While this ``rebellion'' primarily consisted of
  battles in local legislatures or elections in the 1970s, its echoes can
  be heard till today in events like the Bundy Standoff, with ranchers
  taking up arms against federal employees and occupying federal land
  (Thompson, 2016).
  
  Setting policy aside, this description of Colorado is also confirmed by
  polling data and election results. While being traditionally more
  conservative, inflows of immigration from the South coupled with
  increasing urban liberalization and tourism has led the state from
  leaning republican to being aggressively purple: the quintessential
  swing state. Colorado voted both for and later against Bill Clinton,
  voted for G.W. Bush twice, and has supported democratic presidential
  candidates since (Hamm, 2017). The trend is also, maybe surprisingly,
  consistent when considering both rural and urban voters; the divide that
  is said to plague other states seems to have passed by Colorado.
  Additionally, when polled on trust of federal or local governments,
  Colorado residents are systematically skeptical; in a random sample poll
  conducted by Cronin and Loevy (2012) in 2010, 56\% stated that their
  state officials were lazy, wasteful, and inefficient. However--again
  indicating a libertarian, independent streak--most Coloradoans from 1988
  to today consistently believe that their state is ``on the right
  track.''\footnote{Colorado College Citizens Polls, taken from Cronin et
    al. (Cronin \& Loevy, 2012)}
  
  \subsection{Voting in Colorado}\label{voting-in-colorado}
  
  Each County individually administers local, coordinated, primary, and
  general elections, under the supervision of the Colorado Secretary of
  State. This means that each county individually handles the voters
  registered in that county. Unsurprisingly, the same eight most populous
  counties are also the counties with the majority of registered voters,
  as their registrants comprise 73\% of total Colorado registered voters
  (as of November 2017). As table 3.2 shows, these eight counties have a
  registration rate between 60-80\%, compared to a Colorado-wide rate of
  about 67\%. Registration rates for all counties are also graphically
  depicted in figure 3.2. In terms of Party registration, Colorado as a
  whole leans democratic by a very narrow margin (figure 3.3).
  
  \begin{longtable}[]{@{}lccc@{}}
  \caption{Colorado voter registration for largest counties
  \label{tab:voter_reg}}\tabularnewline
  \toprule
  \begin{minipage}[b]{0.10\columnwidth}\raggedright\strut
  County\strut
  \end{minipage} & \begin{minipage}[b]{0.24\columnwidth}\centering\strut
  Total Registered Voters\strut
  \end{minipage} & \begin{minipage}[b]{0.29\columnwidth}\centering\strut
  County Registration Rate\strut
  \end{minipage} & \begin{minipage}[b]{0.25\columnwidth}\centering\strut
  \% of Statewide Registrants\strut
  \end{minipage}\tabularnewline
  \midrule
  \endfirsthead
  \toprule
  \begin{minipage}[b]{0.10\columnwidth}\raggedright\strut
  County\strut
  \end{minipage} & \begin{minipage}[b]{0.24\columnwidth}\centering\strut
  Total Registered Voters\strut
  \end{minipage} & \begin{minipage}[b]{0.29\columnwidth}\centering\strut
  County Registration Rate\strut
  \end{minipage} & \begin{minipage}[b]{0.25\columnwidth}\centering\strut
  \% of Statewide Registrants\strut
  \end{minipage}\tabularnewline
  \midrule
  \endhead
  \begin{minipage}[t]{0.10\columnwidth}\raggedright\strut
  Adams\strut
  \end{minipage} & \begin{minipage}[t]{0.24\columnwidth}\centering\strut
  270,303\strut
  \end{minipage} & \begin{minipage}[t]{0.29\columnwidth}\centering\strut
  0.61\strut
  \end{minipage} & \begin{minipage}[t]{0.25\columnwidth}\centering\strut
  0.07\strut
  \end{minipage}\tabularnewline
  \begin{minipage}[t]{0.10\columnwidth}\raggedright\strut
  Arapahoe\strut
  \end{minipage} & \begin{minipage}[t]{0.24\columnwidth}\centering\strut
  410,546\strut
  \end{minipage} & \begin{minipage}[t]{0.29\columnwidth}\centering\strut
  0.72\strut
  \end{minipage} & \begin{minipage}[t]{0.25\columnwidth}\centering\strut
  0.11\strut
  \end{minipage}\tabularnewline
  \begin{minipage}[t]{0.10\columnwidth}\raggedright\strut
  Boulder\strut
  \end{minipage} & \begin{minipage}[t]{0.24\columnwidth}\centering\strut
  237,091\strut
  \end{minipage} & \begin{minipage}[t]{0.29\columnwidth}\centering\strut
  0.80\strut
  \end{minipage} & \begin{minipage}[t]{0.25\columnwidth}\centering\strut
  0.06\strut
  \end{minipage}\tabularnewline
  \begin{minipage}[t]{0.10\columnwidth}\raggedright\strut
  Denver\strut
  \end{minipage} & \begin{minipage}[t]{0.24\columnwidth}\centering\strut
  450,616\strut
  \end{minipage} & \begin{minipage}[t]{0.29\columnwidth}\centering\strut
  0.75\strut
  \end{minipage} & \begin{minipage}[t]{0.25\columnwidth}\centering\strut
  0.12\strut
  \end{minipage}\tabularnewline
  \begin{minipage}[t]{0.10\columnwidth}\raggedright\strut
  Douglas\strut
  \end{minipage} & \begin{minipage}[t]{0.24\columnwidth}\centering\strut
  237,659\strut
  \end{minipage} & \begin{minipage}[t]{0.29\columnwidth}\centering\strut
  0.83\strut
  \end{minipage} & \begin{minipage}[t]{0.25\columnwidth}\centering\strut
  0.06\strut
  \end{minipage}\tabularnewline
  \begin{minipage}[t]{0.10\columnwidth}\raggedright\strut
  El Paso\strut
  \end{minipage} & \begin{minipage}[t]{0.24\columnwidth}\centering\strut
  445,708\strut
  \end{minipage} & \begin{minipage}[t]{0.29\columnwidth}\centering\strut
  0.71\strut
  \end{minipage} & \begin{minipage}[t]{0.25\columnwidth}\centering\strut
  0.12\strut
  \end{minipage}\tabularnewline
  \begin{minipage}[t]{0.10\columnwidth}\raggedright\strut
  Jefferson\strut
  \end{minipage} & \begin{minipage}[t]{0.24\columnwidth}\centering\strut
  422,362\strut
  \end{minipage} & \begin{minipage}[t]{0.29\columnwidth}\centering\strut
  0.79\strut
  \end{minipage} & \begin{minipage}[t]{0.25\columnwidth}\centering\strut
  0.11\strut
  \end{minipage}\tabularnewline
  \begin{minipage}[t]{0.10\columnwidth}\raggedright\strut
  Larimer\strut
  \end{minipage} & \begin{minipage}[t]{0.24\columnwidth}\centering\strut
  250,626\strut
  \end{minipage} & \begin{minipage}[t]{0.29\columnwidth}\centering\strut
  0.84\strut
  \end{minipage} & \begin{minipage}[t]{0.25\columnwidth}\centering\strut
  0.06\strut
  \end{minipage}\tabularnewline
  \begin{minipage}[t]{0.10\columnwidth}\raggedright\strut
  Other\strut
  \end{minipage} & \begin{minipage}[t]{0.24\columnwidth}\centering\strut
  1,009,392\strut
  \end{minipage} & \begin{minipage}[t]{0.29\columnwidth}\centering\strut
  ---\strut
  \end{minipage} & \begin{minipage}[t]{0.25\columnwidth}\centering\strut
  0.27\strut
  \end{minipage}\tabularnewline
  \begin{minipage}[t]{0.10\columnwidth}\raggedright\strut
  Colorado\strut
  \end{minipage} & \begin{minipage}[t]{0.24\columnwidth}\centering\strut
  3,734,303\strut
  \end{minipage} & \begin{minipage}[t]{0.29\columnwidth}\centering\strut
  0.67\strut
  \end{minipage} & \begin{minipage}[t]{0.25\columnwidth}\centering\strut
  1.00\strut
  \end{minipage}\tabularnewline
  \bottomrule
  \end{longtable}
  
  \begin{figure}
  
  {\centering \includegraphics[width=0.5\linewidth]{/Users/tdounias/Desktop/Reed_Senior_Thesis/maps/pct_registered_county_map} 
  
  }
  
  \caption[Registration rates per Colorado county]{Registration rates per Colorado county}\label{fig:reg per county map}
  \end{figure}
  
  \begin{figure}
  
  {\centering \includegraphics[width=0.5\linewidth]{/Users/tdounias/Desktop/Reed_Senior_Thesis/maps/party_affiliation_county_map} 
  
  }
  
  \caption[Democratic/Republican party lean per Colorado county]{Democratic/Republican party lean per Colorado county}\label{fig:party reg per county map}
  \end{figure}
  
  In the past 25 years, there have been a series of key changes in the way
  Colorado administers elections, in relation to Vote By Mail and other
  reforms targeted and expanding the democratic franchise. In 1992,
  Colorado introduced no-excuse absentee voting, allowing voters to either
  physically pick up a mail ballot at a Vote Center or County Office, or
  have a ballot mailed to them prior to election day. In 2008, this reform
  was expanded to a permanent Vote-By-Mail system, which gave voters the
  option to be permanently put on a list of addresses that received mail
  ballots prior to the election. The State also entered a transitional
  status to full mail elections, giving counties the option to make all
  coordinated local elections, general elections, and primary elections
  exclusively VBM. In 2013, the Colorado State Legislature passed
  HB13-1303: The Voter Access and Modernized Elections Act, which mandated
  that every voter currently registered receive a mail ballot for all
  future elections. The Act also expanded the use of Vote Centers instead
  of traditional polling places, instituted same-day voter registration,
  and revamped the way active and inactive voter status was designated on
  voter rolls--more on this in future sections. These changes are
  summarized in Table 3.3.
  
  \begin{longtable}[]{@{}cl@{}}
  \caption{Key changes to Colorado elections policy
  \label{tab:elect_policy}}\tabularnewline
  \toprule
  \begin{minipage}[b]{0.07\columnwidth}\centering\strut
  Year\strut
  \end{minipage} & \begin{minipage}[b]{0.87\columnwidth}\raggedright\strut
  Key Changes\strut
  \end{minipage}\tabularnewline
  \midrule
  \endfirsthead
  \toprule
  \begin{minipage}[b]{0.07\columnwidth}\centering\strut
  Year\strut
  \end{minipage} & \begin{minipage}[b]{0.87\columnwidth}\raggedright\strut
  Key Changes\strut
  \end{minipage}\tabularnewline
  \midrule
  \endhead
  \begin{minipage}[t]{0.07\columnwidth}\centering\strut
  1992\strut
  \end{minipage} & \begin{minipage}[t]{0.87\columnwidth}\raggedright\strut
  No Excuse Absentee Statewide Implementation\strut
  \end{minipage}\tabularnewline
  \begin{minipage}[t]{0.07\columnwidth}\centering\strut
  2008\strut
  \end{minipage} & \begin{minipage}[t]{0.87\columnwidth}\raggedright\strut
  Permanent No-Excuse VBM Lists, Option of Full-VBM Elections\strut
  \end{minipage}\tabularnewline
  \begin{minipage}[t]{0.07\columnwidth}\centering\strut
  2013\strut
  \end{minipage} & \begin{minipage}[t]{0.87\columnwidth}\raggedright\strut
  Automatic Mail Ballot System Implemented Statewide, Established Vote
  Centers\strut
  \end{minipage}\tabularnewline
  \bottomrule
  \end{longtable}
  
  \clearpage
  
  \subsection{Colorado as a Case for this
  Thesis}\label{colorado-as-a-case-for-this-thesis}
  
  Colorado presents such an interesting case for research on Vote By Mail
  exactly because it has gone through such a long transitional process to
  reach its current elections system. It has steadily developed voting
  policy through a mixture of state mandates, county action, and outside
  policy motivations. Colorado's streak of independence and direct
  democracy is also very apparent in this shift in electoral practices,
  since they have been passing policies trying to expand participation for
  a very long time. It gives researchers access to approximately 22 years
  during which at least part of the state conducted elections partially by
  mail, making comparative, county- or individual- level case studies
  particularly alluring. Colorado's streak of independence and direct
  democracy is also very apparent in this shift in electoral practices,
  since they have been passing policies trying to expand participation for
  a very long time.
  
  On a more general level, Colorado is interesting exactly because it is
  ``typical'' but with a wild streak. It is typical rocky mountain
  country, great planes country, and liberal urban city but all \emph{in
  one state}. In is libertarian yet increasingly Democratic. It heavily
  relies on state funding for national parks, yet rebels against federal
  land use laws. It is a frontier state with traditional values, that
  overwhelmingly supports marijuana legalization. It is also a consistent
  purple state, with a Democratic Governor and House, but Republican
  Attorney General, Secretary of State and senate. This means that
  Colorado is a combination of distinct national effects, but also local
  effects that make it significantly different from national trends as a
  whole. In this environment, predicting results of policy can be
  difficult, but extremely salient as multiple effects can be tested
  against each other.
  
  \section{Acquiring the Data}\label{acquiring-the-data}
  
  This thesis relies on county and individual level models to draw
  conclusions on voting behaviors, and how they are affected by voting
  method. As such, the data I need will optimally contain the following:
  
  \begin{itemize}
  \item
    \textbf{County and individual level demographic characteristics}:
    race, gender, urban population
  \item
    \textbf{County and individual level voting data}: turnout, party
    registration, total registrants
  \item
    \textbf{Information on individual elections}: date, ballots cast,
    voting methods, county, election descriptions
  \end{itemize}
  
  In the process of my research, I have acquired sufficient data to cover
  the second and third of these areas. I was unable to procure individual
  level data on demographic characteristics apart from gender, age, and
  party registration. However, reasonable conclusions can still be drawn
  from county or precinct aggregates.
  
  \subsection{Sources and first glance}\label{sources-and-first-glance}
  
  I used two sources of data: Colorado voter records procured from the
  Colorado Secretary of State's office, and demographic data from the 2010
  US Census. In the process of procuring these data I was aided by a
  series of other researchers and professionals with experience in the
  field of elections administration; they are mentioned in my
  acknowledgements.
  
  \subsubsection{2010 US Census}\label{us-census}
  
  The US Census is conducted country-wide every ten years, with the goal
  of procuring accurate data on the demographic characteristics of the
  population. The Census uses a combination of federal field workers
  conducting door-to-door canvassing and statistical methods for data
  aggregation. From the 2010 Census--which is publicly available online--I
  get total population counts, characteristics on race, and rural/urban
  population counts for Colorado.
  
  I use two datasets from the Census. For both, the unit of observation is
  one of the 64 counties of Colorado, and both include the same total
  population counts. One contains racial demographic characteristics and
  the other contain percentages of rural and urban populations in each
  county. The racial demographic dataset needed some wrangling work to
  extract a percentage of white residents in each county. Individuals were
  coded as ``white'' when they identified as exclusively white--this
  doesn't include mixed-race individuals reporting white ancestry.
  
  \subsubsection{Colorado Voter Files}\label{colorado-voter-files}
  
  As any state, Colorado maintains a statewide registry of all currently
  registered voters. This registry is typically under the purview of the
  Secretary of State--in this case, Wayne W. Williams. Voter Registration
  Files are constantly updated with new information on existing voters,
  new voters, or with the removal of inactive or otherwise ineligible
  voters. Therefore, this file will be different every time it is accessed
  or shared. Based on when this file is accessed, only a ``snapshot'' of
  the file can be obtained. I have managed to procure ``snapshots'' for
  each year between 2012 and 2017.
  
  Similarly with VRFs, a Voter History File is maintained and constantly
  updated by the state. This file is uniquely connected to its VRF: only
  voters showing up as registrants will have their histories included. I
  have similarly procured ``snapshots'' of the Voter History File for the
  years between 2012 and 2017.
  
  In the Voter Registration files, the unit of observation is the
  individual voter, and all variables are initially coded as character
  strings. Each voter is assigned a unique voter ID, which serves as a
  point of reference between the two files. Broadly speaking, data in this
  file can be divided between three categories: first, personal
  identification information like address, ZIP code, or phone number;
  second, demographic information like age and gender; third, information
  pertinent to elections administration like congressional district, local
  elections for which the individual should receive a ballot, voter ID,
  and party registration. I will further elaborate on relevant variables
  in the wrangling section.
  
  In the Voter History files, the unit of observation here is a single
  ballot cast, and all variables are initially coded as character strings.
  This means that for each voter registered--and so included in the
  VRF--the history file should contain an observation for each time they
  voted. This file includes two types of data: first, identifiers for the
  election like county, date, description, and type; second, identifiers
  for the individual vote including voter ID and voting method.
  
  \section{Wrangling the Data}\label{wrangling-the-data}
  
  The process of ``wrangling'' refers to manipulating the data into a form
  that can then be used for graphing, exploratory data analysis,
  modelling, or presentation. In this case, wrangling also included
  aggregating data across multiple sources and datasets. For this purpose,
  I made heavy use of the tidyverse R package, and in particular the dplyr
  package. In this section I will go through some of the key problems
  encountered during the wrangling of these data, and then discuss the
  final form each variable takes.
  
  \subsection{Initial Problems with the 2017 Voter File and
  Solution}\label{initial-problems-with-the-2017-voter-file-and-solution}
  
  The first major issue I encountered--which merits discussion in its own
  section--derives from the aforementioned fact that the voter records I
  had access to are ``snapshots''. What this means, is that for each
  person in each year of voter registration files, I will have their
  corresponding history files for all ballots they have cast in Colorado,
  but not their own history of registration and migration. If, say, a
  voter moved from Boulder County to Summit County, I would have their
  votes in Boulder County show up in the voter history file, but them
  being registered in Summit. If you recall the turnout calculations
  specified earlier on, this implies an overestimation when looking back
  at elections that happened some time before the date of the
  ``snapshot''. Additionally, ``snapshots'' of current voter files do not
  reflect voters dropping off the rolls for whatever reason--death, moving
  out of the state, long term inactivity, non-confirmable personal data
  etc. Since for these voters the history files would also not be
  included, the issue created is less one of overestimation of turnout
  like before, but just the inclusion of additional room for error that is
  created when subtracting one from the denominator and enumerator of
  turnout.
  
  This was a significant problem from the beginning of this thesis, since
  I started out with only one ``snapshot'' from 2017. After going through
  turnout calculations, a significant majority of counties appeared to
  have turnout exceeding 100\%, particularly for years between 2000 and
  2012. This was, to put it mildly, concerning. With the help of my
  advisers, I was able to procure similar ``snapshots'' for each year
  between 2012-2016. After similar calculations, I returned figure 3.4 for
  the eight most populous counties as described above, including different
  shapes for election type, colors for county, and a vertical line at 2013
  to signify the latest major change in how Colorado administers
  elections.
  
  \begin{figure}
  
  {\centering \includegraphics[width=0.6\linewidth]{/Users/tdounias/Desktop/Reed_Senior_Thesis/plots/colorado_bigeight_turnout_graph} 
  
  }
  
  \caption[Turnout plot for eight largest Colorado counties, 2012-2016]{Turnout plot for eight largest Colorado counties, 2012-2016}\label{fig:big eight turnout plot}
  \end{figure}
  
  To also further illustrate the in-county migration and dropped voter
  problem, I created a graph that includes logged total counts of
  registered voters calculated using the 2017 and the 2012-2016 files. The
  plot also includes a line at y=x. If in-Colorado migration and dropped
  voters are not an issue, most points on this graph should be at this
  line.
  
  \begin{figure}
  
  {\centering \includegraphics[width=0.6\linewidth]{/Users/tdounias/Desktop/Reed_Senior_Thesis/plots/county_migration_A} 
  
  }
  
  \caption[Comparison of registration count methods]{Comparison of registration count methods}\label{fig:county migration A}
  \end{figure}
  
  Two things should be clear from figure 3.5. First, there is significant
  deviation between the counts using just the 2017 file and all files
  across years. Specifically, the 2017 count consistently underestimates
  the total amount of registered voters--this is shown by the red linear
  model smoothing line. This consistent difference confirms the hypothesis
  that there is a substantial benefit to using ``snapshots'' for multiple
  years. Second, counts get more accurate the closer to 2017 we get. This
  should be even more apparent in figure 3.6, which limits the scale to
  only some high registration counties, and adds a shape indicator for
  county.
  
  \begin{figure}
  
  {\centering \includegraphics[width=0.6\linewidth]{/Users/tdounias/Desktop/Reed_Senior_Thesis/plots/county_migration_B} 
  
  }
  
  \caption[Comparison of registration count methods only for a few counties, 2012-2016]{Comparison of registration count methods only for a few counties, 2012-2016}\label{fig:county migration B}
  \end{figure}
  
  Here the structure of the data becomes clear: for each county, there are
  a series of almost vertically distributed points, which get closer to
  the y = x line the closer the counts get to 2017. Through this series of
  tests, it became clear that using multiple years of data was necessary
  in order to conduct an accurate test of my hypotheses. My selection was
  later vindicated, when looking at comparisons between reported rates of
  turnout\footnote{Turnout is calculated over all registered voters} and
  turnout calculated through my dataset for the 2014 midterm election (see
  fig. 3.7).
  
  \begin{figure}
  
  {\centering \includegraphics[width=0.6\linewidth]{/Users/tdounias/Desktop/Reed_Senior_Thesis/plots/Calc_vs_rep_turnout} 
  
  }
  
  \caption[Comparison of reported and calculated turnout for 2014 midterms across county]{Comparison of reported and calculated turnout for 2014 midterms across county}\label{fig:comp turnout 2014}
  \end{figure}
  
  The differences are insignificant. They exist because of ``noise'' added
  on because of errors in the data, misreporting, private voter
  registration files, voters dropped before the ``snapshot'' occurred, and
  other similar factors.
  
  \subsection{Other Wrangling Issues
  Faced}\label{other-wrangling-issues-faced}
  
  Suffice to say, wrangling data was the majority of the work that went
  into this thesis. Doing a full account would probably read like the
  world's most cliche crime novel: a series of elusive final datasets, a
  plucky yet occasionally naive young detective, two wisened mentors,
  clues, dead ends, frustration, compromise, and\ldots{}spreadsheets. I
  will spare the reader the whole story, but I will include a
  non-comprehensive list of some of the difficulties associated with
  wrangling voter files, as it was a crucial part of the learning process
  I underwent while doing my research.
  
  \textbf{Missing Values}: The decision on how to deal with missing
  values--or NAs--in a dataset is a lot more important than it may
  initially seem. A first, intuitive reaction might be to just disregard
  them; however this works under the assumption that there is no structure
  inherent to why these data are missing! To give just two examples, in
  the data I have collected, the PARTY value for the 2015 voter
  registration file is missing. If I excluded all observations with
  missing PARTY values, I would be excluding a fifth of my data. Missing
  values were also present in the VOTING\_METHOD variable of the voter
  history files. While this may have seemed troubling, after closer
  examination it was revealed that the vast majority of such missing
  values was concentrated in Jefferson County, and in elections prior to
  2002. Therefore, these observations could be ignored, since they played
  no role in my final dataset. The conclusion should be that choices made
  on exclusion, inclusion, or estimation of missing data are very
  important, and should be taken with much care and consideration for the
  underlying structure of the data.
  
  \textbf{Data Input Errors}: Is ``Greece'' a legitimate voting method?
  Probably not. However, ``Greece'' did show up as a value in the
  VOTING\_METHOD variable for my 2012 voter history file snapshot. This
  may have occurred for a series of reasons, like data reading issues--the
  data I acquired had changed hands some times, and also changed platforms
  between STATA and R--or issues at time of input--each county counts
  votes individually, and \emph{then} the state aggregates the data--, or
  some bug in my code. Having adequately checked for the later of these
  reasons, I treated all values that seemed more likely than not to be
  errors as NAs. There were not many of these--less than .001\% of my
  data--but they were a hassle to find, analyze, and then recode into some
  useful value.
  
  \textbf{Data Size}: Nothing to write home about here, just an
  observation that multiple voter registration files can be \emph{huge},
  which puts considerable strain on a computer's processing power. This
  means that wrangling has to comprise of a series of careful, deliberate
  moves. Brute force should be discouraged, as a dead end means several
  hours of melodic computer fan panic.
  
  \textbf{Joining, Merging, Spreading, and the Multiplicity of Levels}:
  For the data to end up in any functional shape, it eventually becomes
  necessary to start joining datasets. Thankfully, a clear division of
  modelling tasks between county and individual level models means that
  joining on COUNTY or VOTER\_ID is ideal, and fairly straightforward. As
  will become clear in later sections, I also had to consider the variety
  of different units of observation, specifically: county, individual,
  ballot, election, county-by-election.
  
  \subsection{Final Variable
  Specifications}\label{final-variable-specifications}
  
  After the conclusion of the wrangling process, the resulting dataset
  included a series of discrete and continuous variables. I will briefly
  outline them here, along with their range and values.
  
  \begin{itemize}
  \tightlist
  \item
    VOTER\_ID: Discrete variable, unique value given to each individual
    voter. Useful for merging.
  \item
    COUNTY: Discrete variable, the 64 counties of Colorado.
  \item
    REGISTRATION\_DATE: Discrete variable, date of registration for each
    registrant. Useful to get total registrants on election day.
  \item
    TURNOUT: Continuous variable, in the range {[}0,1{]}. The response
    variable for my county-level models.
  \item
    ELECTION\_TYPE: Discrete variable, the four types of elections:
    Primary, Coordinated, Midterm, Presidential.
  \item
    ELECTION\_DATE: Discrete variable, self-explanatory.
  \item
    VBM\_PCT: Continuous variable, in the range {[}0,1{]}. This is the
    focus of my analysis, as it counts the percentage of total ballots
    that were mail ballots.
  \item
    PCT\_WHITE: Continuous variable, in the range {[}0,1{]}. Percentage of
    white residents per county.
  \item
    PCT\_URBAN: Continuous variable, in the range {[}0,1{]}. Percentage of
    urban residents per county.
  \item
    PARTY: Discrete variable. For each voter, the party they are
    registered with. Can be: Republican, Democrat, Other, or Unaffiliated.
  \item
    GENDER: Discrete binary variable, Male or Female.
  \item
    AGE: The age of the individual registrant.
  \item
    VOTING\_METHOD: The method used by an individual voter to cast their
    ballot. Is coded as either VBM or In Person, according to Table 3.4:
  \end{itemize}
  
  \begin{longtable}[]{@{}lll@{}}
  \caption{Voting method designation table
  \label{tab:voting_method}}\tabularnewline
  \toprule
  \begin{minipage}[b]{0.13\columnwidth}\raggedright\strut
  Method\strut
  \end{minipage} & \begin{minipage}[b]{0.62\columnwidth}\raggedright\strut
  Description\strut
  \end{minipage} & \begin{minipage}[b]{0.16\columnwidth}\raggedright\strut
  Designation\strut
  \end{minipage}\tabularnewline
  \midrule
  \endfirsthead
  \toprule
  \begin{minipage}[b]{0.13\columnwidth}\raggedright\strut
  Method\strut
  \end{minipage} & \begin{minipage}[b]{0.62\columnwidth}\raggedright\strut
  Description\strut
  \end{minipage} & \begin{minipage}[b]{0.16\columnwidth}\raggedright\strut
  Designation\strut
  \end{minipage}\tabularnewline
  \midrule
  \endhead
  \begin{minipage}[t]{0.13\columnwidth}\raggedright\strut
  Absentee Carry\strut
  \end{minipage} & \begin{minipage}[t]{0.62\columnwidth}\raggedright\strut
  Voters who carried an absentee ballot from an early voting
  location\strut
  \end{minipage} & \begin{minipage}[t]{0.16\columnwidth}\raggedright\strut
  VBM\strut
  \end{minipage}\tabularnewline
  \begin{minipage}[t]{0.13\columnwidth}\raggedright\strut
  Absentee Mail\strut
  \end{minipage} & \begin{minipage}[t]{0.62\columnwidth}\raggedright\strut
  Voters who were sent an absentee ballot, and mailed it in\strut
  \end{minipage} & \begin{minipage}[t]{0.16\columnwidth}\raggedright\strut
  VBM\strut
  \end{minipage}\tabularnewline
  \begin{minipage}[t]{0.13\columnwidth}\raggedright\strut
  Early Voting\strut
  \end{minipage} & \begin{minipage}[t]{0.62\columnwidth}\raggedright\strut
  Voters who physically went to an Early Voting location and voted\strut
  \end{minipage} & \begin{minipage}[t]{0.16\columnwidth}\raggedright\strut
  In Person\strut
  \end{minipage}\tabularnewline
  \begin{minipage}[t]{0.13\columnwidth}\raggedright\strut
  In Person\strut
  \end{minipage} & \begin{minipage}[t]{0.62\columnwidth}\raggedright\strut
  Voters who physically went to a polling place and voted on paper\strut
  \end{minipage} & \begin{minipage}[t]{0.16\columnwidth}\raggedright\strut
  In Person\strut
  \end{minipage}\tabularnewline
  \begin{minipage}[t]{0.13\columnwidth}\raggedright\strut
  Mail Ballot\strut
  \end{minipage} & \begin{minipage}[t]{0.62\columnwidth}\raggedright\strut
  Vote By Mail\strut
  \end{minipage} & \begin{minipage}[t]{0.16\columnwidth}\raggedright\strut
  VBM\strut
  \end{minipage}\tabularnewline
  \begin{minipage}[t]{0.13\columnwidth}\raggedright\strut
  Polling Place\strut
  \end{minipage} & \begin{minipage}[t]{0.62\columnwidth}\raggedright\strut
  Traditional polling place voting, discontinued in 2013\strut
  \end{minipage} & \begin{minipage}[t]{0.16\columnwidth}\raggedright\strut
  In Person\strut
  \end{minipage}\tabularnewline
  \begin{minipage}[t]{0.13\columnwidth}\raggedright\strut
  Vote Center\strut
  \end{minipage} & \begin{minipage}[t]{0.62\columnwidth}\raggedright\strut
  Voters who cast their ballots at Vote Centers\strut
  \end{minipage} & \begin{minipage}[t]{0.16\columnwidth}\raggedright\strut
  In Person\strut
  \end{minipage}\tabularnewline
  \bottomrule
  \end{longtable}
  
  \chapter{Model Specification and
  Results}\label{model-specification-and-results}
  
  \section{Modelling Issues}\label{modelling-issues}
  
  \subsection{Lack of variability}\label{lack-of-variability}
  
  To put it very simply, it's not enough to have hundreds of thousands of
  observations if they are all almost identical for some important
  variable. If, for example, my data included a thousand people in
  Jefferson county, and 63 in all other counties of Colorado combined--one
  in each remaining county--, then I would not be able to leverage my data
  to draw conclusions on county-level effects.
  
  As previously stated, I have registration files going back to 2012. From
  these files, I have extracted data for elections going back to
  2010.\footnote{See section 3.3.1; I extracted data limited to this time
    period to avoid accuracy issues with migration and removal of
    inactive/unavailable voters} In order to make inferences on VBM and
  turnout effects I had to have extensive and varied data. I have
  extensive data--over 35 million observations at the individual
  level--but the data substantially lacks variance in voting method. Put
  simply, the vast majority of registrants in Colorado from 2010 onward
  either did not vote at all, or voted by mail. If you recall the changes
  in Colorado election law, in 2008 counties were allowed to conduct all
  mail elections, and no-excuse permanent absentee voting was implemented
  state-wide; then in 2013 Colorado transitioned to full VBM for all
  elections. This means that few people were still using traditional
  polling places or vote centers to cast their ballots. Figure 4.1 shows
  how, after 2013, and even before that in 2011--the coordinated, local
  election for which mail ballots were more convenient for counties--over
  95\% of ballots cast were mail ballots. Only in the general elections of
  2010 and 2012 there is some variance, but mail ballots account for well
  over two thirds of total votes.
  
  \begin{figure}
  
  {\centering \includegraphics[width=0.6\linewidth]{/Users/tdounias/Desktop/Reed_Senior_Thesis/plots/vbm_county_graph} 
  
  }
  
  \caption[Percentage of mail ballots over total ballots by year]{Percentage of mail ballots over total ballots by year}\label{fig:vbm png}
  \end{figure}
  
  This issue is not completely fatal for my county level models. There is
  still variance between counties that have 100\% mail ballots and those
  that are around the 75-85\% margin. For individual level models--where I
  am estimating voting probability--VBM will be an almost perfect
  predictor for voting, and therefore will not present me with any
  substantial analytical result on how it affects voting probability.
  There are some ways to compensate for this issue, which I outline; due
  to time or data constraints, not all of these will be implemented in
  this thesis:
  
  \begin{itemize}
  \item
    \emph{More (Diverse) Data}: It would be amazing to get snapshot data
    of Colorado voter files from, say, 2004 to today, because it would
    allow for an extensive study on how the 2008 and 2013 election laws
    re-shaped voting decisions in the state. It would be amazing, but also
    expensive and very time consuming--involving several purchases of data
    from the Secretary of State of Colorado. Voter registration files also
    tend to get messier the further back one goes, which means that the
    process of cleaning up the data would get substantially harder. It
    would also require more processing power to handle more observations.
    My research here does not do this, as the scope of a senior thesis is
    a lot more limited than such an overarching study that would probably
    be conducted by multiple researchers with several assistants. I do
    however present several replicable materials for such a study, through
    the creation of an \textit{R} package I include on my gitHub page
    along with the final results of this thesis. This thesis does not go
    that far, but it may help similar studies in the future.
  \item
    \emph{Localized, Natural Experiment Studies}: A natural experiment is
    when, due to policy changes and circumstances, a ``control'' and
    ``treatment'' group of such a policy are created in the same
    approximate geographical area. This happens when, for example, only
    some of the counties in a state enact a specific change. Several such
    studies exist already, with some even tackling VBM in Colorado (Keele
    \& Titiunik, 2017), or how turnout rates are affected by new,
    restrictive registration laws (Burden \& Neiheisel, 2013). This method
    would allow for more accurracy in both the individual and county level
    models, and through the existance of a treatment and control group
    would guarantee the variability that I am currently lacking.
  \item
    \emph{Synthetic Control Group}: The synthetic control group method is
    a way of creating a control group when no such group seems to exist.
    It involves gathering a set of characteristics from the treatment
    group members and then using statistical methods to combine them into
    making the appropriate control (McClelland \& Gault, 2017). I will not
    go into the particulars of this method--the sources cited here should
    provide a decent introduction--, but this method has been successful
    in assessing policy effects such as anti-smoking laws (Barr et al.,
    2012), or even motor voter laws in Oregon\footnote{This is in
      refference to a paper presented by advisor Paul Gronke at the 2017
      Elections Sciences Reform and Administration Conference in Portland,
      Oregon.}.
  \end{itemize}
  
  \subsection{Processing Power}\label{processing-power}
  
  These models take a really long time to run, and many times do not even
  reach results. For this purpose, I acquired a MacBook Pro from Reed
  College Computer User Services that substantially increased the amount
  of RAM I had available for this thesis. This was useful, but not enough.
  Particularly the individual level models, for which I have almost 35
  million different observations, do not run. A long term solution to this
  issue could be the use of a virtual machine, or some RStudio server, ar
  Amazon Work Spaces (AWS). For now, I compensated for this problem by
  using stratified sampling to sample a subset of my observations.
  
  The form of stratified sampling I am using is very simple; based on
  county, mail vote, and electoral participation, I use \texttt{dplyr} in
  \textit{R} to draw a sample that contains equal proportions of every
  combination of values of these variables to those in the poriginal
  dataset. If, for example, the original dataset had 2\% of entries being
  voters from Jefferson county that participated using a mail ballot, the
  sampled dataset would have a proportion that is approximatelly equal to
  2\% (Chihara \& Hesterberg, 2011). In this way I draw a sample of around
  400,000 observations from my initial ballot dataset, on which I run all
  my individual models. After checking the variable ratios in sampled and
  population datasets, I found that the differences between ratios had a
  mean and standard deviation of less than a hundredth of a percentile.
  Therefore this sampled dataset could serve as a decent approximation of
  my population.
  
  \section{Models}\label{models}
  
  \subsection{Variable Specification}\label{variable-specification}
  
  I will not go through each individual variable in this section, but will
  briefly describe my procedure on notation for the following models. I
  will include more comments whenever they seem necessary under each
  model. In this thesis I include predictors on a series of variables that
  can be divided into five categories based on unit of observation:
  county, election, individual, local result, and ballot. The last two are
  functions of other units: local result units are equal to the product of
  elections and counties, while ballot units are equal to the number of
  unique individuals multiplied by the number of elections each of them
  was registered in. For notation, I follow this set of rules:
  
  \begin{enumerate}
  \def\labelenumi{\arabic{enumi}.}
  \tightlist
  \item
    If the variable is a response, it is coded \(y\).
  \item
    If the variable is a predictor, it is coded according to Table 4.1.
  \item
    The variable's superscript will provide information on what it
    represents, else it will be explained.
  \item
    All variables represent a single value of that variable unless stated
    otherwise.
  \item
    Unit of observation will also be specified in subscript, according to
    the indices described in Table 4.1. These indices are also used in sum
    notation.
  \item
    All Greek characters represent coefficients to be calculated.
  \item
    By \(k[j]\) I represent the k-value of the j-observation. In this
    case, this would be the county that an individual is registered in.
  \item
    Note that for Local Result level variables, I use \(k,l\) as an
    indice. This is because there are very few variables at this level, it
    is a direct multiplicative product of two other units, and this
    notation avoids confusion with even more indice types.
  \end{enumerate}
  
  \begin{longtable}[]{@{}ccc@{}}
  \caption{Variable names and indices per unit of observation
  \label{tab:units_vars}}\tabularnewline
  \toprule
  \begin{minipage}[b]{0.27\columnwidth}\centering\strut
  Units\strut
  \end{minipage} & \begin{minipage}[b]{0.20\columnwidth}\centering\strut
  Variable\strut
  \end{minipage} & \begin{minipage}[b]{0.15\columnwidth}\centering\strut
  Index\strut
  \end{minipage}\tabularnewline
  \midrule
  \endfirsthead
  \toprule
  \begin{minipage}[b]{0.27\columnwidth}\centering\strut
  Units\strut
  \end{minipage} & \begin{minipage}[b]{0.20\columnwidth}\centering\strut
  Variable\strut
  \end{minipage} & \begin{minipage}[b]{0.15\columnwidth}\centering\strut
  Index\strut
  \end{minipage}\tabularnewline
  \midrule
  \endhead
  \begin{minipage}[t]{0.27\columnwidth}\centering\strut
  Ballot\strut
  \end{minipage} & \begin{minipage}[t]{0.20\columnwidth}\centering\strut
  u\strut
  \end{minipage} & \begin{minipage}[t]{0.15\columnwidth}\centering\strut
  i\strut
  \end{minipage}\tabularnewline
  \begin{minipage}[t]{0.27\columnwidth}\centering\strut
  Individual\strut
  \end{minipage} & \begin{minipage}[t]{0.20\columnwidth}\centering\strut
  z\strut
  \end{minipage} & \begin{minipage}[t]{0.15\columnwidth}\centering\strut
  j\strut
  \end{minipage}\tabularnewline
  \begin{minipage}[t]{0.27\columnwidth}\centering\strut
  County\strut
  \end{minipage} & \begin{minipage}[t]{0.20\columnwidth}\centering\strut
  x\strut
  \end{minipage} & \begin{minipage}[t]{0.15\columnwidth}\centering\strut
  k\strut
  \end{minipage}\tabularnewline
  \begin{minipage}[t]{0.27\columnwidth}\centering\strut
  Election\strut
  \end{minipage} & \begin{minipage}[t]{0.20\columnwidth}\centering\strut
  w\strut
  \end{minipage} & \begin{minipage}[t]{0.15\columnwidth}\centering\strut
  l\strut
  \end{minipage}\tabularnewline
  \begin{minipage}[t]{0.27\columnwidth}\centering\strut
  Local Result\strut
  \end{minipage} & \begin{minipage}[t]{0.20\columnwidth}\centering\strut
  v\strut
  \end{minipage} & \begin{minipage}[t]{0.15\columnwidth}\centering\strut
  k,l\strut
  \end{minipage}\tabularnewline
  \begin{minipage}[t]{0.27\columnwidth}\centering\strut
  General Index\strut
  \end{minipage} & \begin{minipage}[t]{0.20\columnwidth}\centering\strut
  -\strut
  \end{minipage} & \begin{minipage}[t]{0.15\columnwidth}\centering\strut
  i'\strut
  \end{minipage}\tabularnewline
  \bottomrule
  \end{longtable}
  
  \section{County Level Models}\label{county-level-models}
  
  \subsection{Specifications}\label{specifications}
  
  In this section I will go through a step-by step creation of models at
  the county level. County level models use a series of variables at the
  election, county, and local result levels. The response variable is
  always turnout as a local result. If this model is considered at its
  most basic, it could be thought of as an assignment of voting tendencies
  across counties; each county independent of election has a unique range
  of turnout results. In this way it is possible to build a naive,
  baseline model of turnout as follows:
  
  \begin{equation} \tag{Model 1}
  y^{turnout}_{k,l} = \beta_0 + (\sum_{k=1}^{64}\beta_kx_k^{county}),
  \end{equation}
  
  where \(x_k^{county}\) is a series of 64 dummy variables for each county
  of Colorado. Here differences between elections come from normally
  distributed error terms, rather than predictors. I name this
  \textbf{Model 1}, and it does not reflect the data particularly well.
  First off, this model includes the assumption that counties are
  independent of one another, which is probably false; just consider that
  these counties are areas of the same state, in the same country, with
  populations moving between them at regular intervals, and many of them
  covering the same metropolitan area or congressional district.
  Additionally, this model cannot fully calculate relevant coefficients,
  since a number of counties can be represented as perfect linear
  functions of the other variables. This means they will be dropped by
  \textit{R} when the model is called in the \texttt{lm()} function.
  
  A way to fix both these issues is to use a multilevel model with mixed
  effects for county. By constraining coefficients at the county level to
  a set distribution, this model does away with the assumption of
  independence. The other county level predictors help to explain some of
  the unexplained group level variation, which reduces the standard
  deviation of county coefficients and helps provide more exact estimates
  (Gelman \& Hill, 2006). I call this \textbf{Model 2}, which can be
  written as:
  
  \begin{equation} \tag{Model 2}
  y^{turnout}_{k,l} = a_{k} + \beta_{1}x_k^{\%white} + \beta_{2}x_k^{\%urban},
  \end{equation}
  
  \[a_{k} \sim N (\gamma_0, \sigma_{\alpha}^2)\] This model provides a
  more reasonable set of estimates for each county, but still fails at
  providing any sort of guess as to secular trends, time-specific effects,
  election type effects, or mail voting--the variable of interest. I will
  amend this by adding a set of variables at the election and local result
  levels: election type and an interaction term between election type and
  mail voting. This variable should reflect whether turnout effects of
  mail voting are more pronounced in a specific type of election. I call
  this \textbf{Model 3} and it can be specified as follows:
  
  \begin{equation} \tag{Model 3}
  y^{turnout}_{k,l} = a_{k} + \beta_{1}x_k^{\% white} + \beta_{2}x_k^{\% urban} + (\sum_{i'=1}^{4}\beta_{i'+3}w_{i'}^{election type})*(\beta_3v_{k,l}^{\% mail~vote} + 1),
  \end{equation}
  
  \[a_{k} \sim N(\gamma_0, \sigma_{\alpha}^2)\]
  
  where \(w_{i'}^{election type}\) is a series of four dummy variables for
  each type of election (General, Primary, Coordinated, Midterm). This
  model reflects nearly all the information I have available, apart from
  election date. For the incorporation of election dates there are two
  possible alternatives. First, I can simply add a dummy variable for each
  year. This would assume independence between each year, as it would
  specify different, independent ``slopes'' for the seven years I have
  data for--this is like calculating seven different models, one for each
  year. This is not particularly elegant as a solution nor does it reflect
  the fact that years actually are interconnected; of course there can be
  massive shifts in national or regional political climates, but those
  shifts happened \emph{from some baseline}, which is reflected in
  previous years.
  
  These elections can be thought of as systems for which prior condition
  affects future outcomes, and therefore time cannot be modeled as a
  series of independent effects. The solution here is adding a spline
  function for time, using a general additive multilevel model. The most
  commonly used spline function, and the default in the \texttt{gamm4}
  \textit{R} package is a thin plate regression spline, which I also use
  here (S. N. Wood, 2006). More on the subject of splines can be found in
  the Wood (2006) textbook. The model, which I call \textbf{Model 4} can
  be written as follows:
  
  \begin{equation}\tag{Model 4}
  y^{turnout}_{k,l} = a_{k} + \beta_{1}x_k^{\% white} + \beta_{2}x_k^{\% urban} + (\sum_{i'=1}^{4}\beta_{i'+3}w_{i'}^{election type})*(\beta_3v_{k,l}^{\% mail vote} + 1) + s(w^{year}_{l}),
  \end{equation}
  
  \[a_{k} \sim N(\gamma_0, \sigma_{\alpha}^2)\] where \(s()\) is a thin
  plate spline function with seven knots--equal to the number of
  years.\footnote{I used the \texttt{gam.check()} function that is present
    in the \texttt{mgcv} \textit{R} package, whose call determined that
    the number of knots here may be too low. However, given the data
    available to me, I was limited to the inclusion of seven years and as
    such cannot increase the number of knots any further.} A summary of
  these four models is provided in the following table:
  
  \begin{longtable}[]{@{}cc@{}}
  \caption{County level model descriptions
  \label{tab:model_desc_county}}\tabularnewline
  \toprule
  \begin{minipage}[b]{0.15\columnwidth}\centering\strut
  Model No\strut
  \end{minipage} & \begin{minipage}[b]{0.80\columnwidth}\centering\strut
  Model Description\strut
  \end{minipage}\tabularnewline
  \midrule
  \endfirsthead
  \toprule
  \begin{minipage}[b]{0.15\columnwidth}\centering\strut
  Model No\strut
  \end{minipage} & \begin{minipage}[b]{0.80\columnwidth}\centering\strut
  Model Description\strut
  \end{minipage}\tabularnewline
  \midrule
  \endhead
  \begin{minipage}[t]{0.15\columnwidth}\centering\strut
  Model 1\strut
  \end{minipage} & \begin{minipage}[t]{0.80\columnwidth}\centering\strut
  Naive model with only county specific effects\strut
  \end{minipage}\tabularnewline
  \begin{minipage}[t]{0.15\columnwidth}\centering\strut
  Model 2\strut
  \end{minipage} & \begin{minipage}[t]{0.80\columnwidth}\centering\strut
  Multilevel model; added county level predictors\strut
  \end{minipage}\tabularnewline
  \begin{minipage}[t]{0.15\columnwidth}\centering\strut
  Model 3\strut
  \end{minipage} & \begin{minipage}[t]{0.80\columnwidth}\centering\strut
  Multilevel model; added VBM, interaction terms, and election fixed
  effects\strut
  \end{minipage}\tabularnewline
  \begin{minipage}[t]{0.15\columnwidth}\centering\strut
  Model 4\strut
  \end{minipage} & \begin{minipage}[t]{0.80\columnwidth}\centering\strut
  Multilevel General Additive model; added spline function for election
  year\strut
  \end{minipage}\tabularnewline
  \bottomrule
  \end{longtable}
  
  \subsection{Results}\label{results}
  
  The table in this section presents coefficients and standard errors for
  all four county level models. This table does not include any metrics
  for county--either mixed or fixed effects. I have chosen to omit these
  because they firstly are not very relevant to my hypotheses, and
  secondly because they are very extensive--64 coefficients for each of
  the four models. I have also not included any metric for time--here
  measured in years and used only in the fourth model. Both the mixed
  effects for county and the measure for time should be considered as
  controls: the first controls for county-specific trends while still
  restricting these to allow for non-independence, and the second makes
  sure that my results are indicative of a secular trend, independent of
  any shifts along time.
  
  \begin{longtable}[]{@{}ccccc@{}}
  \caption{Estimated county level coefficients
  \label{tab:county_coef}}\tabularnewline
  \toprule
  \begin{minipage}[b]{0.23\columnwidth}\centering\strut
  Variables\strut
  \end{minipage} & \begin{minipage}[b]{0.14\columnwidth}\centering\strut
  Naive Model\strut
  \end{minipage} & \begin{minipage}[b]{0.18\columnwidth}\centering\strut
  County Effects\strut
  \end{minipage} & \begin{minipage}[b]{0.20\columnwidth}\centering\strut
  Election Effects\strut
  \end{minipage} & \begin{minipage}[b]{0.11\columnwidth}\centering\strut
  Time Data\strut
  \end{minipage}\tabularnewline
  \midrule
  \endfirsthead
  \toprule
  \begin{minipage}[b]{0.23\columnwidth}\centering\strut
  Variables\strut
  \end{minipage} & \begin{minipage}[b]{0.14\columnwidth}\centering\strut
  Naive Model\strut
  \end{minipage} & \begin{minipage}[b]{0.18\columnwidth}\centering\strut
  County Effects\strut
  \end{minipage} & \begin{minipage}[b]{0.20\columnwidth}\centering\strut
  Election Effects\strut
  \end{minipage} & \begin{minipage}[b]{0.11\columnwidth}\centering\strut
  Time Data\strut
  \end{minipage}\tabularnewline
  \midrule
  \endhead
  \begin{minipage}[t]{0.23\columnwidth}\centering\strut
  (Intercept)\strut
  \end{minipage} & \begin{minipage}[t]{0.14\columnwidth}\centering\strut
  0.369\strut
  \end{minipage} & \begin{minipage}[t]{0.18\columnwidth}\centering\strut
  0.492\strut
  \end{minipage} & \begin{minipage}[t]{0.20\columnwidth}\centering\strut
  0.455\strut
  \end{minipage} & \begin{minipage}[t]{0.11\columnwidth}\centering\strut
  0.470\strut
  \end{minipage}\tabularnewline
  \begin{minipage}[t]{0.23\columnwidth}\centering\strut
  \strut
  \end{minipage} & \begin{minipage}[t]{0.14\columnwidth}\centering\strut
  (0.60)\strut
  \end{minipage} & \begin{minipage}[t]{0.18\columnwidth}\centering\strut
  (0.045)**\strut
  \end{minipage} & \begin{minipage}[t]{0.20\columnwidth}\centering\strut
  (0.078)**\strut
  \end{minipage} & \begin{minipage}[t]{0.11\columnwidth}\centering\strut
  (0.072)\strut
  \end{minipage}\tabularnewline
  \begin{minipage}[t]{0.23\columnwidth}\centering\strut
  Pct\_white\strut
  \end{minipage} & \begin{minipage}[t]{0.14\columnwidth}\centering\strut
  \strut
  \end{minipage} & \begin{minipage}[t]{0.18\columnwidth}\centering\strut
  0.034\strut
  \end{minipage} & \begin{minipage}[t]{0.20\columnwidth}\centering\strut
  0.033\strut
  \end{minipage} & \begin{minipage}[t]{0.11\columnwidth}\centering\strut
  0.031\strut
  \end{minipage}\tabularnewline
  \begin{minipage}[t]{0.23\columnwidth}\centering\strut
  \strut
  \end{minipage} & \begin{minipage}[t]{0.14\columnwidth}\centering\strut
  \strut
  \end{minipage} & \begin{minipage}[t]{0.18\columnwidth}\centering\strut
  (0.053)\strut
  \end{minipage} & \begin{minipage}[t]{0.20\columnwidth}\centering\strut
  (0.050)\strut
  \end{minipage} & \begin{minipage}[t]{0.11\columnwidth}\centering\strut
  (0.050)\strut
  \end{minipage}\tabularnewline
  \begin{minipage}[t]{0.23\columnwidth}\centering\strut
  Pct\_urban\strut
  \end{minipage} & \begin{minipage}[t]{0.14\columnwidth}\centering\strut
  \strut
  \end{minipage} & \begin{minipage}[t]{0.18\columnwidth}\centering\strut
  -0.118\strut
  \end{minipage} & \begin{minipage}[t]{0.20\columnwidth}\centering\strut
  -0.117\strut
  \end{minipage} & \begin{minipage}[t]{0.11\columnwidth}\centering\strut
  -0.119\strut
  \end{minipage}\tabularnewline
  \begin{minipage}[t]{0.23\columnwidth}\centering\strut
  \strut
  \end{minipage} & \begin{minipage}[t]{0.14\columnwidth}\centering\strut
  \strut
  \end{minipage} & \begin{minipage}[t]{0.18\columnwidth}\centering\strut
  (0.022)**\strut
  \end{minipage} & \begin{minipage}[t]{0.20\columnwidth}\centering\strut
  (0.021)**\strut
  \end{minipage} & \begin{minipage}[t]{0.11\columnwidth}\centering\strut
  (0.021)\strut
  \end{minipage}\tabularnewline
  \begin{minipage}[t]{0.23\columnwidth}\centering\strut
  typeGeneral\strut
  \end{minipage} & \begin{minipage}[t]{0.14\columnwidth}\centering\strut
  \strut
  \end{minipage} & \begin{minipage}[t]{0.18\columnwidth}\centering\strut
  \strut
  \end{minipage} & \begin{minipage}[t]{0.20\columnwidth}\centering\strut
  0.190\strut
  \end{minipage} & \begin{minipage}[t]{0.11\columnwidth}\centering\strut
  0.254\strut
  \end{minipage}\tabularnewline
  \begin{minipage}[t]{0.23\columnwidth}\centering\strut
  \strut
  \end{minipage} & \begin{minipage}[t]{0.14\columnwidth}\centering\strut
  \strut
  \end{minipage} & \begin{minipage}[t]{0.18\columnwidth}\centering\strut
  \strut
  \end{minipage} & \begin{minipage}[t]{0.20\columnwidth}\centering\strut
  (0.070)**\strut
  \end{minipage} & \begin{minipage}[t]{0.11\columnwidth}\centering\strut
  (0.065)\strut
  \end{minipage}\tabularnewline
  \begin{minipage}[t]{0.23\columnwidth}\centering\strut
  typeMidterm\strut
  \end{minipage} & \begin{minipage}[t]{0.14\columnwidth}\centering\strut
  \strut
  \end{minipage} & \begin{minipage}[t]{0.18\columnwidth}\centering\strut
  \strut
  \end{minipage} & \begin{minipage}[t]{0.20\columnwidth}\centering\strut
  0.252\strut
  \end{minipage} & \begin{minipage}[t]{0.11\columnwidth}\centering\strut
  0.070\strut
  \end{minipage}\tabularnewline
  \begin{minipage}[t]{0.23\columnwidth}\centering\strut
  \strut
  \end{minipage} & \begin{minipage}[t]{0.14\columnwidth}\centering\strut
  \strut
  \end{minipage} & \begin{minipage}[t]{0.18\columnwidth}\centering\strut
  \strut
  \end{minipage} & \begin{minipage}[t]{0.20\columnwidth}\centering\strut
  (0.068)**\strut
  \end{minipage} & \begin{minipage}[t]{0.11\columnwidth}\centering\strut
  (0.063)\strut
  \end{minipage}\tabularnewline
  \begin{minipage}[t]{0.23\columnwidth}\centering\strut
  typePrimary\strut
  \end{minipage} & \begin{minipage}[t]{0.14\columnwidth}\centering\strut
  \strut
  \end{minipage} & \begin{minipage}[t]{0.18\columnwidth}\centering\strut
  \strut
  \end{minipage} & \begin{minipage}[t]{0.20\columnwidth}\centering\strut
  -0.071\strut
  \end{minipage} & \begin{minipage}[t]{0.11\columnwidth}\centering\strut
  -0.170\strut
  \end{minipage}\tabularnewline
  \begin{minipage}[t]{0.23\columnwidth}\centering\strut
  \strut
  \end{minipage} & \begin{minipage}[t]{0.14\columnwidth}\centering\strut
  \strut
  \end{minipage} & \begin{minipage}[t]{0.18\columnwidth}\centering\strut
  \strut
  \end{minipage} & \begin{minipage}[t]{0.20\columnwidth}\centering\strut
  (0.069)\strut
  \end{minipage} & \begin{minipage}[t]{0.11\columnwidth}\centering\strut
  (0.062)\strut
  \end{minipage}\tabularnewline
  \begin{minipage}[t]{0.23\columnwidth}\centering\strut
  typeCoordinated*VBM\strut
  \end{minipage} & \begin{minipage}[t]{0.14\columnwidth}\centering\strut
  \strut
  \end{minipage} & \begin{minipage}[t]{0.18\columnwidth}\centering\strut
  \strut
  \end{minipage} & \begin{minipage}[t]{0.20\columnwidth}\centering\strut
  -0.001\strut
  \end{minipage} & \begin{minipage}[t]{0.11\columnwidth}\centering\strut
  0.002\strut
  \end{minipage}\tabularnewline
  \begin{minipage}[t]{0.23\columnwidth}\centering\strut
  \strut
  \end{minipage} & \begin{minipage}[t]{0.14\columnwidth}\centering\strut
  \strut
  \end{minipage} & \begin{minipage}[t]{0.18\columnwidth}\centering\strut
  \strut
  \end{minipage} & \begin{minipage}[t]{0.20\columnwidth}\centering\strut
  (0.067)\strut
  \end{minipage} & \begin{minipage}[t]{0.11\columnwidth}\centering\strut
  (0.058)\strut
  \end{minipage}\tabularnewline
  \begin{minipage}[t]{0.23\columnwidth}\centering\strut
  typeGeneral*VBM\strut
  \end{minipage} & \begin{minipage}[t]{0.14\columnwidth}\centering\strut
  \strut
  \end{minipage} & \begin{minipage}[t]{0.18\columnwidth}\centering\strut
  \strut
  \end{minipage} & \begin{minipage}[t]{0.20\columnwidth}\centering\strut
  0.151\strut
  \end{minipage} & \begin{minipage}[t]{0.11\columnwidth}\centering\strut
  0.087\strut
  \end{minipage}\tabularnewline
  \begin{minipage}[t]{0.23\columnwidth}\centering\strut
  \strut
  \end{minipage} & \begin{minipage}[t]{0.14\columnwidth}\centering\strut
  \strut
  \end{minipage} & \begin{minipage}[t]{0.18\columnwidth}\centering\strut
  \strut
  \end{minipage} & \begin{minipage}[t]{0.20\columnwidth}\centering\strut
  (0.073)*\strut
  \end{minipage} & \begin{minipage}[t]{0.11\columnwidth}\centering\strut
  (0.037)\strut
  \end{minipage}\tabularnewline
  \begin{minipage}[t]{0.23\columnwidth}\centering\strut
  typeMidterm*VBM\strut
  \end{minipage} & \begin{minipage}[t]{0.14\columnwidth}\centering\strut
  \strut
  \end{minipage} & \begin{minipage}[t]{0.18\columnwidth}\centering\strut
  \strut
  \end{minipage} & \begin{minipage}[t]{0.20\columnwidth}\centering\strut
  -0.058\strut
  \end{minipage} & \begin{minipage}[t]{0.11\columnwidth}\centering\strut
  0.109\strut
  \end{minipage}\tabularnewline
  \begin{minipage}[t]{0.23\columnwidth}\centering\strut
  \strut
  \end{minipage} & \begin{minipage}[t]{0.14\columnwidth}\centering\strut
  \strut
  \end{minipage} & \begin{minipage}[t]{0.18\columnwidth}\centering\strut
  \strut
  \end{minipage} & \begin{minipage}[t]{0.20\columnwidth}\centering\strut
  (0.026)\strut
  \end{minipage} & \begin{minipage}[t]{0.11\columnwidth}\centering\strut
  (0.030)\strut
  \end{minipage}\tabularnewline
  \begin{minipage}[t]{0.23\columnwidth}\centering\strut
  typePrimary*VBM\strut
  \end{minipage} & \begin{minipage}[t]{0.14\columnwidth}\centering\strut
  \strut
  \end{minipage} & \begin{minipage}[t]{0.18\columnwidth}\centering\strut
  \strut
  \end{minipage} & \begin{minipage}[t]{0.20\columnwidth}\centering\strut
  -0.089\strut
  \end{minipage} & \begin{minipage}[t]{0.11\columnwidth}\centering\strut
  -0.003\strut
  \end{minipage}\tabularnewline
  \begin{minipage}[t]{0.23\columnwidth}\centering\strut
  \strut
  \end{minipage} & \begin{minipage}[t]{0.14\columnwidth}\centering\strut
  \strut
  \end{minipage} & \begin{minipage}[t]{0.18\columnwidth}\centering\strut
  \strut
  \end{minipage} & \begin{minipage}[t]{0.20\columnwidth}\centering\strut
  (0.028)\strut
  \end{minipage} & \begin{minipage}[t]{0.11\columnwidth}\centering\strut
  (0.027)\strut
  \end{minipage}\tabularnewline
  \bottomrule
  \end{longtable}
  
  Given that, the first observable result is that the percentage of white
  population and the percentage of urban population are fairly stable
  indicators of a small positive and negative shift in turnout
  respectively. The lack of variability between models is not surprising;
  these represent a county-level, time-independent demographic statistic,
  and there would be no reason to assume that part of their effect would
  be subsumed by other variables in models 3 and 4.
  
  Moving on to election type, the first thing to note is that there is no
  typeCoordinated in the table. This is because of the way \textit{R}
  displays and calculates models for discrete variables, when they are
  coded as indicators. The coefficients for the different election types
  should be read as differences from the ``baseline'' that is
  typeCoordinated. First surprising result here is that the coefficient
  for general presidential elections is substantially lower than that of
  midterms. Or rather this would be surprising if we did not notice the
  interaction terms with VBM, which indicate that, after allowing for VBM
  effects, presidential elections do actually have higher turnout in my
  model than midterms do. Other than this, coefficients in model 3 and
  model 4 \emph{individualy} make sense, in the assumed ordering of
  turnout in such elections--presidential, then midterm, then coordinated
  and lastly primary.
  
  Next, taking election type and all interaction terms into consideration,
  let's examine what happens when the spline function for time is
  introduced between models 3 and 4. Most coefficients shift dramatically,
  with the exception of the interaction between coordinated elections and
  VBM. This dramatic shift--between 5 and 15(!) percentage
  points--indicates that several of the effects that the third model
  estimated are actually time-specific trends, and that there is a
  significant difference if we account for them. In the fourth model, the
  coefficients for election type on their own are still indicative of a
  common assumption on turnout in such elections\footnote{Also see Figure
    3.4}. As for interaction terms with VBM, the effect of VBM on primary
  election turnout is almost wiped out entirely, the interaction with
  general election turnout is depleted but still present at around 8\%,
  and coordinated election VBM effects remain virtually non-existent.
  Interestingly, the effect of VBM on midterm turnout switches sign from a
  negative effect of 5\% to a positive effect of around 11\%; making
  midterm elections the most heavily affected by mail voting.
  
  Taking my hypotheses one by one, these models present evidence less in
  favor of H1, than against its alternate, H1'. While mail voting does
  seem to affect turnout in a way consistent across time--see the
  coefficients for VBM effects on general and midterm elections--this
  effect is not particularly more strong than the percentage of urban
  population in each county. This result is ambiguous, and therefore does
  not present concrete evidence as to the validity of H1. Conversly, if
  going off of these models my second and third hypotheses can be
  convincingly rejected. After controlling for time, the effect that VBM
  has on coordinated or primary elections is marginal at best, compared to
  more consistent effects on midterm and general elections. The one point
  in favor of H3 here is that the effect of VBM on midterm elections is
  slightly higher--about 2\%--than the effect on presidential elections in
  model 4. However, this difference is not significant enough to rule in
  favor of H3; if this difference was caused by the lack of presence of
  national effects, it would be significantly more pronounced in primary
  and coordinated elections. As such, I would ascribe this difference to
  some omited variable, or other effect that I do not include in my
  models.
  
  \section{Individual Level Models}\label{individual-level-models}
  
  \subsection{Specifications}\label{specifications-1}
  
  For the rest of this section, assume the following:
  
  \[y_i \sim \text{Bernoulli}(p)\]
  
  Where \(y_i \in \{0,1\}\) is the probability that the i-th ballot was
  completed.
  
  In this section I do not linearly add to the model until it reaches a
  final stage. The reasoning here is that there is no exact linear path to
  follow; there is an overarching unit of observation--the ballot--and all
  the rest are dependent between each other. For instance, adding a
  variable for Party at the ballot level would not significantly change
  the way I later add percentage of white residents at the county level.
  Therefore, the way I proceed is the following: I ``build'' the models
  step by step and separately for each group of variables (grouping by
  unit of observation). Then I present one example of what a model using
  two of these initial ``building blocks'' would look like. Since this is
  fairly generalizable, I then proceed directly to the full model which
  includes all different variables. Some models will be named; the models
  for which I do so are the ones for which an \textit{R} call is
  attempted, and results in either estimation of coefficients or some
  interesting observation on why their call was not possible.
  
  If receiving a ballot with no information, I would predict that the
  probability that an additional ballot was a vote in favor would be equal
  to turnout, as calculated through all other ballots. Therefore:
  
  \[\hat{\mathbb{P}}(y_i = 1) = \frac{\# \text{votes cast}}{\# \text{ballots}}\]
  
  \subsubsection{Estimation with only one type of
  data}\label{estimation-with-only-one-type-of-data}
  
  There are four levels of data I will go through here: County, Election,
  Person, and Ballot.
  
  \subsubsection{County Level}\label{county-level}
  
  Assume that the ballot I am trying to assess completion for has the name
  of the county it is from written on it. There are two ways I can think
  of for predicting \(\mathbb{P(y_i = 1)}\). First, assume that each
  different county has a different, independent \(\mathbb{P(y_i = 1)}\),
  then:
  
  \[\hat{\mathbb{P}}(y_i = 1) \sim \text{logit}^{-1}(\sum_{k = 1}^{64}x_{k,i}\beta_{k})\]
  
  Where k counts over the 64 counties of Colorado, and \(x_{k}\) is an
  indicator variable for each county. If I, quite reasonably, throw away
  the assumption of independence--these counties are, after all, in the
  same state and the same country--I could also fit a mixed effects model
  as such:
  
  \begin{equation} \tag{Model 1}
  \hat{\mathbb{P}}(y_i = 1) \sim \text{logit}^{-1}(a_{k[i]}),
  \end{equation}
  
  \[a_{k} \sim \text{N}(\gamma_0, \sigma_{\alpha}^2)\]
  
  Where \(\alpha_{k[i]}\) varies by county, constrained by its standard
  deviation and \(\gamma_0\), an intercept coefficient. I name this
  \textbf{Model 1}.
  
  Let's say now that along with the one ballot, I was given a short list
  of \(n^{\text{county vars}}\) other county-level variables, be they
  discrete, continuous, or indicators. The two models would then look
  like:
  
  \[\hat{\mathbb{P}}(y_i = 1) \sim \text{logit}^{-1}(\sum_{k = 1}^{64}x_{k}\beta_{k} + \sum_{i'=1}^{n^{\text{county vars}}}x_{k[i], i'}\beta_{i'+64})\]
  
  Where \(x_{k[i], l}\) is the k-th value of the i'-th variable. If, as
  before, I do not assume independence, the model can be written as:
  
  \begin{equation} \tag{Model 2}  
  \hat{\mathbb{P}}(y_i = 1) \sim \text{logit}^{-1}(a_{k[i]}),
  \end{equation}
  
  \[a_{k} \sim \text{N}(\gamma_0 + \sum_{i'=1}^{n^{\text{county vars}}}x_{k[i], i'}\gamma_{i'}, \sigma_{\alpha}^2)\]
  
  In the case of my specific data, for the time being I have county-level
  data for white population and urban population, so
  \(n^{\text{county vars}} = 2\). I name this \textbf{Model 2}
  
  \subsubsection{Individual Level}\label{individual-level}
  
  Assuming that I know the voter ID of the individual that cast their
  ballot, I can treat this piece of information in about the same way that
  I did for county as described above. This means that the following is
  mostly an exercise in maintaining notation constant. For these purposes,
  let \(n^{ID}\) be the number of total unique voter
  IDs--individuals--that I have data on, and j an indice that sums over
  all individuals. Also let \(z_{j}\) be an indicator variable for each
  individual. Then:
  
  \[\hat{\mathbb{P}}(y_i = 1) \sim \text{logit}^{-1}(\sum_{j = 1}^{n^{ID}}z_{j}\beta_{j})\]
  
  And the second model, not assuming independence, would be:
  
  \[\hat{\mathbb{P}}(y_i = 1) \sim \text{logit}^{-1}(\delta_{j[i]}), \]
  \[\delta_{j} \sim \text{N}(\zeta_0, \sigma_{\delta}^2)\]
  
  Again, in a similar way to county level data, there are variables at an
  individual level, thus making it relatively easy to build further
  models. Let's say now that along with the one ballot, I was given a
  short list of \(n^{\text{indiv vars}}\) other individual-level
  variables, be they discrete, continuous, or indicators. The two models
  would then look like:
  
  \[\hat{\mathbb{P}}(y_i = 1) \sim \text{logit}^{-1}(\sum_{j = 1}^{n^{ID}}z_{j}\beta_{j} + \sum_{i'=1}^{n^{\text{indiv vars}}}z_{j[i], i'}\beta_{i'+n^{ID}})\]
  
  Where \(z_{j[i], l}\) is the j-th value of the i'-th variable. If, as
  before, I do not assume independence, the model can be written as:
  
  \begin{equation} \tag{Model 3*}
  \hat{\mathbb{P}}(y_i = 1) \sim \text{logit}^{-1}(\delta_{j[i]}),
  \end{equation}
  
  \[\delta_{j} \sim \text{N}(\zeta_0 + \sum_{i'=1}^{n^{\text{indiv vars}}}z_{j[i], i'}\delta_{i'}, \sigma_{\delta}^2)\]
  
  In the case of my specific data, for the time being I have
  individual-level data for gender, so \(n^{\text{indiv vars}} = 1\). I
  name the combination of this model and Model 2: \textbf{Model 3}.
  
  \subsubsection{Election Level}\label{election-level}
  
  Again as previously, four models come from including election level
  data. The first two are assuming I only knew what specific election the
  ballot comes from. Let \(w_{i'}\) be an indicator variable for each
  election and \(n^{elect}\) the number of elections. The model assuming
  independence, with \(w_{i'}\) being indicator variables for each
  election, is:
  
  \[\hat{\mathbb{P}}(y_i = 1) \sim \text{logit}^{-1}(\sum_{l = 1}^{n^{elect}}w_{l}\beta_{l})\]
  
  Again, as previously, it would be safe to assume that each election is
  not held in a vacuum. Adding mixed effects this model would be:
  
  \[\hat{\mathbb{P}}(y_i = 1) \sim \text{logit}^{-1}(\eta_{l[i]}), \]
  \[\eta_{l} \sim \text{N}(\nu_0, \sigma_{\nu}^2)\]
  
  Again, in a similar way to county and individual level data, I add in
  variables at an election level. Let's say now that along with the one
  ballot, I was given a short list of \(n^{\text{election vars}}\) other
  election-level variables, be they discrete, continuous, or indicators.
  The two models would then look like:
  
  \begin{equation} \tag{Model 4}
  \hat{\mathbb{P}}(y_i = 1) \sim \text{logit}^{-1}(\sum_{l = 1}^{n^{elect}}w_{l}\beta_{l} + \sum_{i'=1}^{n^{\text{election vars}}}w_{l[i], i'}\beta_{i'+n^{elect}})
  \end{equation}
  
  Where \(w_{l[i], i'}\) is the l-th value of the i'-th variable. For the
  time being I have two different variables that describe individual
  elections: date and type. I choose to fit a glm with some smoothing
  spline function for year. This would also include four distinct
  indicators for election type. I name this final model (including a
  smoothing spline for year) \textbf{Model 4}. Model 4 would not be a
  mixed effects model, since all the variability between elections is
  incorporated in election type and election year--with those two
  variables I can fully describe each election.
  
  \subsubsection{Ballot Level}\label{ballot-level}
  
  In this section I assume that the ballot has some key features written
  on it, like the voting method, age, or party registration of the person
  that filled it out. A mixed effects model here would make no sense,
  since all the data is at the same unit of observation. Therefore, when
  adding ballot level variables, the model would look like:
  
  \begin{equation} \tag{Model 5}
  \hat{\mathbb{P}}(y_i = 1) \sim \text{logit}^{-1}(\beta_0 + \sum_{i' = 1}^{n^{\text{ballot vars}}}u_{i,i'}\beta_{l'})
  \end{equation}
  
  Where \(u_{i,i'}\) is the i-th value of the i'-th variable, and
  \(n^{\text{ballot vars}}\) is the number of ballot level variables. For
  now, I have data on voting method, age, and party. Voting method is
  coded as a binary variable with value one if the method was a Mail Vote.
  Party includes four distinct indicators for REP, DEM, Other, and
  Unaffiliated. Age is tricky; for now the options would be: inclusion as
  an integer, inclusion as a cubic polynomial, inclusion as a 2nd degree
  polynomial, inclusion in some form of spline function. I name this
  \textbf{Model 5}, including age as a linear predictor.
  
  \subsubsection{Estimation with two types of
  data}\label{estimation-with-two-types-of-data}
  
  After the work of setting up the four models at four different levels of
  observation, combining them in twos should be fairly straightforward. To
  avoid being needlessly cumulative, I will pursue this combination for
  County and Individual level only--instead of the six different possible
  combinations.
  
  With the assumption that both counties and individuals are independent
  of one another, I proceed to the first type of model:
  
  \begin{multline*}
  \hat{\mathbb{P}}(y_i = 1) \sim \text{logit}^{-1}(\sum_{k = 1}^{64}x_{k}\beta_{k} + \sum_{i'=1}^{n^{\text{county vars}}}x_{k[i], i'}\beta_{i'+64} + \sum_{j = 1}^{n^{ID}}z_{j}\beta_{j + n^{\text{county vars}} + 64} + \\ \sum_{i'=1}^{n^{\text{indiv vars}}}z_{j[i], i'}\beta_{i'+n^{ID} + n^{\text{county vars}} + 64})
  \end{multline*}
  
  This is large and clunky. It includes variables as described above:
  indicators for each county and individual, and all individual or
  county-level variables. For the corresponding mixed-effects model, I
  assume the tree-like structure we discussed on Monday. The hierarchy has
  two ``levels'', with the second level consisting of two different
  regressions.
  
  \[\hat{\mathbb{P}}(y_i = 1) \sim \text{logit}^{-1}(\delta_{j[i]} + a_{k[i]}), \]
  
  \[a_{k} \sim \text{N}(\gamma_0 + \sum_{i'=1}^{n^{\text{county vars}}}x_{k[i], i'}\gamma_{i'}, \sigma_{\alpha}^2)\]
  \[\delta_{j} \sim \text{N}(\zeta_0 + \sum_{i'=1}^{n^{\text{indiv vars}}}z_{j[i], i'}\delta_{i'}, \sigma_{\delta}^2)\]
  
  \subsubsection{Estimation with the full
  dataset}\label{estimation-with-the-full-dataset}
  
  I now proceed to include variables from all units of observation into
  one model. The first model, assuming independence, is:
  
  \begin{multline*}
  \hat{\mathbb{P}}(y_i = 1) \sim \text{logit}^{-1}(\sum_{k = 1}^{64}x_{k}\beta_{*} + \sum_{i'=1}^{n^{\text{county vars}}}x_{k[i], i'}\beta_{*} + \sum_{j = 1}^{n^{ID}}z_{j}\beta_{*} + \sum_{i'=1}^{n^{\text{indiv vars}}}z_{j[i], i'}\beta_{*} + \\
  \sum_{l = 1}^{n^{elect}}w_{l}\beta_{*} + \sum_{i'=1}^{n^{\text{election vars}}}w_{l[i], i'}\beta_{*} + \sum_{i' = 1}^{n^{\text{ballot vars}}}u_{i,i'}\beta_{*})
  \end{multline*}
  
  You will notice that I have omitted the subscript for all beta
  coefficients. This is because after two or three parameters, this
  becomes very, very large. I think it's reasonable to assume increasing
  indexes for different beta coefficients from left to right in this
  expression.
  
  The mixed effects model will again operate on two ``levels'' of
  hierarchy, but the second level will now include three distinct
  regressions. Caveats for variables like age and date should be noted
  from previous sections. This, the most complete model, will be
  \textbf{Model 6}
  
  \begin{equation} \tag{Model 6}
  \hat{p\_vote} \sim \text{logit}^{-1}(\sum_{i' = 1}^{n^{\text{ballot vars}}}u_{i,i'}\beta_{l'} +\delta_{j[i]} + \alpha_{k[i]} + \eta_{l[i]}),
  \end{equation}
  
  \[\alpha_{k} \sim \text{N}(\gamma_0 + \sum_{i'=1}^{n^{\text{county vars}}}x_{k[i], i'}\gamma_{i'}, \sigma_{\alpha}^2)\]
  
  \[\delta_{j} \sim \text{N}(\zeta_0 + \sum_{i'=1}^{n^{\text{indiv vars}}}z_{j[i], i'}\delta_{i'}, \sigma_{\delta}^2)\]
  
  \[\eta_{l} \sim \text{N}(\nu_0 + \sum_{i'=1}^{n^{\text{election vars}}}w_{l[i], i'}\nu_{i'}, \sigma_{\nu}^2)\]
  
  In summary, Table 4.3 includes all noteworthy models from the previous
  section. I add a few models which should be easily understood based on
  the specifications given above.
  
  \begin{longtable}[]{@{}cc@{}}
  \caption{Individual level model descriptions
  \label{tab:model_desc_individual}}\tabularnewline
  \toprule
  \begin{minipage}[b]{0.15\columnwidth}\centering\strut
  Model No\strut
  \end{minipage} & \begin{minipage}[b]{0.80\columnwidth}\centering\strut
  Model Description\strut
  \end{minipage}\tabularnewline
  \midrule
  \endfirsthead
  \toprule
  \begin{minipage}[b]{0.15\columnwidth}\centering\strut
  Model No\strut
  \end{minipage} & \begin{minipage}[b]{0.80\columnwidth}\centering\strut
  Model Description\strut
  \end{minipage}\tabularnewline
  \midrule
  \endhead
  \begin{minipage}[t]{0.15\columnwidth}\centering\strut
  Model 1\strut
  \end{minipage} & \begin{minipage}[t]{0.80\columnwidth}\centering\strut
  Naive model with only county mixed effects\strut
  \end{minipage}\tabularnewline
  \begin{minipage}[t]{0.15\columnwidth}\centering\strut
  Model 2\strut
  \end{minipage} & \begin{minipage}[t]{0.80\columnwidth}\centering\strut
  Multilevel model; added county level predictors\strut
  \end{minipage}\tabularnewline
  \begin{minipage}[t]{0.15\columnwidth}\centering\strut
  Model 3\strut
  \end{minipage} & \begin{minipage}[t]{0.80\columnwidth}\centering\strut
  Multilevel model; individual-level mixed effects and predictors\strut
  \end{minipage}\tabularnewline
  \begin{minipage}[t]{0.15\columnwidth}\centering\strut
  Model 3a\strut
  \end{minipage} & \begin{minipage}[t]{0.80\columnwidth}\centering\strut
  A combination of 2 and 3 without individual-level mixed effects\strut
  \end{minipage}\tabularnewline
  \begin{minipage}[t]{0.15\columnwidth}\centering\strut
  Model 4\strut
  \end{minipage} & \begin{minipage}[t]{0.80\columnwidth}\centering\strut
  General Additive model; election predictors and time smoothing
  splines\strut
  \end{minipage}\tabularnewline
  \begin{minipage}[t]{0.15\columnwidth}\centering\strut
  Model 5\strut
  \end{minipage} & \begin{minipage}[t]{0.80\columnwidth}\centering\strut
  Ballot-level predictors fixed effects model\strut
  \end{minipage}\tabularnewline
  \begin{minipage}[t]{0.15\columnwidth}\centering\strut
  Model 5a\strut
  \end{minipage} & \begin{minipage}[t]{0.80\columnwidth}\centering\strut
  Multilevel model; ballot predictors with county mixed effects\strut
  \end{minipage}\tabularnewline
  \begin{minipage}[t]{0.15\columnwidth}\centering\strut
  Model 6\strut
  \end{minipage} & \begin{minipage}[t]{0.80\columnwidth}\centering\strut
  Multilevel General Additive model; year splines; individual, county
  mixed effects and all predictors\strut
  \end{minipage}\tabularnewline
  \bottomrule
  \end{longtable}
  
  \subsection{Results}\label{results-1}
  
  To deal with the elephant in the room (page?) right off the bat, I do
  not recommend trusting these results wholeheartedly at the current
  stage. While the aforementioned models are all--to some degree--rational
  parametrizations of individual turnout predictors, the leap from theory
  to implimentation has proven too much for most. This is a direct result
  of the first section in this chapter and the problems outlined within.
  The reason I am still providing the results of these models is twofold:
  first, to show that the data I have on its own \emph{can} be used to
  build and run an individual model of turnout regardless of if that model
  is usefull in responding to my hypotheses; second, to validate that the
  problems I outline in the beginning of this section are actually the
  root cause of the issues I'm having, as some results do give insight
  into this.
  
  \begin{figure}
  
  {\centering \includegraphics[width=0.6\linewidth]{/Users/tdounias/Desktop/Reed_Senior_Thesis/plots/roc_curves_indiv} 
  
  }
  
  \caption[ROC Curve for all individual models]{ROC Curve for all individual models}\label{fig:roc the curves}
  \end{figure}
  
  In terms of trusting these results, I am confident in the results of
  models 1 and 3a, somewhat confident in model 5, and less so for models
  2, 3, and 5a. Models 2, 3 and 5a ``failed to converge''; this means that
  the numeric approximation process by which \textit{R} impliments maximum
  likelihood estimation\footnote{Estimation of maximum likelihood here
    uses Adaptive Gaussian-Hermitian Quadrature (AGQ) to estimate
    coefficients (Handayani, Notodiputro, Sadik, \& Kurnia, 2017)} for
  coefficients doesn't give stable results, within certain conditions.
  While model 5 did converge, it suffers from lack of variance in the
  predictor for VBM, as explained in the beginning of this chapter; this
  is the reason why the coefficient for mail vote is so
  disproportionatelly large and variable. Model 6 simply did not run, even
  on a re-sampled dataset. While I can't really derive any conclusions
  from this fact, there is a distinct possibility that this either occured
  due to a lack of processing power, or lack of sufficient data for the
  model estimation to even reach close to convergance.
  
  There are two conclusions that can be derived from these results. None
  of these conclusions are, sadly, related to my initial hypotheses on
  VBM. The first is that the lack of variance in the data and a lack of
  processing power are direct causes of my inability to estimate these
  models. This is apparent in how model 6 does not run, other models do
  not converge, and the coefficient for VBM is absurdly large--since it
  doesn't vary enough even after stratified sampling to account for any
  variance between mail vote and conventional ballots. The second
  conclusion here is that, despite these issues, there are some
  confirmable results on turnout in general that are common between
  individual and county models. For example, across models 2, 3, 3a the
  urban population of a county is a substantial, negative factor in
  probability of voting, while the white population is a very small,
  positive effect\footnote{This did, however, fail to reach statistical
    significance in both county and individual level models. This means
    that the small, positive effect is not that distinguishable from no
    effect at all.}. Same goes for male gender, which is a very small
  negative effect in voting probability as compared to female gender.
  These effects being stable across several models mean that they are
  independent of the additions to those models; for example, gender, urban
  population, and white population have effects that are not accounted for
  when adding individual level mixed effects. Despite not being able to
  assess VBM as a factor of turnout probability, these models of turnout
  at least show that the data does have substantial use for modelling at
  the individual level.
  
  \clearpage
  
  \begin{longtable}[]{@{}cccccccc@{}}
  \caption{Estimated coefficients for Individual level models
  \label{tab:model_indiv_coefs}}\tabularnewline
  \toprule
  \begin{minipage}[b]{0.12\columnwidth}\centering\strut
  Predictor\strut
  \end{minipage} & \begin{minipage}[b]{0.09\columnwidth}\centering\strut
  Model 1\strut
  \end{minipage} & \begin{minipage}[b]{0.09\columnwidth}\centering\strut
  Model 2\strut
  \end{minipage} & \begin{minipage}[b]{0.09\columnwidth}\centering\strut
  Model 3\strut
  \end{minipage} & \begin{minipage}[b]{0.09\columnwidth}\centering\strut
  Model 3a\strut
  \end{minipage} & \begin{minipage}[b]{0.09\columnwidth}\centering\strut
  Model 4\strut
  \end{minipage} & \begin{minipage}[b]{0.09\columnwidth}\centering\strut
  Model 5\strut
  \end{minipage} & \begin{minipage}[b]{0.10\columnwidth}\centering\strut
  Model 5a\strut
  \end{minipage}\tabularnewline
  \midrule
  \endfirsthead
  \toprule
  \begin{minipage}[b]{0.12\columnwidth}\centering\strut
  Predictor\strut
  \end{minipage} & \begin{minipage}[b]{0.09\columnwidth}\centering\strut
  Model 1\strut
  \end{minipage} & \begin{minipage}[b]{0.09\columnwidth}\centering\strut
  Model 2\strut
  \end{minipage} & \begin{minipage}[b]{0.09\columnwidth}\centering\strut
  Model 3\strut
  \end{minipage} & \begin{minipage}[b]{0.09\columnwidth}\centering\strut
  Model 3a\strut
  \end{minipage} & \begin{minipage}[b]{0.09\columnwidth}\centering\strut
  Model 4\strut
  \end{minipage} & \begin{minipage}[b]{0.09\columnwidth}\centering\strut
  Model 5\strut
  \end{minipage} & \begin{minipage}[b]{0.10\columnwidth}\centering\strut
  Model 5a\strut
  \end{minipage}\tabularnewline
  \midrule
  \endhead
  \begin{minipage}[t]{0.12\columnwidth}\centering\strut
  (Intercept)\strut
  \end{minipage} & \begin{minipage}[t]{0.09\columnwidth}\centering\strut
  -0.175\strut
  \end{minipage} & \begin{minipage}[t]{0.09\columnwidth}\centering\strut
  -0.042\strut
  \end{minipage} & \begin{minipage}[t]{0.09\columnwidth}\centering\strut
  0.001\strut
  \end{minipage} & \begin{minipage}[t]{0.09\columnwidth}\centering\strut
  0.001\strut
  \end{minipage} & \begin{minipage}[t]{0.09\columnwidth}\centering\strut
  \strut
  \end{minipage} & \begin{minipage}[t]{0.09\columnwidth}\centering\strut
  -2.478\strut
  \end{minipage} & \begin{minipage}[t]{0.10\columnwidth}\centering\strut
  -1.888\strut
  \end{minipage}\tabularnewline
  \begin{minipage}[t]{0.12\columnwidth}\centering\strut
  \strut
  \end{minipage} & \begin{minipage}[t]{0.09\columnwidth}\centering\strut
  (0.030)\strut
  \end{minipage} & \begin{minipage}[t]{0.09\columnwidth}\centering\strut
  (0.083)\strut
  \end{minipage} & \begin{minipage}[t]{0.09\columnwidth}\centering\strut
  (0.060)\strut
  \end{minipage} & \begin{minipage}[t]{0.09\columnwidth}\centering\strut
  (0.076)\strut
  \end{minipage} & \begin{minipage}[t]{0.09\columnwidth}\centering\strut
  \strut
  \end{minipage} & \begin{minipage}[t]{0.09\columnwidth}\centering\strut
  (0.015)**\strut
  \end{minipage} & \begin{minipage}[t]{0.10\columnwidth}\centering\strut
  (0.238)**\strut
  \end{minipage}\tabularnewline
  \begin{minipage}[t]{0.12\columnwidth}\centering\strut
  Pct\_urban\strut
  \end{minipage} & \begin{minipage}[t]{0.09\columnwidth}\centering\strut
  \strut
  \end{minipage} & \begin{minipage}[t]{0.09\columnwidth}\centering\strut
  -0.423\strut
  \end{minipage} & \begin{minipage}[t]{0.09\columnwidth}\centering\strut
  -0.436\strut
  \end{minipage} & \begin{minipage}[t]{0.09\columnwidth}\centering\strut
  -0.424\strut
  \end{minipage} & \begin{minipage}[t]{0.09\columnwidth}\centering\strut
  \strut
  \end{minipage} & \begin{minipage}[t]{0.09\columnwidth}\centering\strut
  \strut
  \end{minipage} & \begin{minipage}[t]{0.10\columnwidth}\centering\strut
  -0.538\strut
  \end{minipage}\tabularnewline
  \begin{minipage}[t]{0.12\columnwidth}\centering\strut
  \strut
  \end{minipage} & \begin{minipage}[t]{0.09\columnwidth}\centering\strut
  \strut
  \end{minipage} & \begin{minipage}[t]{0.09\columnwidth}\centering\strut
  (0.055)**\strut
  \end{minipage} & \begin{minipage}[t]{0.09\columnwidth}\centering\strut
  (0.059)**\strut
  \end{minipage} & \begin{minipage}[t]{0.09\columnwidth}\centering\strut
  (0.062)**\strut
  \end{minipage} & \begin{minipage}[t]{0.09\columnwidth}\centering\strut
  \strut
  \end{minipage} & \begin{minipage}[t]{0.09\columnwidth}\centering\strut
  \strut
  \end{minipage} & \begin{minipage}[t]{0.10\columnwidth}\centering\strut
  (0.114)**\strut
  \end{minipage}\tabularnewline
  \begin{minipage}[t]{0.12\columnwidth}\centering\strut
  Pct\_white\strut
  \end{minipage} & \begin{minipage}[t]{0.09\columnwidth}\centering\strut
  \strut
  \end{minipage} & \begin{minipage}[t]{0.09\columnwidth}\centering\strut
  0.067\strut
  \end{minipage} & \begin{minipage}[t]{0.09\columnwidth}\centering\strut
  0.075\strut
  \end{minipage} & \begin{minipage}[t]{0.09\columnwidth}\centering\strut
  0.073\strut
  \end{minipage} & \begin{minipage}[t]{0.09\columnwidth}\centering\strut
  \strut
  \end{minipage} & \begin{minipage}[t]{0.09\columnwidth}\centering\strut
  \strut
  \end{minipage} & \begin{minipage}[t]{0.10\columnwidth}\centering\strut
  -0.151\strut
  \end{minipage}\tabularnewline
  \begin{minipage}[t]{0.12\columnwidth}\centering\strut
  \strut
  \end{minipage} & \begin{minipage}[t]{0.09\columnwidth}\centering\strut
  \strut
  \end{minipage} & \begin{minipage}[t]{0.09\columnwidth}\centering\strut
  (0.102)\strut
  \end{minipage} & \begin{minipage}[t]{0.09\columnwidth}\centering\strut
  (0.073)\strut
  \end{minipage} & \begin{minipage}[t]{0.09\columnwidth}\centering\strut
  (0.094)\strut
  \end{minipage} & \begin{minipage}[t]{0.09\columnwidth}\centering\strut
  \strut
  \end{minipage} & \begin{minipage}[t]{0.09\columnwidth}\centering\strut
  \strut
  \end{minipage} & \begin{minipage}[t]{0.10\columnwidth}\centering\strut
  (0.281)\strut
  \end{minipage}\tabularnewline
  \begin{minipage}[t]{0.12\columnwidth}\centering\strut
  genderMale\strut
  \end{minipage} & \begin{minipage}[t]{0.09\columnwidth}\centering\strut
  \strut
  \end{minipage} & \begin{minipage}[t]{0.09\columnwidth}\centering\strut
  \strut
  \end{minipage} & \begin{minipage}[t]{0.09\columnwidth}\centering\strut
  -0.097\strut
  \end{minipage} & \begin{minipage}[t]{0.09\columnwidth}\centering\strut
  -0.094\strut
  \end{minipage} & \begin{minipage}[t]{0.09\columnwidth}\centering\strut
  \strut
  \end{minipage} & \begin{minipage}[t]{0.09\columnwidth}\centering\strut
  \strut
  \end{minipage} & \begin{minipage}[t]{0.10\columnwidth}\centering\strut
  0.094\strut
  \end{minipage}\tabularnewline
  \begin{minipage}[t]{0.12\columnwidth}\centering\strut
  \strut
  \end{minipage} & \begin{minipage}[t]{0.09\columnwidth}\centering\strut
  \strut
  \end{minipage} & \begin{minipage}[t]{0.09\columnwidth}\centering\strut
  \strut
  \end{minipage} & \begin{minipage}[t]{0.09\columnwidth}\centering\strut
  (0.007)**\strut
  \end{minipage} & \begin{minipage}[t]{0.09\columnwidth}\centering\strut
  (0.007)**\strut
  \end{minipage} & \begin{minipage}[t]{0.09\columnwidth}\centering\strut
  \strut
  \end{minipage} & \begin{minipage}[t]{0.09\columnwidth}\centering\strut
  \strut
  \end{minipage} & \begin{minipage}[t]{0.10\columnwidth}\centering\strut
  (0.017)**\strut
  \end{minipage}\tabularnewline
  \begin{minipage}[t]{0.12\columnwidth}\centering\strut
  Republican\strut
  \end{minipage} & \begin{minipage}[t]{0.09\columnwidth}\centering\strut
  \strut
  \end{minipage} & \begin{minipage}[t]{0.09\columnwidth}\centering\strut
  \strut
  \end{minipage} & \begin{minipage}[t]{0.09\columnwidth}\centering\strut
  \strut
  \end{minipage} & \begin{minipage}[t]{0.09\columnwidth}\centering\strut
  \strut
  \end{minipage} & \begin{minipage}[t]{0.09\columnwidth}\centering\strut
  \strut
  \end{minipage} & \begin{minipage}[t]{0.09\columnwidth}\centering\strut
  0.233\strut
  \end{minipage} & \begin{minipage}[t]{0.10\columnwidth}\centering\strut
  0.208\strut
  \end{minipage}\tabularnewline
  \begin{minipage}[t]{0.12\columnwidth}\centering\strut
  \strut
  \end{minipage} & \begin{minipage}[t]{0.09\columnwidth}\centering\strut
  \strut
  \end{minipage} & \begin{minipage}[t]{0.09\columnwidth}\centering\strut
  \strut
  \end{minipage} & \begin{minipage}[t]{0.09\columnwidth}\centering\strut
  \strut
  \end{minipage} & \begin{minipage}[t]{0.09\columnwidth}\centering\strut
  \strut
  \end{minipage} & \begin{minipage}[t]{0.09\columnwidth}\centering\strut
  \strut
  \end{minipage} & \begin{minipage}[t]{0.09\columnwidth}\centering\strut
  (0.021)**\strut
  \end{minipage} & \begin{minipage}[t]{0.10\columnwidth}\centering\strut
  (0.073)**\strut
  \end{minipage}\tabularnewline
  \begin{minipage}[t]{0.12\columnwidth}\centering\strut
  Other\strut
  \end{minipage} & \begin{minipage}[t]{0.09\columnwidth}\centering\strut
  \strut
  \end{minipage} & \begin{minipage}[t]{0.09\columnwidth}\centering\strut
  \strut
  \end{minipage} & \begin{minipage}[t]{0.09\columnwidth}\centering\strut
  \strut
  \end{minipage} & \begin{minipage}[t]{0.09\columnwidth}\centering\strut
  \strut
  \end{minipage} & \begin{minipage}[t]{0.09\columnwidth}\centering\strut
  \strut
  \end{minipage} & \begin{minipage}[t]{0.09\columnwidth}\centering\strut
  -0.085\strut
  \end{minipage} & \begin{minipage}[t]{0.10\columnwidth}\centering\strut
  -0.124\strut
  \end{minipage}\tabularnewline
  \begin{minipage}[t]{0.12\columnwidth}\centering\strut
  \strut
  \end{minipage} & \begin{minipage}[t]{0.09\columnwidth}\centering\strut
  \strut
  \end{minipage} & \begin{minipage}[t]{0.09\columnwidth}\centering\strut
  \strut
  \end{minipage} & \begin{minipage}[t]{0.09\columnwidth}\centering\strut
  \strut
  \end{minipage} & \begin{minipage}[t]{0.09\columnwidth}\centering\strut
  \strut
  \end{minipage} & \begin{minipage}[t]{0.09\columnwidth}\centering\strut
  \strut
  \end{minipage} & \begin{minipage}[t]{0.09\columnwidth}\centering\strut
  (0.073)\strut
  \end{minipage} & \begin{minipage}[t]{0.10\columnwidth}\centering\strut
  (0.073)\strut
  \end{minipage}\tabularnewline
  \begin{minipage}[t]{0.12\columnwidth}\centering\strut
  UAF\strut
  \end{minipage} & \begin{minipage}[t]{0.09\columnwidth}\centering\strut
  \strut
  \end{minipage} & \begin{minipage}[t]{0.09\columnwidth}\centering\strut
  \strut
  \end{minipage} & \begin{minipage}[t]{0.09\columnwidth}\centering\strut
  \strut
  \end{minipage} & \begin{minipage}[t]{0.09\columnwidth}\centering\strut
  \strut
  \end{minipage} & \begin{minipage}[t]{0.09\columnwidth}\centering\strut
  \strut
  \end{minipage} & \begin{minipage}[t]{0.09\columnwidth}\centering\strut
  -0.308\strut
  \end{minipage} & \begin{minipage}[t]{0.10\columnwidth}\centering\strut
  0.325\strut
  \end{minipage}\tabularnewline
  \begin{minipage}[t]{0.12\columnwidth}\centering\strut
  \strut
  \end{minipage} & \begin{minipage}[t]{0.09\columnwidth}\centering\strut
  \strut
  \end{minipage} & \begin{minipage}[t]{0.09\columnwidth}\centering\strut
  \strut
  \end{minipage} & \begin{minipage}[t]{0.09\columnwidth}\centering\strut
  \strut
  \end{minipage} & \begin{minipage}[t]{0.09\columnwidth}\centering\strut
  \strut
  \end{minipage} & \begin{minipage}[t]{0.09\columnwidth}\centering\strut
  \strut
  \end{minipage} & \begin{minipage}[t]{0.09\columnwidth}\centering\strut
  (0.021)**\strut
  \end{minipage} & \begin{minipage}[t]{0.10\columnwidth}\centering\strut
  (0.021)**\strut
  \end{minipage}\tabularnewline
  \begin{minipage}[t]{0.12\columnwidth}\centering\strut
  Mail Vote\strut
  \end{minipage} & \begin{minipage}[t]{0.09\columnwidth}\centering\strut
  \strut
  \end{minipage} & \begin{minipage}[t]{0.09\columnwidth}\centering\strut
  \strut
  \end{minipage} & \begin{minipage}[t]{0.09\columnwidth}\centering\strut
  \strut
  \end{minipage} & \begin{minipage}[t]{0.09\columnwidth}\centering\strut
  \strut
  \end{minipage} & \begin{minipage}[t]{0.09\columnwidth}\centering\strut
  \strut
  \end{minipage} & \begin{minipage}[t]{0.09\columnwidth}\centering\strut
  23.764\strut
  \end{minipage} & \begin{minipage}[t]{0.10\columnwidth}\centering\strut
  26.502\strut
  \end{minipage}\tabularnewline
  \begin{minipage}[t]{0.12\columnwidth}\centering\strut
  \strut
  \end{minipage} & \begin{minipage}[t]{0.09\columnwidth}\centering\strut
  \strut
  \end{minipage} & \begin{minipage}[t]{0.09\columnwidth}\centering\strut
  \strut
  \end{minipage} & \begin{minipage}[t]{0.09\columnwidth}\centering\strut
  \strut
  \end{minipage} & \begin{minipage}[t]{0.09\columnwidth}\centering\strut
  \strut
  \end{minipage} & \begin{minipage}[t]{0.09\columnwidth}\centering\strut
  \strut
  \end{minipage} & \begin{minipage}[t]{0.09\columnwidth}\centering\strut
  (45.255)**\strut
  \end{minipage} & \begin{minipage}[t]{0.10\columnwidth}\centering\strut
  (285.774)**\strut
  \end{minipage}\tabularnewline
  \begin{minipage}[t]{0.12\columnwidth}\centering\strut
  Age\strut
  \end{minipage} & \begin{minipage}[t]{0.09\columnwidth}\centering\strut
  \strut
  \end{minipage} & \begin{minipage}[t]{0.09\columnwidth}\centering\strut
  \strut
  \end{minipage} & \begin{minipage}[t]{0.09\columnwidth}\centering\strut
  \strut
  \end{minipage} & \begin{minipage}[t]{0.09\columnwidth}\centering\strut
  \strut
  \end{minipage} & \begin{minipage}[t]{0.09\columnwidth}\centering\strut
  \strut
  \end{minipage} & \begin{minipage}[t]{0.09\columnwidth}\centering\strut
  0.093\strut
  \end{minipage} & \begin{minipage}[t]{0.10\columnwidth}\centering\strut
  0.086\strut
  \end{minipage}\tabularnewline
  \begin{minipage}[t]{0.12\columnwidth}\centering\strut
  \strut
  \end{minipage} & \begin{minipage}[t]{0.09\columnwidth}\centering\strut
  \strut
  \end{minipage} & \begin{minipage}[t]{0.09\columnwidth}\centering\strut
  \strut
  \end{minipage} & \begin{minipage}[t]{0.09\columnwidth}\centering\strut
  \strut
  \end{minipage} & \begin{minipage}[t]{0.09\columnwidth}\centering\strut
  \strut
  \end{minipage} & \begin{minipage}[t]{0.09\columnwidth}\centering\strut
  \strut
  \end{minipage} & \begin{minipage}[t]{0.09\columnwidth}\centering\strut
  (0.009)**\strut
  \end{minipage} & \begin{minipage}[t]{0.10\columnwidth}\centering\strut
  (0.009)**\strut
  \end{minipage}\tabularnewline
  \begin{minipage}[t]{0.12\columnwidth}\centering\strut
  AUC\strut
  \end{minipage} & \begin{minipage}[t]{0.09\columnwidth}\centering\strut
  0.543\strut
  \end{minipage} & \begin{minipage}[t]{0.09\columnwidth}\centering\strut
  0.543\strut
  \end{minipage} & \begin{minipage}[t]{0.09\columnwidth}\centering\strut
  \strut
  \end{minipage} & \begin{minipage}[t]{0.09\columnwidth}\centering\strut
  0.545\strut
  \end{minipage} & \begin{minipage}[t]{0.09\columnwidth}\centering\strut
  0.733\strut
  \end{minipage} & \begin{minipage}[t]{0.09\columnwidth}\centering\strut
  0.961\strut
  \end{minipage} & \begin{minipage}[t]{0.10\columnwidth}\centering\strut
  0.963\strut
  \end{minipage}\tabularnewline
  \bottomrule
  \end{longtable}
  
  \chapter*{Conclusion}\label{conclusion}
  \addcontentsline{toc}{chapter}{Conclusion}
  
  \setcounter{chapter}{5} \setcounter{section}{0}
  
  At the end of the day I am unable to draw concrete conclusions on my
  hypotheses. My county level models do provide some indication that my
  second and third hypotheses (H2,H3)--relating to stronger VBM effects on
  lower level elections--is probably false for Colorado, when comparing
  general elections to primaries or coordinated local races. I do not
  however draw conclusions on my first hypothesis; while it seems that the
  effect of mail voting is present, and consistent through time, it
  doesn't seem like this effect is particularly different from any other
  effects like urban population. I would say that this is weak evidence in
  favor of H1, but not enough to solidly confirm it.
  
  Resulting from this, the main contribution of my thesis has to be in
  analysis of data wrangling, the construction of multilevel general
  additive models for turnout, and the accompanying \textit{R} package. To
  take these one by one, I have provided arguments in favor of preferring
  multiple snapshots of registration files rather than just the latest
  iteration of the record. I have analyzed the pitfalls that exist in such
  documents, and given specific examples on how this can be dealt with for
  the Colorado data files. I have also provided a set of variable
  specification that can be useful as indicators of the content of these
  data, or the potential uses of voter registration files in other future
  studies. Finally, I have presented potential future solutions to issues
  these data have with variability, and ways to circumvent processing
  power limitations.
  
  Additionally, I have meticulously gone through the creation of
  multilevel general additive models of individual and county level
  turnout. While due to data and processing power limitations I am unable
  to run all these models. this does not mean that they present no value
  to future research. Quite the contrary, they aid in future studies just
  having to go through the data clean-up stage, and then implement my
  models without having to construct them from scratch. In particular,
  mixed effects and general additive models are not widely used in such
  studies, making their presentation and specification rather unique
  regardless of their application in this piece of research.
  
  Lastly, I provide an extensive library of code used to create this
  document and the research I conduct. I have made an \textit{R}
  package--which I named \texttt{riggd}--that includes more than a dozen
  different functions that serve data wrangling and presentation purposes.
  These functions are made for use on Colorado files, but require
  relatively small amounts of changes to be applied to voter files from
  around the US. I also provide code for all tables and graphics that are
  included in this thesis on gitHub, which is a testament to the
  reproducibility and future value of the research I conducted.
  
  I recognize that, despite many obstacles in terms of data or computing
  power, the outcome of this thesis being more constructive rather than
  conclusive is to some extent my fault. There were many problems in this
  thesis that I should have been aware of earlier in the process, which
  may have allowed me to present some concrete result rather than a series
  of tools and methods. However, in the combination of my existing
  conclusions and the materials I have created through this process, it is
  my belief that this thesis does in fact present a step forward in the
  literature, and that it adds to existing quantitative elections studies
  works. This is a small step, but it helps in our understanding of how
  the franchise of democracy governs itself, and what the actual results
  of elections policy are.
  
  \appendix
  
  \backmatter
  
  \chapter{References}\label{references}
  
  \noindent
  
  \setlength{\parindent}{-0.20in} \setlength{\leftskip}{0.20in}
  \setlength{\parskip}{8pt}
  
  \hypertarget{refs}{}
  \hypertarget{ref-aldrich_rational_1993}{}
  Aldrich, J. H. (1993). Rational Choice and Turnout. \emph{American
  Journal of Political Science}, \emph{37}(1), 246--278.
  \url{http://doi.org/10.2307/2111531}
  
  \hypertarget{ref-ansolabehere_quality_2010}{}
  Ansolabehere, S., \& Hersh, E. (2010). The Quality of Voter Registration
  Records: A State-by-State Analysis. \emph{Institute for Quantitative
  Social Science and Caltech/MIT Voting Technology Project Working Paper}.
  Retrieved from
  \url{https://dataverse.harvard.edu/dataset.xhtml?persistentId=hdl:1902.1/18550}
  
  \hypertarget{ref-ansolabehere_validation:_2012}{}
  Ansolabehere, S., \& Hersh, E. (2012). Validation: What Big Data Reveal
  About Survey Misreporting and the Real Electorate. \emph{Political
  Analysis}, \emph{20}(04), 437--459.
  \url{http://doi.org/10.1093/pan/mps023}
  
  \hypertarget{ref-ansolabehere_adgn:_2017}{}
  Ansolabehere, S., \& Hersh, E. D. (2017). ADGN: An Algorithm for Record
  Linkage Using Address, Date of Birth, Gender, and Name. \emph{Statistics
  and Public Policy}, \emph{4}(1), 1--10.
  \url{http://doi.org/10.1080/2330443X.2017.1389620}
  
  \hypertarget{ref-colorado_general_assembly_tabor_nodate}{}
  Assembly, C. G. (n.d.). TABOR. \emph{Colorado General Assembly}.
  Retrieved from
  \url{https://public.tableau.com/views/TABOR/TABORDash?:showVizHome=no:embed=y\&:display_count=no}
  
  \hypertarget{ref-barr_comprehensive_2012}{}
  Barr, C. D., Diez, D. M., Wang, Y., Dominici, F., \& Samet, J. M.
  (2012). Comprehensive Smoking Bans and Acute Myocardial Infarction Among
  Medicare Enrollees in 387 US Counties: 1999--2008. \emph{American
  Journal of Epidemiology}, \emph{176}(7), 642--648.
  \url{http://doi.org/10.1093/aje/kws267}
  
  \hypertarget{ref-bates_fitting_2014}{}
  Bates, D., Mächler, M., Bolker, B., \& Walker, S. (2014). Fitting Linear
  Mixed-Effects Models using lme4. \emph{arXiv:1406.5823 {[}Stat{]}}.
  Retrieved from \url{http://arxiv.org/abs/1406.5823}
  
  \hypertarget{ref-bergman_changing_2011}{}
  Bergman, E., \& Yates, P. A. (2011). Changing Election Methods: How Does
  Mandated Vote-By-Mail Affect Individual Registrants? \emph{Election Law
  Journal: Rules, Politics, and Policy}, \emph{10}(2), 115--127.
  \url{http://doi.org/10.1089/elj.2010.0079}
  
  \hypertarget{ref-berinsky_perverse_2005}{}
  Berinsky, A. J. (2005). The Perverse Consequences of Electoral Reform in
  the United States. \emph{American Politics Research}, \emph{33}(4),
  471--491. \url{http://doi.org/10.1177/1532673X04269419}
  
  \hypertarget{ref-burden_voter_2000}{}
  Burden, B. C. (2000). Voter Turnout and the National Election Studies.
  \emph{Political Analysis}, \emph{8}(4), 389--398.
  \url{http://doi.org/10.1093/oxfordjournals.pan.a029823}
  
  \hypertarget{ref-burden_new_1998}{}
  Burden, B. C., \& Kimball, D. C. (1998). A New Approach to the Study of
  Ticket Splitting. \emph{The American Political Science Review},
  \emph{92}(3), 533--544. \url{http://doi.org/10.2307/2585479}
  
  \hypertarget{ref-burden_election_2013}{}
  Burden, B. C., \& Neiheisel, J. R. (2013). Election Administration and
  the Pure Effect of Voter Registration on Turnout. \emph{Political
  Research Quarterly}, \emph{66}(1), 77--90.
  \url{http://doi.org/10.1177/1065912911430671}
  
  \hypertarget{ref-burden_election_2014}{}
  Burden, B. C., Canon, D. T., Mayer, K. R., \& Moynihan, D. P. (2014).
  Election Laws, Mobilization, and Turnout: The Unanticipated Consequences
  of Election Reform. \emph{American Journal of Political Science},
  \emph{58}(1), 95--109. \url{http://doi.org/10.1111/ajps.12063}
  
  \hypertarget{ref-us_census_bureau_us_2010}{}
  Bureau, U. C. (2010). US Census Bureau QuickFacts: Colorado. Retrieved
  from \url{https://www.census.gov/quickfacts/co}
  
  \hypertarget{ref-campbell_self-interest_2002}{}
  Campbell, A. L. (2002). Self-Interest, Social Security, and the
  Distinctive Participation Patterns of Senior Citizens. \emph{American
  Political Science Review}, \emph{96}(3), 565--574.
  \url{http://doi.org/10.1017/S0003055402000333}
  
  \hypertarget{ref-chen_voter_2013}{}
  Chen, J. (2013). Voter Partisanship and the Effect of Distributive
  Spending on Political Participation. \emph{American Journal of Political
  Science}, \emph{57}(1), 200--217.
  \url{http://doi.org/10.1111/j.1540-5907.2012.00613.x}
  
  \hypertarget{ref-chihara_mathematical_2011}{}
  Chihara, L. M., \& Hesterberg, T. C. (2011). \emph{Mathematical
  Statistics with Resampling and R} (1 edition). Hoboken, N.J: Wiley.
  
  \hypertarget{ref-cronin_colorado_2012}{}
  Cronin, T. E., \& Loevy, R. D. (2012). \emph{Colorado Politics and
  Policy: Governing a Purple State}. Lincoln, UNITED STATES: UNP -
  Nebraska Paperback. Retrieved from
  \url{http://ebookcentral.proquest.com/lib/reed/detail.action?docID=1034959}
  
  \hypertarget{ref-deufel_race_2010}{}
  Deufel, B. J., \& Kedar, O. (2010). Race And Turnout In U.S. Elections
  Exposing Hidden Effects. \emph{Public Opinion Quarterly}, \emph{74}(2),
  286--318. \url{http://doi.org/10.1093/poq/nfq017}
  
  \hypertarget{ref-edelman_analysis_2018}{}
  Edelman, G., \& Glastris, P. (2018). Analysis Letting people vote at
  home increases voter turnout. Here's proof. \emph{Washington Post}.
  Retrieved from
  \url{https://www.washingtonpost.com/outlook/letting-people-vote-at-home-increases-voter-turnout-heres-proof/2018/01/26/d637b9d2-017a-11e8-bb03-722769454f82_story.html}
  
  \hypertarget{ref-edlin_voting_2007}{}
  Edlin, A., Gelman, A., \& Kaplan, N. (2007). Voting as a Rational
  Choice: Why and How People Vote To Improve the Well-Being of Others.
  \emph{Rationality and Society}, \emph{19}(3), 293--314.
  \url{http://doi.org/10.1177/1043463107077384}
  
  \hypertarget{ref-fortier_absentee_2006}{}
  Fortier, J. C. (2006). \emph{Absentee and early voting: Trends,
  promises, and perils}. Washington, DC: AEI Press.
  
  \hypertarget{ref-fowler_habitual_2006}{}
  Fowler, J. H. (2006). Habitual Voting and Behavioral Turnout.
  \emph{Journal of Politics}, \emph{68}(2), 335--344.
  \url{http://doi.org/10.1111/j.1468-2508.2006.00410.x}
  
  \hypertarget{ref-gelman_data_2006}{}
  Gelman, A., \& Hill, J. (2006). \emph{Data Analysis Using Regression and
  Multilevel/Hierarchical Models} (1 edition). Cambridge ; New York:
  Cambridge University Press.
  
  \hypertarget{ref-gerber_identifying_2013}{}
  Gerber, A. S., Huber, G. A., \& Hill, S. J. (2013). Identifying the
  Effect of All-Mail Elections on Turnout: Staggered Reform in the
  Evergreen State\textless{}a
  href=``\#fn2606''\textgreater{}*\textless{}/a\textgreater{}.
  \emph{Political Science Research and Methods}, \emph{1}(1), 91--116.
  \url{http://doi.org/10.1017/psrm.2013.5}
  
  \hypertarget{ref-geys_explaining_2006}{}
  Geys, B. (2006). Explaining voter turnout: A review of aggregate-level
  research. \emph{Electoral Studies}, \emph{25}(4), 637--663.
  \url{http://doi.org/10.1016/j.electstud.2005.09.002}
  
  \hypertarget{ref-gronke_voting_2012}{}
  Gronke, P., \& Miller, P. (2012). Voting by Mail and Turnout in Oregon:
  Revisiting Southwell and Burchett. \emph{American Politics Research},
  \emph{40}(6), 976--997. \url{http://doi.org/10.1177/1532673X12457809}
  
  \hypertarget{ref-gronke_convenience_2008}{}
  Gronke, P., Galanes-Rosenbaum, E., Miller, P. A., \& Toffey, D. (2008).
  Convenience Voting. \emph{Annual Review of Political Science},
  \emph{11}(1), 437--455.
  \url{http://doi.org/10.1146/annurev.polisci.11.053006.190912}
  
  \hypertarget{ref-hamm_how_2017}{}
  Hamm, K. (2017). How Colorado has voted in presidential elections (and
  how its politics have changed) since 1980. \emph{The Denver Post}.
  Retrieved from
  \url{https://www.denverpost.com/2017/12/22/how-colorado-votes/}
  
  \hypertarget{ref-handayani_comparative_2017}{}
  Handayani, D., Notodiputro, K. A., Sadik, K., \& Kurnia, A. (2017). A
  comparative study of approximation methods for maximum likelihood
  estimation in generalized linear mixed models (GLMM). In (p. 020033).
  Jawa Barat, Indonesia. \url{http://doi.org/10.1063/1.4979449}
  
  \hypertarget{ref-hersh_hacking_2015}{}
  Hersh, E. D. (2015). \emph{Hacking the Electorate: How Campaigns
  Perceive Voters}. New York, NY: Cambridge University Press.
  
  \hypertarget{ref-james_introduction_2017}{}
  James, G., Witten, D., Hastie, T., \& Tibshirani, R. (2017). \emph{An
  Introduction to Statistical Learning: With Applications in R} (1st ed.
  2013, Corr. 7th printing 2017 edition). New York: Springer.
  
  \hypertarget{ref-keele_geographic_2017}{}
  Keele, L., \& Titiunik, R. (2017). Geographic Natural Experiments with
  Interference: The Effect of All-Mail Voting on Turnout in Colorado.
  
  \hypertarget{ref-martin_colorado_1962}{}
  Martin, C. (1962). \emph{Colorado politics} (2ond ed.). Denver,
  Colorado: Big Mountain Press. Retrieved from
  \url{http://hdl.handle.net/2027/mdp.39015024371158}
  
  \hypertarget{ref-matsusaka_voter_1999}{}
  Matsusaka, J. G., \& Palda, F. (1999). Voter turnout: How much can we
  explain? \emph{Public Choice}, \emph{98}(3-4), 431--446.
  \url{http://doi.org/10.1023/A:1018328621580}
  
  \hypertarget{ref-mcclelland_synthetic_2017}{}
  McClelland, R., \& Gault, S. (2017). The Synthetic Control Method as a
  Tool to Understand State Policy. \emph{Washington, DC: Urban-Brookings
  Tax Policy Center}.
  
  \hypertarget{ref-mcdonald_what_nodate}{}
  McDonald, M. P. (n.d.). What is the voting-age population (VAP) and the
  voting-eligible population (VEP)? \emph{United States Elections
  Project}. Retrieved from
  \url{http://www.electproject.org/home/voter-turnout/faq/denominator}
  
  \hypertarget{ref-mettler_government_2008}{}
  Mettler, S., \& Stonecash, J. M. (2008). Government Program Usage and
  Political Voice*. \emph{Social Science Quarterly}, \emph{89}(2),
  273--293. \url{http://doi.org/10.1111/j.1540-6237.2008.00532.x}
  
  \hypertarget{ref-neiheisel_impact_2012}{}
  Neiheisel, J. R., \& Burden, B. C. (2012). The Impact of Election Day
  Registration on Voter Turnout and Election Outcomes. \emph{American
  Politics Research}, \emph{40}(4), 636--664.
  \url{http://doi.org/10.1177/1532673X11432470}
  
  \hypertarget{ref-plutzer_becoming_2002}{}
  Plutzer, E. (2002). Becoming a Habitual Voter: Inertia, Resources, and
  Growth in Young Adulthood. \emph{The American Political Science Review},
  \emph{96}(1), 41--56. Retrieved from
  \url{https://www.jstor.org/stable/3117809}
  
  \hypertarget{ref-richey_sean_voting_2008}{}
  Richey Sean. (2008). Voting by Mail: Turnout and Institutional Reform in
  Oregon*. \emph{Social Science Quarterly}, \emph{89}(4), 902--915.
  \url{http://doi.org/10.1111/j.1540-6237.2008.00590.x}
  
  \hypertarget{ref-robert_nay_help_2002}{}
  Robert Nay. (2002). The Help America Vote Act of 2002.
  
  \hypertarget{ref-rosenstone_mobilization_2003}{}
  Rosenstone, S. J. (2003). \emph{Mobilization, participation, and
  democracy in America}. New York: Longman.
  
  \hypertarget{ref-schneider_behavioral_1990}{}
  Schneider, A., \& Ingram, H. (1990). Behavioral Assumptions of Policy
  Tools. \emph{The Journal of Politics}, \emph{52}(2), 510--529.
  \url{http://doi.org/10.2307/2131904}
  
  \hypertarget{ref-smets_embarrassment_2013}{}
  Smets, K., \& Ham, C. van. (2013). The embarrassment of riches? A
  meta-analysis of individual-level research on voter turnout.
  \emph{Electoral Studies}, \emph{32}(2), 344--359.
  \url{http://doi.org/10.1016/j.electstud.2012.12.006}
  
  \hypertarget{ref-national_council_of_state_legislatures_absentee_nodate}{}
  State Legislatures, N. C. of. (n.d.). Absentee and Early Voting.
  \emph{National Council of State Legislatures}. Retrieved from
  \url{http://www.ncsl.org/research/elections-and-campaigns/absentee-and-early-voting.aspx\#a}
  
  \hypertarget{ref-stein_engaging_2008}{}
  Stein, R. M., \& Vonnahme, G. (2008). Engaging the Unengaged Voter: Vote
  Centers and Voter Turnout. \emph{The Journal of Politics}, \emph{70}(2),
  487--497. \url{http://doi.org/10.1017/S0022381608080456}
  
  \hypertarget{ref-thompson_first_2016}{}
  Thompson, J. (2016). The first Sagebrush Rebellion: What sparked it and
  how it ended. Retrieved from
  \url{https://www.hcn.org/articles/a-look-back-at-the-first-sagebrush-rebellion}
  
  \hypertarget{ref-wood_generalized_2006}{}
  Wood, S. N. (2006). \emph{Generalized Additive Models: An Introduction
  with R} (1 edition). Boca Raton, FL: Chapman; Hall/CRC.
  
  \hypertarget{ref-wood_gamm4:_2017}{}
  Wood, S., \& Scheipl, F. (2017, July). Gamm4: Generalized Additive Mixed
  Models using 'mgcv' and 'lme4'. Retrieved from
  \url{https://CRAN.R-project.org/package=gamm4}


  % Index?

\end{document}

