% This is the Reed College LaTeX thesis template. Most of the work
% for the document class was done by Sam Noble (SN), as well as this
% template. Later comments etc. by Ben Salzberg (BTS). Additional
% restructuring and APA support by Jess Youngberg (JY).
% Your comments and suggestions are more than welcome; please email
% them to cus@reed.edu
%
% See http://web.reed.edu/cis/help/latex.html for help. There are a
% great bunch of help pages there, with notes on
% getting started, bibtex, etc. Go there and read it if you're not
% already familiar with LaTeX.
%
% Any line that starts with a percent symbol is a comment.
% They won't show up in the document, and are useful for notes
% to yourself and explaining commands.
% Commenting also removes a line from the document;
% very handy for troubleshooting problems. -BTS

% As far as I know, this follows the requirements laid out in
% the 2002-2003 Senior Handbook. Ask a librarian to check the
% document before binding. -SN

%%
%% Preamble
%%
% \documentclass{<something>} must begin each LaTeX document
\documentclass[12pt,twoside]{reedthesis}
% Packages are extensions to the basic LaTeX functions. Whatever you
% want to typeset, there is probably a package out there for it.
% Chemistry (chemtex), screenplays, you name it.
% Check out CTAN to see: http://www.ctan.org/
%%
\usepackage{graphicx,latexsym}
\usepackage{amsmath}
\usepackage{amssymb,amsthm}
\usepackage{longtable,booktabs,setspace}
\usepackage{chemarr} %% Useful for one reaction arrow, useless if you're not a chem major
\usepackage[hyphens]{url}
% Added by CII
\usepackage{hyperref}
\usepackage{lmodern}
% End of CII addition
\usepackage{rotating}

% Next line commented out by CII
%%% \usepackage{natbib}
% Comment out the natbib line above and uncomment the following two lines to use the new 
% biblatex-chicago style, for Chicago A. Also make some changes at the end where the 
% bibliography is included. 
%\usepackage{biblatex-chicago}
%\bibliography{thesis}


% Added by CII (Thanks, Hadley!)
% Use ref for internal links
\renewcommand{\hyperref}[2][???]{\autoref{#1}}
\def\chapterautorefname{Chapter}
\def\sectionautorefname{Section}
\def\subsectionautorefname{Subsection}
% End of CII addition

% Added by CII 
\usepackage{caption}
\captionsetup{width=5in}
% End of CII addition

% \usepackage{times} % other fonts are available like times, bookman, charter, palatino


% To pass between YAML and LaTeX the dollar signs are added by CII
\title{How I learned to stop worrying and Love Voter Registration Files}
\author{Theodore Dounias}
% The month and year that you submit your FINAL draft TO THE LIBRARY (May or December)
\date{December 2018}
\division{Mathematics and Natural Sciences and History and Social Sciences}
\advisor{Andrew Bray}
%If you have two advisors for some reason, you can use the following
% Uncommented out by CII
\altadvisor{Paul Gronke} 
% End of CII addition

%%% Remember to use the correct department!
\department{Mathematics and Political Science}
% if you're writing a thesis in an interdisciplinary major,
% uncomment the line below and change the text as appropriate.
% check the Senior Handbook if unsure.
%\thedivisionof{The Established Interdisciplinary Committee for}
% if you want the approval page to say "Approved for the Committee",
% uncomment the next line
%\approvedforthe{Committee}

% Added by CII
%%% Copied from knitr
%% maxwidth is the original width if it's less than linewidth
%% otherwise use linewidth (to make sure the graphics do not exceed the margin)
\makeatletter
\def\maxwidth{ %
  \ifdim\Gin@nat@width>\linewidth
    \linewidth
  \else
    \Gin@nat@width
  \fi
}
\makeatother

\renewcommand{\contentsname}{Table of Contents}
% End of CII addition

\setlength{\parskip}{0pt}

% Added by CII

\providecommand{\tightlist}{%
  \setlength{\itemsep}{0pt}\setlength{\parskip}{0pt}}

\Acknowledgements{

}

\Dedication{

}

\Preface{
This is an example of a thesis setup to use the reed thesis document
class.
}

\Abstract{

}

% End of CII addition
%%
%% End Preamble
%%
%

\begin{document}

% Everything below added by CII
      \maketitle
  
  \frontmatter % this stuff will be roman-numbered
  \pagestyle{empty} % this removes page numbers from the frontmatter

  
      \begin{preface}
      This is an example of a thesis setup to use the reed thesis document
      class.
    \end{preface}
  
      \hypersetup{linkcolor=black}
    \setcounter{tocdepth}{3}
    \tableofcontents
  
      \listoftables
  
      \listoffigures
  
  
  
  \mainmatter % here the regular arabic numbering starts
  \pagestyle{fancyplain} % turns page numbering back on

  \chapter*{Introduction}\label{introduction}
  \addcontentsline{toc}{chapter}{Introduction}
  
  The democratic system is based on procedures as much as principles. The
  way that democracies chose to tally the will of the people is always a
  messy, controversial process. Thus the design and implementation of
  voting systems is far from being neutral; the decisions made on who
  votes, and how, when, and where they do so is inherently coupled with
  the outcome. Underlying those decisions is a nebulous, inconclusively
  answered question: are elections fair, and how can we make them more so.
  
  The passage of the Help America Vote Act---or HAVA--(Robert Nay, 2002),
  which mandated states to update and consolidate public voter
  registration files, and created the US Elections Assistance Commission
  that makes available county level data, innovated the way we use data
  based approaches to answer this question. HAVA offered political
  scientists and statisticians direct access to the voting population's
  voting patterns, political registration, age, geolocation and much more;
  information that up to then was only accessible by sampling through
  surveys. The immense leap here happens because true population data does
  away with the need for sampling techniques that are often biased and
  inaccurate. We can now not only get a complete picture of the data, but
  also link and merge with other sources of information such as US Census
  data on religion, race, education, or income---work that has been
  lucrative for firms such as Catalist or Target Smart. By posing
  Political Scientific questions, and trying to respond with rigorous
  statistics, both disciplines tackle these data to face joint problems
  such as quantifying the quality of voter registration files
  (Ansolabehere \& Hersh, 2010), or linking disparate voter records
  (Ansolabehere \& Hersh, 2017).
  
  \chapter{The State of the Literature}\label{rmd-basics}
  
  In this chapter I will go through the exsting literature on Vote-By-Mail
  (VBM). I will define what Vote-By-Mail is; I will then summarize the
  expectations that researchers have of the effects of VBM on turnout,
  based on existing theories of electoral participation. I will continue
  with a summary of previous quantitative research on the effects that VBM
  and similar policies have had on turnout. I will conclude with some more
  general comments on the available data, and literature concerning the
  most commonly used quantitative methods.
  
  \section{What is VBM?}\label{what-is-vbm}
  
  Gronke (2007, 2008), RMStein (1998)
  
  Vote-By-Mail is a process by which voters receive a ballot delivered by
  mail to their homes. Voters then have a variety of options on how to
  return these ballots, ranging from dropping them off at pre-designated
  locations, to mailing them in, to bringing them to a polling place and
  voting conventionally. This varies across states that have implemented
  VBM. Some common forms of the VBM policy are:
  
  \begin{itemize}
  \tightlist
  \item
    \emph{Postal Voting}: All voters receive a ballot by mail, which can
    then be returned to a pre-designated location or mailed in to be
    counted. This is the current system in Oregon, is an option in
    Colorado, and is implimented by a number of counties in California,
    Utah, and Montana.\\
  \item
    \emph{No-Excuse Absentee}: Voters can choose to register as absentee
    voters without giving any reason related to disability, health,
    distance to polling place etc. This is the case in 27 states and the
    District of Columbia.\\
  \item
    \emph{Permanent No-Excuse Absentee}: This is similar to the previous
    system, but allows voters to register as absentees indefinitely,
    without having to renew their registration each year; they become de
    facto all-mail voters. This is in place in Washington, Kansas, and New
    Jersey.\\
  \item
    \emph{Hybrid or Transitionary Systems}: In hybrid systems, voters
    receive a mail ballot but can choose to disregard it and vote
    conventionaly. This is the case in Colorado. Transitionary systems
    exist in states that have chosen to eventually conduct all elections
    by postal voting, but have given counties an adjustment period during
    which this shift is not mandatory, or mandatory only for certain
    elections. This is the case in California, Utah, and Montana.
  \end{itemize}
  
  Vote-By-Mail is also commonly considered a type of early voting, since
  voters receive their ballots around two weeks in advance of election
  day; they are also able to return that ballot whenever they wish within
  that timeframe. This means that Vote-By-Mail can be counted as a
  ``convenience voting'' reform. These are usually impliented by state and
  local governments with the argument that they either expand the
  democratic franchise by bringing in new voters, or by making it more
  likely that current registered voters participate in the electoral
  process.
  
  \section{The Calculus of Voting}\label{the-calculus-of-voting}
  
  \emph{Grimmer (2011), Burden (2013)}
  
  \subsection{Why Turnout Matters}\label{why-turnout-matters}
  
  \emph{Geys (2006), Smets (2013) ++ book sources}
  
  Turnout is the most commonly used measure for electoral participation.
  It is important because it signifies the level of engagement of the
  population with the state, the elevel of incorporation of different
  subgroups of the population into democratic processes, and the
  legitimacy of elected officials. It is widely accepted that turnout
  should be maximized so that the democratic franchise represents the
  majority of citizens. Turnout for an election can be calculated or
  predicted, the difference being that in the former case we use data
  post-election to measure its absolute value, while in the latter we use
  a series of individual and community covariates to infer the levels of
  turnout for a future or past election.\\
  Calculating turnout, at its core, involves the following equation:\\
  \[ \% ~Turnout = \frac{Total~Ballots~Cast}{Measure~of~Total~Voting~Population}\times100\%~~~(1)\]
  
  The choice of numerator is fairly obvious and universal; the
  denominator, however, is a different story. The two main statistics used
  are the total voting age population, and the raw number of registered
  voters in the geographical location we are examining. The total voting
  age population is used as a measure to incorporate the total amount of
  possible voters in a geographical area, and can be measured using data
  from the US Census. This causes some issues with voters that cross over
  to different districts; if someone lives in district A, it is still
  likelly that they are registered to vote in district B. If this is not
  considered, the calculation of voting age population might be
  misrepresentative.\\
  Using registered voters also brings with it two problems. First, the
  calculation necessarily occurs using voter registration files, which
  many times can include discrepancies, like deceased voters, voters
  included in multiple counties, or individual voters included multiple
  times. Furthermore, the total amount of actual voters among registered
  voters can be misrepresentative of democratic participation; consider
  that if a certain minority community has historically low registration
  rates, their lack of engagement will not be included in turnout rates,
  thus misrepresenting the level of inclusion in the district they reside
  in.\\
  The punch line here is that how the turnout statistic is calculated is
  not a clear choice, and will have an impact on how studies are set up.
  To give one example, consider Oregon's Motor Voter program, that
  automatically registers voters when they interact with government
  services, like the DMV. It is conceivable that this reform will
  \emph{decrease} turnout when measured as a percentage of the total
  registered voter count, but \emph{increase} turnout when measured
  against total population. I will specify how I calculate turnout in the
  next chapter.\\
  Statistical models of turnout can be constructed at either the
  individual or community level. At the individual level, a model is built
  to predict the probability of voting for every member of a group, and
  then sum over the members to create an estimate for turnout. Probit or
  Logit models are prefered. At the community level, researchers first
  choose a geographical level at which to calculate, which then
  constitutes the individual observation in the data that is used to
  create the model.\\
  Both these models include a standard set of societal variables--at the
  individual and aggregate level--, policy variables--whether the district
  does Postal Voting, whether Voter ID requirements are particularly
  strict--, and sometmes time-series data--previous levels of turnout--to
  make predictions on turnout levels. This type of analysis is not
  exclusivelly used to predict turnout but also to, as will be later
  shown, draw inferences on the effects that certain explanatory variables
  have on electoral participation.
  
  \subsection{Theories of Voting}\label{theories-of-voting}
  
  \emph{Aldrich (1993), Berinsky (2001, 2005), Edin (2007), Bendor (2003),
  Gerber and Green (2015), Matsusaka (1997), Fowler (2006)}
  
  \subsection{How they Apply to VBM}\label{how-they-apply-to-vbm}
  
  \emph{Berisnky (2005), Banducci (2000), Gronke and Toffey (2008), and
  several applications of sources in above sections.}
  
  \section{Previous Study Results}\label{previous-study-results}
  
  \subsection{General Results}\label{general-results}
  
  \emph{Arcenaux (2012), Bergman (2011), Burden (2014), Edelman + Pantheon
  Analytics (2018), Gerber (2013), Rhine (1995), Neihelsen (2012), Keele
  (2018), Richey (2008), RMStein (1997, 2007), Gronke (2007, 2008, 2012,
  2017)}
  
  \subsection{The Gerber Piece}\label{the-gerber-piece}
  
  \emph{Gerber(2017)}
  
  \section{Voter Registration Files}\label{voter-registration-files}
  
  \subsection{Inaccuracy of Survey Data}\label{inaccuracy-of-survey-data}
  
  \emph{Ansolabehere and Hersch (2012), Burden (2000), Deufel (2010)}
  
  \subsection{The Importance of VRF}\label{the-importance-of-vrf}
  
  \emph{Books, mentioned later}
  
  \section{Common Methods Used and Problems
  Encountered}\label{common-methods-used-and-problems-encountered}
  
  \subsection{Methods}\label{methods}
  
  \begin{itemize}
  \tightlist
  \item
    \emph{Synthetic Control Group}: Abadie (2010), McClleland (2017),
    Gronke (2017)
  \item
    \emph{Record Linkage}: Ansolabehere and Hersch (2017), Harvey (1994,
    97), Koudas (2013)
  \item
    \emph{GLM (Probit/Logit/Poisson)}: Barreto (2004), Dow (2004)
  \item
    \emph{DID}: Bertrand et al. (2002)
  \item
    \emph{E.I.}: King (2013), Burden (1998), Calvo (2003), Chao (2004),
    RMStein (2002)
  \item
    \emph{Mixed-Effects}: Gelman and Hill (2007)
  \item
    \emph{General EDA and Models}: James et al. (2013), Chapman and Hall
    (2017)
  \end{itemize}
  
  \subsection{Issues}\label{issues}
  
  \emph{Grimmer (2015) \{Not always best to do inferential models\},
  Ansolabehere and Hersch (2010) \{Problems with Voter Reg Files\}, other
  sources from the previous section}
  
  \chapter{Hypothesis, Data, and
  Methods}\label{hypothesis-data-and-methods}
  
  \section{The Data}\label{the-data}
  
  \subsection{Source}\label{source}
  
  \subsection{Structure}\label{structure}
  
  \subsection{Wrangling}\label{wrangling}
  
  \section{Hypotheses}\label{hypotheses}
  
  \subsection{Description of Hypotheses}\label{description-of-hypotheses}
  
  \subsection{Criteria}\label{criteria}
  
  \subsection{Expected Results}\label{expected-results}
  
  \section{Methodology}\label{methodology}
  
  \subsection{EDA}\label{eda}
  
  \subsection{Description and Parametrization of
  Models}\label{description-and-parametrization-of-models}
  
  \section{Gerber Replication}\label{gerber-replication}
  
  \chapter{Results}\label{results}
  
  \section{EDA}\label{eda-1}
  
  \section{Model Output}\label{model-output}
  
  \section{Gerber Expansion Results}\label{gerber-expansion-results}
  
  \chapter*{Conclusion}\label{conclusion}
  \addcontentsline{toc}{chapter}{Conclusion}
  
  \setcounter{chapter}{4} \setcounter{section}{0}
  
  \appendix
  
  \backmatter
  
  \chapter{References}\label{references}
  
  \noindent
  
  \setlength{\parindent}{-0.20in} \setlength{\leftskip}{0.20in}
  \setlength{\parskip}{8pt}


  % Index?

\end{document}

