% This is the Reed College LaTeX thesis template. Most of the work
% for the document class was done by Sam Noble (SN), as well as this
% template. Later comments etc. by Ben Salzberg (BTS). Additional
% restructuring and APA support by Jess Youngberg (JY).
% Your comments and suggestions are more than welcome; please email
% them to cus@reed.edu
%
% See http://web.reed.edu/cis/help/latex.html for help. There are a
% great bunch of help pages there, with notes on
% getting started, bibtex, etc. Go there and read it if you're not
% already familiar with LaTeX.
%
% Any line that starts with a percent symbol is a comment.
% They won't show up in the document, and are useful for notes
% to yourself and explaining commands.
% Commenting also removes a line from the document;
% very handy for troubleshooting problems. -BTS

% As far as I know, this follows the requirements laid out in
% the 2002-2003 Senior Handbook. Ask a librarian to check the
% document before binding. -SN

%%
%% Preamble
%%
% \documentclass{<something>} must begin each LaTeX document
\documentclass[12pt,twoside]{reedthesis}
% Packages are extensions to the basic LaTeX functions. Whatever you
% want to typeset, there is probably a package out there for it.
% Chemistry (chemtex), screenplays, you name it.
% Check out CTAN to see: http://www.ctan.org/
%%
\usepackage{graphicx,latexsym}
\usepackage{amsmath}
\usepackage{amssymb,amsthm}
\usepackage{longtable,booktabs,setspace}
\usepackage{chemarr} %% Useful for one reaction arrow, useless if you're not a chem major
\usepackage[hyphens]{url}
% Added by CII
\usepackage{hyperref}
\usepackage{lmodern}
% End of CII addition
\usepackage{rotating}

% Next line commented out by CII
%%% \usepackage{natbib}
% Comment out the natbib line above and uncomment the following two lines to use the new 
% biblatex-chicago style, for Chicago A. Also make some changes at the end where the 
% bibliography is included. 
%\usepackage{biblatex-chicago}
%\bibliography{thesis}


% Added by CII (Thanks, Hadley!)
% Use ref for internal links
\renewcommand{\hyperref}[2][???]{\autoref{#1}}
\def\chapterautorefname{Chapter}
\def\sectionautorefname{Section}
\def\subsectionautorefname{Subsection}
% End of CII addition

% Added by CII 
\usepackage{caption}
\captionsetup{width=5in}
% End of CII addition

% \usepackage{times} % other fonts are available like times, bookman, charter, palatino


% To pass between YAML and LaTeX the dollar signs are added by CII
\title{How I learned to stop worrying and Love Voter Registration Files}
\author{Theodore Dounias}
% The month and year that you submit your FINAL draft TO THE LIBRARY (May or December)
\date{December 2018}
\division{Mathematics and Natural Sciences and History and Social Sciences}
\advisor{Andrew Bray}
%If you have two advisors for some reason, you can use the following
% Uncommented out by CII
\altadvisor{Paul Gronke} 
% End of CII addition

%%% Remember to use the correct department!
\department{Mathematics and Political Science}
% if you're writing a thesis in an interdisciplinary major,
% uncomment the line below and change the text as appropriate.
% check the Senior Handbook if unsure.
%\thedivisionof{The Established Interdisciplinary Committee for}
% if you want the approval page to say "Approved for the Committee",
% uncomment the next line
%\approvedforthe{Committee}

% Added by CII
%%% Copied from knitr
%% maxwidth is the original width if it's less than linewidth
%% otherwise use linewidth (to make sure the graphics do not exceed the margin)
\makeatletter
\def\maxwidth{ %
  \ifdim\Gin@nat@width>\linewidth
    \linewidth
  \else
    \Gin@nat@width
  \fi
}
\makeatother

\renewcommand{\contentsname}{Table of Contents}
% End of CII addition

\setlength{\parskip}{0pt}

% Added by CII

\providecommand{\tightlist}{%
  \setlength{\itemsep}{0pt}\setlength{\parskip}{0pt}}

\Acknowledgements{

}

\Dedication{

}

\Preface{
This is an example of a thesis setup to use the reed thesis document
class.
}

\Abstract{

}

% End of CII addition
%%
%% End Preamble
%%
%

\begin{document}

% Everything below added by CII
      \maketitle
  
  \frontmatter % this stuff will be roman-numbered
  \pagestyle{empty} % this removes page numbers from the frontmatter

  
      \begin{preface}
      This is an example of a thesis setup to use the reed thesis document
      class.
    \end{preface}
  
      \hypersetup{linkcolor=black}
    \setcounter{tocdepth}{3}
    \tableofcontents
  
      \listoftables
  
      \listoffigures
  
  
  
  \mainmatter % here the regular arabic numbering starts
  \pagestyle{fancyplain} % turns page numbering back on

  \chapter*{Introduction}\label{introduction}
  \addcontentsline{toc}{chapter}{Introduction}
  
  The democratic system is based on procedures as much as principles. The
  way that democracies chose to tally the will of the people is always a
  messy, controversial process. Thus the design and implementation of
  voting systems is far from being neutral; the decisions made on who
  votes, and how, when, and where they do so is inherently coupled with
  the outcome. Underlying those decisions is a nebulous, inconclusively
  answered question: are elections fair, and how can we make them more so.
  
  The passage of the Help America Vote Act---or HAVA--(\emph{HAVA}, 2002),
  which mandated states to update and consolidate public voter
  registration files, and created the US Elections Assistance Commission
  that makes available county level data, innovated the way we use data
  based approaches to answer this question. HAVA offered political
  scientists and statisticians direct access to the voting population's
  voting patterns, political registration, age, geolocation and much more;
  information that up to then was only accessible by sampling through
  surveys. The immense leap here happens because true population data does
  away with the need for sampling techniques that are often biased and
  inaccurate. We can now not only get a complete picture of the data, but
  also link and merge with other sources of information such as US Census
  data on religion, race, education, or income---work that has been
  lucrative for firms such as Catalist or Target Smart. By posing
  Political Scientific questions, and trying to respond with rigorous
  statistics, both disciplines tackle these data to face joint problems
  such as quantifying the quality of voter registration files
  (Ansolabehere \& Hersh, 2010), or linking disparate voter records
  (Ansolabehere \& Hersh, 2017).
  
  \chapter{The State of the Literature}\label{rmd-basics}
  
  In this chapter I will go through the existing literature on
  Vote-By-Mail (VBM).I will first go through some general literature on
  theories of voting decisions. I will define what Vote-By-Mail is; I will
  then summarize the expectations that researchers have of the effects of
  VBM on turnout, based on existing theories of electoral participation. I
  will continue with a summary of previous quantitative research on the
  effects that VBM and similar policies have had on turnout. I will
  conclude with some more general comments on the available data, and
  literature concerning the most commonly used quantitative methods.
  
  \section{Deciding to Vote}\label{deciding-to-vote}
  
  \subsection{Why Turnout Matters}\label{why-turnout-matters}
  
  Turnout is the most commonly used measure for electoral participation.
  It is important because it signifies the level of engagement of the
  population with the state, the level of incorporation of different
  subgroups of the population into democratic processes, and the
  legitimacy of elected officials. It is widely accepted that turnout
  should be maximized so that the democratic franchise represents the
  majority of citizens. Turnout for an election can be calculated or
  predicted, the difference being that in the former case we use data
  post-election to measure its absolute value, while in the latter we use
  a series of individual and community covariates to infer the levels of
  turnout for a future or past election.
  
  Calculating turnout, at its core, involves the following equation:\\
  \[ \% ~Turnout = \frac{Total~Ballots~Cast}{Measure~of~Total~Voting~Population}\times100\%~~~(1)\]
  
  The choice of numerator is fairly obvious and universal; the
  denominator, however, is a different story. The two main statistics used
  are the total voting age population, and the raw number of registered
  voters in the geographical location we are examining. The total voting
  age population is used as a measure to incorporate the total amount of
  possible voters in a geographical area, and can be measured using data
  from the US Census. This causes some issues with voters that cross over
  to different districts; if someone lives in district A, it is still
  likely that they are registered to vote in district B. If this is not
  considered, the calculation of voting age population might be
  misrepresentative.
  
  Using registered voters also brings with it two problems. First, the
  calculation necessarily occurs using voter registration files, which
  many times can include discrepancies, like deceased voters, voters
  included in multiple counties, or individual voters included multiple
  times. Furthermore, the total amount of actual voters among registered
  voters can be misrepresentative of democratic participation; consider
  that if a certain minority community has historically low registration
  rates, their lack of engagement will not be included in turnout rates,
  thus misrepresenting the level of inclusion in the district they reside
  in.
  
  The punch line here is that how the turnout statistic is calculated is
  not a clear choice, and will have an impact on how studies are set up.
  To give one example, consider Oregon's Motor Voter program, that
  automatically registers voters when they interact with government
  services, like the DMV. It is conceivable that this reform will
  \emph{decrease} turnout when measured as a percentage of the total
  registered voter count, but \emph{increase} turnout when measured
  against total population. I will specify how I calculate turnout in the
  next chapter.\textbf{Need sourcing for this}
  
  Statistical models of turnout can be constructed at either the
  individual or community level. At the individual level, a model is built
  to predict the probability of voting for every member of a group, and
  then sum over the members to create an estimate for turnout. Probit or
  Logit models are preferred. At the community level, researchers first
  choose a geographical level at which to calculate, which then
  constitutes the individual observation in the data that is used to
  create the model.
  
  Both these models include a standard set of societal variables--at the
  individual and aggregate level--, policy variables--whether the district
  does Postal Voting, whether Voter ID requirements are particularly
  strict--, election-specific variables--closeness of election or campaign
  expenditure--and sometimes time-series data--previous levels of
  turnout--to make predictions on turnout levels. This type of analysis is
  not exclusively used to predict turnout but also to, as will be later
  shown, draw inferences on the effects that certain explanatory variables
  have on electoral participation.
  
  Through meta-analyses on studies of turnout, it is possible to get a
  clear picture on what variables effect individual and collective choices
  to turn out. Three such studies are conducted by Geys (2006), Geys and
  Cancela (2016), and Smets (2013). Geys includes 83 studies of national
  US elections in his initial meta-analysis (Geys, 2006), later increasing
  that number to 185 (Geys and Cancela, 2016) and adding local elections.
  On aggregate-level models for national elections they conclude that
  competitiveness, campaign financing, and registration policy have the
  most pronounced effects, while on the sub-national level there are more
  pronounced effects for societal variables and characteristics of
  election administration (spending, voting policy, etc.). Smets and Van
  Ham (2013) examine individual-level predictors for turnout in a similar
  meta-analysis, and conclude that ``age and age squared, education,
  residential mobility, region, media exposure, mobilization (partisan and
  nonpartisan), vote in previous election, party identification, political
  interest, and political knowledge'' (Smets \& Ham, 2013) are the most
  significant explanatory variables for turnout, along with income and
  race. \textbf{\emph{I will add sources from books here}}. I will specify
  the model I will use for turnout in the second chapter.
  
  \subsection{Theories of Voting}\label{theories-of-voting}
  
  Here I take one step back from turnout, and examine the theories
  surrounding individual choices to vote or abstain. There are three main
  theories outlined in the literature on why individuals chose to vote.
  While there is some overlap, the following are mostly distinct:
  
  \begin{itemize}
  \item
    \emph{Decision ``at the margins''}: In his 1993 study, Aldrich posits
    that voting is a low cost-low benefit behavior. Therefore, he
    continues, voting is a decision that individuals make ``at the
    margins''; in most people, the urge to vote is not overwhelmingly
    strong, and therefore individuals will vote when it is convenient to
    them, when they are motivated by a competitive race, when policies are
    put in place to help them, and when they are subjected to GOTV (Get
    Out the Vote) efforts. For Aldrich, this is corroborated by the fact
    that most turnout models present consistent, yet weak, relational
    variables; if decisions are made ``at the margins'', then no single
    predictor would have an overwhelming result. This is also supported by
    Matsusaka (1997), and Burden \& Neiheisel (2012). Matsusaka expresses
    support for a more ``random'' process of voting, where turnout models
    are ambiguous because of the difficulty that predicting ``at the
    margins'' entails (Matsusaka \& Palda, 1999). Burden \& Neiheisel
    (2013) also demonstrate support for Aldrich's thesis by using data
    from Wisconsin to calculate a net negative effect of 2\% on turnout
    due to a similar slight shift in turnout.(Aldrich, 1993; Neiheisel \&
    Burden, 2012)
  \item
    \emph{Habitual Voting}: While Aldrich supports that there is no single
    overwhelming predictor of turnout, Fowler (2006) posits that future
    voting behavior can be strongly predicted using individual voting
    history. This leads to the conclusion that individuals are set to
    either be habitual voters, or habitual non-voters (Plutzer, 2002) by
    their upbringing and social circumstances, locking them into distinct
    groups. (Fowler, 2006)
  \item
    \emph{Social/Structural Voting}: Close to habitual voting are those
    that support a model of social and structural voting; these
    researchers claim that the decision to vote or not is deeply rooted in
    socioeconomic factors, which means that the divide between
    traditionally voting and non-voting groups can only be bridged by
    directly dealing with the socioeconomic divide between them (Berinsky,
    2005; Edlin, Gelman, \& Kaplan, 2007 ). Their reasoning is that ``at
    the margins'' voting only addresses groups that do not face
    significant burdens against voting--like the working poor, or
    marginalized racial groups--, and are usually already registered.
    Similarly, they address habitual voting claims by arguing that they
    are too short-sighted; individuals themselves might be habitually
    voting, but their decision to do so is rooted in strong societal and
    policy factors.
  \end{itemize}
  
  \section{From Theory to Policy}\label{from-theory-to-policy}
  
  \subsection{Voting Styles}\label{voting-styles}
  
  I have already flagged in my introduction the reason why theories behind
  voting choice matter: each construct an image of the electorate that
  reacts differently to policy change around voting. They are all an
  answer to the fact that voting policy, and how we conduct elections, is
  not value neutral but has implications for turnout, which in turn has
  implications on the franchise of democracy.
  
  In trying to respond to the issues set up by theoretical paradigms,
  different states--both in the world and US contexts--have adapted to
  different ways of conducting elections. In the US, voting styles can be
  simplified into three categories:
  
  \begin{itemize}
  \item
    \emph{In-Person Election Day}, for which all individuals are required
    to vote at a polling place, on a single election day. There can be
    some leeway for overseas voters, or excused absentee voters, but the
    vast majority of people will have to be present to vote in a
    particular time frame.
  \item
    \emph{In-Person Early Voting}, for which all individuals must vote in
    person at a polling place or vote center, but the timeframe for voting
    extends for around two weeks, not a single day.
  \item
    \emph{Vote-By-Mail, Absentee Early Voting}, for which individuals have
    a clear, no-excuse-necessary option for not being present when they
    vote, or for filing in a mailed ballot and droping it off at
    designated locations.
  \end{itemize}
  
  For the purposes of this thesis I will examine the latter category, and
  specifically Vote-By-Mail. The reason behind this is that the model of
  in-person, election day voting is usually seen as the baseline, the
  ``vanilla'' way of conducting elecions if you will. Therefore it has
  been of interest for researchers to examine if other systems can
  outperform that baseline. Specifically, it is most interesting to
  examine voting styles that are heralded for their expansion of turnout,
  to see whether popular beliefs on their benefits and drawbacks hold; if
  they are different from the base model of conducting American elections,
  or if they present new challenges and unique selling points.
  Vote-By-Mail is particularly intersting because it is quickly taking the
  form of a trend in state elections, as more and more states are
  enforcing more open models of VBM. In the next section, I will more
  closely examine the particulars of Vote-By-Mail.
  
  \subsection{What is VBM?}\label{what-is-vbm}
  
  Vote-By-Mail is a process by which voters receive a ballot delivered by
  mail to their homes. Voters then have a variety of options on how to
  return these ballots, ranging from dropping them off at pre-designated
  locations, to mailing them in, to bringing them to a polling place and
  voting conventionally. This varies across states that have implemented
  VBM. Some common forms of the VBM policy are:
  
  \begin{itemize}
  \tightlist
  \item
    \emph{Postal Voting}: All voters receive a ballot by mail, which can
    then be returned to a pre-designated location or mailed in to be
    counted. This is the current system in Oregon, is an option in
    Colorado, and is implemented by a number of counties in California,
    Utah, and Montana.\\
  \item
    \emph{No-Excuse Absentee}: Voters can choose to register as absentee
    voters without giving any reason related to disability, health,
    distance to polling place etc. This is the case in 27 states and the
    District of Columbia.\\
  \item
    \emph{Permanent No-Excuse Absentee}: This is similar to the previous
    system, but allows voters to register as absentees indefinitely,
    without having to renew their registration each year; they become de
    facto all-mail voters. This is in place in Washington, Kansas, and New
    Jersey.\\
  \item
    \emph{Hybrid or Transitionary Systems}: In hybrid systems, voters
    receive a mail ballot but can choose to disregard it and vote
    conventionally. This is the case in Colorado. Transitional systems
    exist in states that have chosen to eventually conduct all elections
    by postal voting, but have given counties an adjustment period during
    which this shift is not mandatory, or mandatory only for certain
    elections. This is the case in California, Utah, and Montana.
  \end{itemize}
  
  Vote-By-Mail is also commonly considered a type of early voting, since
  voters receive their ballots around two weeks in advance of election
  day; they are also able to return that ballot whenever they wish within
  that time-frame. This means that Vote-By-Mail can be counted as a
  ``convenience voting'' reform. These are usually implemented by state
  and local governments with the argument that they either expand the
  democratic franchise by bringing in new voters, or by making it more
  likely that current registered voters participate in the electoral
  process.
  
  \subsection{How Theories Apply to VBM}\label{how-theories-apply-to-vbm}
  
  Under Aldrich's paradigm, vote by mail would not effect significant
  change in voting behaviour. The whole concept of a decision ``at the
  margins'' is that the forces at play when an individual decides to vote
  are overwhelmingly strong both ways, so any effect that policy can have
  will minimaly shift these margins. If, for example, we take a
  presidential eleciton the forces at play include the media, national
  committees, social effects etc. In this environment some added
  convenience does not significantly add to an individual's decision to
  turn out. However, this would indicate that at a local level, where
  national and media effects are less strong, the effect of VBM on turnout
  might be more significant. The effect would be present for all groups,
  not only those currently registered, since voting would be easier
  uniformly.
  
  If we asume habitual voting, the conclusion on VBM would differ
  significantly. In this case, the effect to be considered is how VBM
  impacts already formed habits around voting. It could be argued that VBM
  has no effect, which follows if we assume that voting habits formed do
  not shift if the mode of voting changes. It could also be argued that
  VBM might have a negative effect on turnout in the short term, because
  it disrupts the habit of election day for a readjustment period, before
  people settle into new groups of habitual voters and non-voters, adapted
  to the new policy context.
  
  Under social and structural voting contexts, VBM retains rather than
  stimulates new voters (Berinsky, 2005). This means that already
  registered and semi-active voters are more likely to participate, but
  there is no significant change in the amount of new voters entering the
  franchise. This would mean that traditional forms of voting policy that
  emphasize access to the polls will do nothing to bring in
  disenfranchised people, and potentially hide the problem under an
  inflated turnout statistic calculated on registered voters. Berinsky in
  particular emphasizes the need for a shift towards voter education,
  rather than early voting or VBM policies (Berinsky, 2005).
  
  \section{Previous Study Results}\label{previous-study-results}
  
  In this section I will go through previous results from studies of
  Vote-By-Mail. I will also include a series of studies that are not
  necessarily about VBM, but have either been conducted in Vote-By-Mail
  states, or have to do with early voting which, as I have mentioned, is
  frequently linked to VBM. Most studies include a set of models or
  predictions of turnout, which are split into individual or county level
  results. I will group the studies according to whether the result shows
  a negative or positive effect on turnout.
  
  \subsection{General Results on VBM}\label{general-results-on-vbm}
  
  I will start with studies that show a negative effect on turnout.
  Bergman (2011) uses a series of logit models of individual voting
  probability in California, during a period where part of the state
  conducted VBM elections, while others maintained traditional voting.
  This is called a ``quasi-experiment'', and is frequent throughout the
  literature. Bergman's results show a statistically significant drop in
  voting probability in VBM counties (Bergman \& Yates, 2011). Using a
  similar method, Keele (2018) takes a single city in Colorado, Basalt
  City, which is divided into two different voting districts using
  different voting systems. The conclusion is, again, a 2-4\% drop in
  turnout along the VBM part of the city (Keele \& Titiunik, 2017). Burden
  et al. (2014) takes a different approach, using country-wide election
  data from 2004 and 2008 presidential elections, and compares districts
  based on early voing practices. Their results show a significant drop in
  turnout, which can be associated to VBM as well due to its closeness to
  EV (Burden, Canon, Mayer, \& Moynihan, 2014).
  
  In contrast, Gerber et al. (2013), applying both individual and
  county-level models for the state of Washington, reach the conclusion
  that VBM increases turnout by around 2-4\%; they use the same
  quasi-experimental model that offers itself to researchers in states
  that are under transitionary systems (Gerber, Huber, \& Hill, 2013).
  R.M. Stein also reaches a similar conclusion when examining Colorado's
  practice of ``vote centers'', which are non-precinct attached polling
  places, which can service multiple counties (Stein \& Vonnahme, 2008). I
  include this paper here due to the link that voting centers have with
  VBM, as they serve as drop-off points for mail-in ballots. Richey (2008)
  examines the effects that Oregon's VBM program has on turnout by using
  past elections data, concluding a 10\% positive trend associated with
  the policy (Richey Sean, 2008). This effect is studied again by Gronke
  et al.(2012) who find a similar positive effect with much lower
  magnitude, which might point to a novelty effect: the existence of
  diminishing returns in turnout after the implementation of this policy
  (Gronke \& Miller, 2012). Gronke et al. (2017), again studying Oregon
  but focusing on Oregon's Motor Voter program, find evidence of positive
  association to turnout {[}@{]}. I include these effects due to Oregon
  being an exclusively VBM state, and because this paper uses a
  ``synthetic control group'' model, which \textbf{might} be discussed in
  a following section. Lastly, I include a study conducted by Pantheon
  Analytics on Colorado, which compares actual turnout to predicted levels
  for VBM counties in Colorado. The results show a positive effect of
  approximately 3.3\% due to VBM (Edelman \& Glastris, 2018).
  
  The conclusion to be drawn from this section is that results on VBM vary
  significantly. There are multiple studies, using multiple methods, on
  multiple states, with multiple results. This only adds to the importance
  of being careful when constructing models and hypotheses to test VBM's
  effects on turnout, as assumptions made in the process can critically
  impact the results.
  
  \subsection{The Gerber Piece}\label{the-gerber-piece}
  
  This is conditional on if I do a replication of Gerber.
  
  \section{Voter Registration Files}\label{voter-registration-files}
  
  \subsection{Inaccuracy of Survey Data}\label{inaccuracy-of-survey-data}
  
  Apart from Voter Registration Files, the main source of data on the
  American electorate is national surveys, like the American NAtional
  Election Studies' survey (ANES), or the Cooperative Congressional
  Elections Study (CCES). These are post-election surveys, distributed to
  voters, which include fields associated directly with
  voting--participation, precinct, which party you voted for--and
  indirectly, through questions on societal factors like race, income, or
  gender. On the surface these seem like a better source of data, since no
  record linkage or ecological inference need be made to connect
  individual voters with an extensive list of covariates. There is,
  however, a significant problem with these data: survey misreporting.
  
  Even without resorting to advanced statistical or data gathering
  methods, the fact that the CCES and NES often misrepresent the
  electorate is apparent just through looking at turnout statistics; both
  show higher turnout than what the true value, calculated from the
  population, was. When looking at surveys a bit closer, using either
  private, extensive data files like Catalyst (Ansolabehere \& Hersh,
  2012) or validated voter files from the late 20th century (Deufel \&
  Kedar, 2010), the results show consistent misreporting among certain
  groups, that tend to either be politically engaged non-voters or
  minorities and low socioeconomic status individuals. This gap, according
  to Deufel et al. (2010), has served to propagate societal stereotypes
  and class entrenchment into studies on turnout, which in turn
  negativelly effect policy, since research using the ANES and CCES are
  widely used to study turnout among the groups that are consistently
  misreporting. Admittedly, the fact that misreporting happens among
  specific groups does open the way for statistical methods to compensate
  for the bias intoduced, but for the purpose of my thesis I will prefer
  the use of VRF.
  
  A last issue with surveys worth mentioning is that they are contingent
  on quantity of responses as well as quality. There is no guarantee that
  the CCES or NES will receive enough responses to correctly infer
  population-wide statistics; something which is more likelly for the
  American Community Survey or the Census, whcih are backed by the
  legitimacy of the federal government. Survey under-reporting is directly
  linked with the practices of the groups conducting the survey, and as
  such is hard to control for after the results are published (Burden,
  2000).
  
  \subsection{The Importance of VRF}\label{the-importance-of-vrf}
  
  As mentioned in my introduction, access to voter registration files has
  provided researchers with unique insight into the voting process.
  Quantitative research has expanded significantly, for three key reasons.
  First, VRF data exists in a consolidated, state-wide format at least for
  national elections. This means that the process of data collection
  involves interaction with significantly fewer government agencies, and a
  data wrangling process that can be quickly adapted to a set format. This
  is, of course, not to say that the process of data collection and
  handling doesn't still pose a significant challenge, as will become
  apparent in my second chapter. Second, there is a huge benefit attached
  to the fact that VRF data describes the whole population, rather than a
  sample. As mentioned in the previous section, survey data might give
  more insight into variables not included in VRF, but that comes at a
  steep cost for accuracy. Using VRF, the problem of self-reporting bias
  is eliminated for some studies, and transformed into a problem of record
  linkage and ecological inference for others (Ansolabehere \& Hersh,
  2017, Burden \& Kimball (1998)). Third, wide public access means
  reproducibility and accessibility, which translates into greater
  accountability for researchers. This effect is important, even if
  mitigated somewhat by private data companies and access fees.
  
  \section{Common Methods Used and Problems
  Encountered}\label{common-methods-used-and-problems-encountered}
  
  \subsection{Methods}\label{methods}
  
  \begin{itemize}
  \tightlist
  \item
    \emph{Synthetic Control Group}: Abadie (2010), McClleland (2017),
    Gronke (2017)
  \item
    \emph{Record Linkage}: Ansolabehere and Hersch (2017), Harvey (1994,
    97), Koudas (2013)
  \item
    \emph{GLM (Probit/Logit/Poisson)}: Barreto (2004), Dow (2004)
  \item
    \emph{DID}: Bertrand et al. (2002)
  \item
    \emph{E.I.}: King (2013), Burden (1998), Calvo (2003), Chao (2004), Rm
    Stein (2002)
  \item
    \emph{Mixed-Effects}: Gelman and Hill (2007)
  \item
    \emph{General EDA and Models}: James et al. (2013), Chapman and Hall
    (2017)
  \end{itemize}
  
  \subsection{Issues}\label{issues}
  
  \emph{Grimmer (2015) \{Not always best to do inferential models\},
  Ansolabehere and Hersch (2010) \{Problems with Voter Reg Files\}}
  
  \chapter{Data Collection and
  Description}\label{data-collection-and-description}
  
  \section{Colorado}\label{colorado}
  
  \subsection{Demographics and
  Characteristics}\label{demographics-and-characteristics}
  
  \subsection{Voting in Colorado}\label{voting-in-colorado}
  
  \section{The Data}\label{the-data}
  
  \subsection{Source}\label{source}
  
  \subsection{First-Glance Description}\label{first-glance-description}
  
  \subsection{Wrangling Difficulties and
  Solutions}\label{wrangling-difficulties-and-solutions}
  
  \chapter{Hypothesis and Methods}\label{hypothesis-and-methods}
  
  \section{Hypotheses}\label{hypotheses}
  
  \subsection{Description of Hypotheses}\label{description-of-hypotheses}
  
  \subsection{Criteria}\label{criteria}
  
  \subsection{Expected Results}\label{expected-results}
  
  \section{Methodology}\label{methodology}
  
  \subsection{EDA}\label{eda}
  
  \subsection{Description and Parametrization of
  Models}\label{description-and-parametrization-of-models}
  
  \section{Gerber Replication}\label{gerber-replication}
  
  \chapter{Results}\label{results}
  
  \section{EDA}\label{eda-1}
  
  \section{Model Output}\label{model-output}
  
  \section{Gerber Expansion Results}\label{gerber-expansion-results}
  
  \chapter*{Conclusion}\label{conclusion}
  \addcontentsline{toc}{chapter}{Conclusion}
  
  \setcounter{chapter}{4} \setcounter{section}{0}
  
  \appendix
  
  \backmatter
  
  \chapter{References}\label{references}
  
  \noindent
  
  \setlength{\parindent}{-0.20in} \setlength{\leftskip}{0.20in}
  \setlength{\parskip}{8pt}
  
  \hypertarget{refs}{}
  \hypertarget{ref-aldrich_rational_1993}{}
  Aldrich, J. H. (1993). Rational choice and turnout. \emph{American
  Journal of Political Science}, \emph{37}(1), 246--278.
  \url{http://doi.org/10.2307/2111531}
  
  \hypertarget{ref-ansolabehere_quality_2010}{}
  Ansolabehere, S., \& Hersh, E. (2010). The quality of voter registration
  records: A state-by-state analysis. \emph{Institute for Quantitative
  Social Science and Caltech/MIT Voting Technology Project Working Paper}.
  Retrieved from
  \url{https://dataverse.harvard.edu/dataset.xhtml?persistentId=hdl:1902.1/18550}
  
  \hypertarget{ref-ansolabehere_validation:_2012}{}
  Ansolabehere, S., \& Hersh, E. (2012). Validation: What big data reveal
  about survey misreporting and the real electorate. \emph{Political
  Analysis}, \emph{20}(4), 437--459.
  \url{http://doi.org/10.1093/pan/mps023}
  
  \hypertarget{ref-ansolabehere_adgn:_2017}{}
  Ansolabehere, S., \& Hersh, E. D. (2017). ADGN: An algorithm for record
  linkage using address, date of birth, gender, and name. \emph{Statistics
  and Public Policy}, \emph{4}(1), 1--10.
  \url{http://doi.org/10.1080/2330443X.2017.1389620}
  
  \hypertarget{ref-bergman_changing_2011}{}
  Bergman, E., \& Yates, P. A. (2011). Changing election methods: How does
  mandated vote-by-mail affect individual registrants? \emph{Election Law
  Journal: Rules, Politics, and Policy}, \emph{10}(2), 115--127.
  \url{http://doi.org/10.1089/elj.2010.0079}
  
  \hypertarget{ref-berinsky_perverse_2005}{}
  Berinsky, A. J. (2005). The perverse consequences of electoral reform in
  the united states. \emph{American Politics Research}, \emph{33}(4),
  471--491. \url{http://doi.org/10.1177/1532673X04269419}
  
  \hypertarget{ref-burden_voter_2000}{}
  Burden, B. C. (2000). Voter turnout and the national election studies.
  \emph{Political Analysis}, \emph{8}(4), 389--398.
  \url{http://doi.org/10.1093/oxfordjournals.pan.a029823}
  
  \hypertarget{ref-burden_new_1998}{}
  Burden, B. C., \& Kimball, D. C. (1998). A new approach to the study of
  ticket splitting. \emph{American Political Science Review},
  \emph{92}(3), 533--544. \url{http://doi.org/10.2307/2585479}
  
  \hypertarget{ref-burden_election_2014}{}
  Burden, B. C., Canon, D. T., Mayer, K. R., \& Moynihan, D. P. (2014).
  Election laws, mobilization, and turnout: The unanticipated consequences
  of election reform. \emph{American Journal of Political Science},
  \emph{58}(1), 95--109.
  
  \hypertarget{ref-deufel_race_2010}{}
  Deufel, B. J., \& Kedar, O. (2010). Race and turnout in u.S. elections
  exposing hidden effects. \emph{Public Opinion Quarterly}, \emph{74}(2),
  286--318. \url{http://doi.org/10.1093/poq/nfq017}
  
  \hypertarget{ref-edelman_analysis_2018}{}
  Edelman, G., \& Glastris, P. (2018). Analysis letting people vote at
  home increases voter turnout. here's proof. \emph{Washington Post}.
  Retrieved from
  \url{https://www.washingtonpost.com/outlook/letting-people-vote-at-home-increases-voter-turnout-heres-proof/2018/01/26/d637b9d2-017a-11e8-bb03-722769454f82_story.html}
  
  \hypertarget{ref-edlin_voting_2007}{}
  Edlin, A., Gelman, A., \& Kaplan, N. (2007). Voting as a rational
  choice: Why and how people vote to improve the well-being of others.
  \emph{Rationality and Society}, \emph{19}(3), 293--314.
  \url{http://doi.org/10.1177/1043463107077384}
  
  \hypertarget{ref-fowler_habitual_2006}{}
  Fowler, J. H. (2006). Habitual voting and behavioral turnout.
  \emph{Journal of Politics}, \emph{68}(2), 335--344.
  \url{http://doi.org/10.1111/j.1468-2508.2006.00410.x}
  
  \hypertarget{ref-gerber_identifying_2013}{}
  Gerber, A. S., Huber, G. A., \& Hill, S. J. (2013). Identifying the
  effect of all-mail elections on turnout: Staggered reform in the
  evergreen state\textless{}a
  href=``\#fn2606''\textgreater{}*\textless{}/a\textgreater{}.
  \emph{Political Science Research and Methods}, \emph{1}(1), 91--116.
  \url{http://doi.org/10.1017/psrm.2013.5}
  
  \hypertarget{ref-geys_explaining_2006}{}
  Geys, B. (2006). Explaining voter turnout: A review of aggregate-level
  research. \emph{Electoral Studies}, \emph{25}(4), 637--663.
  \url{http://doi.org/10.1016/j.electstud.2005.09.002}
  
  \hypertarget{ref-gronke_voting_2012}{}
  Gronke, P., \& Miller, P. (2012). Voting by mail and turnout in oregon:
  Revisiting southwell and burchett. \emph{American Politics Research},
  \emph{40}(6), 976--997. \url{http://doi.org/10.1177/1532673X12457809}
  
  \hypertarget{ref-keele_geographic_2017}{}
  Keele, L., \& Titiunik, R. (2017). Geographic natural experiments with
  interference: The effect of all-mail voting on turnout in colorado.
  
  \hypertarget{ref-matsusaka_voter_1999}{}
  Matsusaka, J. G., \& Palda, F. (1999). Voter turnout: How much can we
  explain? \emph{Public Choice}, \emph{98}(3), 431--446.
  \url{http://doi.org/10.1023/A:1018328621580}
  
  \hypertarget{ref-neiheisel_impact_2012}{}
  Neiheisel, J. R., \& Burden, B. C. (2012). The impact of election day
  registration on voter turnout and election outcomes. \emph{American
  Politics Research}, \emph{40}(4), 636--664.
  \url{http://doi.org/10.1177/1532673X11432470}
  
  \hypertarget{ref-plutzer_becoming_2002}{}
  Plutzer, E. (2002). Becoming a habitual voter: Inertia, resources, and
  growth in young adulthood. \emph{The American Political Science Review},
  \emph{96}(1), 41--56. Retrieved from
  \url{http://www.jstor.org/stable/3117809}
  
  \hypertarget{ref-richey_sean_voting_2008}{}
  Richey Sean. (2008). Voting by mail: Turnout and institutional reform in
  oregon*. \emph{Social Science Quarterly}, \emph{89}(4), 902--915.
  \url{http://doi.org/10.1111/j.1540-6237.2008.00590.x}
  
  \hypertarget{ref-smets_embarrassment_2013}{}
  Smets, K., \& Ham, C. van. (2013). The embarrassment of riches? A
  meta-analysis of individual-level research on voter turnout.
  \emph{Electoral Studies}, \emph{32}(2), 344--359.
  \url{http://doi.org/10.1016/j.electstud.2012.12.006}
  
  \hypertarget{ref-stein_engaging_2008}{}
  Stein, R. M., \& Vonnahme, G. (2008). Engaging the unengaged voter: Vote
  centers and voter turnout. \emph{The Journal of Politics}, \emph{70}(2),
  487--497. \url{http://doi.org/10.1017/S0022381608080456}
  
  \hypertarget{ref-robert_nay_help_2002}{}
  The help america vote act of 2002, Pub. L. No. HR3529 (2002).


  % Index?

\end{document}

